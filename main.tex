\documentclass[a4paper,11pt,twoside]{include/ThesisStyle}
%\documentclass[10pt,twoside]{include/ThesisStyle}

% =============================================================================


\newcommand{\thesistitle}{Virtualization Support\\for\\Application Runtime Specialization and Extension}
\newcommand{\thesisauthor}{Guillermo Polito}
\newcommand{\thesisurl}{http://rmod.inria.fr/}
\newcommand{\thesissubtitle}{}
\newcommand{\thesisdate}{13. April 2015}

% =============================================================================

\input{include/header}
\graphicspath{{.}{figure/}}

% Uncomment to disable the comments
\setboolean{showcomments}{false}

\begin{document}

% =============================================================================
\pagenumbering{alph}
\include{include/creativecommons}
\input{chapter-header.tex}


\newcommand \logoInria{../include/logos/Inria}
\newcommand \logoLifl {../include/logos/logoCRIStAL}
\newcommand \logoUSTL {../include/logos/logo-lille1}
\newcommand \logoMinesDouai {../include/logos/Mines_Douai_grand}
	
% =============================================================================
\ThesisTitle{\thesistitle \\\vspace{5mm}
{\Large Virtualisation pour Specialisation et Extension d'Environnements d'Execution}}
\ThesisDate{\thesisdate}
\ThesisAuthor{\thesisauthor} 
\ThesisOrderId{41719}
\ThesisLilleI

\President{Loïc LAGADEC   \hspace{12.5mm} Professeur -- ENSTA Bretagne} 
\Rapporteurs{Oscar NIERSTRASZ   \hspace{4.5mm} Professeur -- Universität Bern \\\\
Christophe DONY   \hspace{8.5mm} Professeur -- Universit\'e Montpellier}
\Examinateurs{Jan VITEK \hspace{20mm} Professeur -- Northeastern University } 

\Directeur{
	Stéphane DUCASSE  \hspace{5mm} Directeur de recherche -- INRIA Lille Nord-Europe
}

\Coencadreur{Noury BOURAQADI \hspace{4mm} Professeur -- Mines Douai\\\\
Luc FABRESSE  \hspace{12.5mm} Enseignant-Chercheur -- Mines Douai}

\MakeThesisTitlePage 

% =============================================================================
\thispagestyle{empty}
\vspace*{\fill}


\noindent Copyright \copyright{} 2015 by \thesisauthor \\

\noindent RMoD \\
Inria Lille -- Nord Europe \\
Parc Scientifique de la Haute Borne\\
40, avenue Halley \\
59650 Villeneuve d'Ascq \\
France \\
\url{http://rmod.inria.fr/}

% =============================================================================

\vskip 1cm

\parbox{\textwidth}{
\mbox{
    \CcGroupBySa{scale=1.0}{0.95ex}
}
\parbox[c]{0.75\textwidth}{
	This work is licensed under a \href{http://creativecommons.org/licenses/by-sa/4.0/}{\emph{Creative Commons Attribution–ShareAlike 4.0 International License}}.
}}


% =============================================================================
\input{chapter-footer.tex}

% =============================================================================
\pagenumbering{roman}
\dominitoc

% =============================================================================
\cleardoublepage
\chapter*{Acknowledgments}

I would like to thank my thesis supervisors Stéphane Ducasse, Noury Bouraqadi and Luc Fabrese for allowing me to do in the first place a Ph.D at the RMoD and CAR teams, and for their advice during this thesis.\\

\noindent I thank the thesis reviewers and jury members Oscar Nierstrasz, Christophe Dony, Jan Vitek and Loïc Lagadec for accepting being part of the jury of my defense.% reviewing my thesis and providing me valuable feedback.\\

%\noindent I would like to express my gratitude to Igor Stasenko for providing \B and \NB, and Guido Chari for allowing me to use \WF for validation purposes.\\

\noindent I specially thank Santiago, Pablo, Alexis, Esteban, Camille, Charlotte and all others that game me their support me during the time of this thesis.\\

\noindent I would also like to thank the Pharo community for their support and help during my learning. I expect part of this thesis is of value for them.\\
%\noindent For remarks on earlier versions of this thesis I thank Stefan Marr and Damien Pollet.



% =============================================================================
\chapter*{Abstract}

An application runtime is the set of software elements that represent an application during its execution. Application runtimes should be adaptable to different contexts. Advances in computing technology both in hardware and software indeed demand it. For example, on one side we can think on extending a programming language to enhance the developers' productivity. On the other side we can also think on transparently reducing the memory footprint of applications to make them fit in constrained resource scenarios \eg low networks or limited memory availability.

We propose \Vtt, a virtualization infrastructure for object-oriented high-level languages runtimes. \Vtt provides a general purpose infrastructure to control and manipulate object-oriented runtimes in different situations. A first-class representation of an object-oriented runtime, namely an \emph{object space}, provides a high-level API that allows the manipulation of such runtime and clarifies the contract between the language and the virtual machine. A hypervisor is the client of an object space and manipulates it either directly through mirror objects~\cite{Brac04b}, either by executing arbitrary expressions into it.

We implemented a \Vtt prototype on Pharo. We show with this prototype that this infrastructure supports language \emph{bootstrapping} and application runtime \emph{tailoring}. Using bootstrapping we describe an object-oriented high-level language initialization in terms of itself. A bootstrapped language takes benefit of its own abstractions and shows easier to extend. We bootstrapped four languages presenting different programming models \eg traits~\cite{Scha03a}, first-class instance variables~\cite{Verw11a} and mirror-based reflection~\cite{Brac04b}.  Application runtime tailoring is a technique that generates a specialized application by extracting the elements of a program that are used during execution. A tailored application encompasses only the classes and methods it needs and avoids the code bloat that appears from the usage of third-party libraries and frameworks. Our run-fail-grow tailoring technique based on \Vtt succeeds in creating specialized versions of applications, saving between a 95\% and 99\% of memory in comparison with Pharo's official distribution.

\paragraph{Keywords:} application runtimes, object-oriented, high-level, virtualization, first-class runtimes, object spaces, bootstrapping, tailoring.

% =============================================================================
\chapter*{Résumé}

Un environnement d'exécution est l'ensemble des éléments logiciels qui représentent une application pendant son éxecution. Les environnements d'exécution doivent être adaptables à differents contextes. Les progrès des technologies de l'information, tant au niveau logiciel qu'au niveau matériel, rendent ces adaptations nécessaires. Par exemple, nous pouvons envisager d'étendre un language de programmation pour améliorer la productivité des developpeurs. Aussi, nous pouvons envisager de réduire la consommation memoire des applications de manière transparente afin de les adapter à certaines contraintes d'exécution \eg des réseaux lents ou de la mémoire limités.

Nous proposons \Vtt, une infrastructure pour la virtualisation d'environnement d'execution de langages orienté-objets haut-niveau. \Vtt fournit une infrastructure généraliste pour le contrôle et la manipulation d'environnements d'exécution pour différentes situations. Une représentation de 'premier-ordre' de l'environnement d'exécution orienté-objet, que nous appelons \emph{object space}, fournit une interface haut-niveau qui permet la manipulation de ces environnements et clarifie le contrat entre le langage et la machine virtuelle. Un hyperviseur est client d'un object space et le manipule soit directement au travers d'objets 'miroirs'~\cite{Brac04b}, soit en y exécutant des expressions arbitraires.

Nous avons implementé un prototype de \Vtt sur Pharo. Nous montrons au travers de notre prototype que cet infrastructure supporte le \emph{bootstrapping}~(\ie l'amorçage ou initialisation circulaire) des languages et le \emph{tailoring}~(\ie la construction sur-mesure ou 'taille') d'environnement d'éxecution. En utilisant l'amorçage nous initialisons un language orienté-objet haut-niveau qui est auto-décrit. Un langage amorcé profite des ses propres abstractions se montrant donc plus simple à étendre. Nous avons amorcé quatre langages qui présentent des modèles de programmation différents \eg avec des 'traits'~\cite{Scha03a}, avec des variables d'instance de 'premier-ordre'~\cite{Verw11a} ou avec une couche réflexive basé sur le concept de 'miroirs'~\cite{Brac04b}. La taille d'environnements d'éxecution est une technique qui génère une application spécialisé en extrayant seulement le code utilisé pendant l'éxecution d'un programme. Une application taillée inclut seulement les classes et méthodes qu'elle nécessite, et évite que des librairies et des frameworks externes surchargent inutilement la base de code. Notre technique de taille basé sur \Vtt, que nous appelons \emph{run-fail-grow}~(\ie éxecuter-échouer-grandir), créé des versions spécialisées des applications, en sauvant entre un 95\% et 99\% de la memoire en comparaison avec la distribution officielle de Pharo.

% =============================================================================
\renewcommand{\baselinestretch}{1}\normalsize
\tableofcontents
%\listoffigures
%\listoftables

% =============================================================================
\renewcommand{\baselinestretch}{1.2}\normalsize
\mainmatter
\pagenumbering{arabic}
% =============================================================================

\input{chapter-header.tex}
% =============================================================================
\chapter{Introduction}
\chaplabel{introduction}
\minitoc
% =============================================================================


%Reflective systems are those that reason about and act upon themselves \cite{Smit84a}. A causal connection exists between the program and its representation inside the program itself as a meta-program \cite{Maes87a}. This reflective architecture introduces self-references:  an object-oriented system is composed by objects, which are instances of classes, which are also objects, and so on. These self-references, also known as meta-circularities \cite{Chib96a}, allow the manipulation of several meta-levels on one infrastructure.
%
%Reflective systems traditionally modify their self-representation to evolve and define new abstractions. However, the self-modification approach of evolution has many drawbacks, such as making difficult the self-surgery operations~\cite{Casa09a} or the lose of the reproducibility of the system. On the other hand, non-reflective systems develop an evolution approach by recreation. Whenever a change has to be made to the system, a new system is created with the new changes applied. This approach solves many of the drawbacks of the reflective approach.
%
%\gp{add some sentences on why it is important to evolve, which kind of software artifacts we would like to evolve, why it is challenging}

% =============================================================================
%\section{The need for Software Evolution}
% =============================================================================

An application runtime is the set of software elements that denote an application during its execution. Narrowing to high-level object-oriented applications, an application runtime includes \eg loaded libraries, classes and methods, created objects and threads. Within an application runtime, we find the \emph{language runtime}. The language runtime is the subset of the application runtime that defines the concepts and behavior available in the language we use \ie the set of structures and constructs that describe the language internals. The language runtime implements the model that the language proposes to the developers.

Manipulating and modifying these application and language runtimes, and therefore their language models, is becoming important in the last years. Multicore hardware brought new problems on concurrency and parallelism; the \emph{cloud} increases the need of software adaptation for application migration and resource tailoring; new resource constrained devices such as phones are widely spread, making ubiquitous computing a real concern. These new technologies present new challenges to software and language developers. The software we use, and in particular the programming languages and tools we use should be easily tailorable to support many of the new challenges that come with new technology and needs.

We observe, however, that software is not usually designed and thought to easily accept changes. Applications and languages should be either engineered from scratch with change in mind, or a lot of reengineering effort should be invested in them to perform changes. We need tools and methodologies that support such changes~\cite{Nier08b}. The languages and applications we develop should be adaptable to new situations and scenarios.

To address this goal we propose \emph{\Vtt}: a runtime virtualization infrastructure. \Vtt virtualizes an application runtime for its control and manipulation with the ultimate goal of generating specialized versions of it. Manipulations are transparent for the virtualized runtime. We show how \Vtt simplifies the generation of application and language runtimes in two different approaches: language runtime (re)creation by bootstrapping allows us to modify and redefine the concepts a language proposes to its users; application runtime extraction reduces the memory consumption caused by the code units of an application.

%For example, an application runtime should be easily tailorable to consume less resources. We can observe that deployed applications contain a set of \emph{code units} such as classes and methods that tend to occupy more memory~(primary and secondary) than necessary.
%This problem shows itself more evident and harder to control under the usage of third party software. 
%Third party libraries and frameworks are designed in a generic fashion that allows multiple usages and functionalities, while applications use only few of them. 
%Examples are logging libraries, web application frameworks or object-relational mappers.
%Unused code units represent serious drawbacks in constrained devices. 
%First, unused code units may forbid the deployment into a constrained resource device.
%It may also interfere with the deployment and usage of other applications, because of large memory footprints in both secondary~(disk storage) and primary~(RAM) memory~\cite{Mart12a} or the presence of slow networks in the case of rich web applications.
%Second, some deployment targets may have an infrastructure designed in such a manner that forbids the deployment of large applications. For example, the Android's Dalvik VM restricts an application to deploy only 65536 methods.


% =============================================================================
%\section{The cloud and Mobile code}
% =============================================================================

%\gp{explain why code mobility is important!}

%Another example of support that should be brought to user applications is \emph{code mobility}. Code mobility is a mechanism that allows the migration of programs between different environments. This problem is important in the context of ubiquitous systems and virtualization technology. Code mobility provides support for \eg load balancing, adjusting an application's resources dynamically and functionality customization. However, applications must have support to rebind a piece of code or object to another location~\cite{Fugg98a}.

% =============================================================================
\section{Application Runtimes: Concepts and Terminology}
% =============================================================================

In the context of application runtime manipulation, we face the following question: \emph{What are the elements of high-level programming languages we should focus on?} High-level programs are inherent complex pieces of software. 
This section proposes a dissection of a high-level language application runtime and states with it the terminology used during the rest of this dissertation. We made this dissection with the objective of understanding the relationship between the software elements. We believe that understanding these relationships is important in the context of this thesis. Figure \ref{fig:whatToEvolve} shows a schema of this dissection.

\begin{figure}[!ht]
\begin{center}
\includegraphics[width=0.5\linewidth]{elements_to_evolve}
\caption{\textbf{Dissection of a running program.}\label{fig:whatToEvolve} }
\end{center}
\end{figure}

We identify inside a running application the following elements. First, the application runtime contains the \emph{application specific} and the \emph{language} runtimes that represent respectively the elements of the application under execution and the language that provides support for it. The Virtual Machine~(\VM) provides as well an execution support for the application runtime. The overlapping shows the interrelations between these three components. 

Note that some authors may include the \VM inside a broader runtime system. In this thesis, however, we narrow our study to those programs that run on top of a \VM because most of modern object-oriented languages~(\eg Java, Python, Ruby, JavaScript, C\#) are \VM-based. Additionally, this is the reason why we use the terminology \emph{application runtime} instead of runtime system and we consider this component separately from the \VM.

\subsection{Application Runtime}

The application runtime is the set of software elements that constitute an application during its execution. An application runtime includes software elements that describe the application's structure during its execution, such as its libraries, classes and methods, and elements related with the application's execution, such as threads, the execution stack with its activation records and created objects. In the context of this thesis, we do not consider the Virtual Machine as part of the application runtime, but as a machine that supports its execution.

The subset of the application runtime that describes the language we use is the \emph{language runtime}. The language runtime includes the set of structures and constructs that describe the language concepts and behavior. This language runtime is the representation at runtime of the model that the language proposes to the developers. For example, Smalltalk and Ruby propose that an application is structured in classes with implicit metaclasses \ie a class is instance of a (\emph{meta-})class that describes its behavior. Therefore, they also contain structures that represent the concepts of a \ct{Class} and \ct{Metaclass} and control the implicit creation of a metaclass for each new class. Additionally, a language runtime is not only composed of classes but it also includes objects. For example, it contains a table of unique strings or \emph{symbols} used to univocally name objects. A language runtime is usually common to different applications in the same language: it is always present and contains always the same elements.

Within an application runtime we identify also the \emph{application specific runtime}. The application specific runtime is the subset of the application runtime that includes the elements that belong to the particular application. It contains, in other words, the classes written by the application developer. The application specific runtime follows the model and concepts imposed by the language \ie it is expressed in terms of the language runtime, and thus coupled to it. For example, the application specific classes of a Smalltalk or Ruby application have an implicitly created metaclass.

\subsection{Virtual Machine}

High-level languages virtual machines~(\VMs) provide support for executing an application runtime.
Likewise, an application runtime must satisfy the \VM's interface to be executed.
This interface denotes the \emph{execution model} of a \VM and includes the format in which the runtime elements are laid out in memory~(objects, classes, methods), the followed object model (\eg class-based or prototype-based) and the instruction set that the runtime must implement.

\VMs concentrate several complex and interconnected elements with two main purposes: first it is to abstract applications from details such as memory management or machine specifics; second it is to do that while also obtaining good performance.
Early \VMs focused on interpreting an abstract instruction set (bytecodes).
On the one hand the bytecodes guarantee certain platform independence by abstracting away from the \CPU specific instruction set.
On the other hand bytecodes allow one to encode complex operations into little space both serving the hard memory constraints of the hardware and simplifying the design of a compiler.
Obviously this abstraction gain comes at a cost, and ever since the first \VMs were built, research and industry strive to reduce the interpretation overhead.

%This goes even so far that specialized hardware is conceived to match the performance requirements \cite{Unga84a,Stef84a,McGh98a,Clic05a}.

To improve performance some \VMs use a Just-In-Time compiler (\JIT) that dynamically generates native code from bytecode \cite{Deut84a}.
In this case the bytecode becomes an intermediate representation (\IR) for a bigger compiler infrastructure.
However, \JIT compilers are notoriously complex as they crosscut many \VM components and abstraction layers. They have to access high-level information from the running bytecodes and manage native code at the same time.
Similar complexity applies to the automatic memory management present in most high-level language \VMs.
Garbage Collectors (\GC) evolved from simple helpers to complex software artifacts that for instance support concurrent garbage collection \cite{Clic05a}.



%These complex \VMs are the enablers of many of the features in our programming languages.
%However, their complexity constrain the changes we can apply to our application runtimes.
%For example, adding a new field into the classes of an application runtime may impact the bytecode interpreter, the JIT, the GC, and so on.
%These constraints denote the \emph{execution model} of a \VM \ie the contract required to a language runtime to execute it.
%This execution model includes the format in which the runtime elements are layout in memory~(objects, classes, methods), the followed object model (\eg class-based or prototype-based) and the instruction set that the runtime must implement.

\section{Motivation: Changing Technologies}

\section{Problem Statement}

This thesis focuses on the generation of specialized high-level object-oriented application and language runtimes for languages that run on a \VM. In this context, we observe the following problems:

\begin{description}

\item[The \VM defines the language model.] Often, the \VM is the software component in charge of initializing the application and language runtime~(and thus its language model). Changing it requires us to have access to the \VM code, and forces us to modify such representation without the proper level of abstraction.

\item[Unclear \VM-Language interface.] The relationship between the language and the \VM, particularly its execution model, often remains unclear. This unclear interface comes from hardcoded assumptions in the multiple \VM components. Its clients are exposed to the complexities of the \VM internals when it comes to change the application runtime.


%\item[Mismatch between a language model and the \VM execution model.] The execution model imposed by a \VM is often more general than the particular languages that run of top of it. This means that a \VM can support different language models on top of it as long as they satisfy its execution model. However, the \VM often fixes the language model during its language initialization step, preventing us to easily modify it.

\end{description}

\noindent Then, we pose the following research question:

\begin{center}\emph{What is an infrastructure that can better support the creation of specialized application runtimes?}
\end{center}

We would like to have an infrastructure featuring safe application runtime manipulation, and allowing both the specialization and extension of application and language runtimes.

\section{Contributions}

To solve the stated problems, we make the following claim: \newline

\begin{center}\blockquote{\emph{First-class runtimes support better the manipulation of Application Runtimes and thus, the generation of specialized versions of them.}}
\end{center}

First-class runtimes should provide a clear and high-level \VM-Language interface. This interface must hide the internal details of the \VM complexities and allow us to easily manipulate the elements inside an application runtime. At the same time, it should ensure that the \VM execution model is honored during such manipulations.

The contribution of this thesis is three-fold. The main contribution is \Vtt, an \emph{infrastructure for application runtime virtualization}. This infrastructure allows us to manipulate and control a virtualized application runtime through a first-class runtime object, namely an object space.
We validate this language virtualization infrastructure by exploring two approaches for application runtime generation:
\begin{description}
\item[Language Runtime Bootstrapping.] We developed a bootstrapping process for an object-oriented high-level language. Bootstrapping is an explicit process that describes how a language runtime is created using the same language that it generates at the end. Bootstrapping provides us with full control on what are the elements installed in the application runtime at the end.
\item[Application Runtime Tailoring.] We developed a novel dynamic application runtime tailoring approach named Run-Fail-Grow~(RFG). RFG starts with an empty application runtime and it installs elements inside it as they are needed during at runtime. With RFG we can create a specialized runtime that contains only the used elements of an application.
\end{description}

% =============================================================================
\section{Thesis Outline}
% =============================================================================
%\sm{This dissertation structure is different to what I am used to. At least the way you announce the purpose of the chapters is not what I would expect.
%In my diss, everything revolves around one thesis, here, it is a number of things listed one after another, don't see the central motive I would expect}

\gp{write from scratch, left for the end}
\begin{description}
\item[\chapref{background}] 

\item[\chapref{benzo}] 
	
\item[\chapref{ffi}] 

\item[\chapref{validation}] 

\item[\chapref{conclusion}] 

\end{description}


% =============================================================================
\input{chapter-footer.tex}
% =============================================================================

\input{chapter-header.tex}
% ===========================================================================
\chapter{State of the Art}
\chaplabel{background}
\minitoc
% ===========================================================================
\introduction
% ===========================================================================
	
	
% ===========================================================================
\newpage
% ===========================================================================
\section{On Evolving a Language Kernel}

The evolution of a language kernel presents two main challenges to solve: changing the elements of compliance with its Virtual Machine and changing elements having an impact on the system's meta-circularities.

\subsection{Changing the Language-VM Interface}

\gp{maybe this first part explaining problems could be part of the problem statement in the introduction}
%\gp{not sure if defining metacircularities is useful}The causal connections define circular definitions in the system, namely \emph{metacircularities}~\cite{Chib96a, Denk08b}.
Modifying a language kernel to adapt it to these and other situations~(either to do bug fixing or add new features to it) becomes a cumbersome task for some of the following reasons: 
\begin{description}
\item[High amount of low level code.] The initialization of the language's circular definitions is often done in the VM by using a low level language, often a mixture of C, C++ and Assembly. This poses a high entry barrier to developers not versed in such kind of languages.
\item[Language definition is scattered.] The language kernel under initialization is defined by both source code files written in the same language and imperative code present in the VM. Although splitted and scattered, they define the same language and a change in one of these parts may impact on the other one.
\item[Runtime initialization mixes concerns.] The initialization of the language kernel is mixed ~(and coupled) with the initialization of the rest of the runtime system. This introduces an undesired coupling that affects the development and the understanding of the code.
\end{description}

\gp{write the related work of the new bootstrap paper: metacircular vas help to solve the gap between the language kernel and the VM, and so, easing the change of interface. See Jikes RVM, SqueakVM, Klein VM allow some grade of bootstrap or manipulation of the underlying runtime system}

\subsection{Changing a Language's Metacircularities}

Some methods or classes in a Smalltalk image are very sensitive to changes, either because they are used pervasively throughout the system, or take part in important subsystems like the user interface, the compiler, the debugger, the metaclass hierarchy, or classes that are well-known to the virtual machine.
This is because Smalltalk has a single scope where not only everything is visible, but also reflectively accessible.
Any breakage in this kind of code usually leads to spectacular failure. A simple example would be adding a breakpoint in the iterator method \ct{Array>>do:}. In Pharo Smalltalk, adding this breakpoint impacts about 90000 \ct{Array} instances and the image freezes \cite{Denk08b}.
It is thus very difficult to debug or change this code in a realistic setting, without risking to impact the whole image.

But the fact is, some evolutions do require changes to this sensitive code.
If temporary breakage is necessary, then the maintainers must find a way to apply the changes with reduced tools and extra care.
Alternatively, some images are destined to run under restricted conditions that make it impractical to include a complete set of development tools, or simply to access them: for instance, images running on remote servers do not have an active graphical interface\footnote{Remote display solutions like VNC do exist, but suffer from usability and portability problems.}.
This makes development, testing, and maintenance impractical or even impossible.

Current solutions or workarounds are to work on a renamed copy of the classes, or to remotely control a separate image via remote objects.
However, in the first case we only delay the problem, because the complete impact of changes cannot be assessed until the copied and modified classes are merged back into the system.
In the latter case, if a change causes the remote image to crash, then it will be impossible to assess the problem.

The most related family of work is virtualization approaches like Xen \cite{Chis07xen}. Virtualization makes it possible to run several operating systems at once on a single physical machine. As these approaches target full operating systems, they rely on support from the hardware platform, and in some cases from the guest OS; they also concentrate on performance and production features, and consider the guest system mostly as a black box. In contrast, ObjectSpaces provide full control and reflective access to their contents.

Changeboxes \cite{Denk07c} encapsulate and scope changes, allowing several versions of a system to coexist in a single runtime environment, effectively adapting version control from static source code to running systems. Changeboxes scope code changes, while ObjectSpaces scope generic object references; also, Changeboxes do not directly address the problem of applying changes to code that is critical to the runtime system itself.

Scoping side-effects has been the focus of two recents works. Worlds~\cite{Wart08a} provide a way to control and scope side-effects in Javascript. Similar to
ObjectSpaces, side-effects are limited to a first-class environment.  Tanter proposed a more flexible scheme: contextual values~\cite{Tant08b} are scoped 
by a very general context function.

One problem meta-circular architectures is that meta-objects rely on the same code they reflect upon; therefore there is a risk of infinite meta-recursion when the meta-level instruments code that it relies upon.
In \cite{Denk08b}, Denker et al solve this problem by tracking the degree of metaness of the execution context. Meta-objects can only reflect on objects of a lower metaness, thus simulating the semantics of an infinite tower of distinct meta-interpreters. The existing work on Meta-context is only concerned with scoping behavioral changes. More work is needed to extend this work to structure. We plan to explore how ObjectSpaces can be used to provide a way to control structural reflective change.

One possibility for implementing ObjectSpaces is to differentiate messages depending on whether the sender and the receiver are in the same space or not. Several works use a similarly extended message lookup.
%
Us~\cite{Smit96a} is a system based on Self that supports subject-oriented
programming~\cite{Harr93a}. Message lookup depends not only on the receiver of a
message, but also on a second object, called the \emph{perspective}.
The perspective allows for layer activation.
ContextL~\cite{Cost05a, Hirs08a} is a language to support Context-Oriented 
Programming (COP). The language provides a notion of \emph{layers}, which
package context-dependent behavioural variations. In practice, the variations
consist of method definitions, mixins and \emph{before} and \emph{after}
specifications. Layers are dynamically enabled or disabled based on the current
execution context. ObjectSpaces provide form a context. The relationship between
context-oriented programming, subjectivity and ObjectSpaces is an interesting
topic of future research.

Gemstone \cite{Otis91a} provides the concept of class versions. Classes are
automatically versioned, but existing instances keep the class (shape and
behavior) of the original definition. Instances can be migrated at any time.
Gemstone provides (database) transaction semantics, thus state can be rolled
back should the migration fail.
Gemstone's class versions extend the usual Smalltalk class evolution mechanism for robustness, 
large datasets, and domain-specific migration policies. In contrast, ObjectSpaces target general 
reflective access and bootstrap-like evolutions of code that is critical to the environment.

In Java, new class definitions can be loaded using a class loader
\cite{Lian98a}. Class loaders define namespaces, a class type is defined by the
name of the class and its class loader. Thus the type system will prohibit
references between namespaces defined by two different loaders. Class loaders
can be used to load new versions of code and allow for these versions to coexist
at runtime, but they do not provide a first-class model of change.
Java also provides JPDA, a remote debugging architecture that specifies a native interface on the debuggee VM, and a matching API for the debugger front-end, running in a separate VM. However, JDPA only supports introspection features like  inspection and monitoring, and very limited intercession \cite{jdpa}.


\section{On Evolving Applications}

 \begin{table*}[ht]
 \small
 	\centering
\includegraphics[width=\linewidth]{criteria_overview}
 	\begin{tabular}{|ccccc>{\columncolor[gray]{0.8}}c|}
	
\hline
 			& \textbf{Dedicated}
 			& \textbf{Static}
			& \textbf{Hybrid}
 			& \textbf{Dynamic}
 			& \textbf{Tornado} \\
 			& \textbf{SDKs}
 			& \textbf{Analysis}
			& \textbf{Analysis}
 			& \textbf{Analysis}
 			& \\
%  \cmidrule(r){2-6}
% \midrule

		Base Library&&&&&\\Support
 			& + & + & + & + & +\\
		\hline
		Third-Party&&&&&\\ Library Support
 			& - & + & + & + & +\\
		\hline
		Legacy Code&&&&&\\ Support
 			& - & + & + & + & + \\
		\hline
		Reflection&&&&&\\ Support
 			& + & - & - & + & + \\
		\hline
		Special&&&&&\\ Infrastructure& - & + & - & - & + \\for Running
 			&&&&&\\
		\hline
		Flexibility
 			& - & - & - & - & +  \\
 	 \hline
 	\end{tabular}
 	\caption{Evaluation criteria applied to related work on deployment code unit tailoring techniques}
 	\label{tb:comparison}
 \end{table*}
 
The reduction of the deployment footprint of object-oriented applications has been subject of interest both in industry and research since many years. In such regard, we identified four different families of solutions for dead code elimination: dedicated SDKs~(cf. section \ref{section:static_selection_rw}), static analyses~(cf. section \ref{section:static_rw}), dynamic analyses~(cf. section \ref{section:dynamic_rw}) and hybrid analyses~(cf. section \ref{section:hybrid_rw}). Table~\ref{tb:comparison} presents a comparison of these techniques, given the criteria defined in section~\ref{sec:criteria}.

\subsection{Dedicated SDKs}%Pre-conceived specialized application-independent platforms}
\label{section:static_selection_rw}

Dedicated SDKs are SDKs containing frameworks and/or libraries prepared to run under specific circumstances. For example, Java Micro Edition~(J2ME)~\cite{JavaME} as the dedicated version of the Java SDK, or Cocoa Touch as the one of Cocoa. These specialized SDKs are reduced platforms to run applications inside mobile and constrained devices. These platforms provide a reduced and fixed set of base libraries defined a priori and in a not customizable way. Applications have to be written especially for them, and thus legacy code and third-party libraries not written especially for it are not compatible. Reflection is available since the statically tailored base libraries are built in a not automatic fashion, and the application code is not tailored.

\subsection{Static Analysis-Based Techniques}\label{section:static_rw}

Static analysis approaches for dead code elimination make use of the static information of a program to select the minimal subset of used elements. The bibliography describes four different algorithms to achieve this goal: unique name, class hierarchy analysis~(CHA), rapid type analysis~(RTA) and reachable members analysis~(RMA) \cite{Baco96a, Titz06a}. These techniques share a common approach, selecting an entry point method of an application and following from it the execution flow using the available static information \ie type annotations, and class and method names, building a call-graph~\cite{ShortGrov97a}.

These techniques have been studied and applied in many environments and languages. Rayside et al.~\cite{ShortRays02a}, Jax~\cite{ShortTip03a} and the ExoVM System~\cite{Titz06a} propose application extraction tools using these techniques for Java applications. Sallenave et al.~\cite{Sall10a} apply RTA to produce smaller .NET assemblies for embedded systems. Bournoutian et al.~\cite{Bour14a} use CHA to optimize on-device Objective-C applications.

In summary, these approaches are based on the static types found either in the source code or byte code. Thus, they are not applicable efficiently in dynamic languages with no static type information. These solutions are valuable as they allow one to tailor base and third-party libraries, and legacy code. Their tailoring approach generates new deployment units that can run on the standard runtime infrastructure. The main drawback of this approach appears in the presence of reflection and configuration files, which will only work with a subset of reflective invocations through complementary analyses on the strings found in the source code. Also, existing solutions in this family lack the flexibility to declare and identify levels of tailoring, making it an "all or nothing".

\gp{Closed World Assumption kills you guys!}

\subsection{Dynamic Analysis-Based Techniques}\label{section:dynamic_rw}

Dynamic analysis techniques use exclusively runtime information~(\ie execution flow, alive objects, execution statistics) to perform dead code elimination. Amongst these, we identify two different approaches: \emph{load on demand} and \emph{code collection}. Load on demand approaches detect during runtime whenever a class or method needs to be installed and request it to a server application. Code collection approaches deploy the full application  and garbage collect unused code based on usage statistics. Related work in this family share a common characteristic: these techniques are used inside ubiquitous systems \ie systems meant to be always connected. Ubiquitous systems, as they are always connected, have a possibility to fallback and recover in the case of incompleteness. However, to focus here on the dead code elimination techniques, we will discuss the incompleteness recovery techniques in section \ref{section:completeness}.

JUCE~\cite{ShortPopa04a,ShortTeod01a} is a platform for ubiquitous devices supporting code load on demand and code collection. Its approach for building up an application is similar to Tornado's. First, it initializes a minimal running application and code is loaded, with a method granularity, from a server located in a different machine. Unused code is collected following usage statistics, and loaded back again on demand if needed.

OLIE~\cite{Gu03a} is an engine that intelligently partitions and offloads objects during runtime to minimize memory consumption. It is part of the adaptive infrastructure for distributed loading (AIDE). In OLIE, offloaded objects are indeed migrated to nearby remote devices. Migrated objects can be accesses later through proxies that perform remote invocations on them.

SlimVM~\cite{Kers09a, Wagn11a} is an ubiquitous system where all code resides on a remote server and is loaded only on demand on small devices. Some static analysis is performed only on the server to reduce the size of the transported code, by identifying most likely needed code. SlimVM changes the class format,  However, on the client side, every code load is done dynamically.

All solutions inside this category share one main property: they require to run the application inside a special infrastructure to apply their techniques \eg specialized VMs implementing remote lazy loading, code collection or special bytecode sets. The main challenge of these solutions resides on applying these techniques while minimizing their impact on performance during the runtime. Additionally, these solutions require their applications to run exclusively inside their infrastructure. Tornado works in the same way as these solutions: it uses a special infrastructure to run the desired application and select the used elements.  However, Tornado provides also with the ability to extract this application and run in \emph{offline} mode, using the non-modified infrastructure.

Regarding dynamic features such as reflection, this kind of solutions are the ones that can, potentially, handle it in the best way since they have in runtime all the information needed to resolve it. JUCE and OLIE, as Tornado, handle naturally reflection as they do not change the runtime representation (which programs make assumptions of, when they use metaprogramming). SlimVM on the other side, had to change the reflection support because they changed the object and class representation on their VM.

Regarding its applicability, SlimVM needs to recompile the whole application into its own format, while OLIE and JUCE, as Tornado, can tailor base and third party libraries without any modifications on it. Thus, the latter two can be applied to legacy code also for free. None of these solutions provide with the ability to select the level of tailoring always working on the full application. In contrast, Tornado uses Seeds to force a minimal subset of elements to be part of the application.


\subsection{Hybrid Analysis-Based Techniques}\label{section:hybrid_rw}

Hybrid analysis techniques mix static and dynamic~(\ie runtime) information to provide better results. The common approach of these is to start an application, such as Tornado does, and pause it after some minimal runtime information is available \ie call stacks are created, some classes are loaded and initialized, and some objects are instantiated. Then, it uses the built stack of alive objects to perform a static analysis, as described in section \ref{section:static_rw}, with concrete type information.

Java in The Small (JITS)~\cite{ShortCour10a} uses a hybrid approach to select the used parts of a program, and then loads them inside a binary image. A specialized VM loads the binary image at startup. JITS's approach tailors base and third-party libraries as well as application specific code. It does not require modifications on the existent application to tailor it, so a legacy application could theoretically be tailored with this approach. JITS does not offer the possibility to configure the tailoring level, since it was designed to be used only in embedded devices where no more than one application would be running. Regarding reflection, JITS presents the same drawbacks as the other static call graph analysis approaches since not all the runtime information about the reflective invocations can be deduced.

% ===========================================================================
\section{Summary and Outlook}
% ===========================================================================


% =============================================================================
\input{chapter-footer.tex}

\part{\VTT: an Application Runtime Virtualization Infrastructure}
\input{chapter-header.tex}

% ===========================================================================
\chapter{An Infrastructure for Software Evolution: Object Spaces}
\minitoc
% ===========================================================================
\introduction
% ===========================================================================


% ===========================================================================
\section{First Class Runtime Systems in a Nutshell: Object Spaces}

We propose the introduction of first class object runtime systems, namely \emph{object spaces}, to aid and support the evolution of object-oriented software.
Object spaces encapsulate an object runtime system and provide a high-level API to query and modify it. An object space consists, then, on two main components: the object runtime system it represents, and the first class objects used for its manipulation. Figure \ref{fig:objectSpaceOverview} gives an overview of the relation between object runtime systems when using object spaces.

\begin{figure}[htb]
\begin{center}
\includegraphics[width=.9\linewidth]{object_space_overview3}
\caption{Object from object runtime system I can manipulate through the Object Space library the object runtime system II.\label{fig:objectSpaceOverview}}
\end{center}
\end{figure}

The first advantage of using object spaces relies on the addition of a new level of abstraction, decoupling our client applications from the internal representation of the second runtime system. The runtime system under manipulation can reside \eg on the same operating system process, on a different process, or even on a simulated runtime system.

Second, it presents a uniform high-level API for the manipulation of its runtime system. This API provides services for the manipulation of execution, code and objects inside the object space~(cf. Sections \ref{sec:membrane} and \ref{sec:execution}). Through this API, an object space enforces safe communication between runtime systems~(cf. Section \ref{sec:communication}) and isolation~(cf. Section \ref{sec:isolation}).

%\begin{figure}[ht]
%\center
%\includegraphics[width=.9\linewidth]{object_space_overview3}
%\caption{\textbf{Solution overview.} \label{fig:objectSpaceOverview}}
%\end{figure}

%A virtual Smalltalk image is an image living inside another Smalltalk image. The container image, the host, observes the virtual image and has complete control over it.
%The main idea is that such tasks difficult to perform due to the reflective architecture are handled by the host image. We transform the critical "self-brain surgery" tasks into safe "brain surgery" ones, by delegating them to another Smalltalk image.

%Oz is a virtual image model and implementation based on \objectspaces~\cite{Casa09a}. Casaccio et al. sketched \objectspaces to solve self-brain surgery. When doing self-brain surgery, the image under modification becomes a \emph{patient} of a \emph{surgeon} image. The patient is included inside the surgeon as an \objectspace. Through this \objectspace, the image gets manipulated by the surgeon, fixed and finally awoken.

%In figure \ref{fig:objectSpaceOverview} we can see an \objectspace represented as the graph within the dotted line, containing a guest smalltalk image a fa\c{c}ade object and a .

%An \objectspace appears as a first-class object which reifies a full Smalltalk image.
%An \objectspace is a fa\c{c}ade object the surgeon uses to reason about and act upon the patient Smalltalk image.

%In this paper, we extend \objectspaces to \textbf{virtual Smalltalk images}. We understand an \objectspace a a full Smalltalk image completely controllable from the outside. Then, an \objectspace exposes its object graph as well as other resources such as its complete execution or files.

%In this section we describe the concepts and design principles guiding our \objectspace solution for virtual Smalltalk images.
%\gp{come back here and finish this part!!!}

%\subsection{Object Spaces Overview} \label{sec:definitions}

%In Oz, an \objectspace is a subsystem of another image. It is an object graph composed by two main elements: a full Smalltalk image (cf. Section \ref{sec:isolation}) and a \emph{"membrane"} of objects controlling that image (cf. Section \ref{sec:membrane}). The image containing an \objectspace is its \emph{host}, while the \objectspace is its \emph{guest}.


%\begin{description}
%\item[Accessing well known objects.] Well known objects such as \ct{nil}, \ct{true} and \ct{false} can be get and set from an object space. The bootstrap process uses it during the well known instances initialization step. Once an object space is configured with such objects, it can execute code using them and overcome the \textbf{unicity hypothesis}.
%\begin{code}
%mirror getNil();
%mirror getTrue();
%mirror getFalse();
%
%void setNil(mirror aNilObject);
%void setTrue(mirror aTrueObject);
%void setFalse(mirror aFalseObject);
%\end{code}
%
%\item[Object allocation.] Object allocation operations provide support for the initialization of the guest language kernel. In particular, \ct{allocateObjectOfSize()} is the unsafe operation used to instantiate the first objects when there are no classes available.
%\begin{code}
%mirror allocateObjectOfSize(int size);
%mirror allocateObjectOfClass(mirror aClass);
%mirror compile(String sourceCode, mirror aClass);
%\end{code}

\section{Manipulation and Monitoring} \label{sec:membrane}

An object space exposes its API through mirrors~\cite{Brac04b}. 

\subsection{Runtime Mirror}

\begin{description}
\item[Runtime system manipulation.] To initialize the classes in the guest language kernel, Oz provides with operations to install classes and obtain the list of classes installed.
\begin{code}
void installClass(mirror aClass);
List<mirror> getClasses();
\end{code}

\end{description}

\subsection{Low Level Mirrors}

Our object space solution involves two different kind of mirrorsMirrors mediate the interaction between guest and host language kernels to keep them \textbf{isolated} from each other \ie object references from the guest language kernel do not leak into the host language kernel and vice-versa. To ensure isolation, mirrors provide the following API to interact with the objects they wrap:

\begin{description}

\item[Class access.] Method installation and object manipulation is achieved by class access operations. In particular, the \ct{setClass()} operation is used in the first steps of the bootstrap process to close the main circularities and set the class of the \ct{nil} object.
\begin{code}
mirror getClass().
void setClass(mirror aClass).
\end{code}

\item[Internal state access.] Every object initialization of the bootstrap process is achieved by altering their internal state. Mirrors provide a low level API to get and set the fields of an object.
\begin{code}
mirror getInstanceVariable(String variableName).
void setInstanceVariable(String variableName, mirror anObject).
\end{code}


%\item[Code execution.] Oz provides with operations to manipulate the processes/threads of the guest language kernel. In particular, the \ct{runForTime()} operation will execute the guest language kernel with whichever processes it has installed, directly on the VM.
%\begin{code}
%List<mirror> getProcesses();
%mirror createProcessDoing(String expression);
%void runForTime(int milliseconds);
%\end{code}
%\caption{\textbf{Code of the Builder implementing the same logic as in \ct{Dictionary>>initialize}.}\label{code:logic_dup2}}
%\end{figure}
\end{description}

\subsection{High Level Mirrors}


% ===========================================================================

\section{Control of Execution} \label{sec:execution}

An \objectspace's execution is fully controllable from the host. The host can introspect and modify an \objectspace processes via mirrors to obtain information such as the method currently on execution, the values on the stack or the current program counter. Besides from those reflective operations, an \objectspace provides also operations to suspend, resume or terminate existing processes, and to install new ones.

The \objectspace provides fine-grained control on the guest execution. An \objectspace controls the amount of CPU used by the guest image. \sd{can you do that for real? CPU control?}\gp{We can give an objectspace a window of 20ms to run and then we get back control so It's our decision if we give it 20 more or not :)}This way, a virtual image can be customized for scenarios like for example testing, CPU usage analysis, or old hardware simulation. For example, it may restrict its processes to run during only 300 milliseconds every second for either.

\gp{example!}

\begin{description}
\item[Code execution.] Oz provides with operations to manipulate the processes/threads of the guest language kernel. In particular, the \ct{runForTime()} operation will execute the guest language kernel with whichever processes it has installed, directly on the VM.
\begin{code}
List<mirror> getProcesses();
mirror createProcessDoing(String expression);
void runForTime(int milliseconds);
\end{code}
%\caption{\textbf{Code of the Builder implementing the same logic as in \ct{Dictionary>>initialize}.}\label{code:logic_dup2}}
%\end{figure}
\end{description}

\section{Safe Communication} \label{sec:communication}

\gp{rewrite: cross message sends, literal translation, and boundary checks}

As explained in Section \ref{sec:isolation}, an objectspace is an isolated object graph in the sense that from the guest image there is no way to reach host objects. However, the opposite relation is possible: the host can manipulate completely the \objectspace.

The communication mechanism between host and guest images is based on the \emph{injection of objects} into the \objectspace. The host may install from simple literal objects such as strings or numbers, up to more complex objects like classes, methods. An \objectspace permits to \emph{send messages to objects} inside itself by injecting process with the specified code. Injected processes may have any arbitrary expression. The membrane objects can retrieve the result from the process' context once the execution is finished.

The \objectspace membrane ensures that object injection honors the transitive closure property. On one side, literal objects from the host are automatically translated to their representation in the \objectspace. An \objectspace implements the operations to transform literal objects (numbers, strings, symbols, some arrays and byte arrays) \emph{from and to} its internal representation.

On the other side, non literal objects are actually not created in the host and injected in the \objectspace. Non literal objects are directly created in the \objectspace, so the task of injecting the new object inside a graph is safe.

\gp{example!}

\section{Isolation through Mirrors} \label{sec:isolation}



Note from the signatures presented that all mirror operations will return a mirror in exchange, and so, interaction with other objects inside the object space is mediated by Oz. Mirrors could give and revoke permissions on the wrapped object and enforce invariants to keep the model consistent~\cite{Teru13a}.
Oz provides also specific kinds of mirrors with high level APIs to manipulate objects with a specific format and/or behavior such as classes, methods, activation records or processes. We do not cover high level mirrors in this paper because they are not relevant to the bootstrap process.

\gp{next is old, see how to merge}

The manipulation of objects inside the \objectspace image cannot be achieved with a traditional message send mechanism. In the normal case, when a message send is performed, the virtual machine takes the selector symbol of the message and lookups in the class hierarchy method dictionaries of the receiver until it finds a method with the \emph{same}~(identical) selector. In our scenario, both host and guest images contain their own \ct{Symbol} class and symbol table. Then, when performing a \emph{cross image-message send} the method lookup mechanism takes a selector symbol from the host, lookups into the guest receiver's hierarchy, and finally fails because  the selector in the guest is (while maybe equals) not identical to the selector in the host. Also, forcing a \emph{cross image-message send} by using a guest's selector can leak host references to the guest: activating a guest method from the host gives the guest complete access to the host through the \ct{thisContext} special variable which reifies the stack on-demand.

To encapsulate and control the basic object manipulation, the object space fa\c{c}ade object provides mirrors~\cite{Brac04b}. Mirrors hide the internal representation of the objects inside the objectspace and expose reflective behavior. The guest is not aware of the existence of these mirrors.

All objects inside an \objectspace and reachable by reference can be retrieved by host's objects through the \objectspace facade and mirrors. 
There is no limitation nor restriction for object access. 
The host manipulates all objects in a homogeneous way through their mirrors. 

Additionally, specific mirrors are provided to manipulate objects with a specific format and/or behavior such as \ct{Class}, \ct{Metaclass}, \ct{MethodDictionary}, \ct{CompiledMethod}, \ct{MethodContext}, and \ct{Process}.



% =============================================================================
\input{chapter-footer.tex}
\input{chapter-header.tex}
% ===========================================================================
\chapter{The \Vtt Prototype}
\minitoc
% ===========================================================================
\introduction
% ===========================================================================

We implemented \Vtt on the Pharo platform. Our solution virtualizes Pharo runtimes and provides, as already described, the ability to manipulate their object graph and control their execution. Our implementation includes a language side library containing the object space and language hypervisor related classes. Our main modification to Pharo's runtime is an extension to its stack-based \VM. This \VM is a version of Pharo's \VM without a Just In Time~(JIT) compiler. This extensions are meant to allow the co-existence of many runtime systems on the same \VM and to allow us to easily modify the \VM setup interface.

An object space VM-Setup Interface in \Vtt~(Section \ref{sec:setup_interface}) is implemented by exposing the \VM interpreter state. The Pharo \VM interpreter state is hold inside an array, so called \emph{special objects array}, which contains references to the objects that the \VM may need at runtime. We expose and modify this array through an object space. The special objects array also provides support for the execution of several runtimes on top of the same \VM. In our solution, the \VM has a single bytecode interpreter shared amongst the different running runtimes. To allow the execution of a different runtime, we exchange the interpreter state for the state of its corresponding runtime in an atomic operation that we call \emph{context switch}~(Section \ref{sec:context_switch}).

\Vtt mirror implementation~(Section \ref{sec:implementation_mirrors}) does not require particular changes in the Pharo \VM. We base our mirror implementation in two existing \VM primitive operations that allows us to execute a primitive on an arbitrary object by changing the primitive's receiver. Additionally, the already existent reifications of objects such as \ct{CompiledMethod}, \ct{Process}~(threads) or \ct{Context}~(stack frames or activation records), allows us to manage such runtime elements using the same mirror mechanism and avoid specific solutions.

\Vtt's presents a memory layout where the objects of both the virtualized and hypervisor co-exist in the same heap~(Section \ref{sec:memory}). This means that there is no need for special support on \emph{cross-runtime references} for implementing \eg mirrors, and that we can reuse the existing memory management in the virtual machine almost transparently.
However, this solution forbids us to analyze the memory usage of the virtualized runtime and has an impact on the GC.

Finally, This chapter finishes, with means of completion, by presenting the non-implemented aspects of our solution~(Section \ref{sec:not_yet_implemented}).

\section{\VM-Setup Interface}\label{sec:setup_interface}

Pharo \VM holds the state of the \VM-setup interface inside a \emph{special objects array} object. Pharo's special objects array contains 56 entries whose indexes are well known by the \VM for their access at runtime. To implement our \VM-setup interface, we access and modify this special object array. Following, we detail the special objects array entries that we expose for our solution.

\begin{description}
\item[Special Instances.] Special instances such as \ct{nil}, \ct{true} and \ct{false} are directly pushed by the \VM interpreter at runtime instead of residing in a method literal list. A flyweight \ct{Character} table contains the first 256 character objects to ensure their identity and save memory.

\item[System Dictionary.] The system dictionary contains all installed classes in the system. An object space uses this entry to query the installed classes and to install new ones. The \VM does not make any special use of this object as it is managed directly from the language.

\item[Process Scheduler.] The process scheduler contains all existing processes~(threads) in the runtime. We can install and remove process from the process scheduler. The \VM uses this same process scheduler to manage the runtime's execution.

\item[Symbol Table.] The symbol table gathers is the set of \emph{unique} strings in the system. Symbols are mainly used to denote method signatures and ensure reference equality during the method lookup. An object space uses the symbol table to map symbols between runtimes and so it ensures no duplicate symbols are created. The \VM does not make any special use of this object as it is managed directly from the language.

\item[Literal Classes.] Literal classes are the classes of literal objects. Literal objects are those present directly in a method's literal list~(\eg numbers, strings, literal arrays), or those objects that the \VM creates at runtime~(\eg BlockClosures). On one side, the \VM uses the classes available in this list to directly instantiate objects of these types or perform safety checks at runtime~(which cannot be performed at compile time because of reflection or the dynamically-typed nature of the language). On the other side, an object space uses these well-known classes to perform transformations between objects in one runtime to another \eg translate a string from the hypervisor runtime to a string in the virtualized runtime.

\item[Special Selectors.] The special selectors denote those messages that the \VM will send to the image under special situations. This is the case of the \ct{doesNotUnderstand} selector that is sent to the receiver object when the method lookup fails finding a method under a message-send. Other selectors in this category are (a)\ct{cannotInterpret} which is sent when a class in the middle of the method lookup has no method dictionary, (b) \ct{mustBeBoolean} which is sent when a branch operation founds a non-boolean object, (c) \ct{cannotReturn} and \ct{aboutToReturn} are selectors used when contexts are finalized and (d)\ct{run:with:in:} is the message used to implement method wrappers.

\end{description}



%Particularly, the special objects array contains a process scheduler object and its corresponding process objects, implementing green threads. Pharo virtual machine has a single threaded nature and uses green threads to organize its execution.

\section{Cycle Execution and Context Switch} \label{sec:context_switch}

The Pharo \VM has single threaded execution. Only one operating system thread is used to execute Pharo code, so process scheduling is handled internally by the virtual machine. Processes scheduled using this approach are also called \emph{green threads}. Green threads provide process scheduling without native operative system support while limiting the proper usage of modern multicore CPUs. By using green threads only one process, the \emph{active process}, is executed at each instant in time. After the active process executes for a given window of time, if there are any waiting processes with greater or equal priority, the active process gets preempted \ie it is suspended and the process with the highest priority becomes the new active process.

In \Vtt, we reused part of the green thread mechanism to schedule runtime execution, which we called \emph{context switch}. A virtualized language runtime runs for a window of time after which it gets preempted if another runtime with highest priority~(the language hypervisor runtime) is waiting. Internally, the \VM has a single bytecode interpreter shared amongst the different running runtimes. To perform the context switch, we exchange the special objects array of the \VM interpreter by the one in the language runtime to resume, and then resume the \VM execution. This solution keeps the single threaded nature of the \VM meaning that when a language runtime is running the others are suspended. On the good side, there are no concurrency problems between the different language runtimes, allowing us to focus on the language specialization features.


\begin{figure*}[htb]
\begin{center}
\includegraphics[width=.8\linewidth]{object_space_context_switch}
\caption{\textbf{Context Switch Internals.}To perform a context switch, we change the special objects array of the \VM's interpreter.\label{fig:context_switch}}
\end{center}
\end{figure*}

Finally, the window time control was implemented reusing the existing process preemption mechanism in the \VM. A separate thread, namely the \emph{heartbeat}, is awaken every 200 milliseconds and activating a flag that indicates a preemption can occur. When the \VM code interpreter arrives to one of the safe suspension points (\ie a point where suspending the current execution will not leave the current process in a incoherent state) such as a back jump or a message-send bytecode, the code interpreter preempts the active process if the corresponding flag was activated. In \Vtt we modified the preemption code to activate another language runtime if available.

\section{Mirror Implementation}\label{sec:implementation_mirrors}

Our implementation of mirrors manipulate the objects inside an object space by using already Pharo \VM primitives. A primitive in Pharo is a function executed at the \VM level. To execute a primitive from the language, this primitive should be available as a method. Then, we have to send the corresponding message to the a object subject of the primitive:

\begin{code}
Array >> size [
    <primitive: 62>
]

{1 . 2 . 3} size.
\end{code}

We want however, to avoid cross-runtime message-sends~(\chapref{vtt}), we bypass the message-send with the combination of two existing primitives:
\begin{description}
	\item \textbf{Execute a given method on an object.} Given a method, it is possible to execute it on an object, avoiding method lookup in the object. In current Pharo \VM, this primitive is implemented in the method \textbf{\ct{receiver:withArguments:executeMethod:}} of the \ct{CompiledMethod} class, with number 188. This method receives as arguments the object on which the primitive will be executed, an array of arguments, and the methods to execute.
	\item \textbf{Execute a primitive on an object.} It is possible to send a message to an object, so a primitive is executed on the receiver. This primitive is implemented in Pharo's \ct{ProtoObject} class as \textbf{\ct{tryPrimitive:withArgs:}} with number 118 and receives as argument the number of the primitive and an array or arguments.
\end{description}

We use primitive \ct{receiver:withArguments:executeMethod:} to execute the primitive method \ct{tryPrimitive:withArgs:} on the object from the guest image, avoiding the \emph{cross image-message send} and executing directly the primitive on the given object.

\begin{code}
CompiledMethod
       receiver: anObjectFromOtherRuntime
       withArguments: { aPrimitiveNumber . anArrayOfArguments }
       executeMethod: (ProtoObject >> #tryPrimitive:withArgs:)
\end{code}

We have one mirror per type of object supported by the Pharo \VM. Notice that as some runtime elements such as methods, activation records or even processes are reified as objects in Pharo, we can manipulate them through mirrors without developing new support for them in the \VM:

\begin{description}
\item[Object mirrors.] Mirrors for objects containing just object references such as an \ct{Array} or an \ct{OrderedCollection}.
\item[Word mirrors.] Mirrors for objects containing only non-reference word fields such as a \ct{Float} or a \ct{WordArray} object.
\item[Byte mirrors.] Mirrors for objects containing non-reference byte size fields such as a \ct{ByteArray} or a \ct{ByteString}. 
\item[Class mirror.] Mirrors for class like objects. They control the manipulation of classes and honor the format of a class in the Pharo \VM which should necessary have as its first three instance variables the superclass of the class, its format and method dictionary.
\item[Method dictionary mirror.] Mirrors for method dictionaries honoring the \VM invariants for method installation \ie using the identity hash of the key object instead for doing the lookup in the method dictionary.
\item[Method mirror.] Mirrors for method objects. Method objects in the Pharo \VM are objects with a mixed format. They contain object references to their literals, and byte fields with their bytecode.
\item[Context mirror.] Mirror for context objects. A context object is reified lazily by the Pharo \VM and contains a variable amount of fields denoting the local variables of a scope.
\item[Process mirror.] Mirror for process objects. Process objects are used to manage the execution of a Pharo runtime. They can be resumed, suspended or finalized.
\end{description}


%Using Oz, we can also create and install new processes inside an \objectspace given a code expression. The creation of a process requires the creation of a compiled method with the code~(bytecode) corresponding to the desired expression and a method context. The compiled method with the code to run is obtained by compiling the expression in the host and creating an \objectspace compiled method. The \objectspace compiled method is then provided with the compiled bytecode and its corresponding literals.

\section{Memory Layout} \label{sec:memory}

In \Vtt, all objects belonging to the different language runtimes share a unique object heap. Objects are mixed in this heap and are objects from the same language runtime not necessarily contiguous, as shown in Figure \ref{fig:heap}. Although they are mixed, they are logically separated as they conform different object graphs. Pharo being a safe-language~\cite{Hawb98a,Hawb02a} prevents us to forge references~(\ie creating references from numbers using pointer arithmetic) and isolates the object graphs of each runtime system.

\begin{figure}[ht]
\begin{center}
\includegraphics[width=.9\linewidth]{object_spaces_heap}
\caption{\textbf{A unique heap containing objects from different language runtimes.} Objects from the language runtime I and language runtime II are mixed in the heap. In this figure, after the \ct{nil}, \ct{true} and \ct{false} instances that belong to language runtime I, follow the corresponding ones of the language runtime II, which can in order be followed by objects of the former, like the string \textbf{`hi'}. \label{fig:heap}}
\end{center}
\end{figure}

This decision is funded on minimizing the changes made to the virtual machine, because of its complex state of the art. Our decision, while easing the development of our solution, has the following impact on it:

\begin{description}
	\item[Reuse memory handling mechanisms.] We use the same existing memory infrastructure as when \Vtt is not used. Existing mechanisms for allocating objects or growing the object memory when a limit is reached can be reused transparently by our implementation. 
	\item[Simplify the object reference mechanism.] Cross-runtime references are normal object references. No extra support from the virtual machine was developed in this regard.
	\item[Shared garbage collection.] Since objects from the different runtimes are mixed in the object memory, and their boundaries are not clear from the memory point of view, the garbage collector~(GC) is shared between them. Every GC run must iterate over all their objects, increasing its time to run.
\end{description}

%This solution seems suitable  J-Kernel \cite{Hawb98a} and Luna \cite{Hawb02a} present a solution similar to ours regarding the memory usage. They are Java solution for isolating object graphs with security purposes. In them, each object graph is called a \emph{protection domain}. All protection domains loaded in a system, and their objects, share the same memory space. 

%The J-Kernel enforces the separation between domains by using the Java type system, the inability of the Java language to forge object references, and by providing capability objects\cite{Levy84a,Mill03a,Spoo00a} enabling remote messaging and controlling the communication. This same separation in Luna \cite{Hawb02a} is achieved by the modification of the type system and the addition in the virtual machine of the \emph{remote reference} concept. In our solution, the separation is given by the same inability to forge object references and the membrane objects that control the communication.

%\section{Creating an \objectspace} \label{sec:object_space_creation}
%
%An \objectspace can be created either from scratch or by loading an existing image. Loading an existing image was implemented as a virtual machine primitive, because the image snapshot is actually a memory snapshot and therefore, easier to handle at VM level. This primitive, implemented with the code shown in Figure~\ref{code:import_image}, reads the snapshot file, puts all objects into the object memory, updates the object references to make them coherent and finally returns the special objects array of the loaded image.
%
%\begin{figure}[htb]
%\begin{code}
%\textbf{primitiveLoadImage}
%    | headerlength bytesRead newImageStart rootOffset oldBaseAddress dataSize rootOop fileObject |
%    
%    "get the reference to the file object"
%    fileObject := self stackValue: 0.
%
%    "Where will we put the new objects"
%    newImageStart := objectMemory startOfFreeSpace.
%
%    "read image header"
%    self readLongFrom: fileObject.
%    headerlength := self readLongFrom: fileObject.
%    dataSize := self readLongFrom: fileObject.
%    oldBaseAddress := self readLongFrom: fileObject.
%    rootOffset :=
%          (self readLongFrom: fileObject) - oldBaseAddress.
%    
%    "seek into the file the start of the objects"
%    self seek: headerlength onFile: fileObject.
%    
%    "grow the heap in the ammount of the image size"
%    objectMemory growObjectMemory: dataSize.
%    
%    "read the file into the free part of the memory"
%    bytesRead := self
%                    fromFile: fileObject
%                    Read: dataSize
%                    Into: newImageStart.
%
%    "tell the vm the free space is now after the loaded objects"
%    objectMemory advanceFreeSpace: dataSize.
%         
%    "update the pointers of the loaded objects"
%    self
%          updatePointersForObjectsPreviouslyIn: oldBaseAddress
%          from: newImageStart
%          until: newImageStart + dataSize.
%    
%    "return the special objects array"
%    rootOop := newImageStart + rootOffset.
%    self pop: 2 thenPush: rootOop.
%\end{code}
%\caption{Implementation of primitive \textbf{\ct{primitiveLoadImage}} that loads an image snapshot into the object memory written in Slang
%\label{code:import_image}}
%\end{figure}
%
%On the other side, creating an \objectspace from scratch can be implemented as a bootstrap of the system, following the process defined in \cite{Poli12a}. The \objectspace provides the \textbf{\ct{createObjectWithFormat:}} method to create an object respecting the given format but with an anonymous class, so we can consider it as a "classless" object. This method is used in the first stage of the bootstrap process, when no classes are available in the \objectspace image yet, to create the \ct{nil} instance~(cf. Figure~\ref{code:bootstrap_nil}) and the first classes~(cf. Figure~\ref{code:bootstrap_classes}). Later, when the classes are available, those objects are set their corresponding ones by using the \textbf{\ct{setClass:}} message.

\section{Benchmarks}

To verify the feasibility of this model, we built our \Vtt prototype in the Pharo language. Our prototype includes the \Vtt class libraries and modification on the JIT-less version of the Pharo \VM.
We benchmarked the virtualized execution of our prototype implementation using a subset of the computer language benchmarks game\footnote{\url{http://benchmarksgame.alioth.debian.org/}}.

We executed each of these benchmarks ten times in two different setups. Results are found in Table \ref{tb:benchmarks}. We first benchmarked the Vanilla JIT-less Pharo \VM without any of our changes, to make it a fair base of comparison with our JIT-less solution. Following, we benchmarked our solution using a null hypervisor that performs no action. This case is meant to measure the execution cycle overhead.

%We executed each of these benchmarks in four different setups. Results are found in Table \ref{tb:benchmarks}. We first benchmarked the Vanilla JIT-less Pharo \VM without any of our changes, to make it a fair base of comparison. Afterwards, we benchmarked our solution in three different setups:

%\begin{description}
%\item[Null Hypervisor.] A hypervisor performing no action. This case is meant to measure the cycle overhead. 
%\item[File Checking Hypervisor.] A hypervisor that checks the existence of a File at each cycle. This case is meant to measure an action that does not include an interaction with the 
%\item[Stack Checking Hypervisor.] A hypervisor checking the execution stack of the virtualized runtime on each monitoring cycle. This benchmark forces in our prototype implementation a reification of the execution contexts~(stack frames) and the creation of several mirror objects.
%\end{description}

%\begin{landscape}
 \begin{table}[ht]

 	\centering
 	\begin{tabular}{|l|c|c|}%c|c|}
			\hline
			\textbf{Benchmark}
 			& \textbf{Vanilla Pharo \VM (1x)}
			& \textbf{Null Hypervisor}\\
%			& \textbf{File Checking}
%			& \textbf{Stack Checking}\\
			
% 			\\& \textbf{}
%			& \textbf{}\\
%			& \textbf{Hypervisor}
%			& \textbf{Hypervisor}\\
		\hline
		RegexDNA & 5711ms +/-22 & 5878ms +/-18 \\\hline% & 6142ms +/-11 & 6555ms +/-424 \\\hline
		KNucleotide & 94.00ms +/-0.85 & 100.1ms +/-3.4\\\hline% & 105.8ms +/-3.2 & 134.4ms +/-3.7 \\\hline
		SpectralNorm & 79.8ms +/-1.4 & 86.6ms +/-1.4\\\hline% & 92.7ms +/-6.0 & 110.7ms +/-1.7 \\\hline
		BinaryTrees & 36.80ms +/-0.65 & 37.10ms +/-0.83 \\\hline%& 35.30ms +/-0.66 & 65.20ms +/-0.93 \\\hline
		Mandelbrot & 422.3ms +/-1.4 & 488.0ms +/-6.5 \\\hline%& 494.9ms +/-3.4 & 632ms +/-23 \\\hline
		ReverseComplement & 5.40ms +/-0.48 & 6.80ms +/-0.24 \\\hline%& 6.80ms +/-0.59 & 8.00ms +/-0.60 \\\hline
		ThreadRing & 2.50ms +/-0.41 & 2.50ms +/-0.56\\\hline% & 2.70ms +/-0.90 & 3.20ms +/-0.80 \\\hline
		PiDigits & 0.90ms +/-0.69 & 0.70ms +/-0.66 \\\hline%& 0.70ms +/-0.66 & 0.90ms +/-0.69 \\\hline
		Meteor & 2260.2ms +/-2.6 & 2398.2ms +/-8.5 \\\hline%& 2479ms +/-16 & 3677ms +/-267 \\\hline
		NBody & 21.10ms +/-0.57 & 21.70ms +/-0.66 \\\hline%& 22.6ms +/-1.4 & 37.7ms +/-1.5 \\\hline
%		FannkuchRedux & 0.00ms +/-0.00 & 0.00ms +/-0.00 & 0.20ms +/-0.36 & 0.00ms +/-0.00 \\\hline
		Fasta & 4.30ms +/-0.54 & 5.20ms +/-0.65 \\\hline%& 5.70ms +/-0.61 & 9.30ms +/-0.39 \\\hline
 	\end{tabular}
	\vspace*{0.2cm}
 	\caption{\small\textbf{Language Virtualization Overhead.} Comparing the cycle overhead from a non virtualized Pharo VM to a virtualized one.\label{tb:benchmarks}}
 \end{table}
%\end{landscape}

These results show that our prototype implementation poses an overhead between 0-9\% for the majority of these benchmarks. Special cases are Mandelbrot~(16\% overhead), ReverseComplement~(26\% overhead) and Fasta~(20\% overhead).
There is however room for improvements due to the absence of a JIT compiler, shared garbage collection and the single threaded nature of the system.
\gp{not sure what else put in here, since I cannot exactly explain why the overhead is there. One is the shared object memory hence shared GC hence biggest GC time}


\section{Non Implemented Aspects} \label{sec:not_yet_implemented}
 
For the sake of completion, we document in this subsection the aspects that have not been yet implemented in our prototype solution.

\subsection{JIT Compilation}

Just In Time~(JIT) compilation is available in the production distribution of the Pharo \VM. The existing JIT compiler doubles the performance of Pharo Stack \VM while adding yet another complex component in the \VM machinery. To reach such performance, it makes assumptions on the memory layout of the running language runtime. For example, it requires highly accessed objects such as \ct{nil}, \ct{true} or \ct{false} to remain in the same memory position to optimize their access.

However, these assumptions that are broken by \Vtt, as it introduces more than one language runtime in the same heap. Therefore, more than one version of special objects such as \ct{nil} are present in the same heap. Moreover, these objects can be moved in memory on GC compaction. Supporting JIT compilation in \Vtt would require a complete reengineering of the existing JIT compiler and possible the \VM.

\subsection{Plugin and Native Libraries State}

Pharo \VM allows to access resources from outside the language runtime through native libraries and \VM plugins. A \VM plugin is a native library that satisfies a particular interfaces to communicate with the \VM. A native library may keep its own state and may not be thought to be loaded several times in the same process. The same problem appears in \VM plugins, which are often developed assuming the existence of only one language runtime so their state is global for the whole \VM.

Our prototype was developed around the modification of the language runtime. As such, our \Vtt prototype does not modify each \VM plugins, nor it handles the case of loading multiple versions of the same native library from different runtimes. The usage of some of these elements in \Vtt may not be fully working.

\subsection{Finalization of External Resources}

Pharo \VM supports the concept of weak references. If an object is only referenced by a weak reference, this object will be garbage collected and this weak reference replaced by a reference to \ct{nil}. However, before garbage collecting the object, the \emph{finalization mechanism} will send the \ct{finalize} message to the object about to be collected to release any resources it may be holding. Finalization is useful to release resources external to the language runtime such as files or sockets. In Pharo, this finalization process is activated from the \VM but executed by the language runtime.

In \Vtt and because of the shared garbage collection, it may happen that a GC activated by one language runtime (the active language runtime) collects an object from another suspended language runtime. In such case, the \VM will activate only the finalization of the active language runtime. Then, the collected object from the suspended language runtime does not have the possibility to finalize its resources. This may cause external resource leaks, since they can be garbage collected but not properly finalized and released.

% ===========================================================================
\section{Conclusion and Summary}

This chapter explores the implementation aspect of our \Vtt prototype on the Pharo programming language. Our prototype includes a language-side library written for the Pharo and modifications in the Pharo \VM. We based our prototype in the \emph{Stack} Pharo \VM flavor. This \VM is the same as the production Pharo \VM except by the absence of a JIT compiler.

We implemented the \VM Setup interface by exposing Pharo's special object array, an array object referencing those particular objects that the \VM access directly at runtime. This special array contains also the state of the code interpreter, as it contains a reference to the process scheduler of the language. Thus, by exchanging the special objects array of the \VM code interpreter we can make a context switch between different language runtimes.

Mirrors are implemented by the combination of existing \VM primitives and do not require new \VM support for their implementation. Our mirror hierarchy covers all kind of objects that the \VM support. Particularly, the already existing reification of execution-related elements such as processes and contexts allows us to implement their manipulation through mirrors and use the same mechanisms as for regular objects.

Regarding the chosen memory layout, in \Vtt all running language runtimes share the same object heap. This decision ease our prototype implementation as no special support for cross-runtime references is needed. However, this has an impact in the time consumed by the GC over a bigger heap.

The benchmarks we run show that the execution cycle control has an acceptable overhead that is for the majority of our benchmarks below the 10\%. However, a fully engineered solution should improve on that an other points, specifically the JIT compiler, the state of native libraries and plugins, the shared GC and the problems in object finalization.

% =============================================================================
\input{chapter-footer.tex}

\input{chapter-header.tex}
% ===========================================================================
\chapter{Evolution by Recreation: Bootstrapping}
\minitoc
% ===========================================================================
\introduction
% ===========================================================================

As described in The Art of the Metaobject Protocol~(AMOP)~\cite{Kicz91a}, a language initialization solves the \emph{bootstrapping issues} of a language kernel. With this purpose the language initialization is often located in the virtual machine~(VM), where it can solve this issue without depending on running code on the language under construction. For example, Figure \ref{code:ruby_hierarchy} shows an excerpt of the code that initializes the class hierarchy in the Ruby \VM written in C\footnote{Taken from the version 2.1 of the Ruby \VM in \url{http://svn.ruby-lang.org/repos/ruby}}. From this code, Ruby's basic class hierarchy is composed by \ct{BasicObject} as its root, followed by \ct{Object}, \ct{Module} and \ct{Class}. These classes are created manually without a class, and once the class \ct{Class} is available, their class references are updated.

\begin{figure}[ht!]
\begin{code}
void Init_class_hierarchy(void) {
    rb_cBasicObject = boot_defclass("BasicObject", 0);
    rb_cObject = boot_defclass("Object", rb_cBasicObject);
    rb_cModule = boot_defclass("Module", rb_cObject);
    rb_cClass =  boot_defclass("Class",  rb_cModule);

    rb_const_set(rb_cObject, rb_intern("BasicObject"),rb_cBasicObject);
    RBASIC_SET_CLASS(rb_cClass, rb_cClass);
    RBASIC_SET_CLASS(rb_cModule, rb_cClass);
    RBASIC_SET_CLASS(rb_cObject, rb_cClass);
    RBASIC_SET_CLASS(rb_cBasicObject, rb_cClass);
}
\end{code}
\caption{\textbf{Code of the Ruby \VM that initializes the class hierarchy (excerpt).} The \VM code fixes the language class hierarchy.\label{code:ruby_hierarchy}}
\end{figure}


As we stated in \chapref{background}, this has many negative consequences. First, the Ruby \VM fixes this basic class hierarchy and prevents us to change it without changing the \VM.
In addition, an unclear separation between the \VM and language concerns in the \VM code makes harder to change and adapt the language to other circumstances. Finally, when this \VM is written in a low-level language, we rely on the tools and abstractions this low-level language provides instead of the more powerful ones from the high-level language. 

%We developed a bootstrap process for an object-oriented language based on the following ideas: we introduce (a) the language definition as a self-description of the bootstrapped language~(a description of its elements and how to build itself, written in itself), (b) a \textbf{first-class runtime} that provides a clear \VM-language interface for its manipulation, and (c) a specialized code interpreter~(the \emph{bootstrapping interpreter}) that executes the language definition~(Figure \ref{fig:bootstrapping_overview}) to initialize the language kernel in the reified runtime through its clear \VM-language interface. These three explicit components allows the separation of concerns and decouple the initialization of the language kernel from the \VM initialization.

%Following those principles, we developed a bootstrap process for an object-oriented language with the following elements~(Figure \ref{fig:bootstrapping_overview}). These three explicit components decouple the initialization of the language kernel from the \VM initialization so we can define different languages on top of the same \VM. Additionally we can modify the language kernel using the abstractions and tools of the high-level language.

% at runtime, the \VM interpreter does not require this particular fixed class hierarchy nor the fact that classes have metaclasses. It is orthogonal.
%This unnecessary coupling has a double impact on the language infrastructure: first, the only means to change the initial class hierarchy is to change the \VM code; second, the \VM can only run this language kernel even if it's execution model is less restrictive.

In this context we pose the following question: \emph{How do we build and change language kernels (potentially reflective ones)?} This chapter presents a high-level low-level programming approach~\cite{Fram09a} for it and revisits an already well-known technique: \textbf{bootstrapping}~(cf. Section \ref{sec:bootstrapping}). While bootstrapping is well known in the context of compilers~(where a compiler can compile itself), we explore and expand this concept in the context of object-oriented languages. In particular, we show how \Vtt represents a robust infrastructure that helps solving the issues that arise from defining a language issues~(Section\ref{sec:bootstrapping_infrastructure}). %infrastructure or the impact it has on the development process of language engineers. 


%The contribution of this article is to show that \textbf{reified runtimes}~(cf. Section \ref{sec:infrastructure}) supports elegantly the bootstrap of object-oriented languages. A reified runtime makes a clear separation of the \VM and language concerns and provides a clear interface. On top of it, a specialised code interpreter executes a self-description of the language. In such a way, we also decouple the language initialization from the \VM initialization.
%
%%The contribution of this paper is the clarification of such issues: we introduce a bootstrap process in the context of an object-oriented language~(cf. Section \ref{sec:bootstrapping}) and we describe which is the infrastructure it needs to solve the bootstrapping issues while decoupling it from the \VM startup~(cf. Section \ref{sec:infrastructure}).
Finally, we validate our work by showing how our solution allows us to create and run three different language kernels on top of the same \VM~(cf. Section \ref{sec:bootstrapping_validation}). These three languages present different meta-models, that enable in them different semantics and reflective features. Note that the limit of our approach is that these bootstrapped language kernels share the same \VM with its execution model~(\eg the object format and the bytecode instruction set). The co-evolution of \VM code and language kernels will be addressed in future work.


% ===========================================================================
\section{Bootstrapping an OO Language Kernel}\label{sec:bootstrapping}

The idea of a \emph{bootstrap} is well known in the context of compilers, where a compiler is considered bootstrapped when it compiles itself. For example, a bootstrapped C compiler is a compiler that, by using its own source code written in C, can produce another compiler with its same behavior. Notice that the compiled source is not a direct description of the C language, but a description of the compiler itself \ie the description of a program that builds a program. Notice as well that the output of this bootstrap is an executable representation of that description \ie the machine code that will be loaded and run by the operating system.

Bootstrapping an object-oriented language kernel does not have to be mistaken with just writing a compiler of the language in the same language. The compiler of a high-level object-oriented language uses to output  bytecode of a class or method, which is an incomplete view of it: it does not describe the relation of this class with other objects during runtime. However, following the idea of the C compiler we can then define the bootstrap process of an object-oriented language kernel as follows:

\begin{definition}[Object-Oriented Language Kernel Bootstrap]
It is a process whose input is the definition of a language behavior written in the same language, and whose output is a runtime representation of this language: the language kernel. 
\end{definition}

The input definition of this bootstrap \textbf{must} include the knowledge to recreate itself: the basic operations to create classes and methods, and initialize the basic structures of the language kernel \eg the runtime table of symbols or its threads. The runtime representation that outputs this process is made of a graph of objects \ie the classes, methods, threads and other objects that allow programs to run.

\subsection{Overview}

Following those principles, we developed a bootstrap architecture for an object-oriented language based on \Vtt~(Figure \ref{fig:bootstrapping_overview}). This architecture presents three explicit components decouple the initialization of the language kernel from the \VM initialization so we can define different languages on top of the same \VM. Additionally we can modify the language kernel using the abstractions and tools of the high-level language.

\begin{description}
\item[Language self-description] The code that defines the initialization of the language kernel is extracted and expressed in the same language it defines. This self-description~(from now on the \emph{language definition}) contains the collection of elements that will be created to define the language and how to build them. Thus, during the language initialization we can benefit from the abstractions and tools of the language we are defining.
\item[Virtualized Bootstrapped Language Runtime.] The bootstrapped language is initialized inside a virtualized runtime. As such, we can use the object space clear \VM-language interface for its manipulation. This also serves the purpose of identifying what are the \VM and language concerns during language initialization to easily decouple them.
\item[Bootstrapping Virtual Interpreter.] A specialized code interpreter~(the \emph{bootstrapping interpreter}) executes the language definition on the runtime through its \VM-language interface. This interpreter allows the execution of the high-level code in the not-yet-built language kernel and avoids logic duplications for the manipulation of objects during bootstrap.
\end{description}

\begin{figure}[ht]
\center
\includegraphics[width=.9\linewidth]{bootstrap_nutshell}
\caption{\textbf{Solution overview.} A bootstrapping interpreter uses the self-description in the language definition to build the language through the a clear \VM-language interface.\label{fig:bootstrapping_overview}}
\end{figure}

The main limitation of our approach is the \VM execution model. Our solution exposes a clear but fixed \VM interface that introduces a common denominator for each of the languages we bootstrapped. The co-evolution of language kernel and \VM will be addressed in future work and is out of the scope of this paper.

%============================================================================
\subsection{A Bootstrapping Process for Reflective Languages}

Bootstrapping requires a language kernel to be self-described. That is, there should exist a definition of the language expressed in the language it defines. This language definition includes the \textbf{base-level} entities of the language~(its classes and methods), and the \textbf{meta-level} entities with the means to define itself from scratch~(\eg a compiler or compiler interface to create methods and classes). The latter includes in addition the code of the bootstrapping process \ie the steps that must be followed to create a coherent and well formed language kernel~(Figure \ref{fig:language_definition}).

\begin{figure}[ht]
\center
\includegraphics[width=.8\linewidth]{building_reflective_language}
\caption{\textbf{The language definition in the bootstrap process.} The bootstrapping interpreter uses the meta-level of the language to build the base-level of the language~(left-side). Afterwards, it may inject the meta-level to make it a reflective language~(right-side).\label{fig:language_definition}}
\end{figure}

Notice that although we use the language meta-level for bootstrapping purposes, it may just be needed at bootstrapping time and not at runtime. Bootstrapping does not require to inject the self-representation of the language into the language kernel. We consider that as a special case: a language kernel that includes its self-description is a \emph{reflective language kernel}.
A language that contains both introspection and full intercession facilities is a fully reflective language. A fully reflective language does not only have the minimal set of elements to run, but also the minimal to be autonomous: it can create classes and methods without any external component~(compiler, class builder, interpreter).

Thus, this language definition has direct impact on the design of the language we bootstrap. Figure \ref{fig:phases} shows the stages of a bootstrap process as it installs classes or packages inside the language kernel. When the bootstrap process starts, the language kernel is not yet able to execute code by itself \eg it cannot resolve the method lookup because the class hierarchy is not created. As we install packages or classes, we arrive at the point where the kernel contains already the minimal set of elements it needs for execution, the \emph{execution point}. Later on, we can install reflective features into the language to get inside the \emph{reflective spectrum} of languages. 

%When bootstrapping a reflective language, the language definition must ensure the causal connections that exist between the program and its representation~\cite{Maes87a}.


\begin{figure}[ht]
\center
\includegraphics[width=.9\linewidth]{bootstrapping_phases}
\caption{\textbf{Bootstrap Phases.} Initially, a language kernel does not contain the minimal elements to execute code. As the bootstrap process installs elements, it reaches the point of execution where it can run autonomously. Later on, when (if) reflective features are installed, the process reaches the reflective spectrum, where a language kernel is considered reflective.\label{fig:phases}}
\end{figure}

%\begin{description}
%\item[The base-level language elements.]  We consider here those elements that are part of the \textbf{base-level}~(\eg basic language classes and libraries) of the language and also the ones in the \textbf{meta-level}~(\eg the reifications of the language runtime, a compiler interface, the class builder).
%
%\item[The meta-level language elements.]

\subsection{The process to create the language kernel}
The bootstrapping process requires a particular order as it sets up very interrelated dependencies~(sometimes meta-circular in the case of reflective languages~\cite{Stra14a,Chib96a,Maes87a,Smit84a}) amongst the language elements. The order and specifics to build the language must be described in the language definition so the bootstrapping interpreter can make use of it to orchestrate the process. This process presents also which are the elements that will be introduced into the built language: it is here where we decide if the output language kernel will be reflective or not.

Following we describe a bootstrap process we developed to bootstrap the Pharo language. A bootstrap process may be different for different languages, as they may contain different meta-models and concepts. As an example, Figure \ref{code:process} shows an excerpt of our particular bootstrap process applied to the Pharo language.

\begin{figure}[ht]
\begin{code}
Bootstrap >> bootstrap [
    "Create the basic language structures"
    nilObject := UndefinedObject basicNew.
    trueObject := True basicNew.
    falseObject := False basicNew.
    
    globalTable := GlobalTable basicNew.
    globalTable at: #GlobalTable put: smalltalk.
    
    SymbolTable initialize.
    
    "Solve the class creation bootstrapping issue"
    Class
        instVarAt: 4
        put: (FixedClassLayout
            withInstVars: #(superclass methodDict format layout ...)).
    
    "Create classes"
    ClassBuilder new
        superclass: nilObject
        subclass: #ProtoObject
        instanceVariableNames: ''.
    ClassBuilder new
        superclass: ProtoObject
        subclass: #Object
        instanceVariableNames: ''.
    ...
]
\end{code}
\caption{\textbf{Code (excerpt) of the Pharo language definition.}\label{code:process}}
\end{figure}

%In \cite{Poli13b} Polito et at. 
%we provided a detailed process to bootstrap an object-oriented reflective language such as Pharo. The bootstrap process is not the contribution of this paper. However, for the sake of completion and to aid the understanding of the rest of the paper, we briefly explain in this section the bootstrap process by the means of an example consisting in the bootstrap of the Pharo language. \gp{send them to read our report with the full blown process}


%\paragraph{Step 1: Generate the guest language AST definitions.}
%
%The builder takes as input the specification of the guest language, parses them and generates abstract syntax trees~(ASTs) of the new language's elements~(\eg classes and methods). 
%These AST objects provide to the bootstrap process with (a) the format and shape of classes and objects, (b) the source code of methods to be compiled inside the object space and used by the AST interpreter and (c) compile-time information such as instance and class variable names.

\paragraph{\textbf{Step 1: Create the first well-known objects}}\label{sec:create_nil}

When the bootstrap process starts, the bootstrapped language kernel is empty \ie there are no objects inside it. 
The first step of the process is to create the \ct{nil}, \ct{true} and \ct{false} objects needed for execution. It is important to create the \ct{nil} object first as the rest of the objects will have their fields initialized to it. \ct{true} and \ct{false} are required for code execution.
%%that are created inside the object space 
%will initially have their fields pointing to the \ct{nil} object from the guest language kernel. Then, the first object that the bootstrap process creates is the \ct{nil} object.
%
%In a class based object-oriented language\footnote{These same problems appear in the case of prototype based languages}, the expected way to create the \ct{nil} instance would be to instantiate it from its class. However, the \ct{nil} class does not exist yet in the guest language kernel. Creating it poses some recursive questions: (a) if the a class is an object, how could we create a class without another class for it? and (b) if that class must be initialized with references to \ct{nil}, how can we reference the correct \ct{nil} object if there is no \ct{nil} instance yet? To solve this issue, a \ct{nil} object is created in the guest kernel without linking it to a class, using an unsafe operation and breaking the language's invariants temporarily.

%An \emph{unsafe} \ct{createAnonymousObject} operation to create an object with no class. This operation takes as input the format describing the object to be created, creates the corresponding object, and outputs a mirror on that object. In our particular implementation, this operation introduces a temporary anonymous class in the guest and creates an object from it. This temporary class keeps some references to the \ct{nil} object from the host, breaking temporarily the isolation property. Using this operation, the guest \ct{nil} object is created. Later on, when the corresponding class of \ct{nil} is created, its class relationship is replaced, the temporary class is discarded (unreferenced and garbage collected), and the isolation is restored~(cf. Section \ref{sec:well_known}). The method responsible of the creation of the \ct{nil} object is shown in Figure \ref{code:nil_creation}. Figure \ref{fig:nil_creation} depicts the state of the guest language after this step.



%\begin{figure}[ht]
%\includegraphics[width=.98\linewidth]{nil_creation}
%\caption{\textbf{State of the guest language after the creation of the classless nil object.} The \ct{nil} object is the only instance in the guest language. It is an object whose class is an anonymous class, marked in gray, pointing to the \ct{nil} object from the host.\label{fig:nil_creation} \cam{I would expect the anonymous class to appear outside the guest system rather than inside}}
%\end{figure}


\paragraph{\textbf{Step 2: Create basic language structures}}

The basic language structures of a language kernel is the minimal structure needed to create all the rest of the elements in the language. For example, the language may have a table of globally accessible objects, and a table of unique strings or symbols. These basic structures should be created from the very beginning, as the rest of the process can rely on them.
%
%
%To explain it by example, we take the well-know example of Smalltalk~(Figure \ref{fig:pharo_metaclass_metacircularity}). In Smalltalk each class has a metaclass, instance of a \ct{Metaclass} class. The \ct{Metaclass} class is an instance of a \ct{Metaclass} metaclass. According to this model, the \ct{Metaclass} class and \ct{Metaclass} metaclass should be created before any other class, and thus, it is for us the \emph{basic metacircularity}. It is worth saying that languages such as Pharo or Ruby~\cite{Mats01a} present similar models. \gp{In JavaScript... ?}
%
%\begin{figure}[ht]
%\begin{center}
%\includegraphics[width=.8\linewidth]{pharo_metaclass_circularity}
%\caption{\textbf{Metaclass metacircularity in Pharo.}\label{fig:pharo_metaclass_metacircularity}}
%\end{center}
%\end{figure}
%
%The creation of the first \ct{Metaclass} class and \ct{Metaclass} metaclass is mutually dependent: each one is an instance of the other, and thus, each one needs the other to be created. To overcome this, we use again the same strategy as for the \ct{nil} object. The builder creates the corresponding \ct{Metaclass} class and metaclass, not linking them to each other. Once both of them are available, we update their class links.

\paragraph{\textbf{Step 3: Create classes}}
We create all the classes that the language definition requires in the language kernel. The bootstrapping interpreter uses the corresponding class building mechanism in the meta-level of the language to create the corresponding classes from their descriptions. Methods are not yet installed in their classes.
%Creating a class is a complex operation at this stage. For example, a class has a name which is a symbol\footnote{a Symbol is a string unique in the system}, which in turn cannot be created yet because the Symbol class does not exist yet. Another example is the usage of complex collections such as Dictionaries in the class definitions for class variables.
%by taking each class' AST from the specification, instantiating a metaclass from the \ct{Metaclass} class and the corresponding class from the new metaclass.
% The \ct{asClassMirror} message is sent to the metaclass mirror, so the builder can instantiate the corresponding class from this metaclass. This sub step is shown in Figure~\ref{code:class_creation}. Once the bootstrap process creates all classes in this way, the guest language contains all its classes but they are still not initialized. Classes are yet not related between them and have no methods installed. Figure \ref{fig:class_creation} shows an example of the state of the guest language after this step is finished. 
%In this step the builder doesn't initialize all classes. It lets their fields referencing to the guest \ct{nil} object. No methods are installed yet.

%\begin{figure}[ht]
%\begin{code}
%newMetaclass := metaclass basicNew asClassMirror.
%newMetaclass format: aClassDefinition classSide format.
% 
%newClass := newClassMetaclass basicNew asClassMirror.
%newClass format: aClassDefinition format.
%\end{code}
%\caption{\textbf{Class creation inside the object space.} The mirror to the \emph{metaclass} allows the instantiation of a new metaclass. This new metaclass is afterwards used to instantiate the corresponding class. Each of these classes contain their corresponding formats. This task is repeated for all the classes defined in the specification. \label{code:class_creation}}
%\end{figure}
%
%\begin{figure}[ht]
%\begin{center}
%\includegraphics[width=.98\linewidth]{class_creation}
%\caption{\textbf{State of classes in the guest language after class creation.}  The metaclass metacircularity is set up. Other classes, such as \ct{Object} or the correct \ct{UndefinedObject} are instances of their respective metaclass, which in turn are  instances of \ct{Metaclass}. All slots of these classes~(in particular the slot indicating the \emph{superclass}) reference the guest \ct{nil} object. At this point, two \ct{UndefinedObject} classes exist in the guest, in gray the one created in section \ref{sec:create_nil} referencing to the host language, in white the one we created in this step. \label{fig:class_creation}}
%\end{center}
%\end{figure}

%\paragraph{Step 5: Initialize well-known instances.}\label{sec:well_known}
%
%In this step the well known instances of the language kernel are instantiated and initialized.
%We set the class link of the \ct{nil} object with its corresponding class which was created in the last step. We create also other well known instances such as the \ct{true} and \ct{false} objects.% The process uses the \ct{setClass:} operation of the mirrors to set the \ct{nil} object its corresponding class and restore the isolation of the new language. The \ct{true} and \ct{false} instances are instantiated using their corresponding classes in the object space. 
%The code executing this step is shown Figure~\ref{code:nil_fixation}. 
%After this step, \ct{nil} is not a \emph{classless} object anymore. Also, the guest language is \textbf{transitively closed}, although it is not completely initialized as shown in Figure \ref{fig:well_known_instances}.

%\begin{figure}[ht]
%\begin{code}
%theNil setClass: nilClass.
%theTrue := trueClass basicNew.
%theFalse := falseClass basicNew.
%\end{code}
%\caption{\textbf{initialising the nil, true and false instances.} Fixing the nil reference to the new nil class that was just created. The \ct{setClass:} operation of the mirrors is used. Also, the \ct{true} and \ct{false} instances are created by instantiating their corresponding classes. \label{code:nil_fixation}}
%\end{figure}

%\begin{figure}[ht]
%\begin{center}
%\includegraphics[width=.98\linewidth]{well_known_instances}
%\caption{\textbf{The guest language is finally transitively closed.} After setting the \ct{nil} object with its corresponding class from the guest language, the graph is transitively closed.\label{fig:well_known_instances}}
%\end{center}
%\end{figure}

%\begin{figure}[ht]
%\includegraphics[width=.98\linewidth]{class_initialization}
%\caption{\textbf{State of classes in the guest language after class initialization.}  Each class references to its corresponding superclass. The superclass of \ct{Object} is \ct{nil} as it is the root of the inheritance chain. Class state is initialized~(\eg their names initialized as symbols) except their methods. \label{fig:class_initialization}}
%\end{figure}

\paragraph{\textbf{Step 4: Installing methods}}

We compile~(if needed) and install each of the methods present in the language definition into their respective classes. Method literals are bound to their corresponding literals or global objects (\eg classes).

%\begin{figure}[ht]
%\includegraphics[width=.98\linewidth]{method_installation}
%\caption{\textbf{State of classes after method installation.} Each class has a method dictionary. Each method dictionary references the installed methods. A method object contains the bytecode and references the literals it uses.\label{fig:method_installation}}
%\end{figure}


\paragraph{\textbf{Step 5: initialization}}
%With all the classes and methods from the language specification installed, the structural part of the language is already set up. 
This last step consists in the execution of the class initialize methods to set up elements such as character tables, well-known float values~(\eg NaN or Infinity) and the thread machinery. This means that at this point, our language kernel should be able to execute code by itself.
\newline

While such a bootstrap is not difficult to express per se, it raises the question of how it can be executed. This is challenging especially, since for example we need classes to be defined to execute this exact bootstrap description. This question leads to ask ourselves about the infrastructure required to be able to manage this and other different bootstraps. The following section describes the infrastructure we built to solve this problem.

% ===========================================================================
\section{Bootstrapping with \Vtt}\label{sec:bootstrapping_infrastructure}

We implemented a bootstrapping infrastructure on \Vtt. In our solution, the bootstrapped language kernel will be created inside the virtualized runtime. This virtualized runtime is initially empty and a bootstrapping hypervisor will fill it with new objects before running the virtualized environment. The main component of the bootstrapping hypervisor is a \emph{bootstrapping interpreter}. The bootstrapping interpreter is a specialised AST interpreter that executes the code available in the language definition~(Figure \ref{fig:objectSpaceOverview}).

%  on \emph{object spaces} and abstract interpretation~(cf. Figure \ref{fig:objectSpaceOverview}). An object space is a first class representation of an \emph{object runtime system}: it is an object that provide a high level API to manipulate an object runtime system. An object space \textbf{isolates} its represented object runtime system by using mirror objects~\cite{Brac04b}~(cf. Section \ref{sec:mirrors}). Abstract Syntax Tree~(AST) interpretation solves the remaining issues. First, the combination of object spaces and the AST interpreter allow multiple object runtime systems to co-exist and execute independently, overcoming the \textbf{unicity hyphotesis}~(cf. Section \ref{section:object_spaces} and Section \ref{sec:ast_interpreter}). Second, with AST interpretation we can execute code inside the guest language kernel, using the language specification as a source for both compilation and execution, \textbf{avoiding logic duplications}.
%
%In this section, we present how our solution supports the bootstrap process introduced in Section~\ref{sec:process} and solves our stated challenges. We provide the API of both our object spaces and mirror objects, and how our AST interpreter interacts with the object space infrastructure.

\begin{figure}[ht]
\center
\includegraphics[width=.9\linewidth]{object_space_bootstrap_overview}
\caption{\textbf{Solution overview.} An object space provides with a clear interface to manipulate the bootstrapped language kernel. The AST interpreter interprets the language definition and uses the object space interface to manipulate the language kernel.\label{fig:objectSpaceOverview}}
\end{figure}

To support the bootstrapping process in the time, our bootstrapping infrastructure supports also the idea of continuous bootstrapping \ie integrating the bootstrapping process in a continuous integration environment where changes are common and bootstrapping must be performed continuously.

\subsection{Virtualized Runtime for Bootstrapping}\label{section:object_spaces}

We based our infrastructure on the object space model~\cite{Poli13a}. Following this model, an object space is a first-class representation of an object runtime system, meant for its manipulation and control. The object space model provides a framework to encapsulate and manipulate the runtime system where our language kernel will be bootstrapped. When bootstrapping, this object space is initially empty and the bootstrapping interpreter fills it with classes as the bootstrap process advances.

We use an object space with two main purposes: first, as a \VM interface with the bootstrapping interpreter, allowing us to bootstrap our language kernel directly inside a \VM heap; second, once the language kernel is bootstrapped we use it to initialize the \VM with it and run it. By introducing object spaces into the bootstrapping infrastructure, we clarify the interface between the language kernel and our \VM, allowing us to have different language models as long as they respect this interface. The object space model presents an \ct{objectSpace} interface that provides with general operations and mirror objects~\cite{Brac04b} with operations to manipulate individual objects. Following, we describe this interface, focusing on those operations used by the bootstrap process.

\begin{code}
objectSpace {
    /* For bootstrapping purposes when no class is available */
    mirror allocateObjectOfSize(int size);
}
\end{code}

\subsection{The Bootstrapping Interpreter}\label{sec:ast_interpreter}

The bootstrapping interpreter is a code interpreter, potentially written in language different from the bootstrapped one, that interprets code expressed in the bootstrapped language. Its design present the following important points that allow it to execute code inside the language kernel before it reaches the execution point:

\begin{description}
\item[Alternative method lookup.] Before reaching the execution point, the class hierarchy of the language kernel is incomplete, or part of its methods is not yet installed. An alternative method lookup mechanism is put in place in the bootstrapping interpreter to allow message sending before we reach the execution point: methods are looked up in the definition of the language instead of the hierarchy in the language kernel; a mapping is kept between classes created in the language kernel and their definitions in the language definition to know where the lookup should start.

\item[Automatic class stubs.] The bootstrapping interpreter does also solve most of the well known bootstrapping issues~(\eg how to create a class before a class exists) in a generic way by using class stubs. When an inexistent class is needed during the bootstrap process, the interpreter creates an empty class to take it place respecting the \VM format for it. The interpreter will be able to create instances of this class and map it to its corresponding definition to perform the method lookup. This class cannot, however, initially perform reflective operations as it does not contain any reflective information. When the real class is created later on in the process, it replaces the stub.

\end{description}

\begin{figure}[!ht]
\center
\includegraphics[width=.8\linewidth]{interpretation}
\caption{\textbf{The Bootstrapping interpreter in action.} A stub class is created for a non existent class. Each class is mapped to its description in the language definition. The lookup is then performed inside the language definition. Once the method is found, it is executed inside the language kernel.\label{fig:interpretation}}
\end{figure}

Figure \ref{fig:interpretation} illustrates with an example the behavior of the interpreter, particularly in the execution of the \ct{"Object new"} expression. First, if the class \ct{Object} does not exist, it create a stub \ct{Object} class and maps it to its corresponding definition in the language definition. To interpret the \ct{new} message, the interpreter performs the method lookup from the class of the object in the language definition. As the class from the language kernel and the language definition are mapped, the interpreter knows where to start the method lookup. Finally, the found method is executed in the language kernel and an instance of the \ct{Object} class is created.



By using the bootstrapping interpreter to bootstrap, all the executed logic comes from a single source: the language definition. This avoids  major code and logic duplications as the only one point for extension or modification of the bootstrapped language is its definition. Figure~\ref{code:logic_dup3} illustrates how we can use the interpreter to use the \ct{Dictionary} definition from the language and avoid duplications shown in Section \ref{sec:problems}.


\begin{figure}[ht]
\begin{code}
Bootstrap>>createDictionaryWith: n
    "Create a dictionary in the new language kernel"
    ^ interpreter
            execute: 'Dictionary new: size'
            binding: { 'size' -> n }.
\end{code}
\caption{\textbf{Avoiding logic duplications with the bootstrapping interpreter.} This example shows how the bootstrapping interpreter does not duplicate the logic of the \ct{Dictionary>>initialize} method, but uses it instead.\label{code:logic_dup3}}
\end{figure}


\subsection{Continuous Bootstrapping}

Building continuously a language kernel provides the language engineers with the same benefits of continuously building another application: automated integration and testing, quick and continuous feedback on the applied changes. This continuous feedback should give the language developer with the information and tools to resolve conflicts and problems: it should clearly show which was the \emph{impact} of such change in the process. A change introduced in the language impacts directly on the definition of the language~(Figure \ref{fig:impact}). The changed definition is used in turn by the bootstrap process to bootstrap the new version of the language kernel, thus the change has also an indirect impact on the bootstrapped language. 

\begin{figure}[ht]
\center
\includegraphics[width=0.7\linewidth]{impact}
\caption{\textbf{How a change impacts the bootstrap process.} A change in the language may impact directly in the definition of the language, which in turn impacts in the bootstrapped language.\label{fig:impact}}
\end{figure}

However, not every change in the language definition may impact the bootstrap process: not all the code in the language definition is used during the bootstrap~(\ie executed by the bootstrapping interpreter) and not every change impacts in the behavior of the process (\eg changing the set of final classes introduced by the bootstrap). With this purpose we introduced as a second output of our bootstrapping interpreter, an execution trace containing all the language elements that were used to bootstrap: any change on these elements may have an impact on the process. Then, to produce useful feedback for the changes made by a language developer, an \emph{impact resolver} measures the impact of a change in the bootstrap process by comparing the introduced change to the previous bootstrap execution~(Figure \ref{fig:resolving_impact}).

Our bootstrapping infrastructure measures the impact by making a diff between the traced and changed language elements. In case a change breaks the bootstrap process, the language engineer has enough information to spot the problem and act on it.

\begin{figure}[ht]
\center
\includegraphics[width=.8\linewidth]{resolving_impact}
\caption{\textbf{How a change impacts the bootstrap process.} The bootstrap process execution is traced. An impact resolver decides if the introduced change will impact in the bootstrap process or not.\label{fig:resolving_impact}}
\end{figure}

% ===========================================================================
\section{Validation} \label{sec:bootstrapping_validation}

In this section we present our results while bootstrapping three different three different case study languages.
As all our languages share the same \VM, we start this section by describing the execution model of this \VM and its impact on the bootstrapping languages.
To reuse the parsing infrastructure and the bootstrapping interpreter, the three bootstrapped languages share also the same syntax: a Smalltalk syntax. Although these similarities, each of the three language kernels possess different model and semantics: Pharo is a fully-reflective language composed of classes and traits with first class slots and object layouts \cite{Verw11a}; \emph{Metatalk}~\cite{Papo11a} is a language that fully decomposes the meta-level from the base-level using mirrors, allowing us to bootstrap a reflective and a non-reflective version of it; \emph{Candle} is a partially reflective \ct{Smalltalk-80} based mini-kernel that includes introspection and some self-modification features. Figure \ref{fig:languages_spectrum} shows how these three languages are placed in the language spectrum.% These results show that our process can produce different systems when fed with different specifications. The resulting bootstrapped systems are available at \url{http://ci.inria.fr/rmod/view/Oz/}.

\begin{figure}[ht]
\center
\includegraphics[width=.8\linewidth]{languages_by_reflectiveness}
\caption{\textbf{Bootstrapped Languages Spectrum.} How the languages we bootstrapped are placed in the phases and reflective spectrum. In particular, Metatalk with and without its mirrors is in different extremes of the spectrum.\label{fig:languages_spectrum}}
\end{figure}

Finally in this section we present some measurements. To keep bootstrapping practical, we optimized the critical parts of the process for both the language user and the language engineer. On one side language users do not usually search to modify the language kernel but to use it, independently of the language initialization process it provides. To suit this scenario we do not build the language kernel each time: we generate a snapshot with a cached version of it. On the other side we find language designers/engineers whose job is to change the language kernel. For them the bootstrap process must provide with an acceptable development cycle for activities like debugging. With this case in mind, we optimized the \emph{bootstrapping interpreter} with a dynamic compilation technique. Each of the measurements we present below were made on a 2.2 Ghz Intel Core i7 machine with memory 8 Gb 1333 Mhz DDR3.


\subsection{Pharo Execution Model in a Nutshell}\label{sec:pharo_execution_model}

To understand the common denominator between the three language kernels we bootstrapped, we describe briefly the execution model imposed by the Pharo \VM. The Pharo \VM features a bytecode-based stack interpreter with a generational garbage collector and a JIT compiler. For the interested reader, several publications describe its details and how it evolved during time~\cite{Gold83a,Inga97a,Mira11a}. On the execution model side, this \VM imposes us the following contract:

\begin{description}

\item[Object Format.] All objects in a Pharo \VM have a header and a list of fields. The object header is one, two or three words long and describes amongst others how large is the field list of the object, if those fields contain weak or strong pointers, and which is the class of the object.

\item[Object Model.] The Pharo \VM enforces, in its lowest level, a class-based object oriented model with single inheritance. Each object has a reference to its class~(that appears inside its header). Additionally, each class has three mandatory fields: the class format is used to create new instances and the class' superclass and a method dictionary are used during the method lookup. The \VM during its execution does not enforce the existence of metaclasses nor a particular class hierarchy. This simple model allows one to implement language extensions such as Traits~\cite{Scha03a}.

\item[Bytecode set.] The Pharo \VM constrains methods to a single bytecode set, based on its stack machine. This means that every language that is meant to run on top of this \VM must be compiled to this bytecode set, independently of its original syntax and semantics. 

\end{description}

\subsection{Case Study I: Pharo}\label{sec:bootstrap_pharo}

Pharo~\cite{Blac09a} is an object-oriented reflective Smalltalk-inspired programming language. As it is a Smalltalk-80 inspired language, its class model includes implicit metaclasses: each class has its own metaclass, an instance of \ct{Metaclass}. Pharo also extends the execution model its \VM provides with traits~\cite{Scha03a} and class extensions~(\ie the ability to add methods to a class that belongs to another package). Finally Pharo has first class instance variables (slots) structured in object layouts \cite{Verw11a}. Figure \ref{fig:pharo_simplified_model} shows how the elements of the language are related to each other; the diagram is not meant to reflect the actual class graph but the language concepts.

\begin{figure}[ht]
\center
\includegraphics[width=.7\linewidth]{pharo_simplified_model}
\caption{\textbf{Simplified Pharo object model schema.} In Pharo each class has a metaclass. Metaclasses are defined circularly. Both classes and metaclasses makes use of trait objects to define part of their behavior. Classes also has layout that organises first class instance variables (slots). This schema does not represent the actual object graph, but a simplified picture.\label{fig:pharo_simplified_model}}
\end{figure}

Pharo is a fully-reflective language, placed at the end of the reflective spectrum. The Pharo language includes introspection in the kernel itself, and also self-modification stratified in three levels: object mutation facilities, a class builder and a compiler. The main challenge in Pharo is that the kernel itself of Pharo is defined by Traits: \eg the Trait class uses a Trait. First class slots also add to the self-description of the language. This introduces new bootstrapping issues that must be resolved at bootstrapping time.

%Our resulting language kernel includes the packages defining Pharo's class model, traits, collections, the process scheduling library, the compiler and the class builder. The two latter allow the system to be extensible without external tools. With this selection we bootstrapped a language kernel that represents the 19\% of the original language kernel.
%The memory  the resulting bui language is 2MB, contrasting its 22MB original counterpart. \gp{remeasure it. Do we care about size?}

%Regarding its health, the boostrapped kernel can be tested using the SUnit testing framework.
%Unit tests of the kernel itself are loaded using the binary loader and run in the new system.
%Using this same mechanism, core packages like the compiler are able to be tested isolated from other libraries.

%A peculiarity of this system is that it is capable of bootstrapping a copy of itself.
%This is achieved by loading the binary packages of hazelnut and using it's own specification in the building process.
%Regarding the size of our obtained kernel, which is certainly not yet the minimal possible, our results shows that the design of the language kernel should be refined to create an even cleaner version.

\subsection{Case Study II: Metatalk} \label{sec:bootstrap_metatalk}

Metatalk~\cite{Papo11a} is a reflective language where reflection is fully decomposed in explicit meta-objects, namely mirrors~\cite{Brac04b}. Metatalk makes the usage of reflection explicit: a program's execution takes place in the base-level of the language kernel, and it jumps to a meta-level when a mirror is used. Metatalk class model is simpler than Smalltalk's class model. It does not impose metaclasses. Instead, all classes are instances of the single \ct{Class} class. If there is a need for metaclasses~(to share behavior between classes), the developer can write its own explicit metaclasses~(Figure \ref{fig:metatalk_simplified_model}).

Metatalk mirrors decompose reflective behavior as well as the language meta-information \ie class' names, field order and names amongst others are part of its mirrors, and thus, they belong to the meta-level. When there is not a need for reflection, a Metatalk program can discard its meta-level with all the meta-information in it. This decomposition allows us to bootstrap Metatalk with or without its meta-level. This results in two different language kernels: Metatalk base-level has no reflection at all, while Metatalk with both the base and the meta level is a fully-reflective language.

\begin{figure}[ht]
\center
\includegraphics[width=.7\linewidth]{metatalk_simplified_model}
\caption{\textbf{Simplified Metatalk object model schema.} In Metatalk classes have no implicit metaclass. All classes share the same class. Mirrors are simple objects, thus instances of classes, that reflect on a class and contain their metadata. This schema does not represent the actual object graph, but a simplified picture.\label{fig:metatalk_simplified_model}}
\end{figure}

Metatalk's can be bootstrapped in two different ways. A non-reflective bootstrap initializes only the main classes of the language but does not create its meta-level. The non-reflective bootstrap does not contain mirrors. A second bootstrap creates a reflective Metatalk, which based on the latter one introduces the mirror instances with their corresponding metadata. We could bootstrap easily Metatalk in such a way due to the clear decomposition of its reflective elements. 

%%%%%%%%%%%%%%%%%%% Case of study and Results 2 %%%%%%%%%%%%%%%%%%%%
\subsection{Case Study III: Candle} \label{sec:bootstrap_candle}

Candle is a Smalltalk-based language with a micro language kernel. Its class model includes implicit metaclasses as Smalltalk's and Pharo's one. However, Candle has no support for traits or slots~(Figure \ref{fig:candle_simplified_model}). We built Candle's language kernel by adapting MicroSqueak~\cite{Malo11a} to run on top of the Pharo \VM. This micro language kernel was designed with the explicit goal of being the minimal distribution for the Squeak Smalltalk language.

\begin{figure}[ht]
\center
\includegraphics[width=.6\linewidth]{candle_simplified_model}
\caption{\textbf{Simplified Candle object model schema.} Candle follows a more traditional Smalltalk-80 model. In Candle each class has a metaclass. Metaclasses are defined circularly. There are no traits. This schema does not represent the actual object graph, but a simplified picture.\label{fig:candle_simplified_model}}
\end{figure}


Candle is a partially reflective language defined by a total of 49 classes and a reduced set of methods. Candle includes a minimal core of the language, a basic collection library and basic file IO support. It also provides with object introspection and mutation facilities. It does not include, however, a class builder or compiler to extend itself.%A bootstrapped Candle kernel presents a memory footprint of 80KB, with potential applications in embedded devices with little available memory.\gp{remeasure it. Do we care about size?}

\subsection{Measurements}

In this section we present the benchmarks we did to measure the bootstrap time of each of our three languages using our standard infrastructure. Table \ref{tb:measurements} shows the time to bootstrap each of the three languages using an unoptimised AST interpreter. This time comprehends the entire bootstrap process: from parsing the code in the language definition to its complete setup. We executed each of these benchmarks 10 times. The results table puts also the results in context: it presents how many code entities~(classes, traits, mirrors) and methods are built for each language. Notice that the bootstrapping time depends on the amount of elements it builds and also on their complexity. For example, creating a class in Pharo involves a biggest graph of objects than in the other two languages (because of the introduction of traits and class layouts). Section \ref{sec:optimisations} introduces two optimizations we did based on these measurements, that focus on the startup time and the development cycle of the bootstrap. 

 \begin{table}[ht]
 \small
 	\centering
 	\begin{tabular}{|l|c|c|}
			\hline
			\textbf{Language}
			& \xspace\textbf{Code entities / Methods}\xspace
			& \xspace\textbf{Bootstrap time}\\
		\hline
		Pharo & 626* / 6812 & 9004756ms +/-621265 \\\hline
		Candle & 100* / 875 & 86747ms +/-8060 \\\hline
		Metatalk w/o mirrors & 25 / 114 & 957ms +/-112 \\\hline
		Metatalk reflective & 58* / 166 & 13697ms +/-61 \\\hline
 	\end{tabular}
		\vspace*{0.2cm}
 	\caption{\small\textbf{Building Benchmarks.} Comparing the execution time of the bootstrapped languages using AST interpretation and partial evaluation. (*) Pharo and Candle have implicit metaclasses, meaning that for each created class, an associated metaclass is created even if not necessary. Metatalk introduces a mirror object for each of the classes in the language.\label{tb:measurements}}
 \end{table}

We can observe from our measurements that bootstrapping Metatalk takes in average 1 second if no mirrors are created and 13 in the reflective Metatalk case. Candle bootstrap is slower, in the order of 1 minute and a half, mainly because it contains eight times more methods than the Metatalk. We can see that a plain AST-based bootstrapping interpreter has a a bigger impact in the bootstrap time if the language contains complex structures to initialize.  Indeed, creating a Pharo class using the AST interpreter is an operation that takes in average 17 seconds, because each class contains a reification of its memory layout and slots~\cite{Verw11a}. This problem is aggravated by the high amount of classes and methods in this language definition.

Particularly about bootstrapping Pharo, a lack of modularity of the language impacts in the amount of code elements we have to build. Pharo's language kernel is historically a monolithic system which precludes us to build a minimal system. In fact, the Pharo language kernel we are bootstrapping represents a subset of the full Pharo language as it is distributed.

\subsection{Optimisations}\label{sec:optimisations}

To be useful in practice, we understand that the bootstrap process should have the following two properties: (a) be fast enough to provide a good feedback loop and allow debugging to the language engineer and (b) provide a short startup time for the language users. Optimising a bootstrap process is indeed a challenge since we cannot optimise it statically by fixing the meta-level semantics, as changing them is the main purpose of the bootstrap. In the following sections we show how snapshotting and dynamic compilation aid in these two optimisation scenarios. 

\begin{description}
\item[Enhancing Bootstrap Time: Dynamic Compilation.]
Since the main purpose of the bootstrap process is to easily change the meta-level semantics and structure of the language entities we cannot fix them statically to optimize them. In exchange, we chose to optimize the interpretation cycle using a dynamic compiler. The dynamic compiler compiles the interpreted code on demand. This compiled code is cached and executed directly on the \VM bypassing the interpretation step in following executions. We implemented dynamic compilation to optimize Pharo as it presents the worse of our results~(cf. Table \ref{tb:dynamic_compilation}). We reduced the total bootstrap time by a factor of 2.85. Additionally, we observed a mayor improvement on class creation, where the time improves from 17 to less than half a second. Class creation has a great impact on the Pharo's total bootstrap time, as it is executed 313 times. Contrastingly, the initial setup of the language structures~(\eg the symbol and character table, the initial threads) is executed only once where the cost of our dynamic compilation implementation increases the execution time. Please notice that the current implementation does not optimize method compilation nor parsing, meaning there is still a room for improvement.

 \begin{table}[ht]
 \small
 	\centering
 	\begin{tabular}{|l|c|c|c|}
			\hline
			\textbf{Case}
 			& \textbf{AST Interpretation}
			& \textbf{Dynamic Compilation}
			& \textbf{Gain Factor}\\
		\hline
		Total Bootstrap & 9004756ms +/-621265 & 3158525ms +/-219334 & 2.85x\\\hline
 		Initial Setup (Symbol table, etc.) & 247621ms +/-9875 & 319630ms +/-40333 & 0.77x\\\hline
		Creation of one class & 17216ms +/-1401 & 432ms +/-189 & 39.85x\\\hline
 	\end{tabular}
	\vspace*{0.2cm}
 	\caption{\small\textbf{Comparison of bootstrap time in absence and presence of dynamic compilation.}\label{tb:dynamic_compilation}}
 \end{table}

\item[Optimising Startup Time: Snapshotting.]\label{sec:snapshot}
The user of a programming language is concerned about writing applications that run on this programming language instead of changing the programming language. From a user perspective the initialization of the language is transparent within the startup of an application. It should be however fast and ensure always the same state.
The language initialization present in state of the art \VMs~(Section \ref{sec:intro}) provides both properties. Bootstrapping, in the sense of this paper, turns this process slower due to the interpretation step.

For language users, we overcome this slow-down by \emph{caching} the result of our bootstrap process in a snapshot. Thus, we bootstrap a language kernel only when we change it, and otherwise we load the cached version. Caching keeps both properties of application startup: it guarantees the same state and it is faster. Table \ref{tb:startup} shows a comparison in the startup time of our \VM loading Pharo and Candle using snapshots, in contrast with Ruby. We measured the startup times by running each of them 10 times and making an average. From the results, we observe our startup time is bigger than ruby's but still reasonable, under the half of a second.

 \begin{table}[ht]
 \small
 	\centering
 	\begin{tabular}{|l|c|}
			\hline
			\textbf{Language}
 			& \textbf{Startup time}\\
		\hline
		Ruby &  64ms +/-7.1\\\hline
		Pharo & 280.8ms +/-3.4\\\hline
		Candle & 186ms +/-7.6\\\hline
		Metatalk w/o mirrors &202ms +/-13\\\hline
		Metatalk reflective &205ms +/-11\\\hline
 	\end{tabular}
	\vspace*{0.2cm}
 	\caption{\small\textbf{Startup time in perspective.} Comparing the startup time of a ruby application with the same in Pharo or Candle using a snapshot.\label{tb:startup}}
 \end{table}

Implementation-wise, the snapshot we used is a memory dump of the \VM heap. This heap will contain all the objects, classes and methods we created during the bootstrap. At load time, the memory dump is restored into memory and the \VM internals are re-configured to use this heap using the \VM setup interface~(Section \ref{section:object_spaces}). This idea is the same used by languages such as Smalltalk, Lisp, Javascript in V8 or the JikesRVM~\cite{Alpe00a}. Loading a binary image is as fast as reading the file and putting its contents inside the \VM's heap.

\end{description}

% ===========================================================================
\section{Conclusion and Summary}

% =============================================================================
\input{chapter-footer.tex}
\input{chapter-header.tex}
% ===========================================================================
\chapter{Evolution by Extraction: Tailoring}
\minitoc
% ===========================================================================
\introduction
% ===========================================================================

Deployed object-oriented applications often contain \emph{code units}~(e.g. packages, classes, methods) that the running application never uses~(Section~\ref{sec:problem}).
This problem shows itself more evident and harder to control under the usage of third party software. 
Third party libraries and frameworks are designed in a generic fashion that allows multiple usages and functionalities, while applications use only few of them. 
Examples are logging libraries, web application frameworks or object-relational mappers.

Unused deployed code units have an undesired impact when targeting a constrained infrastructure. 
Constrained devices may have restrictive hardware such as low primary or secondary memory, or even software impositions such as the Android's Dalvik VM restriction to deploy only 65536 methods\footnote{According to dalvik's bytecode documentation~(\url{http://source.android.com/devices/tech/dalvik/dalvik-bytecode.html}), the source register accepts values between 0 and 65535.}. Big JavaScript mashup applications have an impact on loading time due to network speed and parsing time.
These limitations may forbid the deployment of applications that contain lots of code units, or limit the amount of applications and content a user can have in its device.

Existing solutions to this problem propose to eliminate dead code by extracting used code units of an application, and thus reduce their size and memory footprint. The majority of the solutions in the field propose to automatically detect and extract used code units, so called \emph{tailoring}, with static call graph construction as the most dominant technique~\cite{Grov97a}. 
These static approaches present limitations in the presence of dynamic features such as reflection~\cite{Livs05a}, or in the absence of static type annotations. Additionally, they do not allow the user to customize the process of selection to cover different levels of an application's code \ie a user may want to extract only the used application specific code and let third-party and base-language libraries untouched; another user may want apply the process to the whole of the application.

This chapter describes the \emph{run-fail-grow} (RFG) technique: an alternative solution to dead code elimination that identifies at runtime those code units that are actually used in an application~(Section~\ref{sec:model}).
For such a task, in RFG we launch a \emph{reference} application containing all code units~(base libraries, third party libraries and application code) and a \emph{nurtured} application containing a minimal set of code units we want to ensure in our application, so called a \emph{seed}.
RFG consists in ``growing'' the nurtured application into a deployable specialized version of the reference application. 
RFG runs the nurtured version of the application and feeds it with the code units that were detected as missing with a failure.

The resulting deployable application only embeds the seed and used code.
By carefully choosing the seed, the user customizes the scope of the tailoring process making possible different levels of tailoring.
For example, a seed that includes all base libraries makes the tailoring process to only select used code in the application-specific part; whereas an empty seed makes the tailoring process to select used code in all parts: base libraries, application libraries and application-specific part.
% For example, if the seed includes the language base libraries, it ensures that the deployed application will have it all whereas an empty seed will result that only part of the base libs 
% Afterwards, deployment units are created from this shrank application to only contain the used code units.
% Using these compacted deployment units leverages the targeted device limitations. 
The dynamic nature of our solution allows its usage in dynamically-typed languages, and applications using reflection. Our solution does not need to modify the original application thanks to its run-fail-grow approach.
It also successfully deals with applications that make use of programming language features such as reflection or open classes.

We developed Tornado, an RFG implementation using \Vtt~(Section~\ref{sec:implementation}). In Tornado, the nurtured application resides inside a virtualized runtime. We use \Vtt to monitor the execution of the nurtured application to install code when we need to. We validate our work by conducting several experiments (Section~\ref{sec:results}) and comparing our solution to existing related work on application tailoring~(Section~\ref{sec:related_work}).
Finally, we discuss some aspects and trade-offs of the run-fail-grow approach and our implementation~(Section~\ref{sec:discussion}).

%Section~\ref{sec:problem} illustrates the problem of having unused code in application deployment units, it then describes the challenges of application tailoring and finally it presents a list of criteria for evaluating tailoring techniques.
%Section~\ref{sec:model} thoroughly describes our \emph{run-fail-grow} approach to tailor applications.
%Section~\ref{sec:implementation} presents Tornado, our run-fail-grow implementation in the Pharo programming language.
%Section~\ref{sec:results} reports on results of experiments we conducted on two different applications. 
%Section~\ref{sec:related_work} presents an evaluation of our solution and compares it to related work.
%Section~\ref{sec:discussion} discusses some aspects and trade-offs of the run-fail-grow approach and our implementation.
%Finally, Section~\ref{sec:conclusion} concludes this paper and presents some future work.


% ===========================================================================
\section{Problem of Unused Code Units}\label{sec:problem}
Deployed applications contain a set of code units such as classes and methods.
At run-time, required code units are loaded into RAM according to some \emph{loading strategy}.
Scripting languages such as JavaScript, Ruby, Python or PHP have an explicit loading strategy: they load and run a script file and all its declarations when an explicit import statement is found. 
Java uses a transparent code unit loading strategy based on class loaders~\cite{Lian98a}: a class loader loads classes on demand in a transparent way for the application code.
Besides the loading strategy, an application may load a code unit that is only partially used, such as a class with some methods that are never executed.
Additionally, in both situations the deployed application tends to occupy more secondary memory than necessary since the full set of code units should be available to be loaded as we cannot anticipate their usage.


%Figure~\ref{fig:unusedCodeUnits} shows a typical deployment scenario with unused code units.
%
%This problem becomes more evident with the inclusion of third party libraries~(and frameworks). 
%Third party libraries provide often a great variety of code units, usually designed in a generic fashion. 
%They allow multiple usages, while applications tend to use only some of them. 
%Additionally, application developers do not often modify and customize third party libraries to fit their needs but use them as black boxes. 
%Modifying them would mean to lose compatibility with the original development branch of the library and having deep knowledge on the library.
%
%\begin{figure}[ht]
%\begin{center}
%\includegraphics[width=1\linewidth]{components}
%\caption{\textbf{Unused Code Units.} Package P1 contains class A which is used during runtime and class B which is never needed and thus, not loaded. Package P2 contains class C which is partially used~(it contains methods that are never invoked) though it's completely loaded. Class D is loaded because an instance of it is created, but it is never used.\label{fig:unusedCodeUnits}}
%\end{center}
%\end{figure}

Unused code units represent serious drawbacks in constrained devices. 
First, unused code units may forbid the deployment into a constrained resource device.
It may also interfere with the deployment and usage of other applications, because of large memory footprints in both secondary~(disk storage) and primary~(RAM) memory~\cite{Mart12a} or the presence of slow networks in the case of rich web applications.
Second, some deployment targets may have an infrastructure designed in such a manner that forbids the deployment of large applications. For example, the Android's Dalvik VM restricts an application to deploy only 65536 methods per application.

% \noury{I would remove this paragraph}\lf{I rewrote it but yes we can discard it}
% The rest of this section is organized as follows. 
% Section~\ref{sec:example_intro} illustrates these problems through an example and shows what the ideal case would be. 
% Section~\ref{sec:challenges} shows the challenges involved in controlling unused code units in object-oriented applications. 
% Finally, Section~\ref{sec:criteria} defines evaluation criteria that an ideal solution should fulfill.

%\begin{description}
%\item[Wasting storage space.] A deployment unit or some of the code units it contains may be never loaded by the runtime environment. Those unused code units stay in secondary memory wasting storage space that could be used for other applications.
%
%\item[Large memory footprints.] Big loading granularities produce the load of several code units at the same time. When that is the case, the runtime environment may load code units which will never be used. These unused code units produce larger memory footprints unnecessarily.
%
%\item[Mismatch with the runtime representation.] The deployment units contain code units that usually describe the structural~(\eg packages, classes) and behavioral~(\eg methods, scripts) parts of an application. However, the main code units in charge of code execution during an application's runtime are objects, in turn organized in graphs. Objects and object graphs are not directly represented in the deployment units, but through the code that creates them inside scripts and methods. This mismatch between the deployment code units and the code units existing in runtime poses the problem of unused objects: an application may create, store or cache objects that it may not use, and thus, waste memory in them\gp{cite mariano?}.
%\end{description}

\subsection{A Motivating Example} \label{sec:example_intro}

To clearly show the problem, consider the application using a logging library in Figure~\ref{fig:example_dead_code}. In this figure, we emphasize in gray the unused code units that can safely be removed.
%An interface is present in the diagram to show polymorphism between two classes that do not share a class inheritance hierarchy. 
%However, some languages, such as the dynamically typed ones, may not need to represent it in the source code.

\begin{figure}[ht]
\begin{center}
\includegraphics[width=.7\linewidth]{example_dead_code}
\caption{\small\textbf{Example of unused code units.} In gray, the unused code units that can safely be removed.\label{fig:example_dead_code}}
\end{center}
\end{figure}

Figure~\ref{fig:code_example1} shows the code of this application, written in the Pharo Smalltalk language\footnote{To those not versed in Smalltalk-like syntax, these are the equivalents to Java that are required for this example:\\
\ct{new SomeClass();} -> \ct{SomeClass new.}\\
\ct{this.simpleMethod();} -> \ct{self simpleMethod.}\\
\ct{other.methodWithArg(arg);} -> \ct{other methodWithArg: arg.}\\
\ct{/*a comment*/} -> \ct{"a comment"}\\
\ct{"a string"} -> \ct{'a string'}
}. This application contains a \ct{MainApp} class with a \ct{start} method, which is the entry point of our application. The \ct{start} method creates an instance of \ct{Stdout\-Logger} and logs the application's start and end. In turn, the \ct{StdoutLogger} uses the \ct{stdout} global instance to log in the standard output the current time and the message. To print the time, the \ct{StdoutLogger} makes use of the \ct{Time} class from the base libraries of the language. Note that for the sake of clarity, we didn't include in the example all base libraries, though, in modern programming languages they represent a large codebase with several features going from networking to multithreading. For example, Java 8 SE contains 4240 classes\footnote{according to the javadoc API}, and the development edition of Pharo 3.0~\cite{Blac09a} contains 4115 classes and traits.

\begin{figure}[ht]
\begin{code}
MainApp>>start (
    logger := StdoutLogger new.
    logger log: 'Application has started'.
    "do something"
    logger log: 'Application has finished'. )

!\unusedcode{StdoutLogger>>newLine (}!
!\unusedcode{~~~stdout newLine. )}!

StdoutLogger>>log: aMessage (
    stdout nextPutAll: Time now printString.
    stdout nextPutAll: aMessage.
    stdout newLine. )
    
!\unusedcode{RemoteLogger>>log: aMessage (}!
!\unusedcode{~~~| socket |}!
!\unusedcode{~~~socket := self newSocket.}!
!\unusedcode{~~~socket nextPutAll: Time now printString.}!
!\unusedcode{~~~socket nextPutAll: aMessage.}!
!\unusedcode{~~~socket newLine. )}!

!\unusedcode{RemoteLogger>>newSocket (}!
!\unusedcode{~~~"...."}!
!\unusedcode{~~~"creates an instance of socket given some configuration" )}!
\end{code}

\caption{ \small\textbf{Code of the example logging application.} In gray, methods not used by the application.\label{fig:code_example1}}
\end{figure}

In this example we can detect the following unused code units, shown in grey in Figure~\ref{fig:example_dead_code} and Figure~\ref{fig:code_example1}:
\begin{enumerate}
\item The logger library includes two logging classes~(\ct{Stdout\-Logger} and \ct{RemoteLogger}). Only the \ct{StdoutLogger} is used and thus, the \ct{RemoteLogger} class can be discarded.
\item Since the \ct{MainApp} class does not use the \ct{Socket} class nor the \ct{RemoteLogger} class~(the only user of the \ct{Socket} class), the \ct{Socket} class can be discarded.
\item No class in the application makes use of the \ct{Date} class, according that it is not used in the base libs either. Then, this class can be safely removed.
\item The method \ct{newLine}~(lines 7-8 of Figure~\ref{fig:code_example1}) of the \ct{StdoutLogger} class is not used and can be also removed.
\item The \ct{StdoutLogger} class uses the \ct{Time} class to print the current time. Then, all code units that are not related to the \ct{Time now} resolution or printing~(\ie time arithmetic) could be considered as unused.
\end{enumerate}

We would like to generate a new version of this application not containing these unused code units while keeping the application's behavior. We call this technique \emph{application tailoring}.

\subsection{Challenges of Application Tailoring} \label{sec:challenges}

A lot of work exists on the tailoring of statically-typed applications~\cite{Cour10a,Rays02a,Tip03a,Popa04a,Teod01a}, where type annotations aid in the resolution of which piece of code will be used at runtime. 
However, static analysis is not an option in the context of dynamically-typed languages or in the presence of meta-programming and reflection~\cite{Livs05a}.
%~\cite{Mart12a}
In this context of dynamically-typed and object-oriented programs that may use reflection, we identify the following main challenges in creating tailored applications:

\begin{description}

\item[Language Runtime Unused Code Units.] As the core point of this thesis, we would like to extract not only application code but also code that belongs to the language base libraries, including its core meta-model. This presents the problems already stated in \chapref{background}.

\item[Dynamic typing.] Dynamically-typed languages cannot benefit from the most powerful static analysis due to the absence of type annotations. Name-based static analyses~(static analyses that build a simpler call graph based only on method names) can be used on them, but are not as efficient. Static techniques to detect code unit usage, such as call-graph analysis, need the support of more dynamic techniques \eg tracking runtime information, following the application's execution flow, or performing symbolic execution.

\item[Polymorphism and inheritance.] Polymorphism in object-oriented languages allows a code unit to treat objects of different concrete types in the same way as soon as they share a common interface. Inheritance plays a similar role: any class can extend another class and provide different behavior while sharing a common API.
As a consequence, both polymorphism and inheritance make the behavior of a program more difficult to predict by just statically analyzing its code units~\cite{Taen89a}.

\item[Application runtime configuration.] Modern applications often contain libraries and frameworks besides their proper code. 
To make these different code units fit together, applications rely on heavy configurations. 
These configurations are usually present in configuration files looked up dynamically by the application. 
Based on these configurations, the dependency injection pattern is usually used to dynamically set up the application. 
This recurrent and standard process for configuring applications implies that static analysis will be inefficient to detect used code units without library-specific knowledge.
 % The configuration code unit adds another dynamic element to the application, making its behavior more complex to predict with static approaches.

%\item[Application configuration granularity.] An application's configuration is not often granular. Libraries and frameworks may initialize during their startup lots of objects that are not used during the application's life cycle. These configuration objects may remain in static/class fields cached until the application is stopped, impacting in the memory footprint \gp{this is a problem, not a challenge. The challenge is to be able to detect unused objects in addition to methods}.

\item[Reflection.] Reflection makes static analysis inoperative by allowing an application to execute unanticipated pieces of code. 
Any \ct{String} resulting from a program execution or program configuration can denote a message send\footnote{We refer method invocations as message sends because they represent better from our understanding the dynamic property of the invocation.}, the name of a class to be instantiated, or even a script to be executed. Reflection is indeed important to cover, since it is a broadly used tool in industrial applications with object relational mappers such as Hibernate or Glorp and web frameworks such as Ruby On Rails, Struts or Seaside.
\end{description}

\subsection{Evaluation Criteria for Application Tailoring} \label{sec:criteria}


This section presents properties that we consider the most relevant to evaluate techniques addressing the issue of unused deployment code units.

\begin{description}

\item[Reflection.] An ideal tailoring solution should handle correctly reflective code and resolve the unanticipated code executions in the same way as the application would do during runtime.

\item[Base Library Specialization.] A programming language contains several base libraries covering very common and generic tasks. Not all the code units in these libraries are used in an application. An ideal tailoring solution should tailor base libraries of the language to reduce an application's deployment memory footprint.

\item[Third Party Libraries Specialization.] Applications use several third-party libraries and frameworks covering several aspects of application development such as user interfaces, persistence or publication of services. Third party libraries contain large code bases and many dependencies. Thus, an ideal tailoring solution should consider the existence of third-party software.

\item[Legacy Code.] An ideal tailoring solution should be applicable on already existing applications and not require modifications on them.

\item[Dedicated Infrastructure for Deployment.] An ideal tailoring solution should produce a version of the application that is able to run on the official production infrastructure~(such as the VM) without overhead.

\item[Flexibility.] An ideal solution for tailoring an application should support many different levels of application. Some applications may not need to tailor base libraries because they are shared with other applications. However, tailoring base libraries may be useful on those applications residing alone in constrained devices.

\item[Applicable without type annotations.] An ideal tailoring solution should be applicable to dynamic languages with no type annotations.

\item[Full Coverage.] An ideal tailoring solution will guarantee that all code units selected as part of the deployable application are those needed. That is, it does not contain extra code units, nor it misses code units.

\end{description}


% ===========================================================================
\section{Run-Fail-Grow: a Dynamic Approach} \label{sec:model}

\subsection{Run-Fail-Grow in a Nutshell}

We propose run-fail-grow (RFG) for tailoring. Briefly, RFG works by launching a \emph{reference} application encompassing the full application with all its code units~(base libraries, third party libraries and application code) and a \emph{nurtured} application that has only part of its required code units installed. The nurtured application is run, and when a failure is detected because it misses a code unit, we install into it the corresponding code unit from the reference application. Thus, the nurtured application grows progressively as missing code units are found and solved.
Once finished, the nurtured application is ready to be deployed on target devices. Figure~\ref{fig:runfail} depicts the basics of our run-fail-grow approach.

\begin{figure}[ht]
\begin{center}
\includegraphics[width=.7\linewidth]{runfail}
\caption{\small \textbf{Application tailoring with a run-fail-grow approach.} We~(1) run the nurtured application and~(2) detect the missing units on failure. For each failure,~(3) we copy missing code units from the reference application and then~(4) the execution is resumed (just before the failure point) until the process finishes. \label{fig:runfail}}
\end{center}
\end{figure}

On one side, we start the reference application and pause it at the point where either it contains all its collection of code units, either we can load them dynamically~(under a lazy loading strategy). The reference application remains paused to avoid to mutate its state during the tailoring process. Pausing consists in suspending all processes and threads from the application.

On the other side, we fill initially the nurtured application with those code units we want to ensure in the final application \ie the seed. This seed allows us to specify the level of specialization of our deployable application.
By using a seed that contains all base libraries, RFG will only affect the application specific code units and third-party libraries.
However, by using an empty seed, it will also tailor base libraries~(cf. Section \ref{sec:seeds}).

Running the nurtured application consists in two main steps. We first install in the nurtured application one or more \emph{application's entry points} in the form of threads, and then we start to run it.
The execution of these entry points results into sending messages to those objects that start the application. These messages will produce \emph{missing code failures} when the respective classes and methods to resolve the message are not available.
We detect the missing code failures and solve them by fetching the needed code units from the reference application and install them into the nurtured application. RFG installs only code units on demand \ie the content of installed classes and objects is not installed with them but deferred until it is actually needed; methods are not installed until they are invoked.
The process repeats until we end it explicitly, so we can interact with the application as part of the process. Ideally, the nurtured application reaches a stable point where it needs no more code units.
The nurtured application is then ready for deployment.

The dynamic nature of RFG tackles all our challenges. Missing code units are detected and resolved at runtime, where two main elements are available: the exact messages that are sent with their corresponding receiver and arguments, and their concrete types\footnote{We consider the exact class of an object its concrete type}. The methods and classes to install can be easily deduced from the available concrete types, \emph{without depending on type declarations} nor \emph{guessing in case of polymorphism or inheritance}. \emph{Application configurations are honored} since the code that reads and interprets them is actually executed, without the need of custom code for them. \emph{Reflection is supported for free} since reflection invocations are treated as simple message sends and executed as any other code, and strings composed dynamically by the application are available at runtime. 


%%%%%%%%%% 


\subsection{Run-Fail-Grow through an example}
We illustrate in this section the ideas behind RFG with the example introduced in Section~\ref{sec:example_intro}. For the sake of clarity, in this example we will tailor the application's code units and not the base libraries \ie the seed includes the base libraries.

\paragraph{Setup of the Environment.} First, we launch the reference application~(cf. Figure~\ref{fig:example_reference}) and the nurtured application~(cf. Figure~\ref{fig:example_all} Step 0). We fill the nurtured application with a seed containing the language base libraries. Thus, each application has its own copy of the base libraries, as shown in this case with the \ct{Date} and \ct{Time} classes and the \ct{stdout} object.

\begin{figure}[ht]
\begin{center}
\includegraphics[width=.5\linewidth]{example_reference}
\caption{\small\textbf{Reference application with all code units.}\label{fig:example_reference}}
\end{center}
\end{figure}


\begin{figure*}[ht]
\begin{center}
\includegraphics[width=.322\linewidth]{example_before}
\includegraphics[width=.33\linewidth]{example_starting_point}
\includegraphics[width=.33\linewidth]{example_dnu_trap_start}
\includegraphics[width=.33\linewidth]{example_shadow_trap}
\includegraphics[width=.325\linewidth]{example_dnu_trap}
\includegraphics[width=.33\linewidth]{example_finished}
\caption{\small\textbf{The nurtured application at different steps of tailoring.} \label{fig:example_all}}
\end{center}
\end{figure*}

\paragraph{Install the application's entry point.} We install into the nurtured application our application's entry point \ie a \ct{MainApp} instance~(\ct{aMainApp}) and a process that will execute the statement \ct{"aMainApp start"}~(cf. Figure~\ref{fig:example_all} Step 1). Note that although we are referencing an instance of the class \ct{MainApp}, the \ct{MainApp} class is not installed yet.

%\begin{figure}[ht]
%\begin{center}
%\includegraphics[width=.8\linewidth]{example_starting_point}
%\caption{\textbf{Installing an entry point.} The nurturer installs the entry point of the application and starts its execution sending it the \ct{start} message.\label{fig:example_starting_point}}
%\end{center}
%\end{figure}

When the execution starts, the \ct{mainApp} instance receives the \ct{start} message, and we detect the \ct{MainApp} class and its \ct{start} method as a missing code unit failure. We install these two missing code units~(cf. Figure~\ref{fig:example_all} Step 2) and finally the \ct{MainApp>>start} method is activated and starts running.

%\begin{figure}[ht]
%\begin{center}
%\includegraphics[width=.8\linewidth]{example_dnu_trap_start}
%\caption{\textbf{Activating the entry point.} The nurturer installs the \ct{MainApp} class and the \ct{start} method on demand.\label{fig:example_dnu_trap_start}}
%\end{center}
%\end{figure}

\paragraph{Activating the start method.}
The method \ct{start} defined in Figure~\ref{fig:code_example1} is executed, as we can see in Figure~\ref{fig:example_all} Step 2. During the execution of its first statement~(line 2 Figure~\ref{fig:code_example1}) we detect a missing code unit failure: The \ct{StdoutLogger} class does not exist. Thus, before continuing, we install a \ct{StdoutLogger} class with the same shape as its reference counterpart~(cf. Figure~\ref{fig:example_all} Step 3). This class does not contain, however, all the methods nor the meta-data (\eg superclass, package, subclasses) from the reference class since they may not be necessary.

%a \emph{shadow} of the \ct{Formatter} class~(Figure~\ref{fig:example_shadow}).\gp{I don't like the word shadow here, since it does not go with the tailor metaphor} The \ct{Formatter} shadow is a proxy object the tailor will use to know if and when the \ct{Formatter} class is used.

%\begin{figure}[ht]
%\begin{center}
%\includegraphics[width=.8\linewidth]{example_shadow_trap}
%\caption{\textbf{Activating the \ct{new} method.} The nurturer installs a the \ct{StdoutLogger} class and sends it the \ct{new} message.\label{fig:example_shadow_trap}}
%\end{center}
%\end{figure}

%Once the \ct{Formatter} shadow is available, the tailor can continue the execution and send the message \ct{new} to the shadow object. At this point, the tailor finds out that the receiver of the \ct{new} message is a shadow object, and thus, it replaces the shadow object by

%Note that copies of classes are treated specially: they contain a method dictionary where methods are installed, but since no instance of \ct{Formatter} received yet a message, its method dictionary is empty.

Once we install the \ct{StdoutLogger} class, we resume the execution. The first statement results into a new \ct{StdoutLogger} instance. Not that the \ct{new} method is not installed because this method is part of the language base library, already available in the seed. 
% The next expression of the \ct{start} method is now ready to be executed.

During the second statement's execution~(line 3 Figure~\ref{fig:code_example1}), we detect a missing code unit failure on the \ct{log:} message~(cf. Figure~\ref{fig:example_all} Step 4): the corresponding method is not installed in the \ct{StdoutLogger} class. We install the method inside the corresponding class and resume the execution. This time the method is found, and the \ct{log:} method is activated.

%\begin{figure*}[ht]
%\begin{center}
%\includegraphics[width=0.65\linewidth]{example_dnu_trap}
%\caption{\textbf{Activating the \ct{log:} method.} The message \ct{log: 'Application has started'} is sent to the \ct{logger} object. \label{fig:example_dnu_trap}}
%\end{center}
%\end{figure*}


Once the \ct{log:} method finishes, the execution returns to the \ct{start} method. There, the third statement~(line 5 Figure~\ref{fig:code_example1}) is executed with no intervention of our technique, since the \ct{log:} method is already available. Figure~\ref{fig:example_all} Step 5 shows the final state of the nurtured application: it contains only the methods and classes that are actually used by the application. Leaf objects used during the process have been garbage collected.

%\begin{figure}[ht]
%\begin{center}
%\includegraphics[width=.8\linewidth]{example_finished}
%\caption{\textbf{Final state of the nurtured application.} The nurturing has finished and the resulting application is tailored.\label{fig:example_finished}}
%\end{center}
%\end{figure}

\subsection{Detecting Missing Code Units}\label{sec:model_detail}


RFG depends on getting notified when a missing code unit failure appears. RFG's algorithm is based on traps to achieve this task, as shown in Algorithm~\ref{alg:tailoring_process}. Traps are placeholders that are installed in the nurtured application in the place of real elements. They are triggered whenever the application tries to access them. In case a trap is triggered, we stop the nurtured application execution, we install the missing code units replacing their corresponding traps, and finally resume the execution from the moment immediately before the trap was triggered. Traps are installed dynamically in the nurtured application following the information flow of the application \eg when a method \ct{A} is installed some traps are installed on it to capture possible missing code unit failures it may cause.

We identified the following as the basic traps that are necessary to tailor an application:

\begin{description}
\item[Missing object trap.] A \emph{missing object} trap captures messages sent to objects that do not yet exist inside the nurtured application such as classes. When RFG finds one of these traps, its responsibility is to install the corresponding object. The object installed is a \emph{partial} clone of the original object \ie not all of its state is installed, instead it contains traps to capture the access to its class and fields. When a method refers to a static variable, we install one of this traps for it.

\item[Missing method trap.] A \emph{missing method} trap captures me\-thod invocations whose methods are not defined in the nurtured application yet. When the application execution triggers one of these traps, RFG installs the corresponding method in the class hierarchy of the object. In case of missing classes, RFG installs them too. Missing method traps capture also overridden methods. If an overridden method is not trapped, the method lookup may find a superclass implementation and execute it, resulting into an unexpected behavior. Figure~\ref{fig:need_override} illustrates this problem: the class \ct{B} from the reference application contains an override, while it is not present in the nurtured application. If no trap is placed to capture the override, the method \ct{doSomething} from class \ct{A} would be executed, thus changing the semantics of our application.

\end{description}

\begin{algorithm}[ht]
 %\KwData{this text}
 %\KwResult{how to write algorithm with \LaTeX2e }
 Initialize reference application\;
 Initialize nurturing application with the seed\;
 Install entry point(s)\;
 \While{not finished}{
  run the nurtured application\;
  \If{trap was activated}{
   install missing code units\;
   restart message send;
  }
 }
 \caption{\small An abstract view of the run-fail-grow process \label{alg:tailoring_process}}
\end{algorithm}


\begin{figure}[ht]
\begin{center}
\includegraphics[width=0.7\linewidth]{need_override}
\caption{\small\textbf{The need of overriding traps.} \small Method traps should capture the overridden \ct{doSomething} message-send to avoid the superclass method to be executed wrongly.\label{fig:need_override}}
\end{center}
\end{figure}



\subsection{Customizing Dead Code Elimination with Seeds}\label{sec:seeds}

The level of tailoring of RFG can be specified using seeds. A seed is a collection of code units whose installation is forced into the nurtured application. These code units are available for the nurtured application and thus, accessing them does not trigger missing code unit failures.

A seed can contain any arbitrary code unit, including package, classes, methods and even already initialized objects. Seeds are useful to cover different tailoring scenarios. Let's take as a first example a smartphone where the base libraries of the language are already available, so they are shared amongst the many applications installed in it. When targeting such a smartphone, base libraries are already present and we do not need to produce an specialized version of them, but specialize only third part libraries and application code. In this case, we use a seed providing the language base libraries. Let's take as a second example a constrained device robot-like which will contain only our application. When targeting this robot as deployment scenario, we want to specialize all the code to deploy including base libraries. In such a case, the seed is empty to allow the RFG algorithm to work on every code unit.

Figure \ref{fig:nurturing_map_model} presents the tailoring map showing two examples of usage of seeds. Each application contains code units corresponding for the base libraries, third party libraries and application code. In the left the seed covers base and third party libraries, thus RFG applies and selects a subset of the application code units only. In the right, the seed covers only the base libraries, thus RFG applies and select a subset of the code units from the third party libraries and application code.

\begin{figure}[ht]
\begin{center}
\includegraphics[width=.7\linewidth]{nurture_map}
\caption{\small\textbf{Tailoring Map.} A tailoring map describes which code units of an application are included in the seed~(in gray), which ones are subject to the RFG technique~(in white) and the amount of them that are finally selected~(within the thick area).
\label{fig:nurturing_map_model}}
\end{center}
\end{figure}


%============================================================================
\section{Tornado: RFG in \Vtt} \label{sec:implementation}

\subsection{RFG's Implementation Requirements}\label{sec:requirements}

We identify the following requirements for a development platform to implement RFG.
%Some platforms present all of these elements while some others present only part of them. In the latter case, the missing elements should be developed as part of the solution.
In this and the following sections we explain how we fulfilled each of these requirements and how we put them together to implement our solution. Note that these features are only needed to implement RFG and prepare an application for deployment. Once RFG is applied, we must be able to deploy our application on the standard platform infrastructure~(virtual machine, operating system), without special support.

\begin{description}
\item[Control application's execution.] RFG requires full control on an application's execution. It needs to be able to suspend all threads of an application when a trap is triggered, and to resume them once the trap is handled.

\item[Capture message-sends.] RFG requires to intercept an application's execution at runtime to detect missing code unit failures, and thus, to implement traps. In particular, it needs the ability to intercept all message sends of the application, and in particular method invocations.

\item[Install and Query Code at Runtime.] RFG requires a platform where it is possible to install code and query the installed code at runtime. Classes, methods and objects have to be installed at any moment of the execution, including the modification of classes that already contain instances, or objects that are already cloned. Also, we need to fetch the code units installed in the reference application 
\end{description}

\subsection{Tornado's Overview}\label{sec:infrastructure}

We implemented our RFG technique as a tool called \emph{Tornado}. Tornado is implemented using \Vtt, to tailor applications written in the Pharo programming language.
%Pharo is a reflective and dynamic programming language inspired from Smalltalk.
Tornado's architecture combines \Vtt~(cf. Section \ref{sec:oz}) and Ghost proxies~(cf. Section \ref{sec:proxies}) illustrated in Figure \ref{fig:tornado_code units}. The nurtured and reference applications application are hosted inside virtualized runtimes. Tornado provides with a hypervisor that initiates and pauses the reference and nurtured applications. Tornado's hypervisor runs and monitors the nurtured application. It installs traps on the nurtured application as Ghost proxies, and use the object space interface to query and install code units it. Following, we detail how we fulfilled each of RFG's requirements in our solution:

\begin{figure}[ht]
\begin{center}
\includegraphics[width=0.8\linewidth]{tornado_components2}
\caption{\small\textbf{Tornado's architecture overview.} Tornado controls both the reference and nurtured applications through Oz. Traps are installed into the nurtured application with the Ghost library.\label{fig:tornado_code units}}
\end{center}
\end{figure}

\begin{description}
\item[Execution Cycles.] We use \Vtt execution cycles to monitor and control the execution of a nurtured application. When a cycle is finished the Tornado hypervisor checks if the virtualized runtime is suspended on a trap. In such case, it installs the corresponding code unit and resumes the execution with another cycle.

\item[Advanced proxies.] Pharo's libraries includes Ghost~\cite{Mart11a}, an advanced proxy implementation. Ghost allows one to capture all kind of message sends, intercept particular method executions, and even to proxy classes and special objects. We use the Ghost model to implement execution traps.

\item[Object Space Runtime Manipulation.] An object space provide already with operations to query and install the classes and methods in the virtualized runtime.

\end{description}

\subsection{Execution Traps with Ghost Proxies} \label{sec:proxies}

Implementing execution traps such as the ones described in Section~\ref{sec:model_detail} requires a powerful intercession module or library. Traps must capture \emph{all} message sends to objects provided by the language runtime as well as the application objects, including classes~(for example for the case of class messages or static methods). They must capture \emph{self} and \emph{super} message sends, as well as overrides and particular method invocations.

To achieve this, we implemented a set of proxies following the Ghost model~\cite{Mart11a}~(similar to JavaScript proxies~\cite{Vanc10a}). Ghost proposes a low-memory footprint, general purpose proxy implementation for the Smalltalk language supporting the creation of proxies for normal objects as well as classes and methods. 
% These proxies allow the interception of all message sends.
% When a Ghost proxy captures an invocation, it triggers the behavior of a \emph{handler}. 
% The Ghost proxy library provides already several kinds of handlers including for example message forwarding or logging.
Ghost proxies allow the detection of all situations corresponding to our traps.
Tornado handles a table relating each proxy to the code unit or object it represents in the reference application.
%Ghost proxies capture all messages.
Additionally, each proxy is attached to a \emph{handler} that may perform some action when the proxy receives a message.
We rely on this concept to perform the right action for each trap.
We discuss below the different kinds of proxies and handlers we use and how they support RFG.

\begin{description}
\item[Missing object trap.] This trap is implemented as a proxy taking the place of a normal object. This trap is triggered when the proxy receives a message.
Its handler replaces the proxy by a copy of the original object from the reference application.
The copy is created, and all references to the proxy are replaced by references to this new object, which is achieved through the \ct{become:} facility of the Pharo language that dynamically swaps object references.
Each field and the class of this new installed object are installed as new missing object traps.

\item[Missing method trap.]  We implemented the missing method trap in Tornado as a class proxy located at the top of the class hierarchy. Whenever a message is sent to an object, the VM looks up the method in the object's class hierarchy. This trap is triggered whether a message arrives to the top of the hierarchy, meaning that there was no method for it in the hierarchy. When triggered, the handler installs the classes part of the hierarchy of this method and the missing method in its corresponding class. If no method is found to install, Tornado sends the \ct{doesNotUnderstand:} message~(an equivalent to \eg Ruby's \ct{method\_missing} and Python's \ct{\_\_getattr\_\_}) to honor the dynamic semantics of Pharo.

\item[Missing override trap.] We implemented missing override traps in Tornado using method proxies. Method proxies are placed in the method dictionaries of classes containing overriden methods, taking the place of the original method.  When Tornado installs a class into the nurtured application that contains overridden methods in the reference object space, it installs into this class a method proxy for each of its overridden methods. This trap is triggered whenever the method proxy it is about to be executed. The handler of this trap compiles a new method with the same source as the corresponding method from the reference application and installs it inside the nurtured object space.

\item[Primitive methods trap.] Primitive method traps are implementation specific related to the Pharo language. Pharo's primitive operations such as number arithmetic are implemented through primitive methods. Primitive methods are implemented in the Virtual Machine and do often access directly the fields of its receiver and arguments by forging references and manipulating directly the memory, bypassing our traps. Thus, we face an issue when a \emph{missing object trap} proxy is the argument of such a method: the VM can modify this proxy without activating the trap. Primitive method traps are method proxies that decorate Pharo's primitive methods. When they are triggered because one of these methods is about to be executed this trap's handler triggers each of the missing object traps received as arguments, if any. In this way, Tornado forces the installation of the arguments and the primitive is executed with actual objects instead of proxies, as expected.

\end{description}

% \subsection{Traps and Proxies}\label{sec:traps}
% Tornado traps notify the nurturer when some object is missing. In such case, the tailor installs the missing object and continues the execution from the point where it was trapped. Traps are implemented in Tornado through Ghost like proxies. Each trap in tornado is a Ghost proxy. Tornado implements a custom interception handler: every time a trap is activated, the object space where the trap is located gets paused and the control is returned to the tailor to treat the trap. Following, we explain each of the traps implementations and how they are treated by the tailor:
% 
% \begin{description}
% \item[Not-installed trap.] A trap for an object or class not yet installed in the system is a simple proxy. If the proxy receives a message, the tailor installs the object the proxy represents and replaces the proxy by the new object. If the new object has fields, these fields will be propagated as proxies or installed depending on its particular mapping~(cf. section~\ref{sec:mappings}). In our particular implementation, the replacement of the proxy by its new counterpart is done through the \ct{become:} facility of the Pharo language that allows pointer swapping.
% \item[Does not understand trap.] We implemented the does not understand trap in tornado as a class proxy situated on the top of the class hierarchy. Whenever a message is sent to an object, the method lookup mechanism searches a method with the same firm in the object's class hierarchy, starting from the object's class up to the top. Our does not understand trap captures whether a message arrives to the top of the hierarchy, meaning that there was no method for it in the hierarchy. When this trap is activated, the tailor looks in the model application for the method to install, installs it in the nurtured application and finally restarts the execution from the call-site that activated the trap.
% 
% \item[Override trap.] We implemented override traps in tornado using method proxies. Method proxies are activated whenever the proxified method is about to be executed. When Tornado installs a class that contains overridden methods in the reference object space, it also installs into it one method proxy for each of its overridden methods. In such a way, the method lookup mechanism finds the overridden traps and returns the control to the tailor. In turn, the tailor installs the method corresponding to the trap and restarts the execution from the call-site that activated the trap.
% 
% \item[Primitive methods trap.] \gp{il faut l'ecrire}
% 
% \end{description}
% 
% \gp{Traps are placed in the tailoring application so they get activated only when the special cases are given. They do not represent a penalty on runtime during the tailoring process.}
% 
% Note that every time a trap is captured and treated, restarting the execution must handle correctly self and super message-sends\footnote{A super message-send starts the method lookup from the superclass of the class where the actual method in execution is located, instead of the class of the message receiver}. Thus, Tornado always restarts the execution respecting the call-site and the class where the method lookup mechanism started to maintain the program's semantics, regardless the kind of trap that was activated.


\subsection{Object Installation and Propagation Rules} \label{sec:mappings}

As we explained before, Tornado installs all objects inside the nurtured application on demand, as \emph{partial copies}, \ie the objects referenced by the original object will not be copied along with it by default, but traps are placed instead of them. %That is, no object, class or method is installed unless it is needed by the application to run. The need for installing some element is detected by means of traps~(cf. Section~\ref{sec:model_detail}).
%The tailored application contains during the tailored process two kind of objects: application objects that are copies of the model application objects, and proxy objects representing traps.
When Tornado installs an object inside the nurtured application, this new object has the same format and size as its original counterpart. \emph{Propagation rules} determine how each of the object's fields are propagated on installation. Tornado provides the following propagation rules to customize installation:

\begin{description}
\item[Missing object trap.] This is the \emph{default propagation rule} and end user applications can usually be tailored with just them. This propagation rule installs a missing object trap in each field of the object that is being installed.
\item[Materialization.] This propagation rule forces the installation of the object referenced by the field. This is used for those cases where some structure should be guaranteed to the Virtual Machine \eg The first three fields of class objects~(superclass, format and method dictionary) cannot be proxified because they are used by the VM for the method lookup. The same happens with other objects reifying low level concepts such as activation records or semaphores.
\item[Swapping.] This propagation rule forces the reference of the object installed be swapped to another object's reference. The usual use case of this rule is replacing some object reference by \ct{nil}, and so, to force lazy initializations.
\end{description}

% Figure \ref{fig:character_mapping} shows an example of the mapping of the \ct{Character} class which forces the materialization of the character and digit tables class variables.
%
%\begin{figure}[ht]
%\small
%\begin{code}
%Character >> tornadoMappingFor: aTornado
%	^ (super tornadoMappingFor: aTornado)
%			mapClassPoolWith: { 
%				#CharacterTable -> ToMaterializationPropagationStrategy new.
%				#DigitValues -> ToMaterializationPropagationStrategy new
%			}; yourself
%\end{code}
%\caption{\small\textbf{Mapping for class \ct{Character}.} Mapping forcing the installation of the character and digit tables needed by the VM. \label{fig:character_mapping}}
%\end{figure}

\subsection{Object Identity and Proxies}

Tornado takes care of the identity of objects with an identity table. The identity table is important because Tornado works at the object granularity. Due to the inherent graph nature of object-oriented programs, an object being installed may reference another object that is already installed inside the nurtured application.

Identity is also important to preserve in the presence of proxies. Tornado guarantees that identity checks always preserve object identity by following the following invariant: \emph{An object and its proxy do not exist concurrently in the nurtured application.} That is, the nurtured application contains either the object, either its proxy, but not both at the same point in time. When the proxy is replaced by the actual object's copy, all references to the proxy are swapped to references to the new object. The proxy is no longer referenced and thus, garbage collected. This invariant guarantees that identity checks that should be \ct{true} will indeed be \ct{true} because either the compared references point both to the same proxy, or both to the same copy.

\subsection{Implementing Seeds in Tornado}

Tornado's seeds specify the level of tailoring. The seeds are in charge of initializing the nurtured application's virtualized runtime with the elements we want to ensure on it. Our current prototype supports two ways of describing and building seeds: 

\begin{description}
\item[Loading an already existing memory snapshot.] The nurtured application's object space is initialized by loading an already existing snapshot or image~(\ie this is an image in the same sense as Smalltalk or Lisp). This technique consists in using a memory dump from an object heap containing all the classes and objects desired in the seed. This memory snapshot should follow Pharo's object format. 
\item[Creating all seed code units from scratch.] The nurtured application's object space is initialized with objects built from scratch. This technique uses a bootstrapping process as described in \chapref{bootstrapping}.
\end{description}

%Note that a seed can indeed contain any arbitrary code units and objects. They are not restricted to have only base or third party libraries. The selection or extraction of what is included as part of a seed is application dependent and orthogonal to the run-fail-grow process. We show an example of it in Section \ref{sec:results}.

\subsection{Preparing the Application for Deployment}\label{sec:deploy}

Once Tornado is stopped, the nurtured application contains all the code units needed to run. Tornado procceeds to prepare the application for deployment \ie it removes all trap leftovers and extracts the nurtured application. Tornado identifies the traps by the presence of proxies and replaces the references to those proxies by references to another object, defaulting to the \ct{nil} object. Proxy objects do not then represent a drawback in space consumption because they are garbage collected. Once the traps are removed, the nurtured application keeps no dependencies to Tornado nor its infrastructure. Thus, the application can run outside the Oz infrastructure with no performance penalties.

Finally, Tornado extracts the application code units using one of two different techniques: (a) the creation of a snapshot file containing all code units and already initialized objects; or (b) build a static description of the application containing the code for all classes and methods that should be part of it.


\section{Experiments and Results}\label{sec:results}

\subsection{Experiment's Methodology}
We evaluate Tornado by conducting five experiments that tailor different Pharo applications, with increasing requirements. We chose our experiments under the objectives of (a) understand how minimal are the applications we can tailor, (b) explore how successfully we address the challenges we stated in Section \ref{sec:challenges} and (c) exercise those cases that push to the limit the interaction between the language and the VM. Each of our experiments is detailed in the following sections.

Our experiment methodology consists in the following steps:

\begin{enumerate}
\item \textbf{Setting up a seed for the application.} Most of our experiments use what we already called an \emph{empty seed}. This seed is, however, not completely empty. The empty seed we used contains some minimal infrastructural objects that are needed for language-VM interaction, and is therefore 10KB large. Our last experiment, the largest one, uses both this empty seed and an additional seed containing the base libraries. 
\item \textbf{Preparing the application entry points.} This step consists in the installation of the one or more processes that will run our application.
\item \textbf{Run the application.} The application is run, its threads executed. In particular in our last experiment~(an interactive web application), we interact with our application through a web browser. 
\item \textbf{Stop and extract the application.} Once the tailoring process finishes, we stop Tornado and extract the resulting application by making a snapshot of it in a Pharo image file. We test the generated snapshots to verify they work properly, either by using the application or debugging them when they involve no I/O. We evaluate the behavior of the tailored application under the assumption that only the features we used during the tailoring should work.
\item \textbf{Perform measurements.} Finally, to present our results we measure the size of the generated snapshots files and compare them against two different Pharo distributions prepared for production, as we present in Section \ref{sec:results_discussion}.
\end{enumerate}

\subsection{Experiment I: Adding Two Numbers}

The smallest~(in terms of size) interesting program to tailor is adding two numbers, without the involvement of any I/O \ie an application just executing the \ct{"2 + 3"} statement as entry point. Tailoring this program is challenging because it stresses the infrastructure by installing only the minimal elements an application needs to run. It makes evident how small a tailored application can be. Additionally, it is interesting since it makes use of the following features of the Pharo language and infrastructure:

\begin{description}
\item[Immediate objects.] Immediate objects are objects encoded in the object reference instead of being allocated in the heap. Immediate objects do not contain a reference to their class in the object header, as there is no object header. Instead, the object reference where the object is encoded contains a bit tag that the VM uses to identify the immediate object. This means that the Pharo VM must acknowledge the immediate object classes~(or their proxies) in order to send messages to these immediate objects. In this experiment we use immediate small integers, instances of \ct{SmallInteger}.
\item[Special selectors.] The method selector \ct{+} is a special selector for the Pharo VM. Special selectors are optimized as they are broadly used messages, for example for arithmetics. First, they are implemented as special bytecodes to avoid method lookup. If the special bytecode cannot be executed because some VM assertions are not valid~(\eg class and object format assumptions), the VM performs the default method lookup. In this experiment the VM should take care of small integer arithmetic \ie it should fulfill all VM assumptions and not perform a method lookup; Tornado should install no extra methods nor classes.
\end{description}

\subsection{Experiment II: Factorial of a little number}

The following experiment in incremental complexity is the factorial of a small number, again without the involvement of any I/O \ie an application executing the \ct{"10 factorial"} statement as its entry point. Factorial uses arithmetic as the latter experiment~(sums and multiplications), while it also adds the following interesting cases:

\begin{description}
\item[Method lookup.] The \ct{factorial} message is sent to a small integer but not optimized as it is not a special selector. Thus, the VM look ups the corresponding method up in its class hierarchy. The method \ct{factorial} is defined in its superclass~(\ct{Integer}).
\item[Recursion.] The factorial implementation in Pharo base libraries is recursive. Additionally, this recursion activates the \ct{factorial} method many times, creating many activation records in the VM. Activation records are reified lazily whenever it is accessed reflectively, or the stack depth is deeper than the maximum supported.
\end{description}

\subsection{Experiment III: Factorial of a large number}

Following, we experimented with an application whose entry point was the \ct{"100 factorial"} statement. This application does not either make use of any I/O. The factorial of a large integer creates eventually integers that exceed 32 bits, and thus, do not fit as immediate small integers. This experiment adds the following interesting cases:
\begin{description}
\item[Large integers.] Large integers in Pharo are represented, in contrast to immediate small integers, as objects allocated in the heap with their own object header and arbitrary length. Large integers are created automatically by the VM when the result of some integer calculation produces a number that overflows 31 bits. That is, the LargeInteger class (or its proxy) should be available to the VM in order to instantiate the correct object.  Additionally, large integers implement their arithmetic methods by calling primitives from external plugins~(the large integers plugin).
\item[Polymorphism.] The introduction of large integers introduces also \emph{polymorphism} between them and the immediate small integers. They share the same class hierarchy~(\ct{Integer} is the superclass of \ct{SmallInteger} and \ct{LargePositiveInteger}), being the method \ct{factorial} implemented in the superclass and having each of the subclasses their own implementation of the arithmetic methods for adding and multiplying.
\end{description}

\subsection{Experiment IV: Reflective invocations} \label{sec:results_helloworld}

The fourth experiment introduces reflective invocations. Figure \ref{fig:reflective_invocations} introduces the code we used for this experiment. The \ct{User} class in our example has two fields (\ct{name} and \ct{age}), and four methods. Two of these methods (\ct{age} and \ct{name}) return directly the field with the same name, the method \ct{hasWritePermissions} is annotated with the \ct{property} annotation (a pragma in Pharo's terminology) and the method \ct{isMinor} is a normal method. We introduce also \ct{PropertyExtractor} class with the responsibility of returning the name of those methods that are properties of an object \ie all methods that only return a field, and all those methods annotated with the \ct{property} annotation. The statement we introduced as the entry point for this experiment is \ct{"PropertyExtractor new extractPropertiesFrom: User new"}.

\begin{figure}[ht]
\small
\begin{code}
Object subclass: #User
	instanceVariableNames: 'name age'.

User>>age (
	^ age )

User>>hasWritePermissions (
	<property>
	^ true )

User>>name (
	^ name )

PropertyExtractor>>extractPropertiesFrom: anObject (
	^ anObject class methods
		select: [ :each | each isReturnField
			or: [ each pragmas anySatisfy: [ :pragma | pragma keyword = #property ] ] ]
		thenCollect: [ :each | each selector ] )

\end{code}
\caption{\small \textbf{Code of the reflective invocations experiment.} The \ct{PropertyExtractor} class does the reflective invocations, the \ct{User} is the class we will be reflecting on.\label{fig:reflective_invocations}}
\end{figure}

This experiment evaluates how do we handle the reflective abilities of Tornado's RFG. The \ct{PropertyExtractor} queries the methods from the \ct{User} class, that are included as part of the tailored application~(since they receive the messages \ct{isReturnField} and \ct{pragmas}). These reflective invocations include: (a) access to an object's class, (b) access class methods and (c) query those methods to know if they correspond to the criteria of the \ct{PropertyExtractor}.

\subsection{Experiment V: Adding I/O} \label{sec:results_helloworld}

A fifth experiment introduces I/O to each of the previous experiments, adding a statement printing to the standard output the obtained results. Figure~\ref{fig:hello_world_entry_point} shows the code from our entry point in the case of summing up two numbers. The entry points for the other experiments have the same structure, differing only on the expression that is printed~(the \ct{"1+2"} expression in this case). Notice that the \ct{FileStream} class needs to be initialized before the proper printing into the stdout stream because the code needed for class initialization is not installed by default in the empty seed.

\begin{figure}[ht]
\small
\begin{code}
FileStream startUp: true.
FileStream stdout 
	nextPutAll: (1 + 2) asString;
	crlf.
\end{code}
\caption{\small \textbf{Entry point of the experiment that sums two numbers and prints the result in the standard output stream.}\label{fig:hello_world_entry_point}}
\end{figure}

In this experiment, besides testing the proper usage of I/O streams such as the standard output stream, we evaluate the ability of Tornado to handle \textbf{platform identification}. The \ct{stdout} stream initialization for Pharo is done by the File package written in Pharo, and it depends on which is the current operating system. This experiment shows that Tornado prepares tailored versions of applications to run on a single operating system or platform.

\subsection{Experiment VI: Seaside Web Application}



Our last experiment consists is tailoring a web application using Seaside application framework~\cite{Duca07a}. Seaside is a web application framework featuring continuations thanks to stack reification. We configured it with its default values, without making any customizations. The web application under tailoring has a single webpage that allows one to send requests to the web server to increase or decrease a counter. This experience shows that Tornado works in presence of \textbf{threading}. The Seaside application framework makes use of Pharo processes. One process listens incoming connections and opens new processes to handle requests. Seaside uses semaphores to synchronize processes and wait for incoming data from sockets.

For this case, we proceeded to do two different experiences, with two different seeds. We first used the empty seed~(\emph{Seaside Web Application A}), as in the previous experiments, and then used a seed containing all Pharo base libraries~(\emph{Seaside Web Application B}). For reasons of space, the details of how the entry points are initialized for these both seeds can be found in our technical report~\cite{Poli14a}.

\subsection{Results} \label{sec:results_discussion}

We gathered our experiments' results into Table \ref{tb:results}. This table shows:
\begin{description}
\item[Experiment.] The name of the experiment under evaluation, followed by our measurements.
\item[Reference Application.] The size of the reference application containing all of its code units, in KB. We present in here two different sizes: the size of an experimental shrunk version of the Pharo distribution called \emph{PharoKernel}, which was developed independently from us by Pavel Krivanek; and between parenthesis the size of the official Pharo distribution prepared for production: Pharo allows one to prepare a snapshot for production. This option cleans some caches and removes some well known objects and classes from the system, thus, freeing space.
\item[Seed.] The size in KB of the chosen seed for the experiment.
\item[Nurtured Application.] The final size of the nurtured application once Tornado finishes its process and the application is extracted.
\item[Saved.] The percentage of space saved using the smallest reference application size. We chose the smallest reference application to avoid biased results in our favor. We calculated this percentage using the following equation:

\begin{equation*}
saved = 100 - \frac{100*(nurtured - seed)}{reference - seed}
\end{equation*}

\begin{table}[ht]
 \small
 	\centering
 	\begin{tabular}{|l|cccc>{\columncolor[gray]{0.8}}c|}
		\hline
			Experiment
 			& \textbf{Reference App}
			& \textbf{Seed}
			& \textbf{Nurtured}
			& \textbf{Installed}
			& \textbf{Saved(\%)}\\
			
 			& \textbf{\emph{Shrunk(Prod.)}}
			& \textbf{Size}
			& \textbf{App}
			& \textbf{Code}
			& \textbf{}\\
		\hline
		Sum Two
 			&  3799 (12873) & 10 & 11 & 1 & \textbf{99.97\%}\\
		Numbers (I)
 			& &&&&\\
		\hline
		Fact 10 (II)
 			& 3799 (12873) & 10 & 15 & 5 & \textbf{99.87\%}\\
		\hline
		Fact 100 (III)
 			& 3799 (12873) & 10 & 18 & 8 & \textbf{99.79\%}\\
		\hline
		Reflective
 			& 3799 (12873) & 10 & 32 & 22 & \textbf{99.42\%}\\
		App (IV)&&&&&\\
		\hline
		(I) + I/O
 			& 3799 (12873) & 10 & 81 & 71 & \textbf{98.13\%}\\
		\hline
		(II) + I/O
 			& 3799 (12873) & 10 & 82 & 72 & \textbf{98.10\%}\\
		\hline
		(III) + I/O
 			& 3799 (12873) & 10 & 89 & 79 & \textbf{97.92\%}\\
		\hline
		(IV) + I/O
 			& 3799 (12873) & 10 & 95 & 85 & \textbf{97.76\%}\\
		\hline
		Seaside Web
 			& 20254 (17250) & 10 & 573 & 563 & \textbf{96.73\%}\\
		App A&&&&&\\
		\hline
		Seaside Web
 			& 20254 (17250) & 12872 & 13090 & 218 & \textbf{95.02\%}\\
		App B&&&&&\\
		\hline
 	\end{tabular}
 	\caption{\small\textbf{Results of the tailored experiments.} Sizes are displayed in KB. The percentage of saved space does not take into account the seed, as it is not subject of Tornado and it is shared by both the reference and nurtured application.}
 	\label{tb:results}
 \end{table}


Note that we subtract the size of the seed from both the nurtured and reference applications sizes, since the seed is shared between both. That way, we compare only those parts of the application that were subject of the RFG algorithm.

\end{description}

 \begin{table}[ht]
 \small
 	\centering
 	\begin{tabular}{|lc|}
			\hline
			\textbf{Component}
 			& \textbf{Size~(KB)}\\
		\hline
		Pharo Base Libraries & 12872\\\hline
		Seaside Application Framework Libraries & 4378\\\hline
		Seaside Web App & 47\\\hline
		Reflective Invocations App & 104\\\hline
 	\end{tabular}
 	\caption{\small\textbf{Component sizes in our experiments. Size presented in KB.} \label{tb:tailored_components}}
 	\label{tb:basic_sizes}
 \end{table}

Table \ref{tb:tailored_components} shows the size in KB of the code units from we used in our experiments. This table details the size of the Pharo base libraries, third party libraries such as Seaside and our particular experiments, which aid in the understanding of the results. We obtained this sizes by measuring the size of the code units once loaded in memory.

\paragraph{Discussion of Results.}
Our experiments show that Tornado aggressively reduces the size of code units required for an application. Our examples save from 95\% to 99\% of space, compared with their reference application~(which contains all base libraries and third party libraries in case of Seaside). Our first three experiments (the sum of two numbers, and the factorial of 10 and 100) show that Tornado succeeds to create minimal deployment versions of our applications, having into account that our seed forces a minimal of 10KB in each of them. The reflective application is indeed also minimal, but bigger than the other three, as Tornado installs inside the nurtured application (a) all the code that is accessed by reflection and (b) code from the collections package to iterate the methods of a class.

We detect a notorious grow in size when adding I/O to our experiments, which varies from 63KB to 71KB extra. According to the list of installed code units, we identity a problem in the design of the I/O streams library from Pharo: a set of character tables meant for character encoding and conversion are initialized, even if not all of them are later on used by the application. This problem shows that this part of Pharo base libraries should be rethought.

The Seaside experiments show that Tornado can be used in a complex setting such as a web application that runs a web server, while still achieving good results. It is interesting to note, from the comparison of both experiments, that more of half of the size of the final nurtured application in \emph{Seaside Web Application A} seems to be in the base libraries, as the amount of installed code is reduced when introducing the base libraries seed.

\paragraph{Comparison with a Dedicated Platform.}

To have a broader view of our results, we compare them to MicroSqueak~\cite{Malo11a}. MicroSqueak is a dedicated platform that runs on the Pharo platform \ie a specialized platform containing an alternative implementation of base libraries, as Java Micro Edition~(J2ME)~\cite{JavaME} is for Java. MicroSqueak was designed with the explicit goal to be the smallest practical Squeak kernel. It contains a total of 49 classes with a reduced set of methods. It offers a minimal core of the language, a basic collection library and basic file IO support. MicroSqueak presents a minimal memory footprint of 80KB, when we build an application that performs no computation.

On one side, Tornado ensures smaller memory footprints when working on small applications. On the other side, MicroSqueak presents crucial differences with Pharo base libraries: it does not provide the same libraries~(\eg it does not contain socket support) and it does not ensure the same API of those libraries that it contains. Thus, applications such as the one in our Seaside experiment cannot run on top of MicroSqueak without a dedicated version of the Seaside framework.

\section{Comparison of Tornado with Related Work} \label{sec:related_work}



\subsection{Evaluation of Tornado}

We start this section evaluating Tornado according to the criteria we defined in Section \ref{sec:criteria}, so we can in the following sections discuss it and compare it with other approaches. Table~\ref{tb:comparison} shows an overview of the criteria defined in Section~\ref{sec:criteria} and their possible values to evaluate tailoring solutions.
In this section we focus on the evaluation of Tornado that is summarized in the latest column of Table~\ref{tb:comparison}.

\belowrulesep=0pt
\aboverulesep=0pt

\begin{table}[ht]
 \small
 	\centering
 	\begin{tabular}{|c|cccc>{\columncolor[gray]{0.8}}c|}
	
\hline
 			& \textbf{Dedicated}
 			& \textbf{Static}
			& \textbf{Hybrid}
 			& \textbf{Dynamic}
 			& \textbf{Tornado} \\
 			& \textbf{platforms}
 			& \textbf{Analysis}
			& \textbf{Analysis}
 			& \textbf{Analysis}
 			& \\
  \cmidrule(r){2-6}
% \midrule

		Base Libraries
 			& + & + & + & + & +\\
		\hline
		Third-Party
		& & & & & \\Libraries
 			& - & + & + & + & +\\
		\hline
		Legacy Code
 			& - & + & + & + & + \\
		\hline
		Reflection Support
 			& + & - & - & + & + \\
		\hline
		Dedicated Deploy
			& & & & & \\
		Infrastructure
 			& - & + & - & - & + \\
		\hline
		Flexibility
 			& - & - & - & - & +  \\
		\hline
		Ensures
		& & & & & \\
		Completeness
 			& - & - & - & - & -  \\
 	 \hline
 	\end{tabular}
	\includegraphics[width=.9\linewidth]{criteria_overview}
 	\caption{Evaluation criteria applied to related work on deployment code unit tailoring techniques}
 	\label{tb:comparison}
 \end{table}

Tornado's model and implementation show themselves as a complete solution in the area of application tailoring. It tailors code units written by the application's developer as well as those from the base language and third-party libraries. There is no special code for managing such cases since Tornado's infrastructure allows the inspection of loaded classes, regardless their origin. This approach, based on runtime execution, offers two main advantages: (a) it does not require modifications in the nurtured application's code allowing its usage on legacy code and libraries in a transparent way, and (b) it supports reflection naturally since the code exercised during the tailoring is the same that will be executed once deployed.

Tornado requires a dedicated infrastructure only during the tailoring: tools to monitor and manipulate the tailoring application. However, once the tailoring is finished and the application reaches a stable point, Tornado extracts and prepares the application to run in the deployment-ready unmodified infrastructure.

Finally, Tornado is a flexible solution in the sense that it allows one to configure the level of tailoring by means of a seed. The seed contains a pre-selection of code units available in the tailoring application before the tailoring starts. In such a way, we can use the seed to specify whether, for example, the base or third-party libraries should be tailored or not.
 
The reduction of the deployment footprint of object-oriented applications has been subject of interest both in industry and research since many years. In such regard, we identified four different families of solutions for dead code elimination: dedicated platforms~(cf. Section \ref{section:static_selection_rw}), static analyses~(cf. Section \ref{section:static_rw}), dynamic analyses~(cf. Section \ref{section:dynamic_rw}) and hybrid analyses~(cf. Section \ref{section:hybrid_rw}). Table~\ref{tb:comparison} presents a comparison of these techniques, given the criteria defined in section~\ref{sec:criteria}.

\subsection{Dedicated platforms}%Pre-conceived specialized application-independent platforms}
\label{section:static_selection_rw}

Dedicated platforms are platforms containing frameworks and/or libraries prepared to run under specific circumstances. For example, Java Micro Edition~(J2ME)~\cite{JavaME} as the dedicated version of the Java platform, or Cocoa Touch as the one of Cocoa. These specialized platforms are reduced platforms to run applications inside mobile and constrained devices. These platforms provide a reduced and fixed set of base libraries defined a priori and in a not customizable way. Applications have to be written especially for them, and thus legacy code and third-party libraries not written especially for it are not compatible. Reflection is available since the statically tailored base libraries are built in a not automatic fashion, and the application code is not tailored.

\subsection{Static Analysis-Based Techniques}\label{section:static_rw}

Static analysis approaches for dead code elimination make use of the static information of a program to select the minimal subset of used elements. The bibliography describes four different algorithms to achieve this goal: unique name, class hierarchy analysis~(CHA), rapid type analysis~(RTA) and reachable members analysis~(RMA) \cite{Baco96a, Titz06a}. These techniques share a common approach, selecting an entry point method of an application and following from it the execution flow using the available static information \ie type annotations, and class and method names, building a call-graph~\cite{Grov97a}.

These techniques have been studied and applied in many environments and languages. Rayside et al.~\cite{Rays02a}, Jax~\cite{Tip03a} and the ExoVM System~\cite{Titz06a} propose application extraction tools using these techniques for Java applications. Sallenave et al.~\cite{Sall10a} apply RTA to produce smaller .NET assemblies for embedded systems. Bournoutian et al.~\cite{Bour14a} use CHA to optimize on-device Objective-C applications. Ole Agesen~\cite{Ages96a} presents in his thesis a static technique applied to Self, a dynamically-typed language. Ole Agesen uses type inference to obtain type information and use it to select which objects to extract.

In summary, these approaches are based on the static types found either in the source code or byte code. Thus, they are not applicable \emph{efficiently} in dynamic languages with no static type declarations. These solutions are valuable as they allow one to tailor base and third-party libraries, and legacy code. Their tailoring approach generates new deployment units that can run on the standard runtime infrastructure. The main drawback of this approach appears in the presence of reflection and configuration files, which will only work with a subset of reflective invocations through complementary analyses on the strings found in the source code. Also, existing solutions in this family lack the flexibility to declare and identify levels of tailoring, making it an "all or nothing".

\subsection{Dynamic Analysis-Based Techniques}\label{section:dynamic_rw}

Dynamic analysis techniques use exclusively runtime information~(\ie execution flow, alive objects, execution statistics) to perform dead code elimination. Amongst these, we identify two different approaches: \emph{load on demand} and \emph{code collection}. Load on demand approaches detect during runtime whenever a class or method needs to be installed and request it to a server application. Code collection approaches deploy the full application  and garbage collect unused code based on usage statistics. Related work in this family share a common characteristic: these techniques are used inside ubiquitous systems \ie systems meant to be always connected. Ubiquitous systems, as they are always connected, have a possibility to fallback and recover in the case of incompleteness. However, to focus here on the dead code elimination techniques, we will discuss the incompleteness recovery techniques in section \ref{sec:discussion}.

\begin{description}
\item[JUCE \cite{Popa04a,Teod01a}.] It is a platform for ubiquitous devices supporting code load on demand and code collection. Its approach for building up an application is similar to Tornado. First, it initializes a minimal running application and code is loaded, with a method granularity, from a server located in a different machine. Unused code is collected following usage statistics, and loaded back again on demand if needed.

\item[OLIE~\cite{Gu03a}.] It is an engine that intelligently partitions and offloads objects during runtime to minimize memory consumption. It is part of the adaptive infrastructure for distributed loading (AIDE). In OLIE, offloaded objects are indeed migrated to nearby remote devices. Migrated objects can be accessed later through proxies that perform remote invocations on them.

\item[SlimVM~\cite{Kers09a, Wagn11a}.] It is an ubiquitous system where all code resides on a remote server and is loaded only on demand on small devices. Some static analysis is performed only on the server to reduce the size of the transported code, by identifying most likely needed code. SlimVM changes the class format. However, on the client side, every code load is done dynamically.

\end{description}

All solutions inside this category share one main property: they require to run the application inside a dedicated infrastructure to apply their techniques \eg dedicated VMs implementing remote lazy loading, code collection or new bytecode sets. The main challenge of these solutions resides on applying these techniques while minimizing their impact on performance during the runtime. Additionally, these solutions require their applications to run exclusively inside their infrastructure. Tornado works in the same way as these solutions: it uses a dedicated infrastructure to run the desired application and select the used elements.  However, Tornado provides also with the ability to extract this application and run in \emph{offline} mode, using the non-modified infrastructure.

Regarding dynamic features such as reflection, these solutions are the ones that can, potentially, handle it in the best way since they have in runtime all the information needed to resolve it. JUCE and OLIE, as Tornado, handle naturally reflection as they do not change the runtime representation (which programs make assumptions of, when they use metaprogramming). SlimVM on the other side, had to change the reflection support because they changed the object and class representation on their VM.

Regarding its applicability, SlimVM needs to recompile the whole application into its own format, while OLIE and JUCE, as Tornado, can tailor base and third party libraries without any modifications on it. Thus, the latter two can be applied to legacy code also for free. None of these solutions provide with the ability to select the level of tailoring always working on the full application. In contrast, Tornado uses seeds to force a minimal subset of elements to be part of the application.


\subsection{Hybrid Analysis-Based Techniques}\label{section:hybrid_rw}

Hybrid analysis techniques mix static and dynamic~(\ie runtime) information to provide better results. The common approach of these is to start an application, such as Tornado does, and pause it after some minimal runtime information is available \ie call stacks are created, some classes are loaded and initialized, and some objects are instantiated. Then, it uses the built stack of alive objects to perform a static analysis, as described in Section \ref{section:static_rw}, with concrete type information.

Java in The Small (JITS)~\cite{Cour10a} uses a hybrid approach to select the used parts of a program, and then loads them inside a binary image. A dedicated VM loads the binary image at startup. JITS's approach tailors base and third-party libraries as well as application specific code. It does not require modifications on the existent application to tailor it, so a legacy application could theoretically be tailored with this approach. JITS does not offer the possibility to configure the tailoring level, since it was designed to be used only in embedded devices where no more than one application would be running. Regarding reflection, JITS presents the same drawbacks as the other static call graph analysis approaches since not all the runtime information about the reflective invocations can be deduced.

\section{Discussions on the run-fail-grow approach} \label{sec:discussion}

\subsection{Ensuring Completeness} \label{section:safety}

Dead code elimination techniques do never ensure completeness by themselves. Static approaches cannot efficiently predict the need of those elements used by reflection, or configured in external files/resources. Dynamic approaches depend on the code coverage of the application during runtime, \ie if the parts of the application that are not used will be not available afterwards. Hybrid approaches share both weaknesses. Orthogonal to the dead code elimination techniques, two complementary mechanisms are used by existing solutions to guarantee \emph{completeness} and avoid runtime errors due to missing code.

\begin{description}
\item[Lazy Loading.] JUCE~\cite{Popa04a,Teod01a} and SlimVM~\cite{Kers09a, Wagn11a}, as well as Tornado, load missing code from remote servers on demand, Marea\cite{Mart12a} implements application-level virtual memory with lazy loading of unloaded unused objects. These different solutions differ on their lazy loading approaches by the granularity they use. JUCE loads code with a method granularity to control memory consumption. SlimVM uses as its main loading granularity, a \emph{basic block} granularity, but they can work at the class and method level also. Marea uses an object-cluster granularity. It loads object graphs containing not only classes but also individual objects, which were unloaded to reduce the application's memory footprint.
\item[Remote Invocations.] OLIE~\cite{Gu03a} uses remote invocations to invoke methods from those objects that where offloaded and migrated to other devices. This approach may introduce several latency problems due to network communications. OLIE tries to minimize it by offloading those elements that degrade less the performance of the system. For that, it takes at runtime object and bandwidth usage statistics.
\end{description}

\subsection{Maximizing the Effectiveness of Run-fail-grow}\label{sec:maximize_effectiveness}

Dynamic techniques, in particular Tornado, depend on the coverage of the application to ensure the code is loaded and available for execution. Application coverage must ensure that every code unit that is interesting to be deployed is covered, including special and boundary cases as well as the straightforward cases. We can enforce the coverage and installation of code with several techniques. 

\begin{description}
\item[Manual Testing.] Manual testing provides a simple but inefficient way to cover an application's code. Its main benefit is that the code units selection is based on user interactions. Its main drawback is the possibility of human omission during the testing, which impacts directly the detection of used code. 
\item[Automated Testing.] Automated testing counters the human omissions by adding repeatability in the generation of the deployment unit. Different levels of testing have different impacts on the coverage and will produce different results. For example, using unit test to cover the application and libraries' code may exercise more code than the one that is actually needed, since they use to test smaller units and tend to cover the whole code. Acceptance tests may not exercise enough parts of the application. UI tests should be considered as part of the solution for maximizing coverage.
\end{description}

%\subsection{Dynamic Code Coverage} \gp{It deserves to be compared with the trace-copy approach I think}

\subsection{Application Designs that get along with Tornado} As show in Section 5.2, the design of the tailored application directly impacts on the results obtained by Tornado. A series of issues appear regarding global state~(\eg class variables and global variables). A first issue is related to the initialization of such a global state~\cite{Unga95a}. Since Tornado follows the application's execution flow, eager initializations force Tornado to install objects and methods that may not be used later by the application. In contrast, lazy initializations will only be triggered on usage. Thus, better results could be obtained if a lazy initialization strategy is adopted for the global state.

A second issue appears with residual side-effects. Our tailoring technique builds the deployment application by running it. Thus, those executed global side-effects may reside in the tailored application. For example, a web application framework may hold a cache of HTTP sessions in a class variable. When the tailoring process finishes, the application will keep this cache if we do not handle the case. Solving this problem in Tornado may require either minimizing global state in an application, or either installing a new entry point to reinitialize such global state when the tailoring is finished \eg clean caches and session dependent state such as file and socket descriptors.

%\paragraph{Handling Reflection.}
\subsection{Modern Language Features}

Tornado handles modern programming language features such as reflection, open classes and class extensions~\cite{Berg03a}~(\ie a package can define methods to classes from other packages) and traits~\cite{Scha03a}, out of the box. Reflective invocations contain all the information they need to be tailored correctly as Tornado works at the runtime of the application. Tornado installs methods from other packages or behavior units such as traits seamlessly because during runtime it knows the exact concrete type of each object involved in the execution. Thus, no extra static or string analysis is needed. This is possible thanks to Ghost proxies~\cite{Mart11a}, which can capture all message sends and specific method invocations.

%\sd{below, fuzzy paragraph}
%In addition, as reflection works with a closed universe assumption \eg asking for all methods of a class will result into all methods that \emph{exist} in the class and will not retrieve those that are not installed or loaded. For those cases in which an application uses such reflective invocations, tornado allows the installation of missing object traps for each method. Then, methods are found by reflective invocations and installed on demand by Tornado if needed.

%\paragraph{Open-Classes and Class Extensions.}
%
%Pharo supports, as other languages such as Ruby, the concept of open classes and class extensions~\cite{Berg03a} \ie a package can define methods to classes from other packages. Tornado needs no special support to manage class extensions. The \emph{missing method} and \emph{override method} traps detect these cases and Tornado installs extension methods on demand as any other method.

%\subsection{Shrinking VMMMMM}

%\subsection{Shrinking program meta-data}
%
%The Pharo programming language has first class representations of classes and methods. This property and Tornado's lazy installation approach provide out of the box the elimination of unused program meta-data \eg unused class variables, literals and symbols are never installed.

%\subsection{The snapshot approach} \gp{maybe we should remove this discussion since it is orthogonal and not strong}Tornado prepares an application for deployment by extracting it in a snapshot file. A snapshot file allows one to deploy already initialized objects, avoiding the installation of the code needed to create them. Additionally, having a snapshot file speeds up the application's startup. Additionally, our tailoring model is not limited to snapshot/image based systems. Tornado can inspect the complete state of the nurtured application thanks to the object runtime manipulation interface~(Oz object spaces in our case). Thus, it can extract all the information it needs from it once the tailoring is finished: which are the classes installed, their methods, and their state. With such information, a static description of the system could be built. In our technical report~\cite{Poli14a} we present a list of the code units installed in the applications from each of our experiments, obtained by inspecting the nurtured application.

\subsection{Easily Managing Base libraries} Most applications do not use the whole base-library collection distributed along with a language. These libraries, representing big code code bases, are then potential candidates for removal. However, in most of the modern object-oriented languages, base language libraries are loaded and initialized by the language's Virtual Machine~(VM) as some times an order has to be ensured or those same code units are used internally by the VM. Thus, the application developer cannot easily manage and customize which of them he wants, since it often requires VM modifications.

Pharo provides the developer with access to the base libraries in the language. Thanks to this ability, Tornado can manage Pharo's base libraries as it manages application code. There is, however, an exception: the code units that belong to the interface between the language kernel~(\ie the minimal language elements that should be available to run) and the VM must be installed and initialized in a particular order and be always ensured. Because of this, we guarantee that the minimal seed, the \emph{empty seed}, contains at least all these needed code units.

%\subsection{Implementing RFG in other technologies}
%
%Tornado is based on an architecture that allows complex manipulations of both the nurtured and the reference applications. In order to implWe identify the following minimal components as part of Tornado's architecture~(cf. Figure~\ref{fig:tornado_code units}):
%
%\begin{description}
%\item[Object runtime manipulation interface.] An object runtime manipulation interface allows one to act over an object runtime system by controlling its runtime execution~(\eg starting, pausing and restarting it, and installing new threads/processes) and perform introspection and intercession~(\eg installing classes and methods, retrieve the loaded classes) on it. Oz object spaces A well known example of such an interface is the JVM TI~(JVM tool interface)~\cite{JVMTI}.
%We use this module to pause the reference application, suspend the execution of the nurtured application when a failure is detected, get the methods to install from the reference application, and install the necessary methods and classes into the nurtured application.
%
%\item[Advanced intercession module.] An advanced intercession module allows advanced reflective capabilities such as modifying an object's behavior during runtime. Tornado uses this module to capture message sends and so to be notified when it finds missing code units. JRebel~\cite{Jreb12a}, Reflectivity~\cite{Denk08a} or Bifrost~\cite{Res12} are examples of such intercession libraries.



% ===========================================================================
\section{Conclusion and Summary}

In this chapter we presented a run-fail-grow~(RFG) approach for application tailoring. RFG tailors an application by starting it and initializing it with a seed that contains the minimal set of code units we want to ensure. Then, we install and execute the application's entry points. As the application executes, missing code units are found and installed on demand, ensuring that only the needed code units are introduced. By following the runtime execution, it supports dynamic features such as reflection and meta-programming.

We implemented RFG in a tool called Tornado based in \Vtt. Tornado succeeds to produce applications with minimal footprint for deployment. Our results show that we manage different extreme and challenging cases with flexibility.

% =============================================================================
\input{chapter-footer.tex}
%\input{chapter-header.tex}
% ===========================================================================
\chapter{Evolution on Runtime: Surgery}
\chaplabel{benzo}
\minitoc
% ===========================================================================
\introduction
% ===========================================================================


% ===========================================================================
\section{Background}

% ===========================================================================
\section{Support for Runtime Modifications}

%============================================================================
\section{Image Surgery}

% ===========================================================================
\section{An Image Surgeon}

% ===========================================================================
\section{Conclusion and Summary}

% =============================================================================
\input{chapter-footer.tex}

\part{Conclusion}
\input{chapter-header.tex}
% =============================================================================
\chapter{Conclusion}
\chaplabel{conclusion}
\minitoc
% =============================================================================
\introduction

This thesis focuses on the ease of manipulation of application runtimes for Object-Oriented languages, particularly reflective ones. Having access and the power to change such application runtimes is indeed an issue for both language and application developers.

Application runtimes are at the center of the activity of language developers. Implementing a language requires designing its execution model and its runtime representation besides its syntax and exposed concepts. Moreover, extending an existing language to add new features requires its runtime to be easily extensible. Also, reflective languages add the challenge of circularities. To perform these tasks, language developers need tools that allow them to modify a language runtime, extend it and address its circularities.

Application developers are also exposed to the modification and specialization of application runtimes. Particularly, the spreading of new constrained devices such as embedded systems or sensor networks, require application developers to reduce the memory occupied by their applications. To address these concerns, application developers should have access and control on the application runtimes they develop. They should be able to shrink them.

To address these issues we introduced \Vtt: an infrastructure for application runtime virtualization. We show that runtime virtualization is a general purpose tool that can be used for different purposes. In particular we use it to explore the two challenges presented above. Our first scenario is language bootstrapping. The second is the tailoring of application runtimes.

% =============================================================================
\section{Contributions}
% =============================================================================

\subsection{\Vtt}
The main contribution of this thesis is \Vtt, a \emph{language virtualization infrastructure}. In \Vtt, a first-class application runtime, namely an object space, allows the manipulation, control and monitoring of a virtualized runtime through a clear API. A first-class language hypervisor implements such runtime manipulations with the expression power and abstractions of the high-level language we are manipulating.

\Vtt allows application runtime manipulation and control through several techniques. Mirrors allow the direct manipulation of objects inside an object space while enforcing the invariants of the \VM execution model. The hypervisor can execute an object space in cycles and perform custom operations between each of those cycles. This allows the hypervisor to control the execution inside an object space. Finally, to execute arbitrary code inside an object space \Vtt provides with process injection and virtual execution. The first one injects code inside an object space so it can run normally. The latter is based on code interpretation and while slower provides finer control on what code is executed. 

\subsection{Bootstrapping}
Bootstrapping is commonly known by its usage on compiler building, where a compiler can compile itself.
It can be generalized to the introduction of any software system to its own building process.
A bootstrap process has the property of describing the system under construction in terms of the system itself. This allows us to easily change and extend this system, taking advantage of its abstractions and tools.

We applied the idea of a bootstrap in object-oriented languages by providing a circular language definition \ie a definition of the language runtime defined in itself.
Our virtualization infrastructure eases the execution of such language definition. First, an object space provides a clear VM-language interface to helps with the manipulation of the language runtime under creation. Second, a bootstrapping virtual interpreter allows the execution of the language definition when the language cannot still be executed.

We used this infrastructure to bootstrap three different object-oriented languages with different programming models: 
\begin{description}
\item[Candle.] A minimal Smalltalk with implicit metaclasses.
\item[Pharo.] The core of the Pharo language which is defined by traits, first-class layouts and first-class variables.
\item[MetaTalk.] A Smalltalk based language that decomposes reflection into mirrors. The language meta information is hosted inside the meta level of the language, which can be dynamically removed.
\end{description}

\subsection{RFG Tailoring} 

Application tailoring is a technique reduces the memory footprint of an application by removing code bloat. Code bloat is an issue in constrained scenarios, when the size of an application limits its deployment. For example, devices with limited memory or web-applications in slow networks. Application tailoring reduces unused code units~(\eg classes, methods) to produce a specialized version of an application for its deployment in such scenarios.

Using \Vtt, we developed run-fail-grow~(RFG), an approach for dynamic application tailoring. RFG tailors an application by starting it inside an initially empty object space. We can additionally ensure a set of code units inside our application by introducing them inside the seed. Then, a set of application entry points describing where the application starts are installed inside the object space and executed. As the application executes, the application will \emph{fail} due to missing code units. RFG reacts to missing code failures by installing the required code units. Then, code units are only installed on demand inside the virtualized runtime. Using this technique, we ensure that only the needed code units are introduced.

By performing during execution, RFG tailors programs written in dynamically typed languages and using features such as reflection and polymorphism. It works transparently, being able to tailor legacy and third-party code without modifying it. Tornado, our RFG implementation, succeeds to produce applications with minimal footprint for deployment. Our results show that we can aggressively tailor challenging cases. In our experiments we observe memory reductions from 95.02\% to 99.99\% when comparing with the production ready Pharo distribution.

% =============================================================================
%\section{Published Papers}
%% =============================================================================
%
%\subsection{Journals}
%
%Guillermo Polito, Stéphane Ducasse, Luc Fabresse, Noury Bouraqadi, and Benjamin Ryseghem. Bootstrapping Reflective Systems: The Case of Pharo. In Science of Computer Programming, 2013. Impact Factor: 0,548\newline
%
%
%
%tornado (under submission spe)
%
%\subsection{Workshops}
%
%Clara Allende, Guillermo Polito. Virtual Smalltalk Images: Model and Applications. In WISIT - Workshop de Ingeniería en Sistemas y Tecnologías de la Información, 2014.
%
%Guillermo Polito, Stéphane Ducasse, Luc Fabresse, and Noury Bouraqadi. Understanding Pharo’s global state to move programs through time and space. In IWST - International Workshop on Smalltalk Technology, Co-located within the 22th International Smalltalk Conference - 2014, 2014.
%
%Guillermo Polito, Stéphane Ducasse, Luc Fabresse, and Noury Bouraqadi. Virtual Smalltalk Images: Model and Applications. In IWST - International Workshop on Smalltalk Technology, Co-located within the 21th International Smalltalk Conference - 2013, 2013.

\section{Future Work}

\Vtt presents a language runtime virtualization model that opens several directions for future work that we consider for exploration.

\begin{description}

\item[Security.] Virtualization opens the door to easily forbid or constraint operations to the virtualized runtime. Sandboxing can then be transparent to the virtualized application, which may believe that it \emph{owns} the entire machine for itself.

\item[Resource Control.] An application could be virtualized to restrict its consumption of critical resources such as CPU or memory. Doing so in a flexible way with our first class runtimes would simplify \eg the creation of simulators for specific platforms such as constrained devices.

\item[Application distribution and migration.] The virtual language runtime API exposed by an object space encapsulates the internals of the implementation. This same API may provide transparent access to remote application runtimes residing in different processes/machines. This could open the possibility of exploring application distribution and migration in the \emph{cloud} at the language level.

\item[Dynamic Adaptation.] Language virtualization can be also useful in the context of dynamic adaptation of applications with almost zero downtime. An object space API is general enough to allow updates to occur at runtime. Additionally, the execution cycles provides with atomicity for doing such changes.

\item[VM-Language Co-Evolution.] Our approach doesn't address the co-evolution of language and VM. However, object spaces make explicit the border-line between the VM and the language to change the latter. A co-evolution implies changing not only language and VM but also their interface.

\end{description}

% =============================================================================
%\section{Software Artifacts}
%
%The research that appears in this thesis was supported and validated by several prototypes implemented in the course of the P.h.d. The produced software artifacts are the following:
%
%\begin{description}
%
%\item[\Vtt.] \Vtt's prototype is the main software artifact that results from this research. This prototype included changes in the Pharo \VM to allow the described co-existence, manipulation and control of a language runtime. \Vtt also includes several language libraries that expose the \VM behavior: object spaces and its mirrros encapsulate a language runtime; a \Vtt-based AST interpreter was created from an existent AST interpreter in Pharo; a heap exporter and importer allowed us to work on existing smalltalk images.
%
%\item[Hazelnut/Seed.] Hazelnut/Seed is our bootstrapping solution for the Pharo language. Hazelnut describes a language runtime and allow its creation through a bootstrapping interpreter based on \Vtt's one.
%
%\item[Tornado.] Tornado is our tailoring implementation on top of \Vtt. Tornado uses the Ghost proxies model to implement execution traps and the \Vtt execution cycle to install code when missing code is detected.
%
%\end{description}

% =============================================================================
\input{chapter-footer.tex}

% =============================================================================

\bibliographystyle{alpha}
\bibliography{others,rmod,scg}

% =============================================================================

\appendix
\input{chapter-header.tex}
% =============================================================================
\appendix
\chapter{Appendix A: \PH Programming Language}
\markboth{Appendix}{Appendix}
\label{appendixa}
%\minitoc
% =============================================================================

%\clearpage

%\vspace*{5cm}
% =============================================================================
%\section{\PH Programming Language}
%\seclabel{pharo}
% =============================================================================


\PH is a \ST inspired object-oriented and dynamically-typed general-purpose language with its own programming environment.
The language has a simple and expressive syntax which can be learned in a few minutes.
Concepts in \PH are very consistent, everything is an object: classes, methods, numbers, strings, even the execution context.

\PH runs on top of a bytecode-based \emph{virtual machine}.
Development takes place in an \emph{image} in which all objects reside.
All these objects can be modified by the programmer, this includes classes and methods.
Hence, we eliminate the typical edit/compile/run cycle and instead incrementally add, remove or modify classes and methods.
It is worth noting that \emph{all} classes can be extended with new methods in \PH.
For instance, one can add new operations on integers or strings, classes that are treated as unchangeable internal objects by many other high-level languages.
For deployment and debugging, the state of a running image can be saved at any point and subsequently restored.

%\newpage

% ---------------------------------------------------------------------------
\section{Minimal Syntax}
% ---------------------------------------------------------------------------

\noindent
\begin{tabularx}{\linewidth}{@{}rX@{}}
	\multicolumn{2}{l}{Reserved Words}\\
	\midrule
	\textcolor{darkRed}{\texttt{nil}} & the undefined object\\
	\textcolor{darkRed}{\texttt{true}}, \textcolor{darkRed}{\texttt{false}} & boolean objects\\
	\textcolor{darkCyan}{\texttt{self}} & the receiver of the current message\\
	\textcolor{darkCyan}{\texttt{super}} & the receiver, in the superclass context\\
	\textcolor{darkCyan}{\texttt{thisContext}} & the current invocation on the call stack \\
	\\
	\multicolumn{2}{l}{Literal Object Syntax}\\
	\midrule
	\textcolor{string}{\texttt{'}{a string}\texttt{'}} & \\
	\textcolor{string}{\texttt{\#symbol}} & unique string \\
	\textcolor{darkRed}{\texttt{\$a}} & the character \textcolor{darkRed}{\texttt{a}} \\
	\textcolor{darkRed}{\texttt{12 2r1100 16rC}} & integers twelve in decimal, binary and hexadecimal encoding\\
	\textcolor{darkRed}{\texttt{3.14 1.2e3}} & floating-point numbers\\
	\texttt{\#(\textcolor{string}{abc} \textcolor{darkRed}{123})} & literal array containing the symbol \textcolor{string}{\texttt{\#abc}} and the number \textcolor{darkRed}{\texttt{123}} \\
	\texttt{\#[\textcolor{darkRed}{12} \textcolor{darkRed}{16rFF}]} & literal byte array containing the bytes/integers \textcolor{darkRed}{12} and \textcolor{darkRed}{255}\\
	\texttt{\{\textcolor{darkBlue}{foo}\,.\ \textcolor{darkRed}{3}\,+\,\textcolor{darkRed}{2}\}} & dynamic array built from 2 expressions\\
	
	\\
	\multicolumn{2}{l}{Reserved Characters in Expressions}\\
	\midrule
	\textcolor{comment}{\texttt{"}{a comment}\texttt{"}} & \\
	\texttt{.} & expression separator (period)\\
	\texttt{;} & message cascade (semicolon)\\
	\texttt{:=} & {assignment} \\
	\texttt{\textasciicircum} & return a result from a method (caret)\\
	\texttt{[\,:\textcolor{darkBlue}{p}\,|\,}\emph{expr}\texttt{\,]} & code block with a parameter \\
	\texttt{|\,\textcolor{darkBlue}{foo bar}\,|} & declaration of two temporary variables\\
	\texttt{<pragma>}, \texttt{<primitive: 3>} & pragma or annotations used in methods, for instances to declare a primitive method.
\end{tabularx}

% ---------------------------------------------------------------------------
\section{Message Sending}
% ---------------------------------------------------------------------------

A method is called by sending a message to an object called the \emph{receiver}.
Each message returns an object.
Messages are modeled from natural languages with a subject a verb and complements. There are three types of messages with descending precedence: unary, binary, and keyword.

\begin{description}
\item[Unary messages] have no arguments.

\begin{alltt}
\textcolor{darkBlue}{Array} new.
\end{alltt}

The first example creates and returns a new instance of the \textcolor{darkBlue}{\texttt{Array}} class, by sending the message \texttt{new} to the class
\textcolor{darkBlue}{\texttt{Array}} that is an object.

\begin{alltt}
#(\textcolor{darkRed}{1 2 3}) size.
\end{alltt}

The second message returns the size of the literal array which is \textcolor{darkRed}{\texttt{3}}.

\item[Binary messages] take only one argument and are named by one or more symbol characters.

\begin{alltt}
\textcolor{darkRed}{3} + \textcolor{darkRed}{4}.
\end{alltt}

The \texttt{+} message is sent to the integer object \textcolor{darkRed}{\texttt{3}} with \textcolor{darkRed}{\texttt{4}} as the argument.

\begin{alltt}
\textcolor{string}{'Hello'}, \textcolor{string}{' World'}.
\end{alltt}

In the second case, the string \textcolor{string}{\texttt{'Hello'}} receives the message \texttt{,} (comma) with the string \textcolor{string}{\texttt{'~World'}} as the argument.

\item[Keyword messages] can take one or more arguments that are inserted in the message name.

\begin{alltt}
\textcolor{string}{'Smalltalk'} allButFirst: \textcolor{darkRed}{5}.
\end{alltt}

The first example sends the message \texttt{allButFirst:} to a string, with the argument \textcolor{darkRed}{\texttt{5}}.
This returns the string \textcolor{string}{\texttt{'talk'}}.

\begin{alltt}
\textcolor{darkRed}{3} to: \textcolor{darkRed}{10} by: \textcolor{darkRed}{2}.
\end{alltt}

The second example sends \texttt{to:by:} to \textcolor{darkRed}{\texttt{3}}, with arguments \textcolor{darkRed}{\texttt{10}} and \textcolor{darkRed}{\texttt{2}}; this returns a collection containing \textcolor{darkRed}{\texttt{3}}, \textcolor{darkRed}{\texttt{5}}, \textcolor{darkRed}{\texttt{7}}, and \textcolor{darkRed}{\texttt{9}}.

\end{description}


% ---------------------------------------------------------------------------
\section{Precedence}
% ---------------------------------------------------------------------------

There is a fixed global precedence when evaluating expressions in \PH: Parentheses\,$>$\,unary\,$>$\,binary\,$>$\,keyword, and finally from left to right.

\begin{alltt}
(\textcolor{darkRed}{10} between: \textcolor{darkRed}{1} and: \textcolor{darkRed}{2}\,+\,\textcolor{darkRed}{4}\,*\,\textcolor{darkRed}{3}) not
\end{alltt}

Here, the messages \texttt{+} and \texttt{*} are sent first, then \texttt{between:and:} is sent, and finally \texttt{not}.
The rule suffers no exception: operators are just binary messages with \emph{no notion of mathematical precedence}, so \texttt{\textcolor{darkRed}{2}\,+\,\textcolor{darkRed}{4}\,*\,\textcolor{darkRed}{3}} reads left-to-right and thus yields \textcolor{darkRed}{18} and not the expected \textcolor{darkRed}{14}!

% ---------------------------------------------------------------------------
\section{Cascading Messages}
% ---------------------------------------------------------------------------

Multiple messages can be sent to the same receiver with \texttt{;}.

\begin{alltt}
\textcolor{darkBlue}{OrderedCollection} new
  add: \textcolor{string}{#abc};
  add: \textcolor{string}{#def};
  add: \textcolor{string}{#ghi}.
\end{alltt}

The message \texttt{new} is sent to \texttt{\textcolor{darkBlue}{OrderedCollection}} which
results in a new collection to which three \texttt{add:} messages are sent with different arguments.
The value of the whole message cascade is the value of the last message sent (here, the symbol \textcolor{string}{\texttt{\#ghi}}).
This example is the equivalent of first assigning the new collection to a temporary variable and sending three separate \texttt{add:} messages:

\begin{alltt}
| newCollection | 
newCollection := \textcolor{darkBlue}{OrderedCollection} new.
newCollection add: \textcolor{string}{#abc}.
newCollection add: \textcolor{string}{#def}.
newCollection add: \textcolor{string}{#ghi}.
\end{alltt}


To return the original receiver of the message cascade (\ie the collection) instead of the last result (\ie \textcolor{string}{\texttt{\#ghi}}), the \texttt{yourself} message is used:

\begin{alltt}
\textcolor{darkBlue}{OrderedCollection} new
  add: \textcolor{string}{#abc};
  add: \textcolor{string}{#def};
  add: \textcolor{string}{#ghi};
  yourself.
\end{alltt}

% ---------------------------------------------------------------------------
\section{Blocks}
% ---------------------------------------------------------------------------

Blocks are objects containing code that is executed on demand,
(anonymous functions or closures).
They are the basis for control structures like conditionals and loops.

\begin{alltt}
\textcolor{darkRed}{2} = \textcolor{darkRed}{2}
  ifTrue: [ \textcolor{darkBlue}{Error} signal: \textcolor{string}{'Help'} ].
\end{alltt}

The first example sends the message \texttt{ifTrue:} to the boolean
\textcolor{darkRed}{\texttt{true}} (computed from \texttt{\textcolor{darkRed}{2} = \textcolor{darkRed}{2}}) with a block as argument.
Because the boolean is \textcolor{darkRed}{\texttt{true}}, the block is executed and an exception is signaled.

\begin{alltt}
\#(\textcolor{string}{'Hello World'} \textcolor{darkRed}{\$!})
  do: [ :\textcolor{darkBlue}{e} | \textcolor{darkBlue}{Transcript} show: \textcolor{darkBlue}{e} ]
\end{alltt}

The next example sends the message \texttt{do:} to an array.
This evaluates the block once for each element, passing it via the \texttt{e} parameter.
As a result, \texttt{\textcolor{string}{Hello~World!}} is printed.


% ---------------------------------------------------------------------------
\section{Methods}
% ---------------------------------------------------------------------------

Methods are first-class objects in \PH and can be inspected and modified on the fly.
Methods are created by saving expressions in the \PH development environment.
Typically methods are printed with a special first line indicating the class the method is installed on and the name or selector it is given.

\begin{alltt}
\textcolor{darkBlue}{Array} >> helpMethod
    \textcolor{darkRed}{2} = \textcolor{darkRed}{2}
        ifTrue: [ \textcolor{darkBlue}{Error} signal: \textcolor{string}{'Help'} ].
\end{alltt}

This example would denote a simple method with a unary selector on the \texttt{Array} class.
This method could be invoked by evaluating \texttt{\textcolor{darkBlue}{Array} new helpMethod}.

Certain methods are marked with a pragma to use predefined primitives from the \VM.
These are used for expressions that cannot be expressed in \PH.
For instance the \texttt{basicNew} which allocates new objects uses the primitive number 70:

\begin{alltt}
\textcolor{darkBlue}{Behavior} >> basicNew
    \textcolor{comment}{"Answer a new instance of this class"}
    <primitive: \textcolor{darkRed}{70}>
    \textcolor{darkBlue}{OutOfMemory} signal.
\end{alltt}

\chapter{Bootstrap Extracts}
\label{appendixbootstrap}

This appendix shows the extracts of classes and/or methods for the languages we bootstrapped.

\section{Pharo Bootstrap Extract}

This section lists the classes that were extracted from Pharo (version 3) for its bootstrapping. For brevity we do not list the methods in these classes. Additionally, our extraction for bootstrap includes all methods in such classes.

\begin{multicols}{2}\noindent\small
\ct{ASTCache}\newline
\ct{Abort}\newline
\ct{AbstractClassInstaller}\newline
\ct{AbstractClassModification}\newline
\ct{AbstractCompiler}\newline
\ct{AbstractFieldModification}\newline
\ct{AbstractLayout}\newline
\ct{AbstractMethodUpdateStrategy}\newline
\ct{AbstractModification}\newline
\ct{AbstractTimeZone}\newline
\ct{AddedField}\newline
\ct{AdditionalMethodState}\newline
\ct{AllProtocol}\newline
\ct{Announcement}\newline
\ct{AnnouncementLogger}\newline
\ct{AnnouncementSet}\newline
\ct{AnnouncementSubscription}\newline
\ct{Announcer}\newline
\ct{AnonymousClassInstaller}\newline
\ct{ArithmeticError}\newline
\ct{Array}\newline
\ct{ArrayedCollection}\newline
\ct{AssertionFailure}\newline
\ct{Association}\newline
\ct{Author}\newline
\ct{AuthorNameRequest}\newline
\ct{Bag}\newline
\ct{Beeper}\newline
\ct{Behavior}\newline
\ct{BitsLayout}\newline
\ct{BlockCannotReturn}\newline
\ct{BlockClosure}\newline
\ct{BlockLocalTempCounter}\newline
\ct{Boolean}\newline
\ct{ByteArray}\newline
\ct{ByteLayout}\newline
\ct{ByteString}\newline
\ct{ByteSymbol}\newline
\ct{CCompilationContext}\newline
\ct{Categorizer}\newline
\ct{CategoryAdded}\newline
\ct{CategoryRemoved}\newline
\ct{CategoryRenamed}\newline
\ct{ChangesLog}\newline
\ct{Character}\newline
\ct{CharacterSet}\newline
\ct{CharacterSetComplement}\newline
\ct{ChronologyConstants}\newline
\ct{CircularHierarchyError}\newline
\ct{Class}\newline
\ct{ClassAdded}\newline
\ct{ClassAnnouncement}\newline
\ct{ClassCategoryReader}\newline
\ct{ClassCommentReader}\newline
\ct{ClassCommented}\newline
\ct{ClassDescription}\newline
\ct{ClassModification}\newline
\ct{ClassModificationPropagation}\newline
\ct{ClassModifiedClassDefinition}\newline
\ct{ClassOrganization}\newline
\ct{ClassRecategorized}\newline
\ct{ClassRemoved}\newline
\ct{ClassRenamed}\newline
\ct{ClassReorganized}\newline
\ct{ClassTrait}\newline
\ct{Collection}\newline
\ct{CollectionIsEmpty}\newline
\ct{CombinedChar}\newline
\ct{CompilationContext}\newline
\ct{CompiledMethod}\newline
\ct{CompiledMethodLayout}\newline
\ct{CompiledMethodTrailer}\newline
\ct{ContextPart}\newline
\ct{Continuation}\newline
\ct{DangerousClassNotifier}\newline
\ct{Date}\newline
\ct{DateAndTime}\newline
\ct{DateParser}\newline
\ct{DebuggerMethodMapOpal}\newline
\ct{DeepCopier}\newline
\ct{DefaultExternalDropHandler}\newline
\ct{Delay}\newline
\ct{DelayWaitTimeout}\newline
\ct{DependentsArray}\newline
\ct{Deprecation}\newline
\ct{Dictionary}\newline
\ct{DomainError}\newline
\ct{DosTimestamp}\newline
\ct{DuplicatedSlotName}\newline
\ct{DuplicatedVariableError}\newline
\ct{Duration}\newline
\ct{DynamicVariable}\newline
\ct{EmptyLayout}\newline
\ct{Error}\newline
\ct{EventManager}\newline
\ct{EventSensorConstants}\newline
\ct{ExactFloatPrintPolicy}\newline
\ct{Exception}\newline
\ct{ExceptionSet}\newline
\ct{ExceptionSetWithExclusions}\newline
\ct{Exit}\newline
\ct{ExpressionEvaluated}\newline
\ct{ExternalDropHandler}\newline
\ct{ExternalSemaphoreTable}\newline
\ct{False}\newline
\ct{FixedLayout}\newline
\ct{Float}\newline
\ct{FloatArray}\newline
\ct{FloatPrintPolicy}\newline
\ct{FloatingPointException}\newline
\ct{Fraction}\newline
\ct{Generator}\newline
\ct{Halt}\newline
\ct{HashTableSizes}\newline
\ct{HashedCollection}\newline
\ct{Heap}\newline
\ct{IRAccess}\newline
\ct{IRBlockReturnTop}\newline
\ct{IRBuilder}\newline
\ct{IRBytecodeDecompiler}\newline
\ct{IRBytecodeGenerator}\newline
\ct{IRBytecodeScope}\newline
\ct{IRInstVarAccess}\newline
\ct{IRInstruction}\newline
\ct{IRInterpreter}\newline
\ct{IRJump}\newline
\ct{IRJumpIf}\newline
\ct{IRLiteralVariableAccess}\newline
\ct{IRMethod}\newline
\ct{IRPop}\newline
\ct{IRPrimitive}\newline
\ct{IRPrinter}\newline
\ct{IRPushArray}\newline
\ct{IRPushClosureCopy}\newline
\ct{IRPushDup}\newline
\ct{IRPushLiteral}\newline
\ct{IRReceiverAccess}\newline
\ct{IRReconstructor}\newline
\ct{IRRemoteArray}\newline
\ct{IRRemoteTempAccess}\newline
\ct{IRReturn}\newline
\ct{IRSend}\newline
\ct{IRSequence}\newline
\ct{IRStackCount}\newline
\ct{IRTempAccess}\newline
\ct{IRTempVector}\newline
\ct{IRThisContextAccess}\newline
\ct{IRTranslator}\newline
\ct{IRVisitor}\newline
\ct{IdentityBag}\newline
\ct{IdentityDictionary}\newline
\ct{IdentitySet}\newline
\ct{IllegalResumeAttempt}\newline
\ct{InMidstOfFileinNotification}\newline
\ct{IncompatibleLayoutConflict}\newline
\ct{InexactFloatPrintPolicy}\newline
\ct{InputEventFetcher}\newline
\ct{InputEventHandler}\newline
\ct{InputEventSensor}\newline
\ct{InstVarRefLocator}\newline
\ct{InstanceModification}\newline
\ct{InstructionClient}\newline
\ct{InstructionPrinter}\newline
\ct{InstructionStream}\newline
\ct{Integer}\newline
\ct{IntegerArray}\newline
\ct{Interval}\newline
\ct{InvalidGlobalName}\newline
\ct{InvalidSlotName}\newline
\ct{InvalidSuperclass}\newline
\ct{Job}\newline
\ct{JobChange}\newline
\ct{JobDetector}\newline
\ct{JobEnd}\newline
\ct{JobNotification}\newline
\ct{JobProgress}\newline
\ct{JobStart}\newline
\ct{JobStartNotification}\newline
\ct{KeyNotFound}\newline
\ct{KeyedTree}\newline
\ct{LargeInteger}\newline
\ct{LargeNegativeInteger}\newline
\ct{LargePositiveInteger}\newline
\ct{LayoutAbstractScope}\newline
\ct{LayoutClassScope}\newline
\ct{LayoutEmptyScope}\newline
\ct{LegacyWeakSubscription}\newline
\ct{LimitedWriteStream}\newline
\ct{LimitingLineStreamWrapper}\newline
\ct{Link}\newline
\ct{LinkedList}\newline
\ct{LocalTimeZone}\newline
\ct{LookupKey}\newline
\ct{Magnitude}\newline
\ct{ManifestASTCore}\newline
\ct{ManifestOpalCompilerCore}\newline
\ct{Margin}\newline
\ct{Matrix}\newline
\ct{Message}\newline
\ct{MessageCatcher}\newline
\ct{MessageNotUnderstood}\newline
\ct{MessageSend}\newline
\ct{Metaclass}\newline
\ct{MethodAdded}\newline
\ct{MethodClassifier}\newline
\ct{MethodContext}\newline
\ct{MethodDictionary}\newline
\ct{MethodModification}\newline
\ct{MethodModified}\newline
\ct{MethodRecategorized}\newline
\ct{MethodRecompileStrategy}\newline
\ct{MethodRemoved}\newline
\ct{Model}\newline
\ct{ModifiedField}\newline
\ct{Monitor}\newline
\ct{MonitorDelay}\newline
\ct{Month}\newline
\ct{MultiByteBinaryOrTextStream}\newline
\ct{Mutex}\newline
\ct{MutexSet}\newline
\ct{NaNException}\newline
\ct{NonBooleanReceiver}\newline
\ct{NonInteractiveTranscript}\newline
\ct{NotFound}\newline
\ct{NotYetImplemented}\newline
\ct{Notification}\newline
\ct{NullStream}\newline
\ct{Number}\newline
\ct{NumberParser}\newline
\ct{OCASTClosureAnalyzer}\newline
\ct{OCASTSemanticAnalyzer}\newline
\ct{OCASTTranslator}\newline
\ct{OCASTTranslatorForEffect}\newline
\ct{OCASTTranslatorForValue}\newline
\ct{OCAbstractLocalVariable}\newline
\ct{OCAbstractMethodScope}\newline
\ct{OCAbstractScope}\newline
\ct{OCAbstractVariable}\newline
\ct{OCBlockScope}\newline
\ct{OCClassScope}\newline
\ct{OCCopyingTempVariable}\newline
\ct{OCInstanceScope}\newline
\ct{OCInstanceVariable}\newline
\ct{OCKeyedSet}\newline
\ct{OCLiteralList}\newline
\ct{OCLiteralSet}\newline
\ct{OCLiteralVariable}\newline
\ct{OCMethodScope}\newline
\ct{OCOptimizedBlockScope}\newline
\ct{OCRequestorScope}\newline
\ct{OCSemanticError}\newline
\ct{OCSemanticWarning}\newline
\ct{OCShadowVariableWarning}\newline
\ct{OCSourceCodeChanged}\newline
\ct{OCSpecialVariable}\newline
\ct{OCTempVariable}\newline
\ct{OCUndeclaredVariable}\newline
\ct{OCUndeclaredVariableWarning}\newline
\ct{OCUninitializedVariableWarning}\newline
\ct{OCUnknownSelectorWarning}\newline
\ct{OCUnusedVariableWarning}\newline
\ct{OCVectorTempVariable}\newline
\ct{Object}\newline
\ct{ObjectFinalizer}\newline
\ct{ObjectFinalizerCollection}\newline
\ct{ObjectLayout}\newline
\ct{OldClassBuilderAdapter}\newline
\ct{OpalCompiler}\newline
\ct{OrderedCollection}\newline
\ct{OrderedIdentityDictionary}\newline
\ct{OutOfMemory}\newline
\ct{PackageInfo}\newline
\ct{PackageOrganizer}\newline
\ct{PharoClassInstaller}\newline
\ct{PluggableDictionary}\newline
\ct{PluggableSet}\newline
\ct{Point}\newline
\ct{PointerLayout}\newline
\ct{PositionableStream}\newline
\ct{Pragma}\newline
\ct{PragmaAdded}\newline
\ct{PragmaAnnouncement}\newline
\ct{PragmaCollector}\newline
\ct{PragmaCollectorReset}\newline
\ct{PragmaRemoved}\newline
\ct{PragmaUpdated}\newline
\ct{PrimitiveFailed}\newline
\ct{Process}\newline
\ct{ProcessLocalVariable}\newline
\ct{ProcessSpecificVariable}\newline
\ct{ProcessorScheduler}\newline
\ct{ProtoObject}\newline
\ct{Protocol}\newline
\ct{ProtocolAdded}\newline
\ct{ProtocolAnnouncement}\newline
\ct{ProtocolOrganizer}\newline
\ct{ProtocolRemovalException}\newline
\ct{ProtocolRemoved}\newline
\ct{PseudoClassOrganization}\newline
\ct{RBArgumentNode}\newline
\ct{RBArrayNode}\newline
\ct{RBAssignmentNode}\newline
\ct{RBAssignmentToken}\newline
\ct{RBBinarySelectorToken}\newline
\ct{RBBlockNode}\newline
\ct{RBBlockReplaceRule}\newline
\ct{RBCascadeNode}\newline
\ct{RBClassReference}\newline
\ct{RBConfigurableFormatter}\newline
\ct{RBErrorToken}\newline
\ct{RBExplicitVariableParser}\newline
\ct{RBIdentifierToken}\newline
\ct{RBKeywordToken}\newline
\ct{RBLiteralArrayNode}\newline
\ct{RBLiteralArrayToken}\newline
\ct{RBLiteralNode}\newline
\ct{RBLiteralToken}\newline
\ct{RBLiteralValueNode}\newline
\ct{RBMessageNode}\newline
\ct{RBMethodNode}\newline
\ct{RBMultiKeywordLiteralToken}\newline
\ct{RBNumberLiteralToken}\newline
\ct{RBParseErrorNode}\newline
\ct{RBParseTreeRewriter}\newline
\ct{RBParseTreeRule}\newline
\ct{RBParseTreeSearcher}\newline
\ct{RBParser}\newline
\ct{RBPatternBlockNode}\newline
\ct{RBPatternBlockToken}\newline
\ct{RBPatternMessageNode}\newline
\ct{RBPatternMethodNode}\newline
\ct{RBPatternParser}\newline
\ct{RBPatternPragmaNode}\newline
\ct{RBPatternScanner}\newline
\ct{RBPatternVariableNode}\newline
\ct{RBPatternWrapperBlockNode}\newline
\ct{RBPragmaNode}\newline
\ct{RBProgramNode}\newline
\ct{RBProgramNodeVisitor}\newline
\ct{RBReadBeforeWrittenTester}\newline
\ct{RBReplaceRule}\newline
\ct{RBReturnNode}\newline
\ct{RBScanner}\newline
\ct{RBSearchRule}\newline
\ct{RBSelfNode}\newline
\ct{RBSequenceNode}\newline
\ct{RBShortAssignmentToken}\newline
\ct{RBSpecialCharacterToken}\newline
\ct{RBStringReplaceRule}\newline
\ct{RBStringReplacement}\newline
\ct{RBSuperNode}\newline
\ct{RBTemporaryNode}\newline
\ct{RBThisContextNode}\newline
\ct{RBToken}\newline
\ct{RBValueNode}\newline
\ct{RBValueToken}\newline
\ct{RBVariableNode}\newline
\ct{RPackage}\newline
\ct{RPackageAnnouncement}\newline
\ct{RPackageConflictError}\newline
\ct{RPackageCreated}\newline
\ct{RPackageOrganizer}\newline
\ct{RPackageRenamed}\newline
\ct{RPackageSet}\newline
\ct{RPackageTag}\newline
\ct{RPackageUnregistered}\newline
\ct{RWBinaryOrTextStream}\newline
\ct{Random}\newline
\ct{ReadStream}\newline
\ct{ReadWriteStream}\newline
\ct{Rectangle}\newline
\ct{RelativeInstructionPrinter}\newline
\ct{RemovedField}\newline
\ct{ScaledDecimal}\newline
\ct{Schedule}\newline
\ct{SelectorException}\newline
\ct{Semaphore}\newline
\ct{SequenceableCollection}\newline
\ct{Session}\newline
\ct{Set}\newline
\ct{SetElement}\newline
\ct{SharedPool}\newline
\ct{SharedQueue}\newline
\ct{ShiftedField}\newline
\ct{ShouldBeImplemented}\newline
\ct{ShouldNotImplement}\newline
\ct{SimulationExceptionWrapper}\newline
\ct{SizeMismatch}\newline
\ct{Slot}\newline
\ct{SlotClassBuilder}\newline
\ct{SlotClassBuilderError}\newline
\ct{SlotClassBuilderWarning}\newline
\ct{SlotNotFound}\newline
\ct{SmallDictionary}\newline
\ct{SmallIdentityDictionary}\newline
\ct{SmallInteger}\newline
\ct{SmallIntegerLayout}\newline
\ct{SmalltalkImage}\newline
\ct{SnapshotDone}\newline
\ct{SortedCollection}\newline
\ct{SparseLargeArray}\newline
\ct{SparseLargeTable}\newline
\ct{Stopwatch}\newline
\ct{Stream}\newline
\ct{String}\newline
\ct{SubclassResponsibility}\newline
\ct{SubscriptOutOfBounds}\newline
\ct{SubscriptionRegistry}\newline
\ct{Symbol}\newline
\ct{SyntaxErrorNotification}\newline
\ct{SystemAnnouncement}\newline
\ct{SystemAnnouncer}\newline
\ct{SystemDictionary}\newline
\ct{SystemNavigation}\newline
\ct{SystemOrganizer}\newline
\ct{SystemVersion}\newline
\ct{TApplyingOnClassSide}\newline
\ct{TBehavior}\newline
\ct{TBehaviorCategorization}\newline
\ct{TClass}\newline
\ct{TClassDescription}\newline
\ct{TComparable}\newline
\ct{TComposingDescription}\newline
\ct{TIRVisitor}\newline
\ct{TRBProgramNodeVisitor}\newline
\ct{TSortable}\newline
\ct{TTransformationCompatibility}\newline
\ct{TextStream}\newline
\ct{ThreadSafeTranscript}\newline
\ct{Time}\newline
\ct{TimeStamp}\newline
\ct{TimeZone}\newline
\ct{TimedOut}\newline
\ct{Timespan}\newline
\ct{Trait}\newline
\ct{TraitAlias}\newline
\ct{TraitBehavior}\newline
\ct{TraitComposition}\newline
\ct{TraitCompositionException}\newline
\ct{TraitDescription}\newline
\ct{TraitException}\newline
\ct{TraitExclusion}\newline
\ct{TraitMethodDescription}\newline
\ct{TraitTransformation}\newline
\ct{True}\newline
\ct{UndefinedObject}\newline
\ct{UnhandledError}\newline
\ct{UnmodifiedField}\newline
\ct{UserInterruptHandler}\newline
\ct{ValueLink}\newline
\ct{ValueNotFound}\newline
\ct{VariableLayout}\newline
\ct{VirtualMachine}\newline
\ct{Warning}\newline
\ct{WeakActionSequence}\newline
\ct{WeakAnnouncementSubscription}\newline
\ct{WeakArray}\newline
\ct{WeakFinalizationList}\newline
\ct{WeakFinalizerItem}\newline
\ct{WeakIdentityKeyDictionary}\newline
\ct{WeakKeyAssociation}\newline
\ct{WeakKeyDictionary}\newline
\ct{WeakKeyToCollectionDictionary}\newline
\ct{WeakLayout}\newline
\ct{WeakMessageSend}\newline
\ct{WeakOrderedCollection}\newline
\ct{WeakRegistry}\newline
\ct{WeakSet}\newline
\ct{WeakSubscriptionBuilder}\newline
\ct{WeakValueAssociation}\newline
\ct{WeakValueDictionary}\newline
\ct{Week}\newline
\ct{WideCharacterSet}\newline
\ct{WideString}\newline
\ct{WideSymbol}\newline
\ct{WordArray}\newline
\ct{WordLayout}\newline
\ct{WriteStream}\newline
\ct{Year}\newline
\ct{ZeroDivide}\newline
\end{multicols}

\section{Candle Bootstrap Extract}

This section lists the methods from Candle for its bootstrapping.

\begin{multicols}{2}\noindent\small
\ct{PCAssociation>>hash}\newline
\ct{PCAssociation>>value:}\newline
\ct{PCAssociation>>key}\newline
\ct{PCAssociation>>key:}\newline
\ct{PCAssociation>>=}\newline
\ct{PCAssociation>>value}\newline
\ct{PCAssociation>>key:value:}\newline
\ct{PCAssociation>>printOn:}\newline
\ct{PCAssociation>><}\newline
\ct{PCOrderedCollection>>remove:ifAbsent:}\newline
\ct{PCOrderedCollection>>last}\newline
\ct{PCOrderedCollection>>size}\newline
\ct{PCOrderedCollection>>errorNoSuchElement}\newline
\ct{PCOrderedCollection>>add:}\newline
\ct{PCOrderedCollection>>at:}\newline
\ct{PCOrderedCollection>>at:put:}\newline
\ct{PCOrderedCollection>>makeRoomAtFirst}\newline
\ct{PCOrderedCollection>>copyFrom:to:}\newline
\ct{PCOrderedCollection>>grow}\newline
\ct{PCOrderedCollection>>setCollection:}\newline
\ct{PCOrderedCollection>>collect:}\newline
\ct{PCOrderedCollection>>makeRoomAtLast}\newline
\ct{PCOrderedCollection>>}\newline\indent\ct{copyReplaceFrom:to:with:}\newline
\ct{PCOrderedCollection>>addFirst:}\newline
\ct{PCOrderedCollection>>do:}\newline
\ct{PCOrderedCollection>>first}\newline
\ct{PCOrderedCollection>>insert:before:}\newline
\ct{PCOrderedCollection>>removeFirst}\newline
\ct{PCOrderedCollection>>removeIndex:}\newline
\ct{PCOrderedCollection>>removeLast}\newline
\ct{PCOrderedCollection>>select:}\newline
\ct{PCIdentityDictionary>>scanFor:}\newline
\ct{PCIdentityDictionary>>keys}\newline
\ct{PCFile>>primWrite:from:startingAt:count:}\newline
\ct{PCFile>>close}\newline
\ct{PCFile>>cr}\newline
\ct{PCFile>>position}\newline
\ct{PCFile>>position:}\newline
\ct{PCFile>>localFolderPath}\newline
\ct{PCFile>>name}\newline
\ct{PCFile>>nextPutAll:}\newline
\ct{PCFile>>primClose:}\newline
\ct{PCFile>>openReadWrite:}\newline
\ct{PCFile>>primGetPosition:}\newline
\ct{PCFile>>size}\newline
\ct{PCFile>>readInto:startingAt:count:}\newline
\ct{PCFile>>primOpen:writable:}\newline
\ct{PCFile>>next:}\newline
\ct{PCFile>>openReadOnly:}\newline
\ct{PCFile>>primRead:into:startingAt:count:}\newline
\ct{PCFile>>primSize:}\newline
\ct{PCFile>>primImageName}\newline
\ct{PCFile>>primSetPosition:to:}\newline
\ct{PCMessage>>lookupClass}\newline
\ct{PCMessage>>sentTo:}\newline
\ct{PCMessage>>arguments}\newline
\ct{PCMessage>>printOn:}\newline
\ct{PCMessage>>selector}\newline
\ct{PCIdentitySet>>scanFor:}\newline
\ct{PCFloat>>ln}\newline
\ct{PCFloat>>printOn:base:}\newline
\ct{PCFloat>>reciprocalLogBase2}\newline
\ct{PCFloat>>sqrt}\newline
\ct{PCFloat>>tan}\newline
\ct{PCFloat>>truncated}\newline
\ct{PCFloat>>reciprocal}\newline
\ct{PCFloat>>raisedTo:}\newline
\ct{PCFloat>>hash}\newline
\ct{PCFloat>>/}\newline
\ct{PCFloat>>rounded}\newline
\ct{PCFloat>>radiansToDegrees}\newline
\ct{PCFloat>>-}\newline
\ct{PCFloat>>abs}\newline
\ct{PCFloat>>=}\newline
\ct{PCFloat>>adaptToInteger:andSend:}\newline
\ct{PCFloat>>arcCos}\newline
\ct{PCFloat>>>}\newline
\ct{PCFloat>>arcTan}\newline
\ct{PCFloat>>cos}\newline
\ct{PCFloat>>floorLog:}\newline
\ct{PCFloat>>significand}\newline
\ct{PCFloat>>timesTwoPower:}\newline
\ct{PCFloat>>fractionPart}\newline
\ct{PCFloat>>exponent}\newline
\ct{PCFloat>>isNaN}\newline
\ct{PCFloat>>asFloat}\newline
\ct{PCFloat>>arcSin}\newline
\ct{PCFloat>>sin}\newline
\ct{PCFloat>>~=}\newline
\ct{PCFloat>>degreesToRadians}\newline
\ct{PCFloat>>isInfinite}\newline
\ct{PCFloat>>log}\newline
\ct{PCFloat>>sign}\newline
\ct{PCFloat>>*}\newline
\ct{PCFloat>><=}\newline
\ct{PCFloat>>>=}\newline
\ct{PCFloat>>exp}\newline
\ct{PCFloat>>negated}\newline
\ct{PCFloat>>absPrintOn:base:}\newline
\ct{PCFloat>>+}\newline
\ct{PCFloat>><}\newline
\ct{PCClass>>name:}\newline
\ct{PCClass>>classSide}\newline
\ct{PCClass>>}\newline\indent\ct{weakSubclass:instanceVariableNames:}\newline\indent\ct{classVariableNames:}\newline
\ct{PCClass>>initFrom:methodDict:}\newline
\ct{PCClass>>}\newline\indent\ct{variableWordSubclass:}\newline\indent\ct{instanceVariableNames:}\newline\indent\ct{classVariableNames:}\newline
\ct{PCClass>>}\newline\indent\ct{newClassBuilderForSubclass:}\newline\indent\ct{instanceVariableNames:}\newline\indent\ct{classVariableNames:}\newline
\ct{PCClass>>instVarNames}\newline
\ct{PCClass>>}\newline\indent\ct{variableByteSubclass:}\newline\indent\ct{instanceVariableNames:}\newline\indent\ct{classVariableNames:}\newline
\ct{PCClass>>name}\newline
\ct{PCClass>>isMeta}\newline
\ct{PCClass>>instVarNames:}\newline
\ct{PCClass>>theNonMetaClass}\newline
\ct{PCClass>>subclass:}\newline\indent\ct{instanceVariableNames:}\newline\indent\ct{classVariableNames:}\newline
\ct{PCClass>>variableSubclass:}\newline\indent\ct{instanceVariableNames:}\newline\indent\ct{classVariableNames:}\newline
\ct{PCClass>>classVariables}\newline
\ct{PCClass>>classVariables:}\newline
\ct{PCForm>>bits}\newline
\ct{PCForm>>fillRectX:y:w:h:}\newline
\ct{PCForm>>setColorR:g:b:}\newline
\ct{PCForm>>primScreenSize}\newline
\ct{PCForm>>beDisplayDepth:}\newline
\ct{PCForm>>depth}\newline
\ct{PCForm>>setWidth:height:depth:}\newline
\ct{PCForm>>height}\newline
\ct{PCForm>>copyX:y:width:height:}\newline
\ct{PCForm>>drawForm:x:y:rule:}\newline
\ct{PCForm>>width}\newline
\ct{PCCompiledMethod>>numLiterals}\newline
\ct{PCCompiledMethod>>objectAt:put:}\newline
\ct{PCCompiledMethod>>header}\newline
\ct{PCCompiledMethod>>objectAt:}\newline
\ct{PCCompiledMethod>>numTemps}\newline
\ct{PCCompiledMethod>>frameSize}\newline
\ct{PCCompiledMethod>>initialPC}\newline
\ct{PCCompiledMethod>>flushCache}\newline
\ct{PCCompiledMethod>>isCompiledMethod}\newline
\ct{PCArray>>hash}\newline
\ct{PCArray>>replaceFrom:to:with:startingAt:}\newline
\ct{PCArray>>asArray}\newline
\ct{PCArray>>asDictionary}\newline
\ct{PCArray>>elementsExchangeIdentityWith:}\newline
\ct{PCArray>>printOn:}\newline
\ct{PCLinkedList>>linkAt:}\newline
\ct{PCLinkedList>>at:putLink:}\newline
\ct{PCLinkedList>>firstLink}\newline
\ct{PCLinkedList>>removeAllSuchThat:}\newline
\ct{PCLinkedList>>copyWith:}\newline
\ct{PCLinkedList>>at:put:}\newline
\ct{PCLinkedList>>first}\newline
\ct{PCLinkedList>>linkOf:ifAbsent:}\newline
\ct{PCLinkedList>>add:after:}\newline
\ct{PCLinkedList>>isEmpty}\newline
\ct{PCLinkedList>>do:}\newline
\ct{PCLinkedList>>indexOf:startingAt:ifAbsent:}\newline
\ct{PCLinkedList>>last}\newline
\ct{PCLinkedList>>removeFirst}\newline
\ct{PCLinkedList>>linksDo:}\newline
\ct{PCLinkedList>>postCopy}\newline
\ct{PCLinkedList>>add:afterLink:}\newline
\ct{PCLinkedList>>removeAll}\newline
\ct{PCLinkedList>>removeLink:}\newline
\ct{PCLinkedList>>linkOf:}\newline
\ct{PCLinkedList>>at:}\newline
\ct{PCLinkedList>>removeLast}\newline
\ct{PCLinkedList>>copyWithout:}\newline
\ct{PCLinkedList>>addLast:}\newline
\ct{PCLinkedList>>add:}\newline
\ct{PCLinkedList>>add:before:}\newline
\ct{PCLinkedList>>add:beforeLink:}\newline
\ct{PCLinkedList>>lastLink}\newline
\ct{PCLinkedList>>linkAt:ifAbsent:}\newline
\ct{PCLinkedList>>collect:}\newline
\ct{PCLinkedList>>addFirst:}\newline
\ct{PCLinkedList>>removeLink:ifAbsent:}\newline
\ct{PCLinkedList>>validIndex:}\newline
\ct{PCLinkedList>>species}\newline
\ct{PCLinkedList>>swap:with:}\newline
\ct{PCLinkedList>>remove:ifAbsent:}\newline
\ct{PCBehavior>>basicNew:}\newline
\ct{PCBehavior>>superclass}\newline
\ct{PCBehavior>>sharedPools}\newline
\ct{PCBehavior>>superclass:}\newline
\ct{PCBehavior>>initialize}\newline
\ct{PCBehavior>>printOn:}\newline
\ct{PCBehavior>>inheritsFrom:}\newline
\ct{PCBehavior>>isCompact}\newline
\ct{PCBehavior>>allInstancesDo:}\newline
\ct{PCBehavior>>name}\newline
\ct{PCBehavior>>someInstance}\newline
\ct{PCBehavior>>format}\newline
\ct{PCBehavior>>allInstances}\newline
\ct{PCBehavior>>isBits}\newline
\ct{PCBehavior>>isBytes}\newline
\ct{PCBehavior>>isPointers}\newline
\ct{PCBehavior>>methodDict}\newline
\ct{PCBehavior>>isVariable}\newline
\ct{PCBehavior>>new}\newline
\ct{PCBehavior>>basicNew}\newline
\ct{PCBehavior>>allInstVarNames}\newline
\ct{PCBehavior>>classPool}\newline
\ct{PCBehavior>>>>}\newline
\ct{PCBehavior>>instSize}\newline
\ct{PCBehavior>>new:}\newline
\ct{PCBehavior>>indexIfCompact}\newline
\ct{PCBehavior>>selectorAtMethod:setClass:}\newline
\ct{PCBehavior>>isBehavior}\newline
\ct{PCBehavior>>lookupSelector:}\newline
\ct{PCBehavior>>canUnderstand:}\newline
\ct{PCBehavior>>setFormat:}\newline
\ct{PCBehavior>>instSpec}\newline
\ct{PCBehavior>>setCompactClassIndex:}\newline
\ct{PCUndefinedObject>>printOn:}\newline
\ct{PCUndefinedObject>>subclass:}\newline\indent\ct{instanceVariableNames:}\newline\indent\ct{classVariableNames:}\newline
\ct{PCUndefinedObject>>isNil}\newline
\ct{PCUndefinedObject>>ifNotNil:}\newline
\ct{PCUndefinedObject>>basicCopy}\newline
\ct{PCUndefinedObject>>ifNil:}\newline
\ct{PCUndefinedObject>>ifNil:ifNotNil:}\newline
\ct{PCLargePositiveInteger>>digitLength}\newline
\ct{PCLargePositiveInteger>>negative}\newline
\ct{PCLargePositiveInteger>>\textbackslash}\newline
\ct{PCLargePositiveInteger>>highBit}\newline
\ct{PCLargePositiveInteger>>negated}\newline
\ct{PCLargePositiveInteger>>/}\newline
\ct{PCLargePositiveInteger>>//}\newline
\ct{PCLargePositiveInteger>>+}\newline
\ct{PCLargePositiveInteger>><=}\newline
\ct{PCLargePositiveInteger>>>}\newline
\ct{PCLargePositiveInteger>>>=}\newline
\ct{PCLargePositiveInteger>>normalize}\newline
\ct{PCLargePositiveInteger>>}\newline\indent\ct{replaceFrom:to:with:startingAt:}\newline
\ct{PCLargePositiveInteger>>digitAt:}\newline
\ct{PCLargePositiveInteger>>abs}\newline
\ct{PCLargePositiveInteger>>bitAnd:}\newline
\ct{PCLargePositiveInteger>>sign}\newline
\ct{PCLargePositiveInteger>>-}\newline
\ct{PCLargePositiveInteger>>*}\newline
\ct{PCLargePositiveInteger>>bitXor:}\newline
\ct{PCLargePositiveInteger>>~=}\newline
\ct{PCLargePositiveInteger>>digitAt:put:}\newline
\ct{PCLargePositiveInteger>>bitOr:}\newline
\ct{PCLargePositiveInteger>>=}\newline
\ct{PCLargePositiveInteger>>bitShift:}\newline
\ct{PCLargePositiveInteger>>quo:}\newline
\ct{PCLargePositiveInteger>><}\newline
\ct{PCSmallInteger>>>}\newline
\ct{PCSmallInteger>><=}\newline
\ct{PCSmallInteger>>\textbackslash\textbackslash}\newline
\ct{PCSmallInteger>>digitAt:}\newline
\ct{PCSmallInteger>>>=}\newline
\ct{PCSmallInteger>>identityHash}\newline
\ct{PCSmallInteger>>printOn:base:}\newline
\ct{PCSmallInteger>>hash}\newline
\ct{PCSmallInteger>>~=}\newline
\ct{PCSmallInteger>>//}\newline
\ct{PCSmallInteger>>=}\newline
\ct{PCSmallInteger>>asFloat}\newline
\ct{PCSmallInteger>>hashMultiply}\newline
\ct{PCSmallInteger>>-}\newline
\ct{PCSmallInteger>>basicIdentityHash}\newline
\ct{PCSmallInteger>>highBit}\newline
\ct{PCSmallInteger>>*}\newline
\ct{PCSmallInteger>>+}\newline
\ct{PCSmallInteger>>basicCopy}\newline
\ct{PCSmallInteger>>bitAnd:}\newline
\ct{PCSmallInteger>>bitOr:}\newline
\ct{PCSmallInteger>>digitAt:put:}\newline
\ct{PCSmallInteger>>isSmallInteger}\newline
\ct{PCSmallInteger>><}\newline
\ct{PCSmallInteger>>bitShift:}\newline
\ct{PCSmallInteger>>quo:}\newline
\ct{PCSmallInteger>>digitLength}\newline
\ct{PCSmallInteger>>/}\newline
\ct{PCSmallInteger>>bitXor:}\newline
\ct{PCInterval>>size}\newline
\ct{PCInterval>>last}\newline
\ct{PCInterval>>at:}\newline
\ct{PCInterval>>at:put:}\newline
\ct{PCInterval>>species}\newline
\ct{PCInterval>>add:}\newline
\ct{PCInterval>>=}\newline
\ct{PCInterval>>first}\newline
\ct{PCInterval>>hash}\newline
\ct{PCInterval>>includes:}\newline
\ct{PCInterval>>setFrom:to:by:}\newline
\ct{PCInterval>>remove:}\newline
\ct{PCInterval>>do:}\newline
\ct{PCInterval>>increment}\newline
\ct{PCInterval>>collect:}\newline
\ct{PCInterval>>printOn:}\newline
\ct{PCMagnitude>>>}\newline
\ct{PCMagnitude>>=}\newline
\ct{PCMagnitude>><=}\newline
\ct{PCMagnitude>>>=}\newline
\ct{PCMagnitude>>between:and:}\newline
\ct{PCMagnitude>>hash}\newline
\ct{PCMagnitude>>min:}\newline
\ct{PCMagnitude>>max:}\newline
\ct{PCMagnitude>><}\newline
\ct{PCSequenceableCollection>>}\newline\indent\ct{indexOf:startingAt:ifAbsent:}\newline
\ct{PCSequenceableCollection>>}\newline\indent\ct{replaceFrom:to:with:}\newline
\ct{PCSequenceableCollection>>}\newline\indent\ct{replaceFrom:to:with:startingAt:}\newline
\ct{PCSequenceableCollection>>last}\newline
\ct{PCSequenceableCollection>>at:ifAbsent:}\newline
\ct{PCSequenceableCollection>>copyWith:}\newline
\ct{PCSequenceableCollection>>}\newline\indent\ct{copyReplaceFrom:to:with:}\newline
\ct{PCSequenceableCollection>>,}\newline
\ct{PCSequenceableCollection>>do:}\newline
\ct{PCSequenceableCollection>>first}\newline
\ct{PCSequenceableCollection>>select:}\newline
\ct{PCSequenceableCollection>>size}\newline
\ct{PCSequenceableCollection>>=}\newline
\ct{PCSequenceableCollection>>copyFrom:to:}\newline
\ct{PCSequenceableCollection>>asArray}\newline
\ct{PCSequenceableCollection>>indexOf:ifAbsent:}\newline
\ct{PCSequenceableCollection>>remove:ifAbsent:}\newline
\ct{PCSequenceableCollection>>swap:with:}\newline
\ct{PCSequenceableCollection>>collect:}\newline
\ct{PCBitBlt>>fillWords:}\newline
\ct{PCBitBlt>>rule:}\newline
\ct{PCBitBlt>>copyBits}\newline
\ct{PCBitBlt>>sourceForm:}\newline
\ct{PCBitBlt>>copyBitsTranslucent:}\newline
\ct{PCBitBlt>>destForm:}\newline
\ct{PCBitBlt>>sourceX:y:}\newline
\ct{PCBitBlt>>destX:y:width:height:}\newline
\ct{PCBitBlt>>fillWords}\newline
\ct{PCBitBlt>>width:height:}\newline
\ct{PCBitBlt>>initialize}\newline
\ct{PCBitBlt>>fillR:g:b:}\newline
\ct{PCBitBlt>>clipX:y:width:height:}\newline
\ct{PCNumber>>adaptToInteger:andSend:}\newline
\ct{PCNumber>>to:by:}\newline
\ct{PCNumber>>to:by:do:}\newline
\ct{PCNumber>>floorLog:}\newline
\ct{PCNumber>>abs}\newline
\ct{PCNumber>>cos}\newline
\ct{PCNumber>>isNumber}\newline
\ct{PCNumber>>arcTan}\newline
\ct{PCNumber>>log}\newline
\ct{PCNumber>>reciprocal}\newline
\ct{PCNumber>>exp}\newline
\ct{PCNumber>>//}\newline
\ct{PCNumber>>/}\newline
\ct{PCNumber>>asInteger}\newline
\ct{PCNumber>>log:}\newline
\ct{PCNumber>>raisedToInteger:}\newline
\ct{PCNumber>>printStringBase:}\newline
\ct{PCNumber>>rem:}\newline
\ct{PCNumber>>roundUpTo:}\newline
\ct{PCNumber>>adaptToFloat:andSend:}\newline
\ct{PCNumber>>rounded}\newline
\ct{PCNumber>>to:do:}\newline
\ct{PCNumber>>arcSin}\newline
\ct{PCNumber>>printOn:}\newline
\ct{PCNumber>>+}\newline
\ct{PCNumber>>\\}\newline
\ct{PCNumber>>ln}\newline
\ct{PCNumber>>degreesToRadians}\newline
\ct{PCNumber>>negated}\newline
\ct{PCNumber>>quo:}\newline
\ct{PCNumber>>radiansToDegrees}\newline
\ct{PCNumber>>tan}\newline
\ct{PCNumber>>-}\newline
\ct{PCNumber>>arcCos}\newline
\ct{PCNumber>>ceiling}\newline
\ct{PCNumber>>negative}\newline
\ct{PCNumber>>to:}\newline
\ct{PCNumber>>roundTo:}\newline
\ct{PCNumber>>truncated}\newline
\ct{PCNumber>>sin}\newline
\ct{PCNumber>>sign}\newline
\ct{PCNumber>>truncateTo:}\newline
\ct{PCNumber>>*}\newline
\ct{PCNumber>>floor}\newline
\ct{PCNumber>>raisedTo:}\newline
\ct{PCNumber>>sqrt}\newline
\ct{PCLargeNegativeInteger>>abs}\newline
\ct{PCLargeNegativeInteger>>negative}\newline
\ct{PCLargeNegativeInteger>>normalize}\newline
\ct{PCLargeNegativeInteger>>sign}\newline
\ct{PCLargeNegativeInteger>>printOn:base:}\newline
\ct{PCLargeNegativeInteger>>negated}\newline
\ct{PCProcess>>nextLink}\newline
\ct{PCProcess>>printOn:}\newline
\ct{PCProcess>>priority}\newline
\ct{PCProcess>>errorHandler:}\newline
\ct{PCProcess>>resume}\newline
\ct{PCProcess>>suspendedContext}\newline
\ct{PCProcess>>suspend}\newline
\ct{PCProcess>>terminate}\newline
\ct{PCProcess>>priority:}\newline
\ct{PCProcess>>nextLink:}\newline
\ct{PCProcess>>initSuspendedContext:}\newline
\ct{PCProcess>>errorHandler}\newline
\ct{PCCharacter>>=}\newline
\ct{PCCharacter>>isLetter}\newline
\ct{PCCharacter>>printOn:}\newline
\ct{PCCharacter>>isSpecial}\newline
\ct{PCCharacter>>isVowel}\newline
\ct{PCCharacter>>to:}\newline
\ct{PCCharacter>>asCharacter}\newline
\ct{PCCharacter>>setValue:}\newline
\ct{PCCharacter>>>}\newline
\ct{PCCharacter>>asLowercase}\newline
\ct{PCCharacter>><}\newline
\ct{PCCharacter>>asInteger}\newline
\ct{PCCharacter>>asString}\newline
\ct{PCCharacter>>asUppercase}\newline
\ct{PCCharacter>>basicCopy}\newline
\ct{PCCharacter>>tokenish}\newline
\ct{PCCharacter>>hash}\newline
\ct{PCCharacter>>digitValue}\newline
\ct{PCCharacter>>asciiValue}\newline
\ct{PCCharacter>>isDigit}\newline
\ct{PCCharacter>>isUppercase}\newline
\ct{PCSet>>fullCheck}\newline
\ct{PCSet>>fixCollisionsFrom:}\newline
\ct{PCSet>>keyAt:}\newline
\ct{PCSet>>copy}\newline
\ct{PCSet>>init:}\newline
\ct{PCSet>>=}\newline
\ct{PCSet>>grow}\newline
\ct{PCSet>>withArray:}\newline
\ct{PCSet>>asArray}\newline
\ct{PCSet>>findElementOrNil:}\newline
\ct{PCSet>>includes:}\newline
\ct{PCSet>>noCheckAdd:}\newline
\ct{PCSet>>asSet}\newline
\ct{PCSet>>remove:ifAbsent:}\newline
\ct{PCSet>>atNewIndex:put:}\newline
\ct{PCSet>>scanFor:}\newline
\ct{PCSet>>add:}\newline
\ct{PCSet>>do:}\newline
\ct{PCSet>>size}\newline
\ct{PCSet>>collect:}\newline
\ct{PCSet>>swap:with:}\newline
\ct{PCReadStream>>peek}\newline
\ct{PCReadStream>>atEnd}\newline
\ct{PCReadStream>>position:}\newline
\ct{PCReadStream>>next}\newline
\ct{PCReadStream>>skip:}\newline
\ct{PCReadStream>>position}\newline
\ct{PCReadStream>>contents}\newline
\ct{PCReadStream>>peekFor:}\newline
\ct{PCReadStream>>size}\newline
\ct{PCReadStream>>on:}\newline
\ct{PCReadStream>>next:}\newline
\ct{PCMetaclass>>theNonMetaClass}\newline
\ct{PCMetaclass>>new}\newline
\ct{PCMetaclass>>initMethodDict:}\newline
\ct{PCMetaclass>>isMeta}\newline
\ct{PCMetaclass>>name}\newline
\ct{PCMetaclass>>soleInstance:}\newline
\ct{PCObject>>ifNil:}\newline
\ct{PCObject>>pointsTo:}\newline
\ct{PCObject>>class}\newline
\ct{PCObject>>ifNotNil:ifNil:}\newline
\ct{PCObject>>isCompiledMethod}\newline
\ct{PCObject>>respondsTo:}\newline
\ct{PCObject>>~~}\newline
\ct{PCObject>>asLink}\newline
\ct{PCObject>>perform:}\newline
\ct{PCObject>>perform:withArguments:inSuperclass:}\newline
\ct{PCObject>>shouldNotImplement}\newline
\ct{PCObject>>instVarAt:}\newline
\ct{PCObject>>someObject}\newline
\ct{PCObject>>printOn:}\newline
\ct{PCObject>>yourself}\newline
\ct{PCObject>>=}\newline
\ct{PCObject>>initialize}\newline
\ct{PCObject>>isBehavior}\newline
\ct{PCObject>>nextInstance}\newline
\ct{PCObject>>become:}\newline
\ct{PCObject>>putAscii:}\newline
\ct{PCObject>>species}\newline
\ct{PCObject>>tryPrimitive:withArgs:}\newline
\ct{PCObject>>basicAt:}\newline
\ct{PCObject>>isNil}\newline
\ct{PCObject>>putcr}\newline
\ct{PCObject>>isNumber}\newline
\ct{PCObject>>errorSubscriptBounds:}\newline
\ct{PCObject>>isContextPart}\newline
\ct{PCObject>>isSelfEvaluating}\newline
\ct{PCObject>>beep}\newline
\ct{PCObject>>nextObject}\newline
\ct{PCObject>>isInteger}\newline
\ct{PCObject>>basicAt:put:}\newline
\ct{PCObject>>handleExceptionName:context:}\newline
\ct{PCObject>>identityHash}\newline
\ct{PCObject>>asString}\newline
\ct{PCObject>>errorNonIntegerIndex}\newline
\ct{PCObject>>perform:withArguments:}\newline
\ct{PCObject>>hash}\newline
\ct{PCObject>>primitiveFailed}\newline
\ct{PCObject>>printString}\newline
\ct{PCObject>>instVarAt:put:}\newline
\ct{PCObject>>~=}\newline
\ct{PCObject>>at:put:}\newline
\ct{PCObject>>shouldBePrintedAsLiteral}\newline
\ct{PCObject>>isSmallInteger}\newline
\ct{PCObject>>error:}\newline
\ct{PCObject>>basicSize}\newline
\ct{PCObject>>subclassResponsibility}\newline
\ct{PCObject>>->}\newline
\ct{PCObject>>at:}\newline
\ct{PCObject>>ifNotNil:}\newline
\ct{PCObject>>basicCopy}\newline
\ct{PCObject>>basicIdentityHash}\newline
\ct{PCObject>>copy}\newline
\ct{PCObject>>doesNotUnderstand:}\newline
\ct{PCObject>>ifNil:ifNotNil:}\newline
\ct{PCObject>>perform:with:}\newline
\ct{PCObject>>==}\newline
\ct{PCObject>>mustBeBoolean}\newline
\ct{PCObject>>putString:}\newline
\ct{PCObject>>isKindOf:}\newline
\ct{PCObject>>errorImproperStore}\newline
\ct{PCByteArray>>asByteArray}\newline
\ct{PCByteArray>>}\newline\indent\ct{replaceFrom:to:with:startingAt:}\newline
\ct{PCByteArray>>asString}\newline
\ct{PCDictionary>>includes:}\newline
\ct{PCDictionary>>remove:ifAbsent:}\newline
\ct{PCDictionary>>associationAt:}\newline
\ct{PCDictionary>>remove:}\newline
\ct{PCDictionary>>removeKey:}\newline
\ct{PCDictionary>>keysDo:}\newline
\ct{PCDictionary>>collect:}\newline
\ct{PCDictionary>>associationAt:ifAbsent:}\newline
\ct{PCDictionary>>keyAtValue:ifAbsent:}\newline
\ct{PCDictionary>>noCheckAdd:}\newline
\ct{PCDictionary>>printOn:}\newline
\ct{PCDictionary>>scanFor:}\newline
\ct{PCDictionary>>errorKeyNotFound}\newline
\ct{PCDictionary>>at:}\newline
\ct{PCDictionary>>at:ifAbsent:}\newline
\ct{PCDictionary>>errorValueNotFound}\newline
\ct{PCDictionary>>keyAt:}\newline
\ct{PCDictionary>>keyAtValue:}\newline
\ct{PCDictionary>>keys}\newline
\ct{PCDictionary>>associationsDo:}\newline
\ct{PCDictionary>>copy}\newline
\ct{PCDictionary>>removeKey:ifAbsent:}\newline
\ct{PCDictionary>>add:}\newline
\ct{PCDictionary>>at:put:}\newline
\ct{PCDictionary>>do:}\newline
\ct{PCDictionary>>includesKey:}\newline
\ct{PCDictionary>>select:}\newline
\ct{PCSemaphore>>=}\newline
\ct{PCSemaphore>>signal}\newline
\ct{PCSemaphore>>initialize}\newline
\ct{PCSemaphore>>critical:}\newline
\ct{PCSemaphore>>hash}\newline
\ct{PCSemaphore>>wait}\newline
\ct{PCValueLink>>value:}\newline
\ct{PCValueLink>>nextLink}\newline
\ct{PCValueLink>>nextLink:}\newline
\ct{PCValueLink>>value}\newline
\ct{PCValueLink>>=}\newline
\ct{PCValueLink>>asLink}\newline
\ct{PCValueLink>>hash}\newline
\ct{PCValueLink>>printOn:}\newline
\ct{PCMethodDictionary>>associationsDo:}\newline
\ct{PCMethodDictionary>>at:put:}\newline
\ct{PCMethodDictionary>>keysDo:}\newline
\ct{PCMethodDictionary>>keyAt:}\newline
\ct{PCMethodDictionary>>do:}\newline
\ct{PCMethodDictionary>>grow}\newline
\ct{PCMethodDictionary>>at:ifAbsent:}\newline
\ct{PCMethodDictionary>>includesKey:}\newline
\ct{PCMethodDictionary>>removeKey:ifAbsent:}\newline
\ct{PCMethodDictionary>>}\newline\indent\ct{keyAtIdentityValue:ifAbsent:}\newline
\ct{PCMethodDictionary>>add:}\newline
\ct{PCMethodDictionary>>copy}\newline
\ct{PCMethodDictionary>>scanFor:}\newline
\ct{PCMethodDictionary>>swap:with:}\newline
\ct{PCBlock>>value:}\newline
\ct{PCBlock>>value:value:}\newline
\ct{PCBlock>>asContext}\newline
\ct{PCBlock>>outerContext}\newline
\ct{PCBlock>>asContextWithSender:}\newline
\ct{PCBlock>>valueWithArguments:}\newline
\ct{PCBlock>>home}\newline
\ct{PCBlock>>ifError:}\newline
\ct{PCBlock>>method}\newline
\ct{PCBlock>>msecs}\newline
\ct{PCBlock>>numArgs}\newline
\ct{PCBlock>>numCopiedValues}\newline
\ct{PCBlock>>value}\newline
\ct{PCContext>>blockCopy:}\newline
\ct{PCContext>>sender}\newline
\ct{PCContext>>isContextPart}\newline
\ct{PCProcessorScheduler>>initProcessLists}\newline
\ct{PCProcessorScheduler>>installStartProcess}\newline
\ct{PCProcessorScheduler>>remove:ifAbsent:}\newline
\ct{PCProcessorScheduler>>installIdleProcess}\newline
\ct{PCProcessorScheduler>>activeProcess}\newline
\ct{PCProcessorScheduler>>highestPriority}\newline
\ct{PCProcessorScheduler>>idleProcess}\newline
\ct{PCCollection>>add:}\newline
\ct{PCCollection>>asByteArray}\newline
\ct{PCCollection>>do:}\newline
\ct{PCCollection>>errorNotFound}\newline
\ct{PCCollection>>collect:}\newline
\ct{PCCollection>>includes:}\newline
\ct{PCCollection>>emptyCheck}\newline
\ct{PCCollection>>asArray}\newline
\ct{PCCollection>>printOn:}\newline
\ct{PCCollection>>select:}\newline
\ct{PCCollection>>detect:ifNone:}\newline
\ct{PCCollection>>sum}\newline
\ct{PCCollection>>isEmpty}\newline
\ct{PCCollection>>asSet}\newline
\ct{PCCollection>>remove:}\newline
\ct{PCCollection>>remove:ifAbsent:}\newline
\ct{PCCollection>>size}\newline
\ct{PCCollection>>errorEmptyCollection}\newline
\ct{PCArrayedCollection>>sort:}\newline
\ct{PCArrayedCollection>>size}\newline
\ct{PCArrayedCollection>>sort}\newline
\ct{PCArrayedCollection>>mergeSortFrom:to:by:}\newline
\ct{PCArrayedCollection>>add:}\newline
\ct{PCArrayedCollection>>}\newline\indent\ct{mergeFirst:middle:last:into:by:}\newline
\ct{PCArrayedCollection>>}\newline\indent\ct{mergeSortFrom:to:src:dst:by:}\newline
\ct{PCWriteStream>>on:}\newline
\ct{PCWriteStream>>pastEndPut:}\newline
\ct{PCWriteStream>>contents}\newline
\ct{PCWriteStream>>nextPut:}\newline
\ct{PCWriteStream>>position:}\newline
\ct{PCWriteStream>>size}\newline
\ct{PCWriteStream>>space}\newline
\ct{PCWriteStream>>nextPutAll:}\newline
\ct{PCString>>>}\newline
\ct{PCString>>}\newline\indent\ct{findString:startingAt:caseSensitive:}\newline
\ct{PCString>>substrings}\newline
\ct{PCString>>size}\newline
\ct{PCString>><}\newline
\ct{PCString>>asSymbol}\newline
\ct{PCString>>}\newline\indent\ct{findSubstring:in:startingAt:matchTable:}\newline
\ct{PCString>>asByteArray}\newline
\ct{PCString>>compare:with:collated:}\newline
\ct{PCString>>findDelimiters:startingAt:}\newline
\ct{PCString>>hash}\newline
\ct{PCString>>indexOfAscii:inString:startingAt:}\newline
\ct{PCString>>findTokens:}\newline
\ct{PCString>>numArgs}\newline
\ct{PCString>><=}\newline
\ct{PCString>>replaceFrom:to:with:startingAt:}\newline
\ct{PCString>>>=}\newline
\ct{PCString>>at:}\newline
\ct{PCString>>indexOf:startingAt:}\newline
\ct{PCString>>printOn:}\newline
\ct{PCString>>at:put:}\newline
\ct{PCString>>translate:from:to:table:}\newline
\ct{PCString>>asLowercase}\newline
\ct{PCString>>asString}\newline
\ct{PCString>>=}\newline
\ct{PCString>>skipDelimiters:startingAt:}\newline
\ct{PCString>>compare:}\newline
\ct{PCString>>indexOf:startingAt:ifAbsent:}\newline
\ct{PCClassBuilder>>isVariable}\newline
\ct{PCClassBuilder>>initialize}\newline
\ct{PCClassBuilder>>instSize}\newline
\ct{PCClassBuilder>>beBytes}\newline
\ct{PCClassBuilder>>name:}\newline
\ct{PCClassBuilder>>classVariableNames:}\newline
\ct{PCClassBuilder>>compactClassIndex}\newline
\ct{PCClassBuilder>>beCompiledMethod}\newline
\ct{PCClassBuilder>>isPointers}\newline
\ct{PCClassBuilder>>isWords}\newline
\ct{PCClassBuilder>>}\newline\indent\ct{isCompiledMethodClassIndex}\newline
\ct{PCClassBuilder>>beVariable}\newline
\ct{PCClassBuilder>>instVarNames:}\newline
\ct{PCClassBuilder>>isCompiledMethod}\newline
\ct{PCClassBuilder>>beWords}\newline
\ct{PCClassBuilder>>isWeak}\newline
\ct{PCClassBuilder>>newClassFormat}\newline
\ct{PCClassBuilder>>bePointers}\newline
\ct{PCClassBuilder>>build}\newline
\ct{PCClassBuilder>>superclass:}\newline
\ct{PCClassBuilder>>beWeak}\newline
\ct{PCClassBuilder>>compactClassIndexFor:}\newline
\ct{PCClassBuilder>>classVariableNames}\newline
\ct{PCSymbol>>flushCache}\newline
\ct{PCSymbol>>printOn:}\newline
\ct{PCSymbol>>initFrom:}\newline
\ct{PCSymbol>>species}\newline
\ct{PCSymbol>>=}\newline
\ct{PCSymbol>>asString}\newline
\ct{PCSymbol>>asSymbol}\newline
\ct{PCSymbol>>hash}\newline
\ct{PCSymbol>>errorNoModification}\newline
\ct{PCSymbol>>replaceFrom:to:with:startingAt:}\newline
\ct{PCSymbol>>at:put:}\newline
\ct{PCSymbol>>basicCopy}\newline
\ct{PCInteger>>bitOr:}\newline
\ct{PCInteger>>=}\newline
\ct{PCInteger>>floor}\newline
\ct{PCInteger>>asCharacter}\newline
\ct{PCInteger>>//}\newline
\ct{PCInteger>>lastDigit}\newline
\ct{PCInteger>>printOn:base:}\newline
\ct{PCInteger>><}\newline
\ct{PCInteger>>truncated}\newline
\ct{PCInteger>>digitDiv:neg:}\newline
\ct{PCInteger>>bitInvert}\newline
\ct{PCInteger>>copyto:}\newline
\ct{PCInteger>>benchFib}\newline
\ct{PCInteger>>rounded}\newline
\ct{PCInteger>>*}\newline
\ct{PCInteger>>benchmark}\newline
\ct{PCInteger>>asInteger}\newline
\ct{PCInteger>>bitClear:}\newline
\ct{PCInteger>>bitShift:}\newline
\ct{PCInteger>>ceiling}\newline
\ct{PCInteger>>digitAdd:}\newline
\ct{PCInteger>>bitXor:}\newline
\ct{PCInteger>>isInteger}\newline
\ct{PCInteger>>growby:}\newline
\ct{PCInteger>>bitAnd:}\newline
\ct{PCInteger>>/}\newline
\ct{PCInteger>>quo:}\newline
\ct{PCInteger>>+}\newline
\ct{PCInteger>>-}\newline
\ct{PCInteger>>digitRshift:bytes:lookfirst:}\newline
\ct{PCInteger>>hash}\newline
\ct{PCInteger>>digitCompare:}\newline
\ct{PCInteger>>digitLogic:op:length:}\newline
\ct{PCInteger>>digitMultiply:neg:}\newline
\ct{PCInteger>>replaceFrom:to:with:startingAt:}\newline
\ct{PCInteger>>>}\newline
\ct{PCInteger>>asFloat}\newline
\ct{PCInteger>>digitSubtract:}\newline
\ct{PCInteger>>timesRepeat:}\newline
\ct{PCInteger>>normalize}\newline
\ct{PCInteger>>growto:}\newline
\ct{PCInteger>>digitLshift:}\newline
\ct{PCProcessList>>first}\newline
\ct{PCProcessList>>remove:ifAbsent:}\newline
\ct{PCProcessList>>removeFirst}\newline
\ct{PCProcessList>>add:}\newline
\ct{PCProcessList>>do:}\newline
\ct{PCProcessList>>isEmpty}\newline
\ct{PCProcessList>>addLast:}\newline
\ct{PCProcessList>>size}\newline
\ct{PCFalse>>\&}\newline
\ct{PCFalse>>not}\newline
\ct{PCFalse>>or:}\newline
\ct{PCFalse>>printOn:}\newline
\ct{PCFalse>>and:}\newline
\ct{PCFalse>>ifTrue:}\newline
\ct{PCFalse>>|}\newline
\ct{PCFalse>>ifFalse:}\newline
\ct{PCFalse>>ifTrue:ifFalse:}\newline
\ct{PCTrue>>or:}\newline
\ct{PCTrue>>ifTrue:}\newline
\ct{PCTrue>>\&}\newline
\ct{PCTrue>>basicCopy}\newline
\ct{PCTrue>>ifTrue:ifFalse:}\newline
\ct{PCTrue>>not}\newline
\ct{PCTrue>>ifFalse:}\newline
\ct{PCTrue>>printOn:}\newline
\ct{PCTrue>>and:}\newline
\ct{PCTrue>>|}\newline
\ct{PCMethodContext>>privRefresh}\newline
\ct{PCMethodContext>>stackp:}\newline
\ct{PCMethodContext>>removeSelf}\newline
\ct{PCMethodContext>>}\newline\indent\ct{setSender:receiver:method:closure:startpc:}\newline
\ct{PCMethodContext>>tempAt:put:}\newline
\ct{PCMethodContext>>asContext}\newline
\ct{PCMethodContext>>method}\newline
\ct{PCMethodContext>>home}\newline
\ct{PCPoint>>abs}\newline
\ct{PCPoint>>degrees}\newline
\ct{PCPoint>>hash}\newline
\ct{PCPoint>>min:}\newline
\ct{PCPoint>>*}\newline
\ct{PCPoint>>-}\newline
\ct{PCPoint>>r}\newline
\ct{PCPoint>>rounded}\newline
\ct{PCPoint>>printOn:}\newline
\ct{PCPoint>>dist:}\newline
\ct{PCPoint>>adaptToFloat:andSend:}\newline
\ct{PCPoint>>theta}\newline
\ct{PCPoint>>asPoint}\newline
\ct{PCPoint>>crossProduct:}\newline
\ct{PCPoint>>max:}\newline
\ct{PCPoint>>y}\newline
\ct{PCPoint>>dotProduct:}\newline
\ct{PCPoint>>/}\newline
\ct{PCPoint>>setX:setY:}\newline
\ct{PCPoint>>+}\newline
\ct{PCPoint>>adaptToInteger:andSend:}\newline
\ct{PCPoint>>truncated}\newline
\ct{PCPoint>>=}\newline
\ct{PCPoint>>setR:degrees:}\newline
\ct{PCPoint>>//}\newline
\ct{PCPoint>>x}\newline
\ct{PCPoint>>negated}\newline
\end{multicols}

\section{MetaTalk Bootstrap Extract}

This section lists the methods from MetaTalk for its bootstrapping. This list includes mirrors and base-level classes.

\begin{multicols}{2}\noindent\small
\ct{ObjectMirror>>initialize}\newline
\ct{ObjectMirror>>classMirror}\newline
\ct{ObjectMirror>>setClassMirror:}\newline
\ct{ObjectMirror>>baseObject}\newline
\ct{ObjectMirror>>setBaseObject:}\newline
\ct{ClassMirror>>atMethod:put:}\newline
\ct{ClassMirror>>baseObject}\newline
\ct{ClassMirror>>super}\newline
\ct{ClassMirror>>allocate}\newline
\ct{ClassMirror>>subClass:instanceVariableNames:}\newline
\ct{ClassMirror>>setClassMirror:}\newline
\ct{ClassMirror>>initialize}\newline
\ct{ClassMirror>>name:}\newline
\ct{ClassMirror>>methods}\newline
\ct{ClassMirror>>classMirror}\newline
\ct{ClassMirror>>newWithBaseObject:}\newline
\ct{ClassMirror>>super:}\newline
\ct{ClassMirror>>instanceVariables:}\newline
\ct{ClassMirror>>methodSourceOf:}\newline
\ct{ClassMirror>>isClass}\newline
\ct{ClassMirror>>indexOf:}\newline
\ct{ClassMirror>>methodAt:}\newline
\ct{ClassMirror>>atMethod:compile:}\newline
\ct{ClassMirror>>instanceVariables}\newline
\ct{ClassMirror>>name}\newline
\ct{ClassMirror>>setBaseObject:}\newline
\ct{File>>nextPutAll:}\newline
\ct{File>>primOpen:writable:}\newline
\ct{File>>position:}\newline
\ct{File>>primWrite:from:startingAt:count:}\newline
\ct{File>>openReadWrite:}\newline
\ct{File>>cr}\newline
\ct{File>>primClose:}\newline
\ct{File>>primSetPosition:to:}\newline
\ct{File>>close}\newline
\ct{File>>size}\newline
\ct{File>>primSize:}\newline
\ct{Char>>asString}\newline
\ct{Char>>asCharacter}\newline
\ct{Point3D>>initialize}\newline
\ct{Point3D>>z:}\newline
\ct{Point3D>>z}\newline
\ct{Point3D>>asString}\newline
\ct{True>>asString}\newline
\ct{True>>not}\newline
\ct{True>>ifTrue:ifFalse:}\newline
\ct{True>>|}\newline
\ct{True>>ifTrue:}\newline
\ct{True>>ifFalse:}\newline
\ct{True>>\&}\newline
\ct{True>>initialize}\newline
\ct{Number>>negated}\newline
\ct{Number>>\/\/}\newline
\ct{Number>>+}\newline
\ct{Number>>\-}\newline
\ct{Number>>asNumber}\newline
\ct{Number>><}\newline
\ct{Number>>\/}\newline
\ct{Number>>decimalDigitLength}\newline
\ct{Number>>\*}\newline
\ct{Number>>\=}\newline
\ct{Number>>to:}\newline
\ct{Number>>asString}\newline
\ct{Number>>>}\newline
\ct{Base>>reflect:}\newline
\ct{Class>>new}\newline
\ct{Class>>new:}\newline
\ct{Class>>allocate}\newline
\ct{Class>>allocate:}\newline
\ct{Block>>value}\newline
\ct{Block>>valueWithArgs:}\newline
\ct{Block>>whileTrue:}\newline
\ct{UndefinedObject>>asString}\newline
\ct{Point>>asString}\newline
\ct{Point>>initialize}\newline
\ct{Point>>y:}\newline
\ct{Point>>x:}\newline
\ct{Point>>x}\newline
\ct{Point>>y}\newline
\ct{False>>ifTrue:}\newline
\ct{False>>ifTrue:ifFalse:}\newline
\ct{False>>asString}\newline
\ct{False>>ifFalse:}\newline
\ct{False>>|}\newline
\ct{False>>\&}\newline
\ct{False>>not}\newline
\ct{False>>initialize}\newline
\ct{Array>>asString}\newline
\ct{Array>>size}\newline
\ct{Array>>indexAndValuesDo:}\newline
\ct{Array>>at:put:}\newline
\ct{Array>>do:}\newline
\ct{Array>>indexOf:}\newline
\ct{Array>>at:}\newline
\ct{AbstractMirror>>baseObject}\newline
\ct{AbstractMirror>>variables}\newline
\ct{AbstractMirror>>at:put:}\newline
\ct{AbstractMirror>>perform:withArguments:}\newline
\ct{AbstractMirror>>setBaseObject:}\newline
\ct{AbstractMirror>>obj:perform:withArguments:}\newline
\ct{AbstractMirror>>obj:atIndex:put:}\newline
\ct{AbstractMirror>>asString}\newline
\ct{AbstractMirror>>isClass}\newline
\ct{AbstractMirror>>classMirror}\newline
\ct{AbstractMirror>>at:}\newline
\ct{AbstractMirror>>baseObject:}\newline
\ct{AbstractMirror>>obj:atIndex:}\newline
\ct{AbstractMirror>>}\newline
\indent\ct{obj:perform:withArguments:lookupClass:}\newline
\ct{AbstractMirror>>setDirectClass:on:}\newline
\ct{AbstractMirror>>baseClass:}\newline
\ct{AbstractMirror>>perform:}\newline
\ct{AbstractMirror>>directClassOn:}\newline
\ct{AbstractMirror>>setClassMirror:}\newline
\ct{MirrorFactory>>setDirectMirror:on:}\newline
\ct{MirrorFactory>>newOn:}\newline
\ct{MirrorFactory>>directClassOn:}\newline
\ct{MirrorFactory>>on:}\newline
\ct{MirrorFactory>>initialize}\newline
\ct{MirrorFactory>>directMirrorOn:}\newline
\ct{MirrorFactory>>setDirectClass:on:}\newline
\ct{Dict>>values}\newline
\ct{Dict>>at:}\newline
\ct{Dict>>keysAndValuesDo:}\newline
\ct{Dict>>remove:}\newline
\ct{Dict>>add:}\newline
\ct{Dict>>keyOf:}\newline
\ct{Dict>>atKey:}\newline
\ct{Dict>>at:put:}\newline
\ct{Dict>>keys}\newline
\ct{Object>>==}\newline
\ct{Object>>print}\newline
\ct{Object>>=}\newline
\ct{Object>>yourself}\newline
\ct{Object>>asList}\newline
\ct{Object>>quit}\newline
\ct{Object>>copy}\newline
\ct{Object>>size}\newline
\ct{Object>>initialize}\newline
\ct{Object>>asString}\newline
\ct{List>>at:}\newline
\ct{List>>tail}\newline
\ct{List>>=}\newline
\ct{List>>at:put:}\newline
\ct{List>>size}\newline
\ct{List>>,}\newline
\ct{List>>isEmpty}\newline
\ct{List>>remove:}\newline
\ct{List>>do:}\newline
\ct{List>>indexAndValuesDo:}\newline
\ct{List>>asList}\newline
\ct{List>>indexOf:}\newline
\ct{List>>head}\newline
\ct{List>>asString}\newline
\ct{List>>basicAddAll:}\newline
\ct{List>>initialize}\newline
\ct{List>>add:}\newline
\ct{String>>,}\newline
\ct{String>>byteAt:put:}\newline
\ct{String>>copyReplaceFrom:to:with:}\newline
\ct{String>>replaceFrom:to:with:startingAt:}\newline
\ct{String>>print}\newline
\ct{String>>size}\newline
\ct{String>>at:}\newline
\ct{String>>asString}\newline
\ct{String>>at:put:}\newline
\ct{String>>==}\newline
\end{multicols}

\chapter{Tornado Result Extracts}
\label{appendixtornado}

This appendix lits the entry points and resulting code units we obtained from three different tailoring cases.

\section{I/O App Extract} \label{app:extraction_helloworld}

This section lists the methods extracted from a nurtured Hello World application using I/O. This case was tailored using an empty seed. The used entry point is the following:

\begin{code}
FileStream startUp: true.
1 to: 10 do: [ :i | FileStream stdout nextPutAll: 'hello'; crlf ].
\end{code}

This list includes all methods installed from the Pharo base libraries and the simple Hello World application.
\begin{multicols}{2}\noindent\small
\ct{Array class>>new:}\newline
\ct{ArrayedCollection>>size}\newline
\ct{Association class>>key:value:}\newline
\ct{Association>>value:}\newline
\ct{Association>>value}\newline
\ct{BlockClosure>>on:do:}\newline
\ct{BlockClosure>>repeat}\newline
\ct{BlockClosure>>valueNoContextSwitch}\newline
\ct{ByteString class>>compare:with:collated:}\newline
\ct{ByteString class>>}\newline
\indent\ct{findFirstInString:inSet:startingAt:}\newline
\ct{ByteString class>>stringHash:initialHash:}\newline
\ct{ByteString>>at:put:}\newline
\ct{ByteString>>at:}\newline
\ct{ByteString>>isByteString}\newline
\ct{ByteString>>replaceFrom:to:with:startingAt:}\newline
\ct{ByteTextConverter class>>}\newline
\indent\ct{unicodeToByteTable}\newline
\ct{ByteTextConverter>>nextPut:toStream:}\newline
\ct{ByteTextConverter>>unicodeToByte:}\newline
\ct{Character class>>cr}\newline
\ct{Character class>>lf}\newline
\ct{Character class>>value:}\newline
\ct{Character>>=}\newline
\ct{Character>>asInteger}\newline
\ct{Character>>asciiValue}\newline
\ct{Character>>charCode}\newline
\ct{Collection>>detect:ifNone:}\newline
\ct{Dictionary>>at:ifAbsent:}\newline
\ct{Dictionary>>at:ifPresent:}\newline
\ct{Dictionary>>at:put:}\newline
\ct{Dictionary>>noCheckAdd:}\newline
\ct{Dictionary>>scanFor:}\newline
\ct{FileStream class>>newForStdio}\newline
\ct{FileStream class>>new}\newline
\ct{FileStream class>>standardIOStreamNamed:forWrite:}\newline
\ct{FileStream class>>startUp:}\newline
\ct{FileStream class>>stdioHandles}\newline
\ct{FileStream class>>stdoutCharacter>>isCharacter}\newline
\ct{FileStream class>>voidStdioFiles}\newline
\ct{FileStream>>collectionSpeciesStandard}\newline
\ct{FileStream>>enableReadBufferingSmalltalkImage>>vm}\newline
\ct{FileStream>>isBinaryStandard}\newline
\ct{FileStream>>next:putAll:startingAt:Standard}\newline
\ct{FileStream>>nextPut:Standard}\newline
\ct{FileStream>>openOnHandle:name:forWrite:Standard}\newline
\ct{FileStream>>primWrite:from:startingAt:count:Standard}\newline
\ct{GreekEnvironment class>>supportedLanguages}\newline
\ct{HashTableSizes class>>atLeast:}\newline
\ct{HashTableSizes class>>sizes}\newline
\ct{HashedCollection class>>newProto}\newline
\ct{HashedCollection>>atNewIndex:put:}\newline
\ct{HashedCollection>>findElementOrNil:}\newline
\ct{HashedCollection>>fullCheck}\newline
\ct{HashedCollection>>grow}\newline
\ct{HashedCollection>>initialize:}\newline
\ct{Integer>>asCharacter}\newline
\ct{JapaneseEnvironment class>>}\newline
\indent\ct{supportedLanguages}\newline
\ct{KoreanEnvironment class>>}\newline
\indent\ct{supportedLanguages}\newline
\ct{LanguageEnvironment class>>}\newline
\indent\ct{currentPlatform}\newline
\ct{LanguageEnvironment class>>}\newline
\indent\ct{defaultSystemConverter}\newline
\ct{LanguageEnvironment class>>}\newline
\indent\ct{initKnownEnvironments}\newline
\ct{LanguageEnvironment class>>}\newline
\noindent\ct{knownEnvironments}\newline
\ct{LanguageEnvironment class>>}\newline
\indent\ct{localeID:}\newline
\ct{LanguageEnvironment>>localeID:}\newline
\ct{Latin1Environment class>>}\newline
\indent\ct{supportedLanguages}\newline
\ct{Latin2Environment class>>}\newline
\indent\ct{supportedLanguages}\newline
\ct{Latin9Environment class>>}\newline
\indent\ct{supportedLanguages}\newline
\ct{Latin9Environment class>>}\newline
\indent\ct{systemConverterClass}\newline
\ct{Locale class>>currentPlatform}\newline
\ct{Locale class>>determineCurrentLocale}\newline
\ct{Locale>>determineLocaleID}\newline
\ct{Locale>>determineLocale}\newline
\ct{Locale>>fetchISO2Language}\newline
\ct{Locale>>languageEnvironment}\newline
\ct{Locale>>localeID:}\newline
\ct{Locale>>localeID}\newline
\ct{Locale>>primCountry}\newline
\ct{Locale>>primLanguage}\newline
\ct{LocaleID class>>isoLanguage:isoCountry:}\newline
\ct{LocaleID class>>isoLanguage:}\newline
\ct{LocaleID class>>isoString:}\newline
\ct{LocaleID>>=}\newline
\ct{LocaleID>>hash}\newline
\ct{LocaleID>>isoCountry}\newline
\ct{LocaleID>>isoLanguage:isoCountry:}\newline
\ct{LocaleID>>isoLanguage}\newline
\ct{LookupKey class>>key:}\newline
\ct{LookupKey>>key:}\newline
\ct{LookupKey>>key}\newline
\ct{Magnitude>>max:}\newline
\ct{MultiByteFileStream>>}\newline
\indent\ct{basicNext:putAll:startingAt:}\newline
\ct{MultiByteFileStream>>basicNextPut:}\newline
\ct{MultiByteFileStream>>converter:}\newline
\ct{MultiByteFileStream>>converter}\newline
\ct{MultiByteFileStream>>}\newline
\indent\ct{installLineEndConventionInConverter}\newline
\ct{MultiByteFileStream>>nextPut:}\newline
\ct{MultiByteFileStream>>nextPutAll:}\newline
\ct{Number>>negative}\newline
\ct{OSPlatform class>>isWin32}\newline
\ct{OSPlatform class>>platformName}\newline
\ct{Object>>=}\newline
\ct{Object>>at:put:Object>>isCharacter}\newline
\ct{Object>>at:}\newline
\ct{Object>>class}\newline
\ct{Object>>hash}\newline
\ct{Object>>isInteger}\newline
\ct{Object>>species}\newline
\ct{Object>>\textasciitilde\textasciitilde}\newline
\ct{OrderedCollection class>>arrayType}\newline
\ct{OrderedCollection class>>new:}\newline
\ct{OrderedCollection class>>new}\newline
\ct{OrderedCollection>>add:}\newline
\ct{OrderedCollection>>addLast:}\newline
\ct{OrderedCollection>>at:}\newline
\ct{OrderedCollection>>ensureBoundsFrom:to:}\newline
\ct{OrderedCollection>>resetTo:}\newline
\ct{OrderedCollection>>reset}\newline
\ct{OrderedCollection>>setCollection:}\newline
\ct{OrderedCollection>>size}\newline
\ct{PositionableStream class>>on:}\newline
\ct{PositionableStream>>isBinary}\newline
\ct{PositionableStream>>on:}\newline
\ct{ProtoObject>>basicIdentityHash}\newline
\ct{ProtoObject>>flag:}\newline
\ct{ProtoObject>>identityHash}\newline
\ct{ProtoObject>>initialize}\newline
\ct{RussianEnvironment class>>supportedLanguages}\newline
\ct{Semaphore>>critical:}\newline
\ct{SequenceableCollection>>copyFrom:to:}\newline
\ct{SequenceableCollection>>copyUpTo:}\newline
\ct{SequenceableCollection>>do:}\newline
\ct{SequenceableCollection>>first:}\newline
\ct{SequenceableCollection>>first}\newline
\ct{SequenceableCollection>>identityIndexOf:ifAbsent:}\newline
\ct{SequenceableCollection>>identityIndexOf:}\newline
\ct{SequenceableCollection>>indexOf:ifAbsent:}\newline
\ct{SequenceableCollection>>second}\newline
\ct{SequenceableCollection>>writeStream}\newline
\ct{SimplifiedChineseEnvironment class>>supportedLanguages}\newline
\ct{SmallInteger>>bitXor:}\newline
\ct{SmallInteger>>hash}\newline
\ct{Stream>>basicNextPut:}\newline
\ct{String class>>new:}\newline
\ct{String class>>with:}\newline
\ct{String>>=}\newline
\ct{String>>compare:with:collated:}\newline
\ct{String>>findDelimiters:startingAt:}\newline
\ct{String>>findTokens:}\newline
\ct{String>>hash}\newline
\ct{String>>indexOf:startingAt:ifAbsent:}\newline
\ct{String>>isString}\newline
\ct{String>>skipDelimiters:startingAt:}\newline
\ct{TextConverter class>>defaultSystemConverter}\newline
\ct{TextConverter class>>initializeLatin1MapAndEncodings}\newline
\ct{TextConverter class>>latin1Encodings}\newline
\ct{TextConverter class>>latin1Map}\newline
\ct{TextConverter>>initialize}\newline
\ct{TextConverter>>installLineEndConvention:}\newline
\ct{TextConverter>>nextPutAll:toStream:}\newline
\ct{TextConverter>>nextPutByteString:toStream:}\newline
\ct{VirtualMachine class>>getSystemAttribute:}\newline
\ct{WriteStream>>contentsStandard}\newline
\ct{WriteStream>>crlf}\newline
\ct{WriteStream>>nextPut:}\newline
\ct{WriteStream>>on:}\newline
\ct{WriteStream>>reset}\newline
\end{multicols}

\section{Seaside Web Application Entry Points} \label{app:extraction_seaside_entry_points}

This section lists the entry points as used to tailor the Seaside web application with a full Pharo seed and an empty seed. The following code snippet corresponds with a full Pharo seed. It only consists in starting the web server as the base libraries are initialized and available in the seed.

\begin{code}
	ZnZincServerAdaptor startOn: 8888.
\end{code}

The following code snippet corresponds with an empty seed.  This entry point includes also the initialization of the minimal runtime needed to do networking.

\begin{code}
	"We initialize some classes of the system"
	SmalltalkImage initializeForTornado.
	Symbol initializeForTornado.
	Object initialize.
	ExternalSemaphoreTable initialize.
	Socket initialize.
	Delay initialize.
	Delay startUp: true.
	Delay shutDown: true.
	OSPlatform initialize.
	DiskStore initialize.
	FileStream initialize.
	NetNameResolver initialize.
	DateAndTime initialize.
	ProcessorScheduler initialize.
	WeakFinalizationList initialize.
	UUIDGenerator initialize.
	WeakArray initialize.
	GRPharoRandomProvider initialize.
	WASlime initialize.
	UIManager basicDefault: DummyUIManager new.
	ZnServer initialize.
	WAServerManager initialize.
	Smalltalk instVarNamed: 'session' put: Smalltalk newSessionObject.
	Smalltalk startupImage: true snapshotWorked: true.
	
	"Finally we start the web server"
	ZnZincServerAdaptor startOn: 8888.
\end{code}

\section{Seaside App A Extract} \label{app:extraction_seaside_empty_seed}

This section lists the methods extracted from the nurtured Web application when using an empty seed. This list includes all methods installed from Seaside framework, the Counter application and the base library of Pharo.

\begin{multicols}{2}\noindent\small
\ct{Array class>>new:}\newline
\ct{Array>>isSelfEvaluating}\newline
\ct{Array>>printOn:}\newline
\ct{Array>>replaceFrom:to:with:startingAt:}\newline
\ct{Array>>shouldBePrintedAsLiteral}\newline
\ct{ArrayedCollection class>>new:withAll:}\newline
\ct{ArrayedCollection class>>new}\newline
\ct{ArrayedCollection class>>with:with:with:}\newline
\ct{ArrayedCollection class>>with:with:}\newline
\ct{ArrayedCollection class>>with:}\newline
\ct{ArrayedCollection>>mergeSortFrom:to:by:}\newline
\ct{ArrayedCollection>>size}\newline
\ct{ArrayedCollection>>sort:}\newline
\ct{Association class>>key:value:}\newline
\ct{Association class>>key:value:}\newline
\ct{Association>>expireWeakKey}\newline
\ct{Association>>expiredWeakKey}\newline
\ct{Association>>key:WeakKey}\newline
\ct{Association>>key:value:WeakKey}\newline
\ct{Association>>keyWeakKey}\newline
\ct{Association>>value:WeakKey}\newline
\ct{Association>>value:}\newline
\ct{Association>>valueWeakKey}\newline
\ct{Association>>value}\newline
\ct{BlockClosure>>argumentCount}\newline
\ct{BlockClosure>>asContextWithSender:}\newline
\ct{BlockClosure>>asContext}\newline
\ct{BlockClosure>>assert}\newline
\ct{BlockClosure>>cull:}\newline
\ct{BlockClosure>>ensure:}\newline
\ct{BlockClosure>>fixCallbackTemps}\newline
\ct{BlockClosure>>forkAt:named:}\newline
\ct{BlockClosure>>forkAt:}\newline
\ct{BlockClosure>>ifCurtailed:}\newline
\ct{BlockClosure>>ifError:}\newline
\ct{BlockClosure>>isClosure}\newline
\ct{BlockClosure>>newProcess}\newline
\ct{BlockClosure>>numArgs}\newline
\ct{BlockClosure>>numCopiedValues}\newline
\ct{BlockClosure>>on:do:}\newline
\ct{BlockClosure>>on:fork:}\newline
\ct{BlockClosure>>outerContext}\newline
\ct{BlockClosure>>renderOn:}\newline
\ct{BlockClosure>>repeatWithGCIf:}\newline
\ct{BlockClosure>>repeat}\newline
\ct{BlockClosure>>startpc}\newline
\ct{BlockClosure>>value:value:value:}\newline
\ct{BlockClosure>>value:value:}\newline
\ct{BlockClosure>>valueNoContextSwitch}\newline
\ct{BlockClosure>>valueWithArguments:}\newline
\ct{BlockClosure>>valueWithPossibleArguments:}\newline
\ct{ByteArray>>asByteArray}\newline
\ct{ByteArray>>replaceFrom:to:with:startingAt:}\newline
\ct{ByteString class>>compare:with:collated:}\newline
\ct{ByteString class>>findFirstInString:inSet:startingAt:}\newline
\ct{ByteString class>>indexOfAscii:inString:startingAt:}\newline
\ct{ByteString class>>stringHash:initialHash:}\newline
\ct{ByteString class>>translate:from:to:table:}\newline
\ct{ByteString>>asByteArray}\newline
\ct{ByteString>>at:put:}\newline
\ct{ByteString>>at:}\newline
\ct{ByteString>>beginsWith:}\newline
\ct{ByteString>>byteAt:put:}\newline
\ct{ByteString>>byteSize}\newline
\ct{ByteString>>}\newline
\indent\ct{findSubstring:in:startingAt:matchTable:}\newline
\ct{ByteString>>}\newline
\indent\ct{findSubstringViaPrimitive:in:startingAt:matchTable:}\newline
\ct{ByteString>>isByteString}\newline
\ct{ByteString>>isOctetString}\newline
\ct{ByteString>>replaceFrom:to:with:startingAt:}\newline
\ct{ByteSymbol class>>stringHash:initialHash:}\newline
\ct{ByteSymbol>>at:}\newline
\ct{ByteSymbol>>}\newline
\indent\ct{findSubstring:in:startingAt:matchTable:}\newline
\ct{ByteSymbol>>isByteString}\newline
\ct{ByteSymbol>>privateAt:put:}\newline
\ct{ByteSymbol>>species}\newline
\ct{ByteSymbol>>string:}\newline
\ct{CNGBTextConverter class>>encodingNames}\newline
\ct{CP1250TextConverter class>>encodingNames}\newline
\ct{CP1253TextConverter class>>encodingNames}\newline
\ct{ChangesLog class>>default}\newline
\ct{ChangesLog>>recordStartupStamp}\newline
\ct{Character class>>codePoint:}\newline
\ct{Character class>>cr}\newline
\ct{Character class>>lf}\newline
\ct{Character class>>space}\newline
\ct{Character class>>value:}\newline
\ct{Character>>=}\newline
\ct{Character>>asCharacter}\newline
\ct{Character>>asInteger}\newline
\ct{Character>>asSymbol}\newline
\ct{Character>>asUppercase}\newline
\ct{Character>>asciiValue}\newline
\ct{Character>>charCode}\newline
\ct{Character>>characterSet}\newline
\ct{Character>>codePoint}\newline
\ct{Character>>digitValue}\newline
\ct{Character>>greaseInteger}\newline
\ct{Character>>isAlphaNumeric}\newline
\ct{Character>>isCharacter}\newline
\ct{Character>>isDigit}\newline
\ct{Character>>isLetter}\newline
\ct{Character>>isOctetCharacter}\newline
\ct{Character>>isSeparator}\newline
\ct{Character>>isVowel}\newline
\ct{Character>>leadingChar}\newline
\ct{Character>>to:}\newline
\ct{Collection class>>withAll:}\newline
\ct{Collection>>addAll:}\newline
\ct{Collection>>allSatisfy:}\newline
\ct{Collection>>anySatisfy:}\newline
\ct{Collection>>asArray}\newline
\ct{Collection>>detect:ifNone:}\newline
\ct{Collection>>emptyCheck}\newline
\ct{Collection>>inject:into:}\newline
\ct{Collection>>isCollection}\newline
\ct{Collection>>isEmptyOrNil}\newline
\ct{Collection>>isEmpty}\newline
\ct{Collection>>noneSatisfy:}\newline
\ct{Collection>>notEmpty}\newline
\ct{Collection>>printElementsOn:}\newline
\ct{Collection>>printNameOn:}\newline
\ct{Collection>>printOn:}\newline
\ct{Collection>>removeAll:}\newline
\ct{Collection>>removeAllFoundIn:}\newline
\ct{Collection>>sorted:}\newline
\ct{Collection>>sorted}\newline
\ct{CommandLineUIManager class>>replacing:}\newline
\ct{CommandLineUIManager>>initialize}\newline
\ct{CommandLineUIManager>>replacing:}\newline
\ct{CompiledMethod>>frameSize}\newline
\ct{CompiledMethod>>header}\newline
\ct{CompiledMethod>>initialPC}\newline
\ct{CompiledMethod>>isPrimitive}\newline
\ct{CompiledMethod>>numLiterals}\newline
\ct{CompiledMethod>>numTemps}\newline
\ct{CompiledMethod>>objectAt:}\newline
\ct{CompiledMethod>>primitive}\newline
\ct{CompoundTextConverter class>>encodingNames}\newline
\ct{ContextPart class>>contextEnsure:}\newline
\ct{ContextPart class>>contextOn:do:}\newline
\ct{ContextPart class>>newForMethod:}\newline
\ct{ContextPart class>>theReturnMethod}\newline
\ct{ContextPart>>activateMethod:withArgs:receiver:class:}\newline
\ct{ContextPart>>activateReturn:value:}\newline
\ct{ContextPart>>at:put:}\newline
\ct{ContextPart>>at:}\newline
\ct{ContextPart>>bottomContext}\newline
\ct{ContextPart>>cut:}\newline
\ct{ContextPart>>doPop}\newline
\ct{ContextPart>>exceptionClass}\newline
\ct{ContextPart>>exceptionHandlerBlock}\newline
\ct{ContextPart>>exceptionHandlerIsActive:}\newline
\ct{ContextPart>>exceptionHandlerIsActive}\newline
\ct{ContextPart>>findContextSuchThat:}\newline
\ct{ContextPart>>findNextHandlerContextStarting}\newline
\ct{ContextPart>>findNextUnwindContextUpTo:}\newline
\ct{ContextPart>>handleSignal:}\newline
\ct{ContextPart>>insertSender:}\newline
\ct{ContextPart>>isDead}\newline
\ct{ContextPart>>jump}\newline
\ct{ContextPart>>methodReturnTop}\newline
\ct{ContextPart>>nextHandlerContext}\newline
\ct{ContextPart>>pop}\newline
\ct{ContextPart>>privSender:}\newline
\ct{ContextPart>>push:}\newline
\ct{ContextPart>>pushTemporaryVariable:}\newline
\ct{ContextPart>>releaseTo:}\newline
\ct{ContextPart>>resume:through:}\newline
\ct{ContextPart>>resume:}\newline
\ct{ContextPart>>return:from:}\newline
\ct{ContextPart>>return:through:}\newline
\ct{ContextPart>>return:}\newline
\ct{ContextPart>>runUntilErrorOrReturnFrom:}\newline
\ct{ContextPart>>sender}\newline
\ct{ContextPart>>singleRelease}\newline
\ct{ContextPart>>stackp:}\newline
\ct{ContextPart>>stepToCallee}\newline
\ct{ContextPart>>step}\newline
\ct{ContextPart>>terminateTo:}\newline
\ct{Date class>>fromSeconds:}\newline
\ct{Date class>>fromString:}\newline
\ct{Date class>>readFrom:}\newline
\ct{Date class>>starting:}\newline
\ct{Date class>>year:month:day:}\newline
\ct{Date>>dayMonthYearDo:}\newline
\ct{Date>>monthIndex}\newline
\ct{Date>>printOn:format:}\newline
\ct{Date>>printOn:}\newline
\ct{DateAndTime class>>clock}\newline
\ct{DateAndTime class>>fromSeconds:offset:}\newline
\ct{DateAndTime class>>fromSeconds:}\newline
\ct{DateAndTime class>>fromString:}\newline
\ct{DateAndTime class>>initializeOffsets}\newline
\ct{DateAndTime class>>initialize}\newline
\ct{DateAndTime class>>localOffset}\newline
\ct{DateAndTime class>>localTimeZone}\newline
\ct{DateAndTime class>>}\newline
\indent\ct{milliSecondsSinceMidnight}\newline
\ct{DateAndTime class>>millisecondClockValue}\newline
\ct{DateAndTime class>>now}\newline
\ct{DateAndTime class>>readFrom:}\newline
\ct{DateAndTime class>>}\newline
\indent\ct{readOptionalSeparatorFrom:}\newline
\ct{DateAndTime class>>}\newline
\indent\ct{readTimezoneOffsetFrom:}\newline
\ct{DateAndTime class>>}\newline
\indent\ct{readTwoDigitIntegerFrom:}\newline
\ct{DateAndTime class>>startUp:}\newline
\ct{DateAndTime class>>todayAtMilliSeconds:}\newline
\ct{DateAndTime class>>waitForOffsets}\newline
\ct{DateAndTime class>>}\newline
\indent\ct{year:month:day:hour:minute:offset:}\newline
\ct{DateAndTime class>>}\newline
\indent\ct{year:month:day:hour:minute:}\newline
\indent\ct{second:nanoSecond:offset:}\newline
\ct{DateAndTime class>>}\newline
\indent\ct{year:month:day:hour:minute:}\newline
\indent\ct{second:offset:}\newline
\ct{DateAndTime class>>}\newline
\indent\ct{year:month:day:hour:minute:second:}\newline
\ct{DateAndTime class>>}\newline
\indent\ct{year:month:day:hour:minute:}\newline
\ct{DateAndTime class>>}\newline
\indent\ct{year:month:day:offset:}\newline
\ct{DateAndTime class>>}\newline
\indent\ct{year:month:day:}\newline
\ct{DateAndTime>><}\newline
\ct{DateAndTime>>asDateAndTime}\newline
\ct{DateAndTime>>asTimeStamp}\newline
\ct{DateAndTime>>asUTC}\newline
\ct{DateAndTime>>dayMonthYearDo:}\newline
\ct{DateAndTime>>dayOfMonth}\newline
\ct{DateAndTime>>dayOfWeekAbbreviation}\newline
\ct{DateAndTime>>dayOfWeekName}\newline
\ct{DateAndTime>>hour24}\newline
\ct{DateAndTime>>hour}\newline
\ct{DateAndTime>>julianDayNumber}\newline
\ct{DateAndTime>>julianDayOffset}\newline
\ct{DateAndTime>>localSeconds}\newline
\ct{DateAndTime>>minute}\newline
\ct{DateAndTime>>monthAbbreviation}\newline
\ct{DateAndTime>>monthName}\newline
\ct{DateAndTime>>month}\newline
\ct{DateAndTime>>normalize:ticks:base:}\newline
\ct{DateAndTime>>offset:DateAndTime>>midnight}\newline
\ct{DateAndTime>>offset}\newline
\ct{DateAndTime>>printHMSOn:}\newline
\ct{DateAndTime>>second}\newline
\ct{DateAndTime>>setJdn:seconds:nano:offset:}\newline
\ct{DateAndTime>>ticks:offset:DateAndTime>>dayOfWeek}\newline
\ct{DateAndTime>>ticks}\newline
\ct{DateAndTime>>year}\newline
\ct{Delay class>>forMilliseconds:}\newline
\ct{Delay class>>handleTimerEvent}\newline
\ct{Delay class>>initialize}\newline
\ct{Delay class>>primSignal:atMilliseconds:}\newline
\ct{Delay class>>restoreResumptionTimes}\newline
\ct{Delay class>>runTimerEventLoop}\newline
\ct{Delay class>>saveResumptionTimes}\newline
\ct{Delay class>>scheduleDelay:}\newline
\ct{Delay class>>shutDown}\newline
\ct{Delay class>>startTimerEventLoop}\newline
\ct{Delay class>>startUp}\newline
\ct{Delay class>>stopTimerEventLoop}\newline
\ct{Delay class>>unscheduleDelay:}\newline
\ct{Delay>>beingWaitedOn:}\newline
\ct{Delay>>beingWaitedOn}\newline
\ct{Delay>>delayDuration}\newline
\ct{Delay>>resumptionTime:}\newline
\ct{Delay>>resumptionTime}\newline
\ct{Delay>>schedule}\newline
\ct{Delay>>setDelay:forSemaphore:}\newline
\ct{Delay>>signalWaitingProcess}\newline
\ct{Delay>>unschedule}\newline
\ct{Delay>>wait}\newline
\ct{DelayWaitTimeout>>isExpired}\newline
\ct{DelayWaitTimeout>>setDelay:forSemaphore:}\newline
\ct{DelayWaitTimeout>>signalWaitingProcess}\newline
\ct{DelayWaitTimeout>>wait}\newline
\ct{Dictionary>>addAll:GRSmall}\newline
\ct{Dictionary>>associationsDo:}\newline
\ct{Dictionary>>at:ifAbsent:GRSmall}\newline
\ct{Dictionary>>at:ifAbsent:Small}\newline
\ct{Dictionary>>at:ifAbsent:}\newline
\ct{Dictionary>>at:ifAbsentPut:GRSmall}\newline
\ct{Dictionary>>at:ifAbsentPut:}\newline
\ct{Dictionary>>at:ifPresent:GRSmall}\newline
\ct{Dictionary>>at:ifPresent:}\newline
\ct{Dictionary>>at:put:GRSmall}\newline
\ct{Dictionary>>at:put:Small}\newline
\ct{Dictionary>>at:put:}\newline
\ct{Dictionary>>at:}\newline
\ct{Dictionary>>do:}\newline
\ct{Dictionary>>findIndexFor:GRSmall}\newline
\ct{Dictionary>>findIndexForKey:Small}\newline
\ct{Dictionary>>fixCollisionsFrom:}\newline
\ct{Dictionary>>includesKey:GRSmall}\newline
\ct{Dictionary>>includesKey:}\newline
\ct{Dictionary>>initialize:GRSmall}\newline
\ct{Dictionary>>initializeSmall}\newline
\ct{Dictionary>>isEmptyGRSmall}\newline
\ct{Dictionary>>keyAtValue:ifAbsent:}\newline
\ct{Dictionary>>keysAndValuesDo:}\newline
\ct{Dictionary>>keysAndValuesDo:}\newline
\ct{Dictionary>>keysAndValuesDo:}\newline
\ct{Dictionary>>keysDo:GRSmall}\newline
\ct{Dictionary>>noCheckAdd:}\newline
\ct{Dictionary>>postCopyGRSmall}\newline
\ct{Dictionary>>privateAt:put:GRSmall}\newline
\ct{Dictionary>>privateAt:put:Small}\newline
\ct{Dictionary>>rehash}\newline
\ct{Dictionary>>removeKey:ifAbsent:}\newline
\ct{Dictionary>>scanFor:}\newline
\ct{Dictionary>>seasideRequestFieldsGRSmall}\newline
\ct{Dictionary>>sizeGRSmall}\newline
\ct{Dictionary>>valuesDo:}\newline
\ct{DiskStore class>>checkVMVersion}\newline
\ct{DiskStore class>>initialize}\newline
\ct{DiskStore class>>reset}\newline
\ct{DiskStore class>>shutDown:}\newline
\ct{DiskStore class>>startUp:}\newline
\ct{DiskStore class>>useFilePlugin}\newline
\ct{Duration class>>days:hours:minutes:}\newline
\indent\ct{seconds:nanoSeconds:}\newline
\ct{Duration class>>days:hours:minutes:}\newline
\indent\ct{seconds:}\newline
\ct{Duration class>>days:}\newline
\ct{Duration class>>nanoSeconds:}\newline
\ct{Duration class>>seconds:nanoSeconds:}\newline
\ct{Duration class>>seconds:}\newline
\ct{Duration>>+}\newline
\ct{Duration>>-}\newline
\ct{Duration>>asDuration}\newline
\ct{Duration>>asMilliSeconds}\newline
\ct{Duration>>asNanoSeconds}\newline
\ct{Duration>>asSeconds}\newline
\ct{Duration>>days}\newline
\ct{Duration>>isZero}\newline
\ct{Duration>>negated}\newline
\ct{Duration>>seconds:nanoSeconds:}\newline
\ct{Duration>>ticks}\newline
\ct{DynamicVariable class>>value:during:}\newline
\ct{DynamicVariable>>value:during:}\newline
\ct{EUCJPTextConverter class>>encodingNames}\newline
\ct{EUCKRTextConverter class>>encodingNames}\newline
\ct{EncodedCharSet class>>charsetAt:}\newline
\ct{EventManager class>>actionMaps}\newline
\ct{EventManagerclass>>flushEvents}\newline
\ct{Exception class>>,}\newline
\ct{Exception class>>handles:}\newline
\ct{Exception class>>signal:}\newline
\ct{Exception class>>signal}\newline
\ct{Exception>>description}\newline
\ct{Exception>>isResumable}\newline
\ct{Exception>>messageText:}\newline
\ct{Exception>>messageText}\newline
\ct{Exception>>printOn:}\newline
\ct{Exception>>privHandlerContext:}\newline
\ct{Exception>>receiver}\newline
\ct{Exception>>resume:}\newline
\ct{Exception>>resumeUnchecked:}\newline
\ct{Exception>>signal:}\newline
\ct{Exception>>signalerContext}\newline
\ct{Exception>>signal}\newline
\ct{ExceptionSet>>,}\newline
\ct{ExceptionSet>>add:}\newline
\ct{ExceptionSet>>handles:}\newline
\ct{ExceptionSet>>initialize}\newline
\ct{ExtendedNumberParser>>allowPlusSign}\newline
\ct{ExtendedNumberParser>>nextNumber}\newline
\ct{ExternalSemaphoreTable class>>clearExternalObjects}\newline
\ct{ExternalSemaphoreTable class>>collectionBasedOn:withRoomFor:}\newline
\ct{ExternalSemaphoreTable class>>freedSlotsIn:ratherThanIncreaseSizeTo:}\newline
\ct{ExternalSemaphoreTable class>>initialize}\newline
\ct{ExternalSemaphoreTable class>>registerExternalObject:}\newline
\ct{ExternalSemaphoreTable class>>safelyRegisterExternalObject:}\newline
\ct{ExternalSemaphoreTable class>>safelyUnregisterExternalObject:}\newline
\ct{ExternalSemaphoreTable class>>slotFor:in:}\newline
\ct{ExternalSemaphoreTable class>>unprotectedExternalObjects:}\newline
\ct{ExternalSemaphoreTable class>>unprotectedExternalObjects}\newline
\ct{ExternalSemaphoreTable class>>unregisterExternalObject:}\newline
\ct{False>>not}\newline
\ct{False>>|}\newline
\ct{FilePath class>>pathName:isEncoded:}\newline
\ct{FilePath>>asSqueakPathName}\newline
\ct{FilePath>>pathName:isEncoded:}\newline
\ct{FilePath>>pathName}\newline
\ct{FileStream class>>flushAndVoidStdioFiles}\newline
\ct{FileStream class>>initialize}\newline
\ct{FileStream class>>shutDown:}\newline
\ct{FileStream class>>startUp:}\newline
\ct{FileStream class>>stdioHandles}\newline
\ct{FileStream class>>voidStdioFiles}\newline
\ct{Float class>>precision}\newline
\ct{Float>>adaptToInteger:andSend:}\newline
\ct{Float>>asFloat}\newline
\ct{Float>>isInfinite}\newline
\ct{Float>>timesTwoPower:}\newline
\ct{Float>>truncated}\newline
\ct{Fraction class>>numerator:denominator:}\newline
\ct{Fraction>>>=}\newline
\ct{Fraction>>reduced}\newline
\ct{Fraction>>setNumerator:denominator:}\newline
\ct{Fraction>>truncated}\newline
\ct{GRCodecStream class>>on:}\newline
\ct{GRCodecStream>>atEnd}\newline
\ct{GRCodecStream>>initializeOn:}\newline
\ct{GRNullCodec class>>codecName}\newline
\ct{GRNullCodec class>>supportsEncoding:}\newline
\ct{GRNullCodec>>encoderFor:}\newline
\ct{GRNullCodec>>url}\newline
\ct{GRNullCodecStream>>nextPutAll:}\newline
\ct{GRObject class>>new}\newline
\ct{GRObject>>initialize}\newline
\ct{GROrderedMultiMap>>allAt:}\newline
\ct{GROrderedMultiMap>>at:add:}\newline
\ct{GRPharoConverterCodecStream class>>}\newline
\indent\ct{on:converter:}\newline
\ct{GRPharoConverterCodecStream>>}\newline
\indent\ct{contents}\newline
\ct{GRPharoConverterCodecStream>>}\newline
\indent\ct{initializeOn:converter:}\newline
\ct{GRPharoConverterCodecStream>>}\newline
\indent\ct{size}\newline
\ct{GRPharoGenericCodec class>>}\newline
\indent\ct{supportedEncodingNames}\newline
\ct{GRPharoGenericCodec class>>}\newline
\indent\ct{supportsEncoding:}\newline
\ct{GRPharoLatin1Codec class>>}\newline
\indent\ct{supportedEncodingNames}\newline
\ct{GRPharoLatin1Codec class>>}\newline
\indent\ct{supportsEncoding:}\newline
\ct{GRPharoPlatform>>addToShutDownList:}\newline
\ct{GRPharoPlatform>>addToStartUpList:}\newline
\ct{GRPharoPlatform>>}\newline
\indent\ct{includesUnsafeUrlCharacter:}\newline
\ct{GRPharoPlatform>>}\newline
\indent\ct{includesUnsafeXmlCharacter:}\newline
\ct{GRPharoPlatform>>semaphoreClass}\newline
\ct{GRPharoRandomProvider class>>initialize}\newline
\ct{GRPharoRandomProvider class>>nextInt:}\newline
\ct{GRPharoRandomProvider class>>randomClass}\newline
\ct{GRPharoRandomProvider class>>startUp}\newline
\ct{GRPharoUtf8Codec class>>basicForEncoding:}\newline
\ct{GRPharoUtf8Codec class>>supportsEncoding:}\newline
\ct{GRPharoUtf8Codec>>decode:}\newline
\ct{GRPharoUtf8Codec>>decoderFor:}\newline
\ct{GRPharoUtf8Codec>>encoderFor:}\newline
\ct{GRPharoUtf8Codec>>name}\newline
\ct{GRPharoUtf8Codec>>url}\newline
\ct{GRPharoUtf8CodecStream>>encodeFast:}\newline
\ct{GRPharoUtf8CodecStream>>next:}\newline
\ct{GRPharoUtf8CodecStream>>nextPut:}\newline
\ct{GRPharoUtf8CodecStream>>nextPutAll:}\newline
\ct{GRPlatform class>>current}\newline
\ct{GRPlatform>>reducedConflictDictionary}\newline
\ct{GRSmallDictionary class>>new}\newline
\ct{GRSmallDictionary class>>new:}\newline
\ct{GRSmallDictionary class>>new}\newline
\ct{GRSmallDictionary class>>withAll:}\newline
\ct{HashTableSizes class>>atLeast:}\newline
\ct{HashTableSizes class>>sizes}\newline
\ct{HashedCollection class>>new:}\newline
\ct{HashedCollection class>>new}\newline
\ct{HashedCollection class>>sizeFor:}\newline
\ct{HashedCollection>>array}\newline
\ct{HashedCollection>>atNewIndex:put:}\newline
\ct{HashedCollection>>findElementOrNil:}\newline
\ct{HashedCollection>>fullCheck}\newline
\ct{HashedCollection>>grow}\newline
\ct{HashedCollection>>initialize:}\newline
\ct{HashedCollection>>size}\newline
\ct{Heap class>>withAll:sortBlock:}\newline
\ct{Heap>>add:}\newline
\ct{Heap>>do:}\newline
\ct{Heap>>downHeapSingle:}\newline
\ct{Heap>>growHeap>>reSort}\newline
\ct{Heap>>growSize}\newline
\ct{Heap>>growTo:}\newline
\ct{Heap>>isEmpty}\newline
\ct{Heap>>privateRemoveAt:}\newline
\ct{Heap>>remove:ifAbsent:}\newline
\ct{Heap>>removeFirst}\newline
\ct{Heap>>setCollection:tally:}\newline
\ct{Heap>>size}\newline
\ct{Heap>>sortBlock:}\newline
\ct{Heap>>sorts:before:}\newline
\ct{Heap>>upHeap:}\newline
\ct{Heap>>updateObjectIndex:}\newline
\ct{ISO885915TextConverter class>>encodingNames}\newline
\ct{ISO88592TextConverter class>>encodingNames}\newline
\ct{IdentitySet>>scanFor:}\newline
\ct{InstructionStream>>interpretExtension:in:for:}\newline
\ct{InstructionStream>>interpretNextInstructionFor:}\newline
\ct{Integer class>>readFrom:base:}\newline
\ct{Integer class>>readFrom:}\newline
\ct{Integer>>*}\newline
\ct{Integer>>+}\newline
\ct{Integer>><}\newline
\ct{Integer>>=}\newline
\ct{Integer>>>>}\newline
\ct{Integer>>asCharacter}\newline
\ct{Integer>>asInteger}\newline
\ct{Integer>>copyto:}\newline
\ct{Integer>>denominator}\newline
\ct{Integer>>digitCompare:}\newline
\ct{Integer>>digitDiv:neg:}\newline
\ct{Integer>>digitMultiply:neg:}\newline
\ct{Integer>>digitSubtract:}\newline
\ct{Integer>>floor}\newline
\ct{Integer>>isFraction}\newline
\ct{Integer>>isInteger}\newline
\ct{Integer>>noMask:}\newline
\ct{Integer>>normalize}\newline
\ct{Integer>>numerator}\newline
\ct{Integer>>printOn:}\newline
\ct{Integer>>printStringBase:length:padded:}\newline
\ct{Integer>>printStringLength:padded:}\newline
\ct{Integer>>quo:}\newline
\ct{Integer>>rounded}\newline
\ct{Integer>>timesRepeat:}\newline
\ct{Integer>>truncated}\newline
\ct{Interval class>>from:to:by:}\newline
\ct{Interval class>>new}\newline
\ct{Interval>>collect:}\newline
\ct{Interval>>setFrom:to:by:}\newline
\ct{Interval>>size}\newline
\ct{Interval>>species}\newline
\ct{KOI8RTextConverter class>>encodingNames}\newline
\ct{LanguageEnvironment class>>}\newline
\indent\ct{defaultFileNameConverter}\newline
\ct{LargeNegativeInteger>>negative}\newline
\ct{LargeNegativeInteger>>normalize}\newline
\ct{LargePositiveInteger>>*}\newline
\ct{LargePositiveInteger>>+}\newline
\ct{LargePositiveInteger>>-}\newline
\ct{LargePositiveInteger>>//}\newline
\ct{LargePositiveInteger>><}\newline
\ct{LargePositiveInteger>>asFloat}\newline
\ct{LargePositiveInteger>>digitAt:}\newline
\ct{LargePositiveInteger>>digitLength}\newline
\ct{LargePositiveInteger>>highBitOfMagnitude}\newline
\ct{LargePositiveInteger>>negated}\newline
\ct{LargePositiveInteger>>negative}\newline
\ct{LargePositiveInteger>>normalize}\newline
\ct{LargePositiveInteger>>quo:}\newline
\ct{LimitedWriteStream>>nextPut:}\newline
\ct{LimitedWriteStream>>nextPutAll:}\newline
\ct{LimitedWriteStream>>setLimit:limitBlock:}\newline
\ct{LookupKey class>>key:}\newline
\ct{LookupKey>>key:}\newline
\ct{LookupKey>>key}\newline
\ct{MacOSXPlatform class>>isActivePlatform}\newline
\ct{MacRomanTextConverter class>>}\newline
\indent\ct{encodingNames}\newline
\ct{Magnitude>>>}\newline
\ct{Magnitude>>between:and:}\newline
\ct{Magnitude>>max:}\newline
\ct{Magnitude>>min:}\newline
\ct{MethodContext class>>}\newline
\indent\ct{sender:receiver:method:arguments:}\newline
\ct{MethodContext>>aboutToReturn:through:}\newline
\ct{MethodContext>>blockReturnTop}\newline
\ct{MethodContext>>closure}\newline
\ct{MethodContext>>contextTag}\newline
\ct{MethodContext>>methodReturnContext}\newline
\ct{MethodContext>>method}\newline
\ct{MethodContext>>privRefresh}\newline
\ct{MethodContext>>receiver}\newline
\ct{MethodContext>>setSender:receiver:method:arguments:}\newline
\ct{MethodContext>>setSender:receiver:method:closure:startpc:}\newline
\ct{MethodContext>>tempAt:put:}\newline
\ct{MethodContext>>tempAt:}\newline
\ct{MethodDictionary>>at:ifAbsent:}\newline
\ct{MethodDictionary>>includesKey:}\newline
\ct{MethodDictionary>>scanFor:}\newline
\ct{Month class>>daysInMonth:forYear:}\newline
\ct{Month class>>indexOfMonth:}\newline
\ct{Month class>>nameOfMonth:}\newline
\ct{Mutex>>critical:}\newline
\ct{Mutex>>initialize}\newline
\ct{NetNameResolver class>>initializeNetwork}\newline
\ct{NetNameResolver class>>initialize}\newline
\ct{NetNameResolver class>>primNameResolverStatus}\newline
\ct{NetNameResolver class>>resolverStatus}\newline
\ct{Number class>>one}\newline
\ct{Number class>>readFrom:}\newline
\ct{Number>>//}\newline
\ct{Number>>\%}\newline
\ct{Number>>\textbackslash\textbackslash}\newline
\ct{Number>>abs}\newline
\ct{Number>>asDuration}\newline
\ct{Number>>floor}\newline
\ct{Number>>fractionPart}\newline
\ct{Number>>integerPart}\newline
\ct{Number>>isNumber}\newline
\ct{Number>>isZero}\newline
\ct{Number>>negated}\newline
\ct{Number>>negative}\newline
\ct{Number>>quo:}\newline
\ct{Number>>raisedToInteger:}\newline
\ct{Number>>rem:}\newline
\ct{Number>>strictlyPositive}\newline
\ct{Number>>to:}\newline
\ct{NumberParser class>>on:}\newline
\ct{NumberParser>>nextElementaryLargeIntegerBase:}\newline
\ct{NumberParser>>nextIntegerBase:}\newline
\ct{NumberParser>>nextUnsignedIntegerBase:}\newline
\ct{NumberParser>>nextUnsignedIntegerOrNilBase:}\newline
\ct{NumberParser>>on:}\newline
\ct{NumberParser>>peekSignIsMinus}\newline
\ct{NumberParser>>readExponent}\newline
\ct{OSPlatform class>>determineActivePlatformStartingAt:}\newline
\ct{OSPlatform class>>initialize}\newline
\ct{OSPlatform class>>isMacOS}\newline
\ct{OSPlatform class>>platformName}\newline
\ct{OSPlatform class>>shutDown:}\newline
\ct{OSPlatform class>>startUp:ISO88597}\newline
\ct{OSPlatform class>>version}\newline
\ct{OSPlatform>>shutDown:}\newline
\ct{OSPlatform>>startUp:}\newline
\ct{Object class>>flushDependents}\newline
\ct{Object class>>flushEvents}\newline
\ct{Object class>>initializeDependentsFields}\newline
\ct{Object class>>initialize}\newline
\ct{Object class>>newFrom:}\newline
\ct{Object>>=}\newline
\ct{Object>>actAsExecutor}\newline
\ct{Object>>as:}\newline
\ct{Object>>asSetElement}\newline
\ct{Object>>asString}\newline
\ct{Object>>assert:}\newline
\ct{Object>>at:put:}\newline
\ct{Object>>at:}\newline
\ct{Object>>basicAt:put:}\newline
\ct{Object>>basicAt:}\newline
\ct{Object>>basicSize}\newline
\ct{Object>>breakDependents}\newline
\ct{Object>>class}\newline
\ct{Object>>copyFrom:}\newline
\ct{Object>>copySameFrom:}\newline
\ct{Object>>copy}\newline
\ct{Object>>enclosedSetElement}\newline
\ct{Object>>encodeOn:}\newline
\ct{Object>>executor}\newline
\ct{Object>>greaseString}\newline
\ct{Object>>hash}\newline
\ct{Object>>instVarAt:put:}\newline
\ct{Object>>instVarAt:}\newline
\ct{Object>>instVarNamed:put:}\newline
\ct{Object>>isArray}\newline
\ct{Object>>isCharacter}\newline
\ct{Object>>isInteger}\newline
\ct{Object>>isKindOf:}\newline
\ct{Object>>isLiteral}\newline
\ct{Object>>isMemberOf:}\newline
\ct{Object>>isSelfEvaluating}\newline
\ct{Object>>isString}\newline
\ct{Object>>isSymbol}\newline
\ct{Object>>myDependents:}\newline
\ct{Object>>notNil}\newline
\ct{Object>>perform:with:}\newline
\ct{Object>>perform:withArguments:inSuperclass:}\newline
\ct{Object>>perform:withArguments:}\newline
\ct{Object>>perform:}\newline
\ct{Object>>postCopy}\newline
\ct{Object>>printOn:}\newline
\ct{Object>>printStringLimitedTo:}\newline
\ct{Object>>printString}\newline
\ct{Object>>readFromString:}\newline
\ct{Object>>renderOn:}\newline
\ct{Object>>respondsTo:}\newline
\ct{Object>>restoreFromSnapshot:}\newline
\ct{Object>>shallowCopy}\newline
\ct{Object>>shouldBePrintedAsLiteral}\newline
\ct{Object>>snapshotCopy}\newline
\ct{Object>>species}\newline
\ct{Object>>split:do:}\newline
\ct{Object>>split:}\newline
\ct{Object>>value}\newline
\ct{Object>>yourself}\newline
\ct{Object>>\textasciitilde=}\newline
\ct{OrderedCollection class>>arrayType}\newline
\ct{OrderedCollection class>>new:}\newline
\ct{OrderedCollection class>>new}\newline
\ct{OrderedCollection>>add:}\newline
\ct{OrderedCollection>>addAll:}\newline
\ct{OrderedCollection>>addAllLast:}\newline
\ct{OrderedCollection>>addFirst:}\newline
\ct{OrderedCollection>>addLast:}\newline
\ct{OrderedCollection>>asArray}\newline
\ct{OrderedCollection>>at:}\newline
\ct{OrderedCollection>>collect:}\newline
\ct{OrderedCollection>>copyEmpty}\newline
\ct{OrderedCollection>>do:}\newline
\ct{OrderedCollection>>ensureBoundsFrom:to:}\newline
\ct{OrderedCollection>>growAtLast}\newline
\ct{OrderedCollection>>makeRoomAtFirst}\newline
\ct{OrderedCollection>>makeRoomAtLast}\newline
\ct{OrderedCollection>>postCopy}\newline
\ct{OrderedCollection>>remove:ifAbsent:}\newline
\ct{OrderedCollection>>removeFirst}\newline
\ct{OrderedCollection>>removeIndex:}\newline
\ct{OrderedCollection>>resetTo:}\newline
\ct{OrderedCollection>>reset}\newline
\ct{OrderedCollection>>reverseDo:}\newline
\ct{OrderedCollection>>select:}\newline
\ct{OrderedCollection>>setCollection:}\newline
\ct{OrderedCollection>>size}\newline
\ct{PositionableStream class>>on:}\newline
\ct{PositionableStream>>atEnd}\newline
\ct{PositionableStream>>isBinary}\newline
\ct{PositionableStream>>isEmpty}\newline
\ct{PositionableStream>>on:}\newline
\ct{PositionableStream>>originalContents}\newline
\ct{PositionableStream>>peekFor:}\newline
\ct{PositionableStream>>peek}\newline
\ct{PositionableStream>>position:}\newline
\ct{PositionableStream>>position}\newline
\ct{PositionableStream>>reset}\newline
\ct{PositionableStream>>skip:}\newline
\ct{PositionableStream>>skipSeparators}\newline
\ct{PositionableStream>>skipTo:}\newline
\ct{Process class>>forContext:priority:}\newline
\ct{Process>>activateReturn:value:}\newline
\ct{Process>>calleeOf:}\newline
\ct{Process>>complete:}\newline
\ct{Process>>isActiveProcess}\newline
\ct{Process>>name:}\newline
\ct{Process>>popTo:}\newline
\ct{Process>>primitiveResume}\newline
\ct{Process>>priority:}\newline
\ct{Process>>priority}\newline
\ct{Process>>psValueAt:put:}\newline
\ct{Process>>psValueAt:}\newline
\ct{Process>>resume}\newline
\ct{Process>>return:value:}\newline
\ct{Process>>suspendedContext:}\newline
\ct{Process>>suspendingList}\newline
\ct{Process>>suspend}\newline
\ct{Process>>terminate}\newline
\ct{ProcessLocalVariable class>>value:}\newline
\ct{ProcessLocalVariable>>value:}\newline
\ct{ProcessSpecificVariable class>>soleInstance}\newline
\ct{ProcessSpecificVariable class>>value}\newline
\ct{ProcessSpecificVariable>>default}\newline
\ct{ProcessSpecificVariable>>value}\newline
\ct{ProcessorScheduler class>>idleProcess}\newline
\ct{ProcessorScheduler class>>initialize}\newline
\ct{ProcessorScheduler class>>}\newline
\indent\ct{relinquishProcessorForMicroseconds:}\newline
\ct{ProcessorScheduler class>>startUp}\newline
\ct{ProcessorScheduler>>activePriority}\newline
\ct{ProcessorScheduler>>activeProcess}\newline
\ct{ProcessorScheduler>>highIOPriority}\newline
\ct{ProcessorScheduler>>highestPriority}\newline
\ct{ProcessorScheduler>>lowIOPriority}\newline
\ct{ProcessorScheduler>>lowestPriority}\newline
\ct{ProcessorScheduler>>terminateActive}\newline
\ct{ProcessorScheduler>>timingPriority}\newline
\ct{ProcessorScheduler>>userInterruptPriority}\newline
\ct{ProtoObject>>basicIdentityHash}\newline
\ct{ProtoObject>>flag:}\newline
\ct{ProtoObject>>identityHash}\newline
\ct{ProtoObject>>initialize}\newline
\ct{ProtoObject>>instVarsInclude:}\newline
\ct{ProtoObject>>isNil}\newline
\ct{ProtoObject>>pointsTo:}\newline
\ct{ProtoObject>>\textasciitilde\textasciitilde}\newline
\ct{Random>>initialize}\newline
\ct{Random>>nextInt:}\newline
\ct{Random>>nextValue}\newline
\ct{Random>>next}\newline
\ct{ReadStream class>>on:from:to:}\newline
\ct{ReadStream>>next}\newline
\ct{ReadStream>>on:from:to:}\newline
\ct{ReadStream>>upTo:}\newline
\ct{ReadStream>>upToEnd}\newline
\ct{Semaphore class>>forMutualExclusion}\newline
\ct{Semaphore class>>new}\newline
\ct{Semaphore>>critical:ifError:}\newline
\ct{Semaphore>>critical:}\newline
\ct{Semaphore>>initSignals}\newline
\ct{Semaphore>>signal}\newline
\ct{Semaphore>>waitTimeoutMSecs:}\newline
\ct{Semaphore>>wait}\newline
\ct{SequenceableCollection class>>new:streamContents:}\newline
\ct{SequenceableCollection class>>ofSize:}\newline
\ct{SequenceableCollection class>>streamContents:limitedTo:}\newline
\ct{SequenceableCollection class>>streamContents:}\newline
\ct{SequenceableCollection class>>streamSpecies}\newline
\ct{SequenceableCollection>>,}\newline
\ct{SequenceableCollection>>allButFirst:}\newline
\ct{SequenceableCollection>>at:ifAbsent:}\newline
\ct{SequenceableCollection>>atAllPut:}\newline
\ct{SequenceableCollection>>copyAfter:}\newline
\ct{SequenceableCollection>>copyFrom:to:}\newline
\ct{SequenceableCollection>>copyReplaceFrom:to:with:}\newline
\ct{SequenceableCollection>>copyUpTo:}\newline
\ct{SequenceableCollection>>do:separatedBy:}\newline
\ct{SequenceableCollection>>do:}\newline
\ct{SequenceableCollection>>doWithIndex:}\newline
\ct{SequenceableCollection>>first:}\newline
\ct{SequenceableCollection>>first}\newline
\ct{SequenceableCollection>>from:to:put:}\newline
\ct{SequenceableCollection>>grownBy:}\newline
\ct{SequenceableCollection>>includes:}\newline
\ct{SequenceableCollection>>indexOf:ifAbsent:}\newline
\ct{SequenceableCollection>>indexOf:startingAt:ifAbsent:}\newline
\ct{SequenceableCollection>>indexOf:}\newline
\ct{SequenceableCollection>>indexOfSubCollection:startingAt:}\newline
\ct{SequenceableCollection>>keysAndValuesDo:}\newline
\ct{SequenceableCollection>>last}\newline
\ct{SequenceableCollection>>readStream}\newline
\ct{SequenceableCollection>>replaceFrom:to:with:}\newline
\ct{SequenceableCollection>>reverseDo:}\newline
\ct{SequenceableCollection>>second}\newline
\ct{SequenceableCollection>>select:}\newline
\ct{SequenceableCollection>>split:indicesDo:}\newline
\ct{SequenceableCollection>>splitOn:}\newline
\ct{SequenceableCollection>>swap:with:}\newline
\ct{SequenceableCollection>>withIndexDo:}\newline
\ct{SequenceableCollection>>writeStream}\newline
\ct{Set class>>new}\newline
\ct{Set>>add:}\newline
\ct{Set>>do:}\newline
\ct{Set>>fixCollisionsFrom:}\newline
\ct{Set>>grow}\newline
\ct{Set>>includes:}\newline
\ct{Set>>noCheckAdd:}\newline
\ct{Set>>remove:ifAbsent:}\newline
\ct{Set>>scanFor:}\newline
\ct{SmallInteger class>>maxValCP1252}\newline
\ct{SmallInteger>>*}\newline
\ct{SmallInteger>>/}\newline
\ct{SmallInteger>>asFloat}\newline
\ct{SmallInteger>>decimalDigitLength}\newline
\ct{SmallInteger>>gcd:}\newline
\ct{SmallInteger>>hashMultiply}\newline
\ct{SmallInteger>>hash}\newline
\ct{SmallInteger>>highBitOfPositiveReceiver}\newline
\ct{SmallInteger>>highBit}\newline
\ct{SmallInteger>>identityHash}\newline
\ct{SmallInteger>>isLarge}\newline
\ct{SmallInteger>>numberOfDigitsInBase:}\newline
\ct{SmallInteger>>printOn:base:length:padded:}\newline
\ct{SmallInteger>>printOn:base:}\newline
\ct{SmallInteger>>printString}\newline
\ct{SmallInteger>>quo:}\newline
\ct{SmalltalkImage class>>initializeForTornado}\newline
\ct{SmalltalkImage>>addDeferredStartupAction:}\newline
\ct{SmalltalkImage>>addToShutDownList:}\newline
\ct{SmalltalkImage>>addToStartUpList:}\newline
\ct{SmalltalkImage>>at:ifAbsent:}\newline
\ct{SmalltalkImage>>at:}\newline
\ct{SmalltalkImage>>clearExternalObjects}\newline
\ct{SmalltalkImage>>}\newline
\indent\ct{executeDeferredStartupActions:}\newline
\ct{SmalltalkImage>>garbageCollectMost}\newline
\ct{SmalltalkImage>>imagePath}\newline
\ct{SmalltalkImage>>includesKey:}\newline
\ct{SmalltalkImage>>installLowSpaceWatcher}\newline
\ct{SmalltalkImage>>isHeadless}\newline
\ct{SmalltalkImage>>isInteractive}\newline
\ct{SmalltalkImage>>}\newline
\indent\ct{logStartUpErrorDuring:into:tryDebugger:}\newline
\ct{SmalltalkImage>>lowSpaceThreshold}\newline
\ct{SmalltalkImage>>lowSpaceWatcher}\newline
\ct{SmalltalkImage>>newSessionObject}\newline
\ct{SmalltalkImage>>primImagePath}\newline
\ct{SmalltalkImage>>primSignalAtBytesLeft:}\newline
\ct{SmalltalkImage>>}\newline
\indent\ct{primitiveGetSpecialObjectsArray}\newline
\ct{SmalltalkImage>>processShutDownList:}\newline
\ct{SmalltalkImage>>processStartUpList:}\newline
\ct{SmalltalkImage>>recordStartupStamp}\newline
\ct{SmalltalkImage>>registerExternalObject:}\newline
\ct{SmalltalkImage>>send:toClassesNamedIn:with:}\newline
\ct{SmalltalkImage>>shutDownImage:}\newline
\ct{SmalltalkImage>>specialObjectsArray}\newline
\ct{SmalltalkImage>>}\newline
\indent\ct{startupImage:snapshotWorked:}\newline
\ct{SmalltalkImage>>unregisterExternalObject:}\newline
\ct{SmalltalkImage>>vm}\newline
\ct{SmalltalkImage>>wordSize}\newline
\ct{Socket class>>acceptFrom:}\newline
\ct{Socket class>>initializeNetwork}\newline
\ct{Socket class>>initialize}\newline
\ct{Socket class>>newTCP}\newline
\ct{Socket class>>register:}\newline
\ct{Socket class>>registry}\newline
\ct{Socket class>>standardTimeout}\newline
\ct{Socket class>>unregister:}\newline
\ct{Socket>>acceptFrom:}\newline
\ct{Socket>>accept}\newline
\ct{Socket>>closeAndDestroy:}\newline
\ct{Socket>>closeAndDestroy}\newline
\ct{Socket>>close}\newline
\ct{Socket>>dataAvailable}\newline
\ct{Socket>>destroy}\newline
\ct{Socket>>initialize:}\newline
\ct{Socket>>isConnected}\newline
\ct{Socket>>isOtherEndClosed}\newline
\ct{Socket>>isValid}\newline
\ct{Socket>>listenOn:backlogSize:}\newline
\ct{Socket>>}\newline\indent\ct{primAcceptFrom:}\newline\indent\ct{receiveBufferSize:}\newline\indent\ct{sendBufSize:}\newline\indent\ct{semaIndex:}\newline\indent\ct{readSemaIndex:}\newline\indent\ct{writeSemaIndex:}\newline
\ct{Socket>>primSocket:receiveDataInto:startingAt:count:}\newline
\ct{Socket>>primSocket:sendData:startIndex:count:}\newline
\ct{Socket>>primSocket:setOption:value:}\newline
\ct{Socket>>primSocketConnectionStatus:}\newline
\ct{Socket>>primSocketDestroy:}\newline
\ct{Socket>>primSocketReceiveDataAvailable:}\newline
\ct{Socket>>primSocketRemoteAddress:}\newline
\ct{Socket>>primSocketSendDone:}\newline
\ct{Socket>>readSemaphore}\newline
\ct{Socket>>register}\newline
\ct{Socket>>remoteAddress}\newline
\ct{Socket>>sendSomeData:startIndex:count:for:}\newline
\ct{Socket>>sendSomeData:startIndex:count:}\newline
\ct{Socket>>setOption:value:}\newline
\ct{Socket>>socketHandle}\newline
\ct{Socket>>unregister}\newline
\ct{Socket>>waitForAcceptFor:}\newline
\ct{Socket>>waitForConnectionFor:ifTimedOut:}\newline
\ct{Socket>>waitForDataFor:ifClosed:ifTimedOut:}\newline
\ct{Socket>>waitForDataFor:}\newline
\ct{Socket>>waitForDisconnectionFor:}\newline
\ct{Socket>>waitForSendDoneFor:}\newline
\ct{SparseLargeTable>>at:}\newline
\ct{SparseLargeTable>>noCheckAt:}\newline
\ct{SparseLargeTable>>pvtCheckIndex:}\newline
\ct{SparseLargeTable>>size}\newline
\ct{SqNumberParser>>allowPlusSign}\newline
\ct{SqNumberParser>>makeIntegerOrScaledInteger}\newline
\ct{SqNumberParser>>readScale}\newline
\ct{Stream>>basicNext}\newline
\ct{Stream>>nextPutAll:}\newline
\ct{Stream>>print:}\newline
\ct{String class>>crlf}\newline
\ct{String class>>empty}\newline
\ct{String class>>new:}\newline
\ct{String:startingAt:}\newline
\ct{String>>=}\newline
\ct{String>>asDateAndTime}\newline
\ct{String>>asDate}\newline
\ct{String>>asLowercase}\newline
\ct{String>>asNumber}\newline
\ct{String>>asString}\newline
\ct{String>>asSymbol}\newline
\ct{String>>asUppercase}\newline
\ct{String>>asZnMimeType}\newline
\ct{String>>asZnUrl}\newline
\ct{String>>compare:caseSensitive:}\newline
\ct{String>>compare:with:collated:}\newline
\ct{String>>convertFromWithConverter:}\newline
\ct{String>>encodeOn:}\newline
\ct{String>>findString:startingAt:caseSensitive:}\newline
\ct{String>>findString:}\newline
\ct{String>>find}\newline
\ct{String>>greaseInteger}\newline
\ct{String>>hash}\newline
\ct{String>>includesSubstring:}\newline
\ct{String>>indexOf:startingAt:ifAbsent:}\newline
\ct{String>>indexOf:startingAt:}\newline
\ct{String>>indexOf:}\newline
\ct{String>>}\newline
\indent\ct{indexOfSubCollection:startingAt:ifAbsent:}\newline
\ct{String>>indexOfSubCollection:}\newline
\ct{String>>isString}\newline
\ct{String>>isWideString}\newline
\ct{String>>match:}\newline
\ct{String>>putOn:}\newline
\ct{String>>renderOn:}\newline
\ct{String>>sameAs:}\newline
\ct{String>>seasideMimeType}\newline
\ct{String>>startingAt:match:startingAt:}\newline
\ct{String>>subStrings:}\newline
\ct{String>>translateFrom:to:table:}\newline
\ct{String>>translateToLowercase}\newline
\ct{String>>translateToUppercase}\newline
\ct{String>>translateWith:}\newline
\ct{String>>trimBoth:}\newline
\ct{String>>trimBoth}\newline
\ct{String>>trimLeft:right:}\newline
\ct{Symbol class>>initializeForTornado}\newline
\ct{Symbol class>>intern:}\newline
\ct{Symbol class>>internCharacter:}\newline
\ct{Symbol class>>lookup:}\newline
\ct{Symbol class>>shutDown:}\newline
\ct{Symbol>>=}\newline
\ct{Symbol>>asString}\newline
\ct{TextConverter class>>allEncodingNames}\newline
\ct{TextConverter class>>encodingNames}\newline
\ct{TextConverter class>>encodingNames}\newline
\ct{TextConverter class>>encodingNames}\newline
\ct{TextConverter class>>encodingNames}\newline
\ct{TextConverter class>>encodingNames}\newline
\ct{TextConverter class>>encodingNames}\newline
\ct{TextConverter class>>latin1Encodings}\newline
\ct{TextConverter class>>latin1}\newline
\ct{GRCodec class>>forEncoding:}\newline
\ct{TextConverter>>initialize}\newline
\ct{TextConverter>>nextFromStream:UTF8}\newline
\ct{Time class>>dateAndTimeFromSeconds:}\newline
\ct{Time class>>dateAndTimeNow}\newline
\ct{Time class>>fromSeconds:}\newline
\ct{Time class>>hour:minute:second:nanoSecond:}\newline
\ct{Time class>>millisecondClockValue}\newline
\ct{Time class>>milliseconds:since:}\newline
\ct{Time class>>millisecondsSince:}\newline
\ct{Time class>>primSecondsClock}\newline
\ct{Time class>>readFrom:}\newline
\ct{Time class>>seconds:nanoSeconds:}\newline
\ct{Time class>>secondsWhenClockTicks}\newline
\ct{Time class>>totalSeconds}\newline
\ct{Time>>hour24}\newline
\ct{Time>>hour}\newline
\ct{Time>>minute}\newline
\ct{Time>>nanoSecond}\newline
\ct{Time>>print24:showSeconds:on:}\newline
\ct{Time>>printOn:}\newline
\ct{Time>>seconds}\newline
\ct{Time>>second}\newline
\ct{Time>>ticks:}\newline
\ct{TimeZone>>offset}\newline
\ct{Timespan class>>starting:duration:}\newline
\ct{Timespan>><}\newline
\ct{Timespan>>dayOfMonth}\newline
\ct{Timespan>>duration:}\newline
\ct{Timespan>>month}\newline
\ct{Timespan>>start:}\newline
\ct{Timespan>>start}\newline
\ct{Timespan>>year}\newline
\ct{True>>ifFalse:}\newline
\ct{True>>not}\newline
\ct{True>>|}\newline
\ct{UIManager class>>basicDefault:}\newline
\ct{UIManager class>>default:}\newline
\ct{WAUnescapedDocument>>initializeWithStream:codec:}\newline
\ct{UIManager class>>default}\newline
\ct{UIManager>>activate}\newline
\ct{UIManager>>beDefault}\newline
\ct{UIManager>>boot:during:}\newline
\ct{UIManager>>deactivate}\newline
\ct{UIManager>>onSnapshot:}\newline
\ct{UTF16TextConverter class>>encodingNames}\newline
\ct{UTF8DecomposedTextConverter class>>encodingNames}\newline
\ct{UUIDGenerator class>>initialize}\newline
\ct{UUIDGenerator class>>startUp}\newline
\ct{UndefinedObject>>encodeOn:}\newline
\ct{UndefinedObject>>isNil}\newline
\ct{UndefinedObject>>notNil}\newline
\ct{UndefinedObject>>seasideUrl}\newline
\ct{UndefinedObject>>shallowCopy}\newline
\ct{Unicode class>>isDigit:}\newline
\ct{Unicode class>>isLetter:}\newline
\ct{Unicode class>>toUppercase:}\newline
\ct{VirtualMachine class>>allocationsBetweenGC:}\newline
\ct{VirtualMachine class>>getSystemAttribute:}\newline
\ct{VirtualMachine class>>interpreterClass}\newline
\ct{VirtualMachine class>>}\newline
\indent\ct{interpreterSourceDate}\newline
\ct{VirtualMachine class>>}\newline
\indent\ct{interpreterSourceVersion}\newline
\ct{VirtualMachine class>>isPharoVM}\newline
\ct{VirtualMachine class>>isRunningCogit}\newline
\ct{VirtualMachine class>>}\newline
\indent\ct{maxExternalSemaphores}\newline
\ct{VirtualMachine class>>parameterAt:put:}\newline
\ct{VirtualMachine class>>parameterAt:}\newline
\ct{VirtualMachine class>>setGCParameters}\newline
\ct{VirtualMachine class>>tenuringThreshold:}\newline
\ct{VirtualMachine class>>version}\newline
\ct{VirtualMachine class>>wordSize}\newline
\ct{WAAccessIntervalReapingStrategy>>}\newline
\indent\ct{defaultConfiguration}\newline
\ct{WAAccessIntervalReapingStrategy>>}\newline
\indent\ct{initialize}\newline
\ct{WAAccessIntervalReapingStrategy>>}\newline
\indent\ct{interval}\newline
\ct{WAAccessIntervalReapingStrategy>>}\newline
\indent\ct{reap}\newline
\ct{WAAccessIntervalReapingStrategy>>}\newline
\indent\ct{stored:key:}\newline
\ct{WAActionCallback>>block:}\newline
\ct{WAActionCallback>>evaluateWithArgument:}\newline
\ct{WAActionCallback>>isEnabledFor:}\newline
\ct{WAActionCallback>>signalRenderNotification}\newline
\ct{WAActionPhaseContinuation>>continue}\newline
\ct{WAActionPhaseContinuation>>handleRequest}\newline
\ct{WAActionPhaseContinuation>>renderContext:}\newline
\ct{WAActionPhaseContinuation>>renderContext}\newline
\ct{WAActionPhaseContinuation>>runCallbacks}\newline
\ct{WAActionPhaseContinuation>>shouldRedirect}\newline
\ct{WAAdmin class>>defaultServerManager}\newline
\ct{WAAdmin class>>serverAdaptors}\newline
\ct{WAAnchorTag>>callback:}\newline
\ct{WAAnchorTag>>tag}\newline
\ct{WAAnchorTag>>url}\newline
\ct{WAAnchorTag>>with:}\newline
\ct{WAApplication>>contentType}\newline
\ct{WAApplication>>doesHandlerSupportCookies:}\newline
\ct{WAApplication>>handleDefault:}\newline
\ct{WAApplication>>handleFiltered:}\newline
\ct{WAApplication>>isApplication}\newline
\ct{WAApplication>>isImplemented:}\newline
\ct{WAApplication>>keyField}\newline
\ct{WAApplication>>libraries}\newline
\ct{WAApplication>>mainClass}\newline
\ct{WAApplication>>mimeType}\newline
\ct{WAApplication>>newSession}\newline
\ct{WAApplication>>resourceBaseUrl}\newline
\ct{WAApplication>>sessionClass}\newline
\ct{WAApplicationConfiguration>>parents}\newline
\ct{WAAttributeSearchContext class>>key:target:}\newline
\ct{WAAttributeSearchContext>>at:ifPresent:}\newline
\ct{WAAttributeSearchContext>>at:put:}\newline
\ct{WAAttributeSearchContext>>attribute}\newline
\ct{WAAttributeSearchContext>>cachedValues}\newline
\ct{WAAttributeSearchContext>>findAttributeAndSelectAncestorsOf:}\newline
\ct{WAAttributeSearchContext>>initializeWithKey:}\newline
\ct{WAAttributeSearchContext>>isAttributeInheritedOn:}\newline
\ct{WAAttributeSearchContext>>isAttributeLocalOn:}\newline
\ct{WAAttributeSearchContext>>key}\newline
\ct{WABrush>>initialize}\newline
\ct{WABrush>>parent}\newline
\ct{WABrush>>setParent:canvas:}\newline
\ct{WABrush>>with:}\newline
\ct{WABufferedResponse class>>on:}\newline
\ct{WABufferedResponse>>contents}\newline
\ct{WABufferedResponse>>destroy}\newline
\ct{WABufferedResponse>>initializeOn:}\newline
\ct{WABufferedResponse>>stream}\newline
\ct{WACache>>at:ifAbsent:}\newline
\ct{WACache>>expiryPolicy}\newline
\ct{WACache>>initializeCollections}\newline
\ct{WACache>>initializeMutex}\newline
\ct{WACache>>initialize}\newline
\ct{WACache>>keyAtValue:ifAbsent:}\newline
\ct{WACache>>keyAtValue:}\newline
\ct{WACache>>keySize}\newline
\ct{WACache>>missStrategy}\newline
\ct{WACache>>notifyRemoved:key:}\newline
\ct{WACache>>notifyRetrieved:key:}\newline
\ct{WACache>>notifyStored:key:}\newline
\ct{WACache>>pluginsDo:}\newline
\ct{WACache>>reapingStrategy}\newline
\ct{WACache>>reap}\newline
\ct{WACache>>removalAction}\newline
\ct{WACache>>setExpiryPolicy:}\newline
\ct{WACache>>setMissStrategy:}\newline
\ct{WACache>>setReapingStrategy:}\newline
\ct{WACache>>setRemovalAction:}\newline
\ct{WACache>>store:}\newline
\ct{WACacheCapacityConfiguration>>describeOn:}\newline
\ct{WACachePlugin>>configuration}\newline
\ct{WACachePlugin>>defaultConfiguration}\newline
\ct{WACachePlugin>>initialize}\newline
\ct{WACachePlugin>>removed:key:}\newline
\ct{WACachePlugin>>retrieved:key:}\newline
\ct{WACachePlugin>>setCache:}\newline
\ct{WACachePlugin>>stored:key:}\newline
\ct{WACacheReapingStrategy>>reap}\newline
\ct{WACallback class>>on:}\newline
\ct{WACallback>>convertKey:}\newline
\ct{WACallback>>evaluateWithFieldValues:}\newline
\ct{WACallback>>key}\newline
\ct{WACallback>>setKey:callbacks:}\newline
\ct{WACallback>>valueForField:}\newline
\ct{WACallbackRegistry>>advanceKey}\newline
\ct{WACallbackRegistry>>handle:}\newline
\ct{WACallbackRegistry>>increaseKey}\newline
\ct{WACallbackRegistry>>initialize}\newline
\ct{WACallbackRegistry>>nextKey}\newline
\ct{WACallbackRegistry>>store:}\newline
\ct{WACanvas>>brush:}\newline
\ct{WACanvas>>flush}\newline
\ct{WACanvas>>nest:}\newline
\ct{WACanvas>>render:}\newline
\ct{WACanvas>>text:}\newline
\ct{WAComponent>>accept:}\newline
\ct{WAComponent>>acceptDecorated:}\newline
\ct{WAComponent>>decoration}\newline
\ct{WAComponent>>initialize}\newline
\ct{WAComponent>>updateStates:}\newline
\ct{WAConfigurationDescription>>add:to:}\newline
\ct{WAConfigurationDescription>>addAttribute:}\newline
\ct{WAConfigurationDescription>>attributes}\newline
\ct{WAConfigurationDescription>>expressions}\newline
\ct{WAConfigurationDescription>>initialize}\newline
\ct{WAConfigurationDescription>>integer:}\newline
\ct{WAConfiguredRequestFilter>>configuration}\newline
\ct{WACounter>>count:}\newline
\ct{WACounter>>decrease}\newline
\ct{WACounter>>increase}\newline
\ct{WACounter>>initialize}\newline
\ct{WACounter>>renderContentOn:}\newline
\ct{WACounter>>states}\newline
\ct{WADefaultScriptGenerator>>close:on:}\newline
\ct{WADefaultScriptGenerator>>open:on:}\newline
\ct{WADevelopmentConfiguration>>parents}\newline
\ct{WADispatcher class>>default}\newline
\ct{WADispatcher>>handleFiltered:named:}\newline
\ct{WADispatcher>>handleFiltered:}\newline
\ct{WADispatcher>>handlerAt:ifAbsent:}\newline
\ct{WADispatcher>>handlerAt:with:}\newline
\ct{WADispatcher>>handlers}\newline
\ct{WADispatcher>>nameOfHandler:}\newline
\ct{WADispatcher>>urlFor:}\newline
\ct{WADocument class>>on:codec:}\newline
\ct{WADocument class>>on:}\newline
\ct{WADocument>>closeWADocument>>destroy}\newline
\ct{WADocument>>initializeWithStream:codec:}\newline
\ct{WADocument>>nextPut:}\newline
\ct{WADocument>>nextPutAll:}\newline
\ct{WADocument>>open:}\newline
\ct{WADynamicVariable class>>use:during:}\newline
\ct{WADynamicVariable class>>value}\newline
\ct{WAEncoder class>>on:table:}\newline
\ct{WAEncoder class>>on:}\newline
\ct{WAEncoder>>initializeOn:table:}\newline
\ct{WAEncoder>>nextPut:}\newline
\ct{WAErrorHandler class>>exceptionSelector}\newline
\ct{WAExampleComponent>>rendererClass}\newline
\ct{WAExceptionFilter>>exceptionHandler}\newline
\ct{WAExceptionFilter>>handleFiltered:}\newline
\ct{WAExceptionHandler class>>context:}\newline
\ct{WAExceptionHandler class>>exceptionSelector}\newline
\ct{WAExceptionHandler class>>handleExceptionsDuring:context:}\newline
\ct{WAExceptionHandler class>>handles:}\newline
\ct{WAExceptionHandler>>handleExceptionsDuring:}\newline
\ct{WAExceptionHandler>>handles:}\newline
\ct{WAExceptionHandler>>initializeWithContext:}\newline
\ct{WAHeaderFields>>checkValue:}\newline
\ct{WAHeaderFields>>privateAt:put:}\newline
\ct{WAHeadingTag>>initialize}\newline
\ct{WAHeadingTag>>level1}\newline
\ct{WAHeadingTag>>level}\newline
\ct{WAHeadingTag>>tag}\newline
\ct{WAHtmlAttributes>>encodeOn:}\newline
\ct{WAHtmlAttributes>>privateAt:put:}\newline
\ct{WAHtmlCanvas>>anchor}\newline
\ct{WAHtmlCanvas>>heading:}\newline
\ct{WAHtmlCanvas>>heading}\newline
\ct{WAHtmlCanvas>>spaceEntity}\newline
\ct{WAHtmlDocument>>scriptGenerator:}\newline
\ct{WAHtmlDocument>>scriptGenerator}\newline
\ct{WAHtmlElement class>>root:}\newline
\ct{WAHtmlElement>>attributeAt:put:}\newline
\ct{WAHtmlElement>>attributes}\newline
\ct{WAHtmlElement>>encodeBeforeOn:}\newline
\ct{WAHtmlElement>>encodeOn:}\newline
\ct{WAHtmlElement>>initializeWithRoot:}\newline
\ct{WAHtmlElement>>isClosed}\newline
\ct{WAHtmlRoot>>add:}\newline
\ct{WAHtmlRoot>>beXhtml10Strict}\newline
\ct{WAHtmlRoot>>bodyAttributes}\newline
\ct{WAHtmlRoot>>closeOn:}\newline
\ct{WAHtmlRoot>>docType:}\newline
\ct{WAHtmlRoot>>htmlAttributes}\newline
\ct{WAHtmlRoot>>initialize}\newline
\ct{WAHtmlRoot>>meta}\newline
\ct{WAHtmlRoot>>openOn:}\newline
\ct{WAHtmlRoot>>title:}\newline
\ct{WAHtmlRoot>>writeElementsOn:}\newline
\ct{WAHtmlRoot>>writeFootOn:}\newline
\ct{WAHtmlRoot>>writeHeadOn:}\newline
\ct{WAHtmlRoot>>writeScriptsOn:}\newline
\ct{WAHtmlRoot>>writeStylesOn:}\newline
\ct{WAHttpVersion class>>fromString:}\newline
\ct{WAHttpVersion class>>major:minor:}\newline
\ct{WAHttpVersion class>>readFrom:}\newline
\ct{WAHttpVersion>>initializeWithMajor:minor:}\newline
\ct{WAInitialRequestVisitor class>>request:}\newline
\ct{WAInitialRequestVisitor>>initializeWithRequest:}\newline
\ct{WAInitialRequestVisitor>>request}\newline
\ct{WAInitialRequestVisitor>>visitPresenter:}\newline
\ct{WAKeyGenerator class>>current}\newline
\ct{WAKeyGenerator>>keyOfLength:}\newline
\ct{WALastAccessExpiryPolicy>>}\newline
\indent\ct{defaultConfiguration}\newline
\ct{WALastAccessExpiryPolicy>>initialize}\newline
\ct{WALastAccessExpiryPolicy>>isExpired:key:}\newline
\ct{WALastAccessExpiryPolicy>>retrieved:key:}\newline
\ct{WALastAccessExpiryPolicy>>stored:key:}\newline
\ct{WALastAccessExpiryPolicy>>timeout}\newline
\ct{WALeastRecentlyUsedExpiryPolicy>>}\newline
\indent\ct{defaultConfiguration}\newline
\ct{WALeastRecentlyUsedExpiryPolicy>>}\newline
\indent\ct{initialize}\newline
\ct{WALeastRecentlyUsedExpiryPolicy>>}\newline
\indent\ct{isExpired:key:}\newline
\ct{WALeastRecentlyUsedExpiryPolicy>>}\newline
\indent\ct{maximumAge}\newline
\ct{WALeastRecentlyUsedExpiryPolicy>>}\newline
\indent\ct{removed:key:}\newline
\ct{WALeastRecentlyUsedExpiryPolicy>>}\newline
\indent\ct{retrieved:key:}\newline
\ct{WALeastRecentlyUsedExpiryPolicy>>}\newline
\indent\ct{stored:key:}\newline
\ct{WAMergedRequestFields class>>on:}\newline
\ct{WAMergedRequestFields>>allAt:}\newline
\ct{WAMergedRequestFields>>at:ifAbsent:}\newline
\ct{WAMergedRequestFields>>includesKey:}\newline
\ct{WAMergedRequestFields>>initializeOn:}\newline
\ct{WAMergedRequestFields>>keysDo:}\newline
\ct{WAMetaElement>>content:}\newline
\ct{WAMetaElement>>contentScriptType:}\newline
\ct{WAMetaElement>>contentType:}\newline
\ct{WAMetaElement>>encodeBeforeOn:}\newline
\ct{WAMetaElement>>responseHeaderName:}\newline
\ct{WAMetaElement>>tag}\newline
\ct{WAMimeType class>>fromString:}\newline
\ct{WAMimeType class>>main:sub:}\newline
\ct{WAMimeType class>>textJavascript}\newline
\ct{WAMimeType class>>textPlain}\newline
\ct{WAMimeType>>charset:}\newline
\ct{WAMimeType>>greaseString}\newline
\ct{WAMimeType>>main:}\newline
\ct{WAMimeType>>main}\newline
\ct{WAMimeType>>parameters}\newline
\ct{WAMimeType>>sub:}\newline
\ct{WAMimeType>>sub}\newline
\ct{WAMutex>>critical:}\newline
\ct{WAMutex>>initialize}\newline
\ct{WAMutualExclusionFilter>>handleFiltered:}\newline
\ct{WAMutualExclusionFilter>>initialize}\newline
\ct{WAMutualExclusionFilter>>shouldTerminate:}\newline
\ct{WANotifyRemovalAction>>removed:key:}\newline
\ct{WAObject>>application}\newline
\ct{WAObject>>requestContext}\newline
\ct{WAObject>>session}\newline
\ct{WAPainter>>renderWithContext:}\newline
\ct{WAPainter>>updateRoot:}\newline
\ct{WAPainter>>updateUrl:}\newline
\ct{WAPainterVisitor>>visitComponent:}\newline
\ct{WAPainterVisitor>>visitDecorationsOfComponent:}\newline
\ct{WAPainterVisitor>>visitPainter:}\newline
\ct{WAPainterVisitor>>visitPresenter:}\newline
\ct{WAPathConsumer class>>path:}\newline
\ct{WAPathConsumer>>atEnd}\newline
\ct{WAPathConsumer>>initializeWith:}\newline
\ct{WAPathConsumer>>next}\newline
\ct{WAPresenter>>childrenDo:}\newline
\ct{WAPresenter>>children}\newline
\ct{WAPresenter>>initialRequest:}\newline
\ct{WAPresenter>>script}\newline
\ct{WAPresenter>>style}\newline
\ct{WAPresenter>>updateRoot:}\newline
\ct{WAPresenter>>updateStates:}\newline
\ct{WAPresenterGuide class>>client:}\newline
\ct{WAPresenterGuide>>client}\newline
\ct{WAPresenterGuide>>initializeWithClient:}\newline
\ct{WAPresenterGuide>>visit:}\newline
\ct{WAPresenterGuide>>visitPainter:}\newline
\ct{WARegistryConfiguration>>parents}\newline
\ct{WARenderContext>>actionBaseUrl:}\newline
\ct{WARenderContext>>actionUrl:}\newline
\ct{WARenderContext>>actionUrl}\newline
\ct{WARenderContext>>callbacks}\newline
\ct{WARenderContext>>defaultVisitor}\newline
\ct{WARenderContext>>destroy}\newline
\ct{WARenderContext>>document:}\newline
\ct{WARenderContext>>document}\newline
\ct{WARenderContext>>initialize}\newline
\ct{WARenderContext>>resourceUrl:}\newline
\ct{WARenderContext>>visitor:}\newline
\ct{WARenderContext>>visitor}\newline
\ct{WARenderLoopConfiguration>>parents}\newline
\ct{WARenderLoopContinuation>>createActionContinuation}\newline
\ct{WARenderLoopContinuation>>createRenderContinuation}\newline
\ct{WARenderLoopContinuation>>presenter}\newline
\ct{WARenderLoopContinuation>>toPresenterSendRoot:}\newline
\ct{WARenderLoopContinuation>>updateRoot:}\newline
\ct{WARenderLoopContinuation>>updateStates:}\newline
\ct{WARenderLoopContinuation>>updateUrl:}\newline
\ct{WARenderLoopContinuation>>withNotificationHandlerDo:}\newline
\ct{WARenderLoopMain>>createRoot}\newline
\ct{WARenderLoopMain>>prepareRoot:}\newline
\ct{WARenderLoopMain>>rootClass}\newline
\ct{WARenderLoopMain>>rootDecorationClasses}\newline
\ct{WARenderLoopMain>>start}\newline
\ct{WARenderPhaseContinuation>>createHtmlRootWithContext:}\newline
\ct{WARenderPhaseContinuation>>createRenderContext}\newline
\ct{WARenderPhaseContinuation>>handleRequest}\newline
\ct{WARenderPhaseContinuation>>}\newline
\indent\ct{processRendering:}\newline
\ct{WARenderVisitor class>>context:}\newline
\ct{WARenderVisitor>>initializeWithContext:}\newline
\ct{WARenderVisitor>>renderContext}\newline
\ct{WARenderVisitor>>visitPainter:}\newline
\ct{WARenderer class>>context:}\newline
\ct{WARenderer>>actionUrl}\newline
\ct{WARenderer>>callbacks}\newline
\ct{WARenderer>>context}\newline
\ct{WARenderer>>document}\newline
\ct{WARenderer>>flush}\newline
\ct{WARenderer>>initializeWithContext:}\newline
\ct{WARenderer>>render:}\newline
\ct{WARenderer>>text:}\newline
\ct{WARequest class>>method:uri:version:}\newline
\ct{WARequest>>at:ifAbsent:}\newline
\ct{WARequest>>cookiesAt:}\newline
\ct{WARequest>>cookies}\newline
\ct{WARequest>>destroy}\newline
\ct{WARequest>>fields}\newline
\ct{WARequest>>headerAt:ifAbsent:}\newline
\ct{WARequest>>headerAt:}\newline
\ct{WARequest>>}\newline
\indent\ct{initializeWithMethod:uri:version:}\newline
\ct{WARequest>>isGet}\newline
\ct{WARequest>>isPrefetch}\newline
\ct{WARequest>>isXmlHttpRequest}\newline
\ct{WARequest>>method}\newline
\ct{WARequest>>postFields}\newline
\ct{WARequest>>queryFields}\newline
\ct{WARequest>>setBody:}\newline
\ct{WARequest>>setCookies:}\newline
\ct{WARequest>>setHeaders:}\newline
\ct{WARequest>>setPostFields:}\newline
\ct{WARequest>>setRemoteAddress:}\newline
\ct{WARequest>>uri}\newline
\ct{WARequest>>url}\newline
\ct{WARequestContext class>>}\newline
\indent\ct{request:response:codec:}\newline
\ct{WARequestContext>>application}\newline
\ct{WARequestContext>>charSet}\newline
\ct{WARequestContext>>codec}\newline
\ct{WARequestContext>>consumer}\newline
\ct{WARequestContext>>destroy}\newline
\ct{WARequestContext>>handlers}\newline
\ct{WARequestContext>>handler}\newline
\ct{WARequestContext>>}\newline
\indent\ct{initializeWithRequest:response:codec:}\newline
\ct{WARequestContext>>newDocument}\newline
\ct{WARequestContext>>push:during:}\newline
\ct{WARequestContext>>request}\newline
\ct{WARequestContext>>respond:}\newline
\ct{WARequestContext>>respond}\newline
\ct{WARequestContext>>responseGenerator}\newline
\ct{WARequestContext>>response}\newline
\ct{WARequestContext>>session}\newline
\ct{WARequestFilter>>handleFiltered:}\newline
\ct{WARequestFilter>>initialize}\newline
\ct{WARequestFilter>>next}\newline
\ct{WARequestFilter>>setNext:}\newline
\ct{WARequestFilter>>updateStates:}\newline
\ct{WARequestHandler>>addFilter:}\newline
\ct{WARequestHandler>>addFilterLast:}\newline
\ct{WARequestHandler>>basicUrl}\newline
\ct{WARequestHandler>>configuration:}\newline
\ct{WARequestHandler>>configuration}\newline
\ct{WARequestHandler>>defaultConfiguration}\newline
\ct{WARequestHandler>>documentClass}\newline
\ct{WARequestHandler>>filters}\newline
\ct{WARequestHandler>>filter}\newline
\ct{WARequestHandler>>handle:}\newline
\ct{WARequestHandler>>initialize}\newline
\ct{WARequestHandler>>isApplication}\newline
\ct{WARequestHandler>>isRoot}\newline
\ct{WARequestHandler>>parent}\newline
\ct{WARequestHandler>>preferenceAt:}\newline
\ct{WARequestHandler>>responseGenerator}\newline
\ct{WARequestHandler>>serverHostname}\newline
\ct{WARequestHandler>>serverPath}\newline
\ct{WARequestHandler>>serverPort}\newline
\ct{WARequestHandler>>serverProtocol}\newline
\ct{WARequestHandler>>setFilter:}\newline
\ct{WARequestHandler>>setParent:}\newline
\ct{WARequestHandler>>url}\newline
\ct{WAResponse class>>messageForStatus:}\newline
\ct{WAResponse class>>statusFound}\newline
\ct{WAResponse class>>statusNotFound}\newline
\ct{WAResponse>>contentType:}\newline
\ct{WAResponse>>contentType}\newline
\ct{WAResponse>>cookies}\newline
\ct{WAResponse>>destroy}\newline
\ct{WAResponse>>found}\newline
\ct{WAResponse>>headerAt:ifAbsent:}\newline
\ct{WAResponse>>headerAt:put:}\newline
\ct{WAResponse>>headers}\newline
\ct{WAResponse>>initializeOn:}\newline
\ct{WAResponse>>initialize}\newline
\ct{WAResponse>>location:}\newline
\ct{WAResponse>>nextPutAll:}\newline
\ct{WAResponse>>notFound}\newline
\ct{WAResponse>>redirectTo:}\newline
\ct{WAResponse>>status:message:}\newline
\ct{WAResponse>>status:}\newline
\ct{WAResponse>>status}\newline
\ct{WAResponseGenerator class>>on:ShiftJIS}\newline
\ct{WAResponseGenerator>>initializeOn:}\newline
\ct{WAResponseGenerator>>notFound}\newline
\ct{WAResponseGenerator>>requestContext}\newline
\ct{WAResponseGenerator>>request}\newline
\ct{WAResponseGenerator>>respond}\newline
\ct{WAResponseGenerator>>response}\newline
\ct{WARoot class>>context:}\newline
\ct{WARoot>>setContext:}\newline
\ct{WAScriptGenerator>>initialize}\newline
\ct{WAScriptGenerator>>loadScripts}\newline
\ct{WAScriptGenerator>>writeLoadScriptsOn:}\newline
\ct{WAScriptGenerator>>writeScriptTag:on:}\newline
\ct{WAServerAdaptor class>>defaultSmall}\newline
\ct{WAServerAdaptor class>>manager:}\newline
\ct{WAServerAdaptor class>>new}\newline
\ct{WAServerAdaptor class>>port:}\newline
\ct{WAServerAdaptor class>>startOn:}\newline
\ct{WAServerAdaptor>>codec}\newline
\ct{WAServerAdaptor>>contextFor:}\newline
\ct{WAServerAdaptor>>defaultPort}\newline
\ct{WAServerAdaptor>>defaultRequestHandler}\newline
\ct{WAServerAdaptor>>handle:}\newline
\ct{WAServerAdaptor>>handlePadding:}\newline
\ct{WAServerAdaptor>>handleRequest:}\newline
\ct{WAServerAdaptor>>initializeWithManager:}\newline
\ct{WAServerAdaptor>>initialize}\newline
\ct{WAServerAdaptor>>manager}\newline
\ct{WAServerAdaptor>>port:}\newline
\ct{WAServerAdaptor>>port}\newline
\ct{WAServerAdaptor>>process:}\newline
\ct{WAServerAdaptor>>requestFor:}\newline
\ct{WAServerAdaptor>>requestHandler}\newline
\ct{WAServerAdaptor>>responseFor:}\newline
\ct{WAServerAdaptor>>start}\newline
\ct{WAServerManager class>>default}\newline
\ct{WAServerManager class>>initialize}\newline
\ct{WAServerManager class>>shutDown}\newline
\ct{WAServerManager class>>startUp}\newline
\ct{WAServerManager>>adaptors}\newline
\ct{WAServerManager>>canStart:}\newline
\ct{WAServerManager>>register:}\newline
\ct{WAServerManager>>start:}\newline
\ct{WASession>>actionField}\newline
\ct{WASession>>actionUrlForContinuation:}\newline
\ct{WASession>>actionUrlForKey:}\newline
\ct{WASession>>application}\newline
\ct{WASession>>clearJumpTo}\newline
\ct{WASession>>createCache}\newline
\ct{WASession>>handleFiltered:}\newline
\ct{WASession>>initializeFilters}\newline
\ct{WASession>>initialize}\newline
\ct{WASession>>isSession}\newline
\ct{WASession>>presenter}\newline
\ct{WASession>>properties}\newline
\ct{WASession>>start}\newline
\ct{WASession>>updateRoot:}\newline
\ct{WASession>>updateStates:}\newline
\ct{WASession>>updateUrl:}\newline
\ct{WASession>>url}\newline
\ct{WASessionContinuation>>basicValue}\newline
\ct{WASessionContinuation>>captureAndInvoke}\newline
\ct{WASessionContinuation>>captureState}\newline
\ct{WASessionContinuation>>redirectToContinuation:}\newline
\ct{WASessionContinuation>>registerForUrl:}\newline
\ct{WASessionContinuation>>registerForUrl}\newline
\ct{WASessionContinuation>>request}\newline
\ct{WASessionContinuation>>respond:}\newline
\ct{WASessionContinuation>>setStates:}\newline
\ct{WASessionContinuation>>states}\newline
\ct{WASessionContinuation>>updateStates:}\newline
\ct{WASessionContinuation>>updateUrl:}\newline
\ct{WASessionContinuation>>value}\newline
\ct{WASessionContinuation>>withUnregisteredHandlerDo:}\newline
\ct{WASlime class>>initialize}\newline
\ct{WASlime class>>startUp}\newline
\ct{WASnapshot>>initialize}\newline
\ct{WASnapshot>>register:}\newline
\ct{WASnapshot>>resetWASnapshot>>restore}\newline
\ct{WATagBrush>>after}\newline
\ct{WATagBrush>>attributes}\newline
\ct{WATagBrush>>before}\newline
\ct{WATagBrush>>closeTag}\newline
\ct{WATagBrush>>document}\newline
\ct{WATagBrush>>isClosed}\newline
\ct{WATagBrush>>openTag}\newline
\ct{WATagBrush>>storeCallback:}\newline
\ct{WATagBrush>>with:}\newline
\ct{WAUpdateRootVisitor class>>root:WeakKey}\newline
\ct{WAUpdateRootVisitor>>initializeWithRoot:}\newline
\ct{WAUpdateRootVisitor>>root}\newline
\ct{WAUpdateRootVisitor>>visitPainter:}\newline
\ct{WAUpdateStatesVisitor class>>snapshot:}\newline
\ct{WAUpdateStatesVisitor>>initializeWithSnapshot:}\newline
\ct{WAUpdateStatesVisitor>>snapshot}\newline
\ct{WAUpdateStatesVisitor>>visitPresenter:}\newline
\ct{WAUpdateUrlVisitor class>>url:}\newline
\ct{WAUpdateUrlVisitor>>initializeWithUrl:}\newline
\ct{WAUpdateUrlVisitor>>url}\newline
\ct{WAUpdateUrlVisitor>>visitPainter:}\newline
\ct{WAUrl class>>absolute:}\newline
\ct{WAUrl class>>decodePercent:}\newline
\ct{WAUrl>>addField:value:}\newline
\ct{WAUrl>>addField:}\newline
\ct{WAUrl>>addToPath:}\newline
\ct{WAUrl>>decode:}\newline
\ct{WAUrl>>decodedWith:}\newline
\ct{WAUrl>>encodeOn:}\newline
\ct{WAUrl>>encodePathOn:}\newline
\ct{WAUrl>>encodeQueryOn:}\newline
\ct{WAUrl>>encodeSchemeAndAuthorityOn:}\newline
\ct{WAUrl>>fragment}\newline
\ct{WAUrl>>initializeFromString:}\newline
\ct{WAUrl>>initialize}\newline
\ct{WAUrl>>parsePath:}\newline
\ct{WAUrl>>parseQuery:}\newline
\ct{WAUrl>>password}\newline
\ct{WAUrl>>path:}\newline
\ct{WAUrl>>pathElementsIn:do:}\newline
\ct{WAUrl>>pathString}\newline
\ct{WAUrl>>path}\newline
\ct{WAUrl>>postCopy}\newline
\ct{WAUrl>>printOn:}\newline
\ct{WAUrl>>queryFields:}\newline
\ct{WAUrl>>queryFields}\newline
\ct{WAUrl>>seasideUrl}\newline
\ct{WAUrl>>slash:}\newline
\ct{WAUrl>>subStringsIn:splitBy:do:}\newline
\ct{WAUrl>>user}\newline
\ct{WAUrlEncoder class>>on:codec:}\newline
\ct{WAUrlEncoder>>nextPutAll:}\newline
\ct{WAUserConfiguration>>addParent:}\newline
\ct{WAUserConfiguration>>canAddParent:}\newline
\ct{WAUserConfiguration>>expressionAt:ifAbsent:}\newline
\ct{WAUserConfiguration>>initialize}\newline
\ct{WAUserConfiguration>>}\newline
\indent\ct{localAttributeAt:ifAbsent:}\newline
\ct{WAUserConfiguration>>parents}\newline
\ct{WAValueExpression>>}\newline
\indent\ct{determineValueWithContext:}\newline
\indent\ct{configuration:}\newline
\ct{WAValueExpression>>value}\newline
\ct{WAValueHolder class>>with:}\newline
\ct{WAValueHolder>>contents:}\newline
\ct{WAValueHolder>>contents}\newline
\ct{WAVisiblePresenterGuide>>visitPresenter:}\newline
\ct{WAVisitor>>visit:WAVisitor>>start:}\newline
\ct{WAXmlDocument>>closeTag:}\newline
\ct{WAXmlDocument>>destroy}\newline
\ct{WAXmlDocument>>}\newline
\indent\ct{initializeWithStream:codec:}\newline
\ct{WAXmlDocument>>}\newline
\indent\ct{openTag:attributes:closed:}\newline
\ct{WAXmlDocument>>}\newline
\indent\ct{openTag:attributes:}\newline
\ct{WAXmlDocument>>openTag:}\newline
\ct{WAXmlDocument>>print:}\newline
\ct{WAXmlDocument>>urlEncoder}\newline
\ct{WAXmlDocument>>xmlEncoder}\newline
\ct{WAXmlEncoder>>nextPutAll:}\newline
\ct{WeakAnnouncementSubscription class>>}\newline
\indent\ct{finalizationList}\newline
\ct{WeakAnnouncementSubscription class>>}\newline
\indent\ct{finalizeValues}\newline
\ct{WeakArray class>>finalizationProcess}\newline
\ct{WeakArray class>>initialize}\newline
\ct{WeakArray class>>restartFinalizationProcess}\newline
\ct{WeakArray class>>startUp:}\newline
\ct{WeakFinalizationList class>>hasNewFinalization}\newline
\ct{WeakFinalizationList class>>initialize}\newline
\ct{WeakFinalizationList class>>startUp:}\newline
\ct{WeakFinalizationList>>swapWithNil}\newline
\ct{WeakFinalizerItem class>>new}\newline
\ct{WeakFinalizerItem>>add:}\newline
\ct{WeakFinalizerItem>>list:object:}\newline
\ct{WeakIdentityKeyDictionary>>compare:to:}\newline
\ct{WeakIdentityKeyDictionary>>startIndexFor:}\newline
\ct{WeakKeyDictionary>>associationsDo:}\newline
\ct{WeakKeyDictionary>>finalizeValues}\newline
\ct{WeakKeyDictionary>>fullCheck}\newline
\ct{WeakKeyDictionary>>grow}\newline
\ct{WeakKeyDictionary>>initialize:}\newline
\ct{WeakKeyDictionary>>noCheckAdd:}\newline
\ct{WeakKeyDictionary>>overridingAt:put:}\newline
\ct{WeakKeyDictionary>>rehash}\newline
\ct{WeakKeyDictionary>>removeKey:ifAbsent:}\newline
\ct{WeakKeyDictionary>>scanFor:}\newline
\ct{WeakKeyDictionary>>scanForKeyOrNil:}\newline
\ct{WeakRegistry>>add:executor:}\newline
\ct{WeakRegistry>>add:}\newline
\ct{WeakRegistry>>finalizeValues}\newline
\ct{WeakRegistry>>protected:}\newline
\ct{WeakRegistry>>remove:ifAbsent:}\newline
\ct{WeakSet>>add:}\newline
\ct{WeakSet>>do:}\newline
\ct{WeakSet>>growTo:}\newline
\ct{WeakSet>>grow}\newline
\ct{WeakSet>>initialize:}\newline
\ct{WeakSet>>like:}\newline
\ct{WeakSet>>noCheckNoGrowFillFrom:}\newline
\ct{WeakSet>>scanFor:}\newline
\ct{WeakSet>>scanForEmptySlotFor:}\newline
\ct{Week class>>nameOfDay:}\newline
\ct{WideString>>at:}\newline
\ct{WideString>>wordAt:}\newline
\ct{WriteStream>><<}\newline
\ct{WriteStream>>contents}\newline
\ct{WriteStream>>cr}\newline
\ct{WriteStream>>growTo:}\newline
\ct{WriteStream>>nextPut:}\newline
\ct{WriteStream>>nextPutAll:}\newline
\ct{WriteStream>>on:}\newline
\ct{WriteStream>>pastEndPut:}\newline
\ct{WriteStream>>reset}\newline
\ct{WriteStream>>size}\newline
\ct{WriteStream>>space}\newline
\ct{ZdcAbstractSocketStream class>>on:}\newline
\ct{ZdcAbstractSocketStream>>SocketSendData:startingAt:count:}\newline
\ct{ZdcAbstractSocketStream>>SocketWaitForData}\newline
\ct{ZdcAbstractSocketStream>>autoFlush:}\newline
\ct{ZdcAbstractSocketStream>>binary}\newline
\ct{ZdcAbstractSocketStream>>bufferSize:}\newline
\ct{ZdcAbstractSocketStream>>close}\newline
\ct{ZdcAbstractSocketStream>>flush}\newline
\ct{ZdcAbstractSocketStream>>initializeBuffers}\newline
\ct{ZdcAbstractSocketStream>>initialize}\newline
\ct{ZdcAbstractSocketStream>>isBinary}\newline
\ct{ZdcAbstractSocketStream>>nextPut:}\newline
\ct{ZdcAbstractSocketStream>>nextPutAll:}\newline
\ct{ZdcAbstractSocketStream>>next}\newline
\ct{ZdcAbstractSocketStream>>on:}\newline
\ct{ZdcAbstractSocketStream>>peek}\newline
\ct{ZdcAbstractSocketStream>>shouldSignal:}\newline
\ct{ZdcAbstractSocketStream>>socketClose}\newline
\ct{ZdcAbstractSocketStream>>}\newline
\indent\ct{socketIsConnected}\newline
\ct{ZdcAbstractSocketStream>>}\newline
\indent\ct{socketIsDataAvailable}\newline
\ct{ZdcAbstractSocketStream>>}\newline
\indent\ct{socketReceiveDataInto:startingAt:count:}\newline
\ct{ZdcAbstractSocketStream>>socket}\newline
\ct{ZdcAbstractSocketStream>>timeout:}\newline
\ct{ZdcAbstractSocketStream>>timeout}\newline
\ct{ZdcIOBuffer class>>on:}\newline
\ct{ZdcIOBuffer class>>onByteArrayOfSize:}\newline
\ct{ZdcIOBuffer>>advanceWritePointer:}\newline
\ct{ZdcIOBuffer>>availableForReading}\newline
\ct{ZdcIOBuffer>>availableForWriting}\newline
\ct{ZdcIOBuffer>>bufferSize}\newline
\ct{ZdcIOBuffer>>buffer}\newline
\ct{ZdcIOBuffer>>compact}\newline
\ct{ZdcIOBuffer>>contentsStart}\newline
\ct{ZdcIOBuffer>>freeSpaceStart}\newline
\ct{ZdcIOBuffer>>isEmpty}\newline
\ct{ZdcIOBuffer>>isFull}\newline
\ct{ZdcIOBuffer>>next:putAll:startingAt:}\newline
\ct{ZdcIOBuffer>>nextPut:}\newline
\ct{ZdcIOBuffer>>next}\newline
\ct{ZdcIOBuffer>>on:}\newline
\ct{ZdcIOBuffer>>peek}\newline
\ct{ZdcIOBuffer>>reset}\newline
\ct{ZdcSimpleSocketStream>>atEnd}\newline
\ct{ZdcSimpleSocketStream>>}\newline
\indent\ct{fillBytes:startingAt:count:}\newline
\ct{ZdcSimpleSocketStream>>}\newline
\indent\ct{fillReadBufferNoWait}\newline
\ct{ZdcSimpleSocketStream>>}\newline
\indent\ct{fillReadBuffer}\newline
\ct{ZdcSimpleSocketStream>>}\newline
\indent\ct{flushBytes:startingAt:count:}\newline
\ct{ZdcSimpleSocketStream>>}\newline
\indent\ct{flushWriteBuffer}\newline
\ct{ZdcSimpleSocketStream>>}\newline
\indent\ct{isConnected}\newline
\ct{ZdcSocketStream>>next:putAll:startingAt:}\newline
\ct{ZnBivalentWriteStream class>>on:}\newline
\ct{ZnBivalentWriteStream>>isBinary}\newline
\ct{ZnBivalentWriteStream>>nextPut:}\newline
\ct{ZnBivalentWriteStream>>nextPutAll:}\newline
\ct{ZnBivalentWriteStream>>on:}\newline
\ct{ZnBivalentWriteStream>>space}\newline
\ct{ZnByteArrayEntity class>>bytes:}\newline
\ct{ZnByteArrayEntity class>>designatedMimeType}\newline
\ct{ZnByteArrayEntity>>bytes:}\newline
\ct{ZnByteArrayEntity>>bytes}\newline
\ct{ZnByteArrayEntity>>writeOn:}\newline
\ct{ZnCharacterEncoder class>>newForEncoding:}\newline
\ct{ZnCharacterEncoder>>beLenient}\newline
\ct{ZnCharacterEncoder>>encodedByteCountForString:}\newline
\ct{ZnConstants class>>defaultHTTPVersion}\newline
\ct{ZnConstants class>>defaultMaximumEntitySize}\newline
\ct{ZnConstants class>>defaultServerString}\newline
\ct{ZnConstants class>>frameworkNameAndVersion}\newline
\ct{ZnConstants class>>frameworkName}\newline
\ct{ZnConstants class>>frameworkVersion}\newline
\ct{ZnConstants class>>httpStatusCodes}\newline
\ct{ZnConstants class>>knownHTTPMethods}\newline
\ct{ZnConstants class>>knownHTTPVersions}\newline
\ct{ZnConstants class>>maximumLineLength}\newline
\ct{ZnConstants class>>remoteAddressHeader}\newline
\ct{ZnEntity class>>byteArrayEntityClass}\newline
\ct{ZnEntity class>>bytes:}\newline
\ct{ZnEntity class>>new}\newline
\ct{ZnEntity class>>stringEntityClass}\newline
\ct{ZnEntity class>>text:}\newline
\ct{ZnEntity class>>textCRLF:}\newline
\ct{ZnEntity class>>type:length:}\newline
\ct{ZnEntity class>>type:}\newline
\ct{ZnEntity>>contentLength:}\newline
\ct{ZnEntity>>contentLength}\newline
\ct{ZnEntity>>contentType:}\newline
\ct{ZnEntity>>contentType}\newline
\ct{ZnEntityReader>>allowsReadingUpToEnd}\newline
\ct{ZnEntityReader>>binary}\newline
\ct{ZnEntityReader>>canReadContent}\newline
\ct{ZnEntityReader>>hasContentLength}\newline
\ct{ZnEntityReader>>headers:}\newline
\ct{ZnEntityReader>>headers}\newline
\ct{ZnEntityReader>>isChunked}\newline
\ct{ZnEntityReader>>readEntity}\newline
\ct{ZnEntityReader>>stream:}\newline
\ct{ZnEntityWriter>>headers:}\newline
\ct{ZnEntityWriter>>headers}\newline
\ct{ZnEntityWriter>>isChunked}\newline
\ct{ZnEntityWriter>>isGzipped}\newline
\ct{ZnEntityWriter>>stream:}\newline
\ct{ZnEntityWriter>>writeEntity:}\newline
\ct{ZnHeaders class>>defaultResponseHeaders}\newline
\ct{ZnHeaders class>>readFrom:}\newline
\ct{ZnHeaders>>acceptEntityDescription:}\newline
\ct{ZnHeaders>>at:add:}\newline
\ct{ZnHeaders>>at:ifAbsent:}\newline
\ct{ZnHeaders>>at:put:}\newline
\ct{ZnHeaders>>clearContentLength}\newline
\ct{ZnHeaders>>clearContentType}\newline
\ct{ZnHeaders>>contentLength:}\newline
\ct{ZnHeaders>>contentType:}\newline
\ct{ZnHeaders>>headersDo:}\newline
\ct{ZnHeaders>>headers}\newline
\ct{ZnHeaders>>includesKey:}\newline
\ct{ZnHeaders>>isDescribingEntity}\newline
\ct{ZnHeaders>>isEmpty}\newline
\ct{ZnHeaders>>normalizeHeaderKey:}\newline
\ct{ZnHeaders>>readFrom:}\newline
\ct{ZnHeaders>>readOneHeaderFrom:}\newline
\ct{ZnHeaders>>removeKey:ifAbsent:}\newline
\ct{ZnHeaders>>writeOn:}\newline
\ct{ZnLineReader class>>on:}\newline
\ct{ZnLineReader>>growBuffer}\newline
\ct{ZnLineReader>>limit:}\newline
\ct{ZnLineReader>>nextLine}\newline
\ct{ZnLineReader>>on:}\newline
\ct{ZnLineReader>>processNext}\newline
\ct{ZnLineReader>>reset}\newline
\ct{ZnLineReader>>store:}\newline
\ct{ZnLogSupport>>debug:}\newline
\ct{ZnLogSupport>>disable}\newline
\ct{ZnLogSupport>>enabled:}\newline
\ct{ZnLogSupport>>enabled}\newline
\ct{ZnLogSupport>>info:}\newline
\ct{ZnLogSupport>>initialize}\newline
\ct{ZnLogSupport>>transaction:}\newline
\ct{ZnManagingMultiThreadedServer>>}\newline
\indent\ct{closeConnections}\newline
\ct{ZnManagingMultiThreadedServer>>}\newline
\indent\ct{closeSocketStream:}\newline
\ct{ZnManagingMultiThreadedServer>>}\newline
\indent\ct{connections}\newline
\ct{ZnManagingMultiThreadedServer>>}\newline
\indent\ct{lock}\newline
\ct{ZnManagingMultiThreadedServer>>}\newline
\indent\ct{socketStreamOn:}\newline
\ct{ZnManagingMultiThreadedServer>>}\newline
\indent\ct{stop:}\newline
\ct{ZnMessage class>>readBinaryFrom:UTF8}\newline
\ct{ZnMessage>>entity:}\newline
\ct{ZnMessage>>entityReaderOn:}\newline
\ct{ZnMessage>>entityWriterOn:}\newline
\ct{ZnMessage>>entity}\newline
\ct{ZnMessage>>hasEntity}\newline
\ct{ZnMessage>>hasHeaders}\newline
\ct{ZnMessage>>headers:}\newline
\ct{ZnMessage>>headersDo:}\newline
\ct{ZnMessage>>headers}\newline
\ct{ZnMessage>>isConnectionClose}\newline
\ct{ZnMessage>>readBinaryFrom:}\newline
\ct{ZnMessage>>readHeaderFrom:}\newline
\ct{ZnMessage>>setConnectionClose}\newline
\ct{ZnMessage>>wantsConnectionClose}\newline
\ct{ZnMessage>>writeOn:}\newline
\ct{ZnMimeType class>>applicationOctetStream}\newline
\ct{ZnMimeType class>>fromString:}\newline
\ct{ZnMimeType class>>main:sub:parameters:}\newline
\ct{ZnMimeType class>>main:sub:}\newline
\ct{ZnMimeType class>>textPlain}\newline
\ct{ZnMimeType>>=}\newline
\ct{ZnMimeType>>asZnMimeType}\newline
\ct{ZnMimeType>>charSet:}\newline
\ct{ZnMimeType>>charSet}\newline
\ct{ZnMimeType>>main:}\newline
\ct{ZnMimeType>>main}\newline
\ct{ZnMimeType>>parameterAt:ifAbsent:}\newline
\ct{ZnMimeType>>parameters:}\newline
\ct{ZnMimeType>>parameters}\newline
\ct{ZnMimeType>>printOn:}\newline
\ct{ZnMimeType>>setCharSetUTF8}\newline
\ct{ZnMimeType>>sub:}\newline
\ct{ZnMimeType>>sub}\newline
\ct{ZnMultiThreadedServer>>augmentResponse:forRequest:}\newline
\ct{ZnMultiThreadedServer>>closeSocketStream:}\newline
\ct{ZnMultiThreadedServer>>exceptionSet:}\newline
\ct{ZnMultiThreadedServer>>executeOneRequestResponseOn:}\newline
\ct{ZnMultiThreadedServer>>executeRequestResponseLoopOn:}\newline
\ct{ZnMultiThreadedServer>>listenLoop}\newline
\ct{ZnMultiThreadedServer>>readRequestBadExceptionSet}\newline
\ct{ZnMultiThreadedServer>>readRequestSafely:}\newline
\ct{ZnMultiThreadedServer>>readRequestTerminationExceptionSet}\newline
\ct{ZnMultiThreadedServer>>serveConnectionsOn:}\newline
\ct{ZnMultiThreadedServer>>workerProcessName}\newline
\ct{ZnMultiThreadedServer>>writeResponseBad:on:}\newline
\ct{ZnMultiThreadedServer>>writeResponseSafely:on:}\newline
\ct{ZnMultiThreadedServer>>writeResponseTerminationExceptionSet}\newline
\ct{ZnMultiValueDictionary>>at:add:}\newline
\ct{ZnMultiValueDictionary>>at:put:}\newline
\ct{ZnMultiValueDictionary>>checkLimitForKey:}\newline
\ct{ZnMultiValueDictionary>>defaultLimit}\newline
\ct{ZnMultiValueDictionary>>initialize:}\newline
\ct{ZnMultiValueDictionary>>keysAndValuesDo:}\newline
\ct{ZnMultiValueDictionary>>limit}\newline
\ct{ZnNetworkingUtils class>>defaultSocketStreamTimeout}\newline
\ct{ZnNetworkingUtils class>>default}\newline
\ct{ZnNetworkingUtils class>>ipAddressToString:}\newline
\ct{ZnNetworkingUtils class>>listenBacklogSize}\newline
\ct{ZnNetworkingUtils class>>serverSocketOn:}\newline
\ct{ZnNetworkingUtils class>>socketBufferSize}\newline
\ct{ZnNetworkingUtils class>>socketStreamOn:}\newline
\ct{ZnNetworkingUtils class>>socketStreamTimeout}\newline
\ct{ZnNetworkingUtils>>bufferSize}\newline
\ct{ZnNetworkingUtils>>serverSocketOn:}\newline
\ct{ZnNetworkingUtils>>setServerSocketOptions:}\newline
\ct{ZnNetworkingUtils>>setSocketStreamParameters:}\newline
\ct{ZnNetworkingUtils>>socketStreamClass}\newline
\ct{ZnNetworkingUtils>>socketStreamOn:}\newline
\ct{ZnNetworkingUtils>>timeout}\newline
\ct{ZnNullEncoder class>>handlesEncoding:}\newline
\ct{ZnPercentEncoder>>characterEncoder:}\newline
\ct{ZnPercentEncoder>>characterEncoder}\newline
\ct{ZnPercentEncoder>>decode:to:}\newline
\ct{ZnPercentEncoder>>decode:}\newline
\ct{ZnPercentEncoder>>encode:to:}\newline
\ct{ZnPercentEncoder>>encode:}\newline
\ct{ZnPercentEncoder>>safeSet:}\newline
\ct{ZnPercentEncoder>>safeSet}\newline
\ct{ZnRequest>>isHttp10}\newline
\ct{ZnRequest>>method}\newline
\ct{ZnRequest>>readHeaderFrom:}\newline
\ct{ZnRequest>>requestLine:}\newline
\ct{ZnRequest>>requestLine}\newline
\ct{ZnRequest>>uri}\newline
\ct{ZnRequest>>wantsConnectionClose}\newline
\ct{ZnRequestLine class>>readFrom:}\newline
\ct{ZnRequestLine>>isHttp10}\newline
\ct{ZnRequestLine>>method:}\newline
\ct{ZnRequestLine>>method}\newline
\ct{ZnRequestLine>>readFrom:}\newline
\ct{ZnRequestLine>>uri:}\newline
\ct{ZnRequestLine>>uri}\newline
\ct{ZnRequestLine>>version:}\newline
\ct{ZnRequestLine>>version}\newline
\ct{ZnResourceMetaUtils class>>}\newline
\indent\ct{decodePercent:}\newline
\ct{ZnResourceMetaUtils class>>}\newline
\indent\ct{encodePercent:safeSet:encoding:}\newline
\ct{ZnResourceMetaUtils class>>}\newline
\indent\ct{parseQueryFrom:}\newline
\ct{ZnResourceMetaUtils class>>}\newline
\indent\ct{queryKeyValueSafeSet}\newline
\ct{ZnResourceMetaUtils class>>}\newline
\indent\ct{urlPathSafeSet}\newline
\ct{ZnResourceMetaUtils class>>}\newline
\indent\ct{writeQueryFields:on:}\newline
\ct{ZnResourceMetaUtils class>>}\newline
\indent\ct{writeQueryFields:withTextEncoding:on:}\newline
\ct{ZnResponse>>setConnectionCloseFor:}\newline
\ct{ZnResponse>>setKeepAliveFor:}\newline
\ct{ZnResponse>>statusLine:}\newline
\ct{ZnResponse>>statusLine}\newline
\ct{ZnResponse>>useConnection:}\newline
\ct{ZnResponse>>writeOn:}\newline
\ct{ZnSeasideServerAdaptorDelegate class>>with:}\newline
\ct{ZnSeasideServerAdaptorDelegate>>adaptor:}\newline
\ct{ZnSeasideServerAdaptorDelegate>>adaptor}\newline
\ct{ZnSeasideServerAdaptorDelegate>>}\newline
\indent\ct{handleRequest:}\newline
\ct{ZnServer class>>defaultServerClass}\newline
\ct{ZnServer class>>initialize}\newline
\ct{ZnServer class>>managedServers}\newline
\ct{ZnServer class>>on:}\newline
\ct{ZnServer class>>shutDown:}\newline
\ct{ZnServer class>>startUp:}\newline
\ct{ZnServer class>>unregister:}\newline
\ct{ZnServer>>authenticator}\newline
\ct{ZnServer>>bindingAddress}\newline
\ct{ZnServer>>delegate:}\newline
\ct{ZnServer>>delegate}\newline
\ct{ZnServer>>maximumEntitySize}\newline
\ct{ZnServer>>optionAt:ifAbsent:}\newline
\ct{ZnServer>>optionAt:ifAbsentPut:}\newline
\ct{ZnServer>>optionAt:put:}\newline
\ct{ZnServer>>port:}\newline
\ct{ZnServer>>port}\newline
\ct{ZnServer>>reader:}\newline
\ct{ZnServer>>reader}\newline
\ct{ZnServer>>stop}\newline
\ct{ZnServer>>unregister}\newline
\ct{ZnServer>>useGzipCompressionAndChunking}\newline
\ct{ZnSignalProgress class>>enabled}\newline
\ct{ZnSingleThreadedServer class>>default}\newline
\ct{ZnSingleThreadedServer class>>on:Latin1}\newline
\ct{ZnSingleThreadedServer>>acceptWaitTimeout}\newline
\ct{ZnSingleThreadedServer>>augmentResponse:forRequest:}\newline
\ct{ZnSingleThreadedServer>>authenticateAndDelegateRequest:}\newline
\ct{ZnSingleThreadedServer>>authenticateRequest:do:}\newline
\ct{ZnSingleThreadedServer>>closeDelegate}\newline
\ct{ZnSingleThreadedServer>>handleRequest:}\newline
\ct{ZnSingleThreadedServer>>handleRequestProtected:}\newline
\ct{ZnSingleThreadedServer>>initializeServerSocket}\newline
\ct{ZnSingleThreadedServer>>isRunning}\newline
\ct{ZnSingleThreadedServer>>logRequest:response:started:}\newline
\ct{ZnSingleThreadedServer>>logRequest:}\newline
\ct{ZnSingleThreadedServer>>logResponse:}\newline
\ct{ZnSingleThreadedServer>>log}\newline
\ct{ZnSingleThreadedServer>>noteAcceptWaitTimedOut}\newline
\ct{ZnSingleThreadedServer>>periodicTasks}\newline
\ct{ZnSingleThreadedServer>>process}\newline
\ct{ZnSingleThreadedServer>>readRequest:}\newline
\ct{ZnSingleThreadedServer>>releaseServerSocket}\newline
\ct{ZnSingleThreadedServer>>serverProcessName}\newline
\ct{ZnSingleThreadedServer>>serverSocket}\newline
\ct{ZnSingleThreadedServer>>socketStreamOn:}\newline
\ct{ZnSingleThreadedServer>>start}\newline
\ct{ZnSingleThreadedServer>>stop:}\newline
\ct{ZnSingleThreadedServer>>withMaximumEntitySizeDo:}\newline
\ct{ZnSingleThreadedServer>>writeResponse:on:}\newline
\ct{ZnStatusLine class>>badRequest}\newline
\ct{ZnStatusLine class>>code:}\newline
\ct{ZnStatusLine>>code:}\newline
\ct{ZnStatusLine>>code}\newline
\ct{ZnStatusLine>>reason}\newline
\ct{ZnStatusLine>>version:}\newline
\ct{ZnStatusLine>>version}\newline
\ct{ZnStatusLine>>writeOn:}\newline
\ct{ZnStringEntity class>>designatedMimeType}\newline
\ct{ZnStringEntity class>>text:}\newline
\ct{ZnStringEntity>>computeContentLength}\newline
\ct{ZnStringEntity>>contentLength}\newline
\ct{ZnStringEntity>>encoder:}\newline
\ct{ZnStringEntity>>encoder}\newline
\ct{ZnStringEntity>>hasEncoder}\newline
\ct{ZnStringEntity>>initializeEncoder}\newline
\ct{ZnStringEntity>>string:}\newline
\ct{ZnStringEntity>>string}\newline
\ct{ZnStringEntity>>writeOn:}\newline
\ct{ZnUTF8Encoder class>>handlesEncoding:}\newline
\ct{ZnUTF8Encoder class>>newForEncoding:}\newline
\ct{ZnUTF8Encoder>>decodeBytes:}\newline
\ct{ZnUTF8Encoder>>encodedByteCountFor:}\newline
\ct{ZnUTF8Encoder>>}\newline
\indent\ct{findFirstNonASCIIIn:startingAt:}\newline
\ct{ZnUTF8Encoder>>}\newline
\indent\ct{next:putAll:startingAt:toStream:}\newline
\ct{ZnUTF8Encoder>>}\newline
\indent\ct{next:putAllASCII:startingAt:toStream:}\newline
\ct{ZnUTF8Encoder>>}\newline
\indent\ct{next:putAllByteString:startingAt:}\newline
\indent\ct{toStream:}\newline
\ct{ZnUTF8Encoder>>}\newline
\indent\ct{nextFromStream:}\newline
\ct{ZnUTF8Encoder>>}\newline
\indent\ct{nextPut:toStream:}\newline
\ct{ZnUnknownHttpMethod class>>method:}\newline
\ct{ZnUnknownHttpMethod>>method:}\newline
\ct{ZnUrl class>>fromString:}\newline
\ct{ZnUrl class>>schemesNotUsingPath}\newline
\ct{ZnUrl>>addPathSegment:}\newline
\ct{ZnUrl>>decodePercent:}\newline
\ct{ZnUrl>>encodePath:on:}\newline
\ct{ZnUrl>>enforceKnownScheme}\newline
\ct{ZnUrl>>hasFragment}\newline
\ct{ZnUrl>>hasHost}\newline
\ct{ZnUrl>>hasPath}\newline
\ct{ZnUrl>>hasPort}\newline
\ct{ZnUrl>>hasQuery}\newline
\ct{ZnUrl>>hasScheme}\newline
\ct{ZnUrl>>hasUsername}\newline
\ct{ZnUrl>>isSchemeUsingPath}\newline
\ct{ZnUrl>>parseFrom:defaultScheme:}\newline
\ct{ZnUrl>>parseFrom:}\newline
\ct{ZnUrl>>parsePath:}\newline
\ct{ZnUrl>>printAuthorityOn:}\newline
\ct{ZnUrl>>printOn:}\newline
\ct{ZnUrl>>printPathOn:}\newline
\ct{ZnUrl>>printPathQueryFragmentOn:}\newline
\ct{ZnUrl>>printQueryOn:}\newline
\ct{ZnUrl>>query:}\newline
\ct{ZnUrl>>query}\newline
\ct{ZnUrl>>scheme}\newline
\ct{ZnUtils class>>httpDate:}\newline
\ct{ZnUtils class>>httpDate}\newline
\ct{ZnUtils class>>nextPutAll:on:}\newline
\ct{ZnUtils class>>signalProgress:total:}\newline
\ct{ZnUtils class>>streamingBufferSize}\newline
\ct{ZnZincServerAdaptor>>basicStart}\newline
\ct{ZnZincServerAdaptor>>configureDelegate}\newline
\ct{ZnZincServerAdaptor>>configureServerForBinaryReading}\newline
\ct{ZnZincServerAdaptor>>defaultCodec}\newline
\ct{ZnZincServerAdaptor>>defaultDelegate}\newline
\ct{ZnZincServerAdaptor>>defaultZnServer}\newline
\ct{ZnZincServerAdaptor>>isRunning}\newline
\ct{ZnZincServerAdaptor>>isStopped}\newline
\ct{ZnZincServerAdaptor>>requestAddressFor:}\newline
\ct{ZnZincServerAdaptor>>requestBodyFor:}\newline
\ct{ZnZincServerAdaptor>>requestCookiesFor:}\newline
\ct{ZnZincServerAdaptor>>requestFieldsFor:}\newline
\ct{ZnZincServerAdaptor>>requestHeadersFor:}\newline
\ct{ZnZincServerAdaptor>>requestMethodFor:}\newline
\ct{ZnZincServerAdaptor>>requestUrlFor:}\newline
\ct{ZnZincServerAdaptor>>requestVersionFor:}\newline
\ct{ZnZincServerAdaptor>>responseFrom:}\newline
\ct{ZnZincServerAdaptor>>server}\newline
\ct{ZnZincServerAdaptor>>shutDown}\newline
\ct{ZnZincServerAdaptor>>startUp}\newline
\end{multicols}


\section{Seaside App B Extract} \label{app:extraction_seaside_pharo_seed}

This section list the methods extracted from the nurtured Web application when using a seed containing all base libraries from Pharo. This list includes all methods installed from Seaside framework and the counter application. The list of methods part of the base library are excluded as it is the same list of the methods found in Pharo base library.
\begin{multicols}{2}\noindent\small
\ct{WAAccessIntervalReapingStrategy>>}\newline
\indent\ct{defaultConfiguration}\newline	
\ct{WAAccessIntervalReapingStrategy>>initialize}\newline	
\ct{WAAccessIntervalReapingStrategy>>interval}\newline	
\ct{WAAccessIntervalReapingStrategy>>reap}\newline	
\ct{WAAccessIntervalReapingStrategy>>stored:key:}\newline	
\ct{WAActionCallback>>block:}\newline	
\ct{WAActionCallback>>evaluateWithArgument:}\newline	
\ct{WAActionCallback>>isEnabledFor:}\newline	
\ct{WAActionCallback>>signalRenderNotification}\newline	
\ct{WAActionPhaseContinuation>>continue}\newline	
\ct{WAActionPhaseContinuation>>handleRequest}\newline	
\ct{WAActionPhaseContinuation>>renderContext:}\newline	
\ct{WAActionPhaseContinuation>>renderContext}\newline	
\ct{WAActionPhaseContinuation>>runCallbacks}\newline	
\ct{WAActionPhaseContinuation>>shouldRedirect}\newline	
\ct{WAAdmin class>>defaultServerManager}\newline	
\ct{WAAdmin class>>serverAdaptors}\newline	
\ct{WAAnchorTag>>callback:}\newline	
\ct{WAAnchorTag>>tag}\newline	
\ct{WAAnchorTag>>url}\newline	
\ct{WAAnchorTag>>with:}\newline	
\ct{WAApplication>>contentType}\newline	
\ct{WAApplication>>doesHandlerSupportCookies:}\newline	
\ct{WAApplication>>handleDefault:}\newline	
\ct{WAApplication>>handleFiltered:}\newline	
\ct{WAApplication>>isApplication}\newline	
\ct{WAApplication>>isImplemented:}\newline	
\ct{WAApplication>>keyField}\newline	
\ct{WAApplication>>libraries}\newline	
\ct{WAApplication>>mainClass}\newline	
\ct{WAApplication>>mimeType}\newline	
\ct{WAApplication>>newSession}\newline	
\ct{WAApplication>>resourceBaseUrl}\newline	
\ct{WAApplication>>sessionClass}\newline	
\ct{WAApplicationConfiguration>>parents}\newline	
\ct{WAAttributeSearchContext class>>key:target:}\newline	
\ct{WAAttributeSearchContext>>at:ifPresent:}\newline	
\ct{WAAttributeSearchContext>>at:put:}\newline	
\ct{WAAttributeSearchContext>>attribute}\newline	
\ct{WAAttributeSearchContext>>cachedValues}\newline	
\ct{WAAttributeSearchContext>>}\newline
\indent\ct{findAttributeAndSelectAncestorsOf:}\newline	
\ct{WAAttributeSearchContext>>initializeWithKey:}\newline	
\ct{WAAttributeSearchContext>>isAttributeInheritedOn:}\newline	
\ct{WAAttributeSearchContext>>isAttributeLocalOn:}\newline	
\ct{WAAttributeSearchContext>>key}\newline	
\ct{WABrush>>initialize}\newline	
\ct{WABrush>>parent}\newline	
\ct{WABrush>>setParent:canvas:}\newline	
\ct{WABrush>>with:}\newline	
\ct{WABufferedResponse class>>on:}\newline	
\ct{WABufferedResponse>>contents}\newline	
\ct{WABufferedResponse>>destroy}\newline	
\ct{WABufferedResponse>>initializeOn:}\newline	
\ct{WABufferedResponse>>stream}\newline	
\ct{WACache>>at:ifAbsent:}\newline	
\ct{WACache>>expiryPolicy}\newline	
\ct{WACache>>initializeCollections}\newline	
\ct{WACache>>initializeMutex}\newline	
\ct{WACache>>initialize}\newline	
\ct{WACache>>keyAtValue:ifAbsent:}\newline	
\ct{WACache>>keyAtValue:}\newline	
\ct{WACache>>keySize}\newline	
\ct{WACache>>missStrategy}\newline	
\ct{WACache>>notifyRemoved:key:}\newline	
\ct{WACache>>notifyRetrieved:key:}\newline	
\ct{WACache>>notifyStored:key:}\newline	
\ct{WACache>>pluginsDo:}\newline	
\ct{WACache>>reapingStrategy}\newline	
\ct{WACache>>reap}\newline	
\ct{WACache>>removalAction}\newline	
\ct{WACache>>setExpiryPolicy:}\newline	
\ct{WACache>>setMissStrategy:}\newline	
\ct{WACache>>setReapingStrategy:}\newline	
\ct{WACache>>setRemovalAction:}\newline	
\ct{WACache>>store:}\newline	
\ct{WACacheCapacityConfiguration>>describeOn:}\newline	
\ct{WACacheMissStrategy>>missed:}\newline	
\ct{WACachePlugin>>configuration}\newline	
\ct{WACachePlugin>>defaultConfiguration}\newline	
\ct{WACachePlugin>>initialize}\newline	
\ct{WACachePlugin>>removed:key:}\newline	
\ct{WACachePlugin>>retrieved:key:}\newline	
\ct{WACachePlugin>>setCache:}\newline	
\ct{WACachePlugin>>stored:key:}\newline	
\ct{WACacheReapingStrategy>>reap}\newline	
\ct{WACallback class>>on:}\newline	
\ct{WACallback>>convertKey:}\newline	
\ct{WACallback>>evaluateWithFieldValues:}\newline	
\ct{WACallback>>key}\newline	
\ct{WACallback>>setKey:callbacks:}\newline	
\ct{WACallback>>valueForField:}\newline	
\ct{WACallbackRegistry>>advanceKey}\newline	
\ct{WACallbackRegistry>>handle:}\newline	
\ct{WACallbackRegistry>>increaseKey}\newline	
\ct{WACallbackRegistry>>initialize}\newline	
\ct{WACallbackRegistry>>nextKey}\newline	
\ct{WACallbackRegistry>>store:}\newline	
\ct{WACanvas>>brush:}\newline	
\ct{WACanvas>>flush}\newline	
\ct{WACanvas>>nest:}\newline	
\ct{WACanvas>>render:}\newline	
\ct{WACanvas>>text:}\newline	
\ct{WAComponent>>accept:}\newline	
\ct{WAComponent>>acceptDecorated:}\newline	
\ct{WAComponent>>decoration}\newline	
\ct{WAComponent>>initialize}\newline	
\ct{WAComponent>>updateStates:}\newline	
\ct{WAConfigurationDescription>>add:to:}\newline	
\ct{WAConfigurationDescription>>addAttribute:}\newline	
\ct{WAConfigurationDescription>>attributes}\newline	
\ct{WAConfigurationDescription>>expressions}\newline	
\ct{WAConfigurationDescription>>initialize}\newline	
\ct{WAConfigurationDescription>>integer:}\newline	
\ct{WAConfiguredRequestFilter>>configuration}\newline	
\ct{WACounter>>count:}\newline	
\ct{WACounter>>decrease}\newline	
\ct{WACounter>>increase}\newline	
\ct{WACounter>>initialize}\newline	
\ct{WACounter>>renderContentOn:}\newline	
\ct{WACounter>>states}\newline	
\ct{WADefaultScriptGenerator>>close:on:}\newline	
\ct{WADefaultScriptGenerator>>open:on:}\newline	
\ct{WADevelopmentConfiguration>>parents}\newline	
\ct{WADispatcher class>>default}\newline	
\ct{WADispatcher>>handleFiltered:named:}\newline	
\ct{WADispatcher>>handleFiltered:}\newline	
\ct{WADispatcher>>handlerAt:ifAbsent:}\newline	
\ct{WADispatcher>>handlerAt:with:}\newline	
\ct{WADispatcher>>handlers}\newline	
\ct{WADispatcher>>nameOfHandler:}\newline	
\ct{WADispatcher>>urlFor:}\newline	
\ct{WADocument class>>on:codec:}\newline	
\ct{WADocument>>close}\newline	
\ct{WADocument>>destroy}\newline	
\ct{WADocument>>initializeWithStream:codec:}\newline	
\ct{WADocument>>nextPut:}\newline	
\ct{WADocument>>nextPutAll:}\newline	
\ct{WADocument>>open:}\newline	
\ct{WADynamicVariable class>>use:during:}\newline	
\ct{WADynamicVariable class>>value}\newline	
\ct{WAEncoder class>>on:table:}\newline	
\ct{WAEncoder class>>on:}\newline	
\ct{WAEncoder>>initializeOn:table:}\newline	
\ct{WAEncoder>>nextPut:}\newline	
\ct{WAErrorHandler class>>exceptionSelector}\newline	
\ct{WAExampleComponent>>rendererClass}\newline	
\ct{WAExceptionFilter>>exceptionHandler}\newline	
\ct{WAExceptionFilter>>handleFiltered:}\newline	
\ct{WAExceptionHandler class>>context:}\newline	
\ct{WAExceptionHandler class>>exceptionSelector}\newline	
\ct{WAExceptionHandler class>>handleExceptionsDuring:context:}\newline	
\ct{WAExceptionHandler class>>handles:}\newline	
\ct{WAExceptionHandler>>handleExceptionsDuring:}\newline	
\ct{WAExceptionHandler>>handles:}\newline	
\ct{WAExceptionHandler>>initializeWithContext:}\newline	
\ct{WAHeaderFields>>checkValue:}\newline	
\ct{WAHeaderFields>>privateAt:put:}\newline	
\ct{WAHeadingTag>>initialize}\newline	
\ct{WAHeadingTag>>level1}\newline	
\ct{WAHeadingTag>>level}\newline	
\ct{WAHeadingTag>>tag}\newline	
\ct{WAHtmlAttributes>>encodeOn:}\newline	
\ct{WAHtmlAttributes>>privateAt:put:}\newline	
\ct{WAHtmlCanvas>>anchor}\newline	
\ct{WAHtmlCanvas>>heading:}\newline	
\ct{WAHtmlCanvas>>heading}\newline	
\ct{WAHtmlCanvas>>spaceEntity}\newline	
\ct{WAHtmlDocument>>scriptGenerator:}\newline	
\ct{WAHtmlDocument>>scriptGenerator}\newline	
\ct{WAHtmlElement class>>root:}\newline	
\ct{WAHtmlElement>>attributeAt:put:}\newline	
\ct{WAHtmlElement>>attributes}\newline	
\ct{WAHtmlElement>>encodeBeforeOn:}\newline	
\ct{WAHtmlElement>>encodeOn:}\newline	
\ct{WAHtmlElement>>initializeWithRoot:}\newline	
\ct{WAHtmlElement>>isClosed}\newline	
\ct{WAHtmlRoot>>add:}\newline	
\ct{WAHtmlRoot>>beXhtml10Strict}\newline	
\ct{WAHtmlRoot>>bodyAttributes}\newline	
\ct{WAHtmlRoot>>closeOn:}\newline	
\ct{WAHtmlRoot>>docType:}\newline	
\ct{WAHtmlRoot>>htmlAttributes}\newline	
\ct{WAHtmlRoot>>initialize}\newline	
\ct{WAHtmlRoot>>meta}\newline	
\ct{WAHtmlRoot>>openOn:}\newline	
\ct{WAHtmlRoot>>title:}\newline	
\ct{WAHtmlRoot>>writeElementsOn:}\newline	
\ct{WAHtmlRoot>>writeFootOn:}\newline	
\ct{WAHtmlRoot>>writeHeadOn:}\newline	
\ct{WAHtmlRoot>>writeScriptsOn:}\newline	
\ct{WAHtmlRoot>>writeStylesOn:}\newline	
\ct{WAHttpVersion class>>fromString:}\newline	
\ct{WAHttpVersion class>>major:minor:}\newline	
\ct{WAHttpVersion class>>readFrom:}\newline	
\ct{WAHttpVersion>>initializeWithMajor:minor:}\newline	
\ct{WAInitialRequestVisitor class>>request:}\newline	
\ct{WAInitialRequestVisitor>>}\newline
\indent\ct{initializeWithRequest:}\newline	
\ct{WAInitialRequestVisitor>>request}\newline	
\ct{WAInitialRequestVisitor>>visitPresenter:}\newline	
\ct{WAKeyGenerator class>>current}\newline	
\ct{WAKeyGenerator>>keyOfLength:}\newline	
\ct{WALastAccessExpiryPolicy>>}\newline
\indent\ct{defaultConfiguration}\newline	
\ct{WALastAccessExpiryPolicy>>initialize}\newline	
\ct{WALastAccessExpiryPolicy>>isExpired:key:}\newline	
\ct{WALastAccessExpiryPolicy>>retrieved:key:}\newline	
\ct{WALastAccessExpiryPolicy>>stored:key:}\newline	
\ct{WALastAccessExpiryPolicy>>timeout}\newline	
\ct{WALeastRecentlyUsedExpiryPolicy>>}\newline
\indent\ct{defaultConfiguration}\newline	
\ct{WALeastRecentlyUsedExpiryPolicy>>}\newline
\indent\ct{initialize}\newline	
\ct{WALeastRecentlyUsedExpiryPolicy>>}\newline
\indent\ct{isExpired:key:}\newline	
\ct{WALeastRecentlyUsedExpiryPolicy>>}\newline
\indent\ct{maximumAge}\newline	
\ct{WALeastRecentlyUsedExpiryPolicy>>}\newline
\indent\ct{removed:key:}\newline	
\ct{WALeastRecentlyUsedExpiryPolicy>>}\newline
\indent\ct{retrieved:key:}\newline	
\ct{WALeastRecentlyUsedExpiryPolicy>>}\newline
\indent\ct{stored:key:}\newline	
\ct{WAMergedRequestFields class>>on:}\newline	
\ct{WAMergedRequestFields>>allAt:}\newline	
\ct{WAMergedRequestFields>>at:ifAbsent:}\newline	
\ct{WAMergedRequestFields>>includesKey:}\newline	
\ct{WAMergedRequestFields>>initializeOn:}\newline	
\ct{WAMergedRequestFields>>keysAndValuesDo:}\newline	
\ct{WAMergedRequestFields>>keysDo:}\newline	
\ct{WAMetaElement>>content:}\newline	
\ct{WAMetaElement>>contentScriptType:}\newline	
\ct{WAMetaElement>>contentType:}\newline	
\ct{WAMetaElement>>encodeBeforeOn:}\newline	
\ct{WAMetaElement>>responseHeaderName:}\newline	
\ct{WAMetaElement>>tag}\newline	
\ct{WAMimeType class>>fromString:}\newline	
\ct{WAMimeType class>>main:sub:}\newline	
\ct{WAMimeType class>>textJavascript}\newline	
\ct{WAMimeType class>>textPlain}\newline	
\ct{WAMimeType>>charset:}\newline	
\ct{WAMimeType>>greaseString}\newline	
\ct{WAMimeType>>main:}\newline	
\ct{WAMimeType>>main}\newline	
\ct{WAMimeType>>parameters}\newline	
\ct{WAMimeType>>sub:}\newline	
\ct{WAMimeType>>sub}\newline	
\ct{WAMutex>>critical:}\newline	
\ct{WAMutex>>initialize}\newline	
\ct{WAMutualExclusionFilter>>handleFiltered:}\newline	
\ct{WAMutualExclusionFilter>>initialize}\newline	
\ct{WAMutualExclusionFilter>>shouldTerminate:}\newline	
\ct{WANotifyRemovalAction>>removed:key:}\newline	
\ct{WAObject>>application}\newline	
\ct{WAObject>>requestContext}\newline	
\ct{WAObject>>session}\newline	
\ct{WAPainter>>renderWithContext:}\newline	
\ct{WAPainter>>updateRoot:}\newline	
\ct{WAPainter>>updateUrl:}\newline	
\ct{WAPainterVisitor>>visitComponent:}\newline	
\ct{WAPainterVisitor>>visitDecorationsOfComponent:}\newline	
\ct{WAPainterVisitor>>visitPainter:}\newline	
\ct{WAPainterVisitor>>visitPresenter:}\newline	
\ct{WAPathConsumer class>>path:}\newline	
\ct{WAPathConsumer>>atEnd}\newline	
\ct{WAPathConsumer>>initializeWith:}\newline	
\ct{WAPathConsumer>>next}\newline	
\ct{WAPathConsumer>>upToEnd}\newline	
\ct{WAPresenter>>childrenDo:}\newline	
\ct{WAPresenter>>children}\newline	
\ct{WAPresenter>>initialRequest:}\newline	
\ct{WAPresenter>>script}\newline	
\ct{WAPresenter>>style}\newline	
\ct{WAPresenter>>updateRoot:}\newline	
\ct{WAPresenter>>updateStates:}\newline	
\ct{WAPresenterGuide class>>client:}\newline	
\ct{WAPresenterGuide>>client}\newline	
\ct{WAPresenterGuide>>initializeWithClient:}\newline	
\ct{WAPresenterGuide>>visit:}\newline	
\ct{WAPresenterGuide>>visitPainter:}\newline	
\ct{WARegistryConfiguration>>parents}\newline	
\ct{WARenderContext>>actionBaseUrl:}\newline	
\ct{WARenderContext>>actionUrl:}\newline	
\ct{WARenderContext>>actionUrl}\newline	
\ct{WARenderContext>>callbacks}\newline	
\ct{WARenderContext>>defaultVisitor}\newline	
\ct{WARenderContext>>destroy}\newline	
\ct{WARenderContext>>document:}\newline	
\ct{WARenderContext>>document}\newline	
\ct{WARenderContext>>initialize}\newline	
\ct{WARenderContext>>resourceUrl:}\newline	
\ct{WARenderContext>>visitor:}\newline	
\ct{WARenderContext>>visitor}\newline	
\ct{WARenderLoopConfiguration>>parents}\newline	
\ct{WARenderLoopContinuation>>}\newline
\indent\ct{createActionContinuation}\newline	
\ct{WARenderLoopContinuation>>}\newline
\indent\ct{createRenderContinuation}\newline	
\ct{WARenderLoopContinuation>>}\newline
\indent\ct{presenter}\newline	
\ct{WARenderLoopContinuation>>}\newline
\indent\ct{toPresenterSendRoot:}\newline	
\ct{WARenderLoopContinuation>>}\newline
\indent\ct{updateRoot:}\newline	
\ct{WARenderLoopContinuation>>}\newline
\indent\ct{updateStates:}\newline	
\ct{WARenderLoopContinuation>>}\newline
\indent\ct{updateUrl:}\newline	
\ct{WARenderLoopContinuation>>}\newline
\indent\ct{withNotificationHandlerDo:}\newline	
\ct{WARenderLoopMain>>createRoot}\newline	
\ct{WARenderLoopMain>>prepareRoot:}\newline	
\ct{WARenderLoopMain>>rootClass}\newline	
\ct{WARenderLoopMain>>rootDecorationClasses}\newline	
\ct{WARenderLoopMain>>start}\newline	
\ct{WARenderPhaseContinuation>>}\newline
\indent\ct{createHtmlRootWithContext:}\newline	
\ct{WARenderPhaseContinuation>>}\newline
\indent\ct{createRenderContext}\newline	
\ct{WARenderPhaseContinuation>>}\newline
\indent\ct{handleRequest}\newline	
\ct{WARenderPhaseContinuation>>}\newline
\indent\ct{processRendering:}\newline	
\ct{WARenderVisitor class>>context:}\newline	
\ct{WARenderVisitor>>initializeWithContext:}\newline	
\ct{WARenderVisitor>>renderContext}\newline	
\ct{WARenderVisitor>>visitPainter:}\newline	
\ct{WARenderer class>>context:}\newline	
\ct{WARenderer>>actionUrl}\newline	
\ct{WARenderer>>callbacks}\newline	
\ct{WARenderer>>context}\newline	
\ct{WARenderer>>document}\newline	
\ct{WARenderer>>flush}\newline	
\ct{WARenderer>>initializeWithContext:}\newline	
\ct{WARenderer>>render:}\newline	
\ct{WARenderer>>text:}\newline	
\ct{WARequest class>>method:uri:version:}\newline	
\ct{WARequest>>at:ifAbsent:}\newline	
\ct{WARequest>>cookiesAt:}\newline	
\ct{WARequest>>cookies}\newline	
\ct{WARequest>>destroy}\newline	
\ct{WARequest>>fields}\newline	
\ct{WARequest>>headerAt:ifAbsent:}\newline	
\ct{WARequest>>headerAt:}\newline	
\ct{WARequest>>initializeWithMethod:uri:version:}\newline	
\ct{WARequest>>isGet}\newline	
\ct{WARequest>>isPrefetch}\newline	
\ct{WARequest>>isXmlHttpRequest}\newline	
\ct{WARequest>>method}\newline	
\ct{WARequest>>postFields}\newline	
\ct{WARequest>>queryFields}\newline	
\ct{WARequest>>setBody:}\newline	
\ct{WARequest>>setCookies:}\newline	
\ct{WARequest>>setHeaders:}\newline	
\ct{WARequest>>setPostFields:}\newline	
\ct{WARequest>>setRemoteAddress:}\newline	
\ct{WARequest>>uri}\newline	
\ct{WARequest>>url}\newline	
\ct{WARequestContext class>>request:response:codec:}\newline	
\ct{WARequestContext>>application}\newline	
\ct{WARequestContext>>charSet}\newline	
\ct{WARequestContext>>codec}\newline	
\ct{WARequestContext>>consumer}\newline	
\ct{WARequestContext>>destroy}\newline	
\ct{WARequestContext>>handlers}\newline	
\ct{WARequestContext>>handler}\newline	
\ct{WARequestContext>>initializeWithRequest:response:codec:}\newline	
\ct{WARequestContext>>newDocument}\newline	
\ct{WARequestContext>>push:during:}\newline	
\ct{WARequestContext>>registry}\newline	
\ct{WARequestContext>>request}\newline	
\ct{WARequestContext>>respond:}\newline	
\ct{WARequestContext>>respond}\newline	
\ct{WARequestContext>>responseGenerator}\newline	
\ct{WARequestContext>>response}\newline	
\ct{WARequestContext>>session}\newline	
\ct{WARequestFilter>>handleFiltered:}\newline	
\ct{WARequestFilter>>initialize}\newline	
\ct{WARequestFilter>>next}\newline	
\ct{WARequestFilter>>setNext:}\newline	
\ct{WARequestFilter>>updateStates:}\newline	
\ct{WARequestHandler>>addFilter:}\newline	
\ct{WARequestHandler>>addFilterLast:}\newline	
\ct{WARequestHandler>>basicUrl}\newline	
\ct{WARequestHandler>>configuration:}\newline	
\ct{WARequestHandler>>configuration}\newline	
\ct{WARequestHandler>>defaultConfiguration}\newline	
\ct{WARequestHandler>>documentClass}\newline	
\ct{WARequestHandler>>filters}\newline	
\ct{WARequestHandler>>filter}\newline	
\ct{WARequestHandler>>handle:}\newline	
\ct{WARequestHandler>>initialize}\newline	
\ct{WARequestHandler>>isApplication}\newline	
\ct{WARequestHandler>>isRoot}\newline	
\ct{WARequestHandler>>parent}\newline	
\ct{WARequestHandler>>preferenceAt:}\newline	
\ct{WARequestHandler>>responseGenerator}\newline	
\ct{WARequestHandler>>serverHostname}\newline	
\ct{WARequestHandler>>serverPath}\newline	
\ct{WARequestHandler>>serverPort}\newline	
\ct{WARequestHandler>>serverProtocol}\newline	
\ct{WARequestHandler>>setFilter:}\newline	
\ct{WARequestHandler>>setParent:}\newline	
\ct{WARequestHandler>>url}\newline	
\ct{WAResponse class>>messageForStatus:}\newline	
\ct{WAResponse class>>statusFound}\newline	
\ct{WAResponse>>contentType:}\newline	
\ct{WAResponse>>contentType}\newline	
\ct{WAResponse>>cookies}\newline	
\ct{WAResponse>>destroy}\newline	
\ct{WAResponse>>found}\newline	
\ct{WAResponse>>headerAt:ifAbsent:}\newline	
\ct{WAResponse>>headerAt:put:}\newline	
\ct{WAResponse>>headers}\newline	
\ct{WAResponse>>initializeOn:}\newline	
\ct{WAResponse>>initialize}\newline	
\ct{WAResponse>>location:}\newline	
\ct{WAResponse>>redirectTo:}\newline	
\ct{WAResponse>>status:message:}\newline	
\ct{WAResponse>>status:}\newline	
\ct{WAResponse>>status}\newline	
\ct{WAResponseGenerator class>>on:}\newline	
\ct{WAResponseGenerator>>expiredRegistryKey}\newline	
\ct{WAResponseGenerator>>initializeOn:}\newline	
\ct{WAResponseGenerator>>requestContext}\newline	
\ct{WAResponseGenerator>>request}\newline	
\ct{WAResponseGenerator>>respond}\newline	
\ct{WAResponseGenerator>>response}\newline	
\ct{WARoot class>>context:}\newline	
\ct{WARoot>>setContext:}\newline	
\ct{WAScriptGenerator>>initialize}\newline	
\ct{WAScriptGenerator>>loadScripts}\newline	
\ct{WAScriptGenerator>>writeLoadScriptsOn:}\newline	
\ct{WAScriptGenerator>>writeScriptTag:on:}\newline	
\ct{WAServerAdaptor class>>default}\newline	
\ct{WAServerAdaptor class>>manager:}\newline	
\ct{WAServerAdaptor class>>new}\newline	
\ct{WAServerAdaptor class>>port:}\newline	
\ct{WAServerAdaptor class>>startOn:}\newline	
\ct{WAServerAdaptor>>codec}\newline	
\ct{WAServerAdaptor>>contextFor:}\newline	
\ct{WAServerAdaptor>>defaultPort}\newline	
\ct{WAServerAdaptor>>defaultRequestHandler}\newline	
\ct{WAServerAdaptor>>handle:}\newline	
\ct{WAServerAdaptor>>handlePadding:}\newline	
\ct{WAServerAdaptor>>handleRequest:}\newline	
\ct{WAServerAdaptor>>initializeWithManager:}\newline	
\ct{WAServerAdaptor>>initialize}\newline	
\ct{WAServerAdaptor>>manager}\newline	
\ct{WAServerAdaptor>>port:}\newline	
\ct{WAServerAdaptor>>port}\newline	
\ct{WAServerAdaptor>>process:}\newline	
\ct{WAServerAdaptor>>requestFor:}\newline	
\ct{WAServerAdaptor>>requestHandler}\newline	
\ct{WAServerAdaptor>>responseFor:}\newline	
\ct{WAServerAdaptor>>start}\newline	
\ct{WAServerManager class>>default}\newline	
\ct{WAServerManager>>adaptors}\newline	
\ct{WAServerManager>>canStart:}\newline	
\ct{WAServerManager>>register:}\newline	
\ct{WAServerManager>>start:}\newline	
\ct{WASession>>actionField}\newline	
\ct{WASession>>actionUrlForContinuation:}\newline	
\ct{WASession>>actionUrlForKey:}\newline	
\ct{WASession>>application}\newline	
\ct{WASession>>clearJumpTo}\newline	
\ct{WASession>>createCache}\newline	
\ct{WASession>>handleFiltered:}\newline	
\ct{WASession>>initializeFilters}\newline	
\ct{WASession>>initialize}\newline	
\ct{WASession>>isSession}\newline	
\ct{WASession>>presenter}\newline	
\ct{WASession>>properties}\newline	
\ct{WASession>>start}\newline	
\ct{WASession>>unknownRequest}\newline	
\ct{WASession>>updateRoot:}\newline	
\ct{WASession>>updateStates:}\newline	
\ct{WASession>>updateUrl:}\newline	
\ct{WASession>>url}\newline	
\ct{WASessionContinuation>>basicValue}\newline	
\ct{WASessionContinuation>>captureAndInvoke}\newline	
\ct{WASessionContinuation>>captureState}\newline	
\ct{WASessionContinuation>>redirectToContinuation:}\newline	
\ct{WASessionContinuation>>registerForUrl:}\newline	
\ct{WASessionContinuation>>registerForUrl}\newline	
\ct{WASessionContinuation>>request}\newline	
\ct{WASessionContinuation>>respond:}\newline	
\ct{WASessionContinuation>>setStates:}\newline	
\ct{WASessionContinuation>>states}\newline	
\ct{WASessionContinuation>>updateStates:}\newline	
\ct{WASessionContinuation>>updateUrl:}\newline	
\ct{WASessionContinuation>>value}\newline	
\ct{WASessionContinuation>>withUnregisteredHandlerDo:}\newline	
\ct{WASnapshot>>initialize}\newline	
\ct{WASnapshot>>register:}\newline	
\ct{WASnapshot>>reset}\newline	
\ct{WASnapshot>>restore}\newline	
\ct{WATagBrush>>after}\newline	
\ct{WATagBrush>>attributes}\newline	
\ct{WATagBrush>>before}\newline	
\ct{WATagBrush>>closeTag}\newline	
\ct{WATagBrush>>document}\newline	
\ct{WATagBrush>>isClosed}\newline	
\ct{WATagBrush>>openTag}\newline	
\ct{WATagBrush>>storeCallback:}\newline	
\ct{WATagBrush>>with:}\newline	
\ct{WATagCanvas>>space}\newline	
\ct{WATagCanvas>>space}\newline
\ct{WAUnescapedDocument>>initializeWithStream:codec:}\newline	
\ct{WAUpdateRootVisitor class>>root:}\newline	
\ct{WAUpdateRootVisitor>>initializeWithRoot:}\newline	
\ct{WAUpdateRootVisitor>>root}\newline	
\ct{WAUpdateRootVisitor>>visitPainter:}\newline	
\ct{WAUpdateStatesVisitor class>>snapshot:}\newline	
\ct{WAUpdateStatesVisitor>>initializeWithSnapshot:}\newline	
\ct{WAUpdateStatesVisitor>>snapshot}\newline	
\ct{WAUpdateStatesVisitor>>visitPresenter:}\newline	
\ct{WAUpdateUrlVisitor class>>url:}\newline	
\ct{WAUpdateUrlVisitor>>initializeWithUrl:}\newline	
\ct{WAUpdateUrlVisitor>>url}\newline	
\ct{WAUpdateUrlVisitor>>visitPainter:}\newline	
\ct{WAUrl class>>absolute:}\newline	
\ct{WAUrl class>>decodePercent:}\newline	
\ct{WAUrl>>addAllToPath:}\newline	
\ct{WAUrl>>addField:value:}\newline	
\ct{WAUrl>>addField:}\newline	
\ct{WAUrl>>addToPath:}\newline	
\ct{WAUrl>>decode:}\newline	
\ct{WAUrl>>decodedWith:}\newline	
\ct{WAUrl>>encodeOn:}\newline	
\ct{WAUrl>>encodePathOn:}\newline	
\ct{WAUrl>>encodeQueryOn:}\newline	
\ct{WAUrl>>encodeSchemeAndAuthorityOn:}\newline	
\ct{WAUrl>>fragment}\newline	
\ct{WAUrl>>initializeFromString:}\newline	
\ct{WAUrl>>initialize}\newline	
\ct{WAUrl>>isSeasideField:}\newline	
\ct{WAUrl>>parsePath:}\newline	
\ct{WAUrl>>parseQuery:}\newline	
\ct{WAUrl>>password}\newline	
\ct{WAUrl>>path:}\newline	
\ct{WAUrl>>pathElementsIn:do:}\newline	
\ct{WAUrl>>path}\newline	
\ct{WAUrl>>postCopy}\newline	
\ct{WAUrl>>printOn:}\newline	
\ct{WAUrl>>queryFields:}\newline	
\ct{WAUrl>>queryFields}\newline	
\ct{WAUrl>>seasideUrl}\newline	
\ct{WAUrl>>slash:}\newline	
\ct{WAUrl>>subStringsIn:splitBy:do:}\newline	
\ct{WAUrl>>user}\newline	
\ct{WAUrlEncoder class>>on:codec:}\newline	
\ct{WAUrlEncoder>>nextPutAll:}\newline	
\ct{WAUserConfiguration>>addParent:}\newline	
\ct{WAUserConfiguration>>canAddParent:}\newline	
\ct{WAUserConfiguration>>expressionAt:ifAbsent:}\newline	
\ct{WAUserConfiguration>>initialize}\newline	
\ct{WAUserConfiguration>>localAttributeAt:ifAbsent:}\newline	
\ct{WAUserConfiguration>>parents}\newline	
\ct{WAValueExpression>>determineValueWithContext:configuration:}\newline	
\ct{WAValueExpression>>value}\newline	
\ct{WAValueHolder class>>with:}\newline	
\ct{WAValueHolder>>contents:}\newline	
\ct{WAValueHolder>>contents}\newline	
\ct{WAVisiblePresenterGuide>>visitPresenter:}\newline	
\ct{WAVisitor>>start:}\newline	
\ct{WAVisitor>>visit:}\newline	
\ct{WAXmlDocument>>closeTag:}\newline	
\ct{WAXmlDocument>>destroy}\newline	
\ct{WAXmlDocument>>initializeWithStream:codec:}\newline	
\ct{WAXmlDocument>>openTag:attributes:closed:}\newline	
\ct{WAXmlDocument>>openTag:attributes:}\newline	
\ct{WAXmlDocument>>openTag:}\newline	
\ct{WAXmlDocument>>print:}\newline	
\ct{WAXmlDocument>>urlEncoder}\newline	
\ct{WAXmlDocument>>xmlEncoder}\newline	
\ct{WAXmlEncoder>>nextPutAll:}\newline	
\ct{ZnSeasideServerAdaptorDelegate class>>with:}\newline	
\ct{ZnSeasideServerAdaptorDelegate>>adaptor:}\newline	
\ct{ZnSeasideServerAdaptorDelegate>>adaptor}\newline	
\ct{ZnSeasideServerAdaptorDelegate>>handleRequest:}\newline	
\ct{ZnZincServerAdaptor>>basicStart}\newline	
\ct{ZnZincServerAdaptor>>configureDelegate}\newline	
\ct{ZnZincServerAdaptor>>configureServerForBinaryReading}\newline	
\ct{ZnZincServerAdaptor>>defaultCodec}\newline	
\ct{ZnZincServerAdaptor>>defaultDelegate}\newline	
\ct{ZnZincServerAdaptor>>defaultZnServer}\newline	
\ct{ZnZincServerAdaptor>>isStopped}\newline	
\ct{ZnZincServerAdaptor>>requestAddressFor:}\newline	
\ct{ZnZincServerAdaptor>>requestBodyFor:}\newline	
\ct{ZnZincServerAdaptor>>requestCookiesFor:}\newline	
\ct{ZnZincServerAdaptor>>requestFieldsFor:}\newline	
\ct{ZnZincServerAdaptor>>requestHeadersFor:}\newline	
\ct{ZnZincServerAdaptor>>requestMethodFor:}\newline	
\ct{ZnZincServerAdaptor>>requestUrlFor:}\newline	
\ct{ZnZincServerAdaptor>>requestVersionFor:}\newline	
\ct{ZnZincServerAdaptor>>responseFrom:}\newline
\ct{ZnZincServerAdaptor>>server}\newline

\end{multicols}

% =============================================================================
\input{chapter-footer.tex}

% =============================================================================
\include{chapter/cv}
\thispagestyle{empty}
% =============================================================================
\end{document}