\input{chapter-header.tex}
% ===========================================================================
\chapter{State of the Art}
\chaplabel{background}
\minitoc
% ===========================================================================
\introduction
% ===========================================================================

This chapter presents the related work to this dissertation. With that in mind, we organize and classify the related work in two different axis \ie virtualization aspects and runtime manipulation. The reader would notice that some of the related works we present may fit into more than one of these axis. However, for the sake of clarity, we emphasize the aspects we believe will clarify the understanding of the rest of this thesis.

This chapter starts by presenting a brief introduction to virtualization, its terminology and moreover its usage. We find amongst those usages, that the \emph{co-existence} and \emph{control} of virtual technology may fit into the manipulation of runtimes.
Following, we dive into related work that implement and apply those two techniques, but not in the runtime manipulation scenario as we understand it.

The second part of this chapter focuses on runtime manipulation of object-oriented applications. We first show related work that focus on the safe modification of language semantics at runtime. Then, we show how metacircular runtimes aim to have an unified language runtime to ease its modification.

This chapter finishes by presenting a detailed description of the problem statement and a final outlook of the upcoming chapters in the light of the found problems.


% ===========================================================================
\newpage
% ===========================================================================

\section{Runtime modification}

\subsection{Safe modification of language semantics}

\subsubsection*{Tower of Interpreters}

\subsection*{Reflectivity}One problem meta-circular architectures is that meta-objects rely on the same code they reflect upon; therefore there is a risk of infinite meta-recursion when the meta-level instruments code that it relies upon.
In \cite{Denk08b}, Denker et al solve this problem by tracking the degree of metaness of the execution context. Meta-objects can only reflect on objects of a lower metaness, thus simulating the semantics of an infinite tower of distinct meta-interpreters. The existing work on Meta-context is only concerned with scoping behavioral changes. More work is needed to extend this work to structure. We plan to explore how ObjectSpaces can be used to provide a way to control structural reflective change.

\subsubsection*{Black}

It has a meta-level interpreter which it can modify from the meta-level. Kind of the tower of interpreters but only one level.

\subsubsection*{SafeTcl}
SafeTcl \cite{Oust-97a} allows one to execute code inside a safe interpreter which execute a safe subset of
Tcl commands. A security policy can be given to the safe interpreter to grant or remove
privileges. Commands can be aliased so the untrusted interpreter call an aliased method and
the command is fully implemented by a trusted interpreter. An untrusted script is
isolated in its interpreter given a few extra commands. Unfortunately SafeTcl, has no memory limit
support and some color specifications can crash the interpreter.

\subsection{Metacircular Runtimes}

% ---------------------------------------------------------------------------
After presented the technical background of metacircular \VMs we are presenting several concrete implementations in more detail.
In this first part present \VMs that focus on compile-time reflection.
In \secref{background-reified-vms} we will then focus on a list of \VMs that reify their components and allow for a more close interaction with the language-side.

\subsubsection*{\Squeak \ST \VM}
\seclabel{background-squeak}
% ---------------------------------------------------------------------------
\sm{I still don't take squeak as metacircular in the strict sense. For me, that means  implemented in itself. Java with java, lisp with lisp, smalltalk with small talk, squeak is implemented in slang, that's C with smalltalk syntax, which is barely executable as smalltalk, PyPy is implemented with RPython, also restricted subset, and with its type system, I would say, a very different language. You can however make the point that RPython is a strict subset of Python, and thus, without changes directly executable. That's not true for Slang...}
The \Squeak \VM\cite{Inga97a} is of importance in the context of this work.
Its core building system is still in active use for the \urlfootnote{\Cog \VM}{http://www.mirandabanda.org/cogblog/} which extends \Squeak with a \JIT.
The \Cog \VM is used as default by the \urlfootnote{\PH}{http://pharo.org/} programming language.
\Squeak is built around a \ST dialect called \Slang that is exported to C to be compiled to the final \VM binary.
Additionally the \Slang sources can be interpreted to provide an interactive simulator of the \VM, including full graphical support.

\Slang is limited to the functionality that can be expressed with standard C code.
\Slang in this case is mostly a high-level C preprocessor.
Even though \Slang basically has the same syntax as \ST it is semantically constrained to expressions that can be resolved statically at compilation or code generation time and are compatible with C.
Hence \Slang's semantics are closer to C than to \ST.
Unlike later metacircular frameworks \Squeak uses little or no compile-time reflection to simplify the \VM designs.
However, class composition help structuring the sources.
Next to the \Slang source which account for the biggest part of the interpreter code some \OS-related code and plugins are written in C.
To facilitate the interaction with the pure C part \Slang supports inline C expressions and type annotations.

A great achievement of the \Squeak \VM is a simulator environment that enables programmers to interact dynamically with the running \VM sources.
The simulator is capable or running a complete \Squeak \ST image including graphical user interface.
This means that programmers can change the sources of the running \VM and see the immediate effects in the simulator.
The simulator itself works by setting up a byte array which servers as native memory.
Then the \VM sources written in \Slang are interpreted by the \VM of the development environment.

We see that \Squeak is a pseudo metacircular \VM that uses an indirect bootstrap process.
The newly created \VM does not absorb any features from the host environment.
Yet according to long-time \VM programmers the \Squeak infrastructure is more productive than a comparable C++ or pure C project.

% ---------------------------------------------------------------------------
\subsubsection*{\Jikes: High-level low-level Programming in with \MMTK}
\seclabel{background-jikes}
% ---------------------------------------------------------------------------
\Jikes (former \textsc{Jalapeño})is an early metacircular research \VM for \Java \cite{Alpe00a}.
The \Jikes \VM features several different garbage collectors and does not execute bytecodes but directly compiles to native code.
With metacircularity in mind \Jikes does not resort to a low-level programming language such as C for these typically low-level \VM components.
Instead they are written in \Java as well using a high-level low-level programming framework.

The \Jikes \VM had performance as a major goal, hence direct unobstructed interaction with the low-level world is necessary using a specialized framework.
High-level low-level programming \cite{Fram09a} is mentioned the first time in the context of the \Jikes \VM project.
The goal of high-level low-level programming is to provide high-level abstractions to simplify low-level programming.
Essentially this is the same motivation that drives the metacircular \VM community.

Frampton et al. present a low-level framework packaged as \ttt{org.vmmagic}, which is used as system interface for \Jikes, an experimental \Java \VM.
Additionally their framework is successfully used in a separate project, the memory management toolkit (\MMTK) \cite{Blac04a} which is used independently in several other projects.
The \ttt{org.vmmagic} package introduces highly controlled low-level interaction in a statically type context.
In their framework, methods have to be annotated to enable the use of low-level functionality.

% ---------------------------------------------------------------------------
\subsubsection*{\Maxine \Java \VM}
\seclabel{background-maxine}
% ---------------------------------------------------------------------------
\Maxine is a metacircular \Java \VM \cite{Wimm13a} focused on an efficient developer experience.
Typically \VM frameworks focus on abstraction at the code-level which should yield simpler code and thus help reducing development efforts.
However, in most situations the programmer is still forced to use existing unspecific tools for instance to debug the \VM.
In contrast to that, the \Maxine \VM provides dedicated tools to interact with the \VM in development.
\Maxine uses abstract and high-level representations of \VM-level concepts and consistently exposes them throughout the development process.
Inspectors at multiple abstraction levels are readily available while debugging, giving insights to the complete \VM state.
\Maxine provides and excellent navigation for generated native code by providing links back to language-side objects as well as other native code and symbols.

Even though the \Maxine projects follows an approach where reflection is only used at compile-time, the development tools themselves provide a live interaction with the running \VM artifact.
However, the \VM itself is not reflective as it is not directly built to reason about itself.
This means that when debugging the \VM it behaves almost like a life \ST image where a complete interaction with the underlying system is possible.
We identify this as crucial, as most of the time is spent debugging, notably on inadequate tools like \ttt{gdb} due to lack of alternatives.
\sm{Connect this to your problem statement (back reference) and, make sure your problem statement does not mix up reflection and observability/interactivity }
Hence having a specific debuggers and inspectors greatly improve the interaction with the \VM artifact.


\subsubsection*{\PyPy Toolchain}
\seclabel{background-pypy}
% ---------------------------------------------------------------------------
\urlfootnote{\PyPy}{http://pypy.org/} is a \Python-based high-level \VM framework \cite{Rigo06a}.
\PyPy's major focus lies on an efficient \Python interpreter.
However, it has been successfully used to build \VMs for other languages including \ST \cite{Bolz08a}.
Interpreters are written in a type-inferable subset of \Python called \RPython.
The underlying \PyPy infrastructure automatically provides memory management and \JIT compilation.
Instead of explicitly providing these features, a \VM developer hints certain information to the \PyPy framework to improve the generation of a \GC or \JIT.

\PyPy follows a different approach from the previously presented \VM generation frameworks.
For instance, in Squeak and \Jikes the final \VM implementation is not much different from an implementation done directly in a low-level language.
The programmer specifies all the components of the \VM explicitly, either by implementing them directly or using a provided library.
Compared to the more static C ans C++ these \VM generation frameworks make the compilation phase more tangible.
\ST in \Squeak or \Java in \Jikes or \Maxine fulfill the purpose of the template system in C++ or the restricted macro system in C.
For the explicit implementation part \PyPy is no different.
However, certain features for the final \VM are directly absorbed from the underlying \PyPy infrastructure.
For instance, the \JIT support or the \GC are not explicitly implemented but provided by the \PyPy framework itself.
This is a big difference to the other \VM frameworks as it allows programmers to write the \VM in a more high-level fashion.
For instance in \Squeak memory allocation, even for \VM-level objects, has to be performed explicitly.
Whereas in \PyPy the garbage collection is left to the underlying \VM building infrastructure.
This approach allows \RPython \VMs to behave like standard \Python programs.

Much like the automatic memory management, \PyPy provides a tracing \JIT generator \cite{Bolz09a}.
By default the \VM programmer does not write an explicit \JIT in \PyPy.
Instead the \VM code is annotated to guide the underlying tracing \JIT generator.
This means a \VM compilation time a specific tracing \JIT is created for the given meta information.
As a result, the \JIT can track high-level loops in the final interpreted language.
Again, this is similar to \PyPy's \GC, both are provided as a service and do not have to be programmed explicitly.
Instead, the \VM programmer tweaks parameters of the \JIT or \GC.

% ---------------------------------------------------------------------------
\gp{Runtime Reified or Self-aware \VMs}
\seclabel{background-reified-vms}
% ---------------------------------------------------------------------------

The \VMs presented so far have little or no self-awareness.
\VM generation frameworks allow a high amount of reflection at \VM compile time.
This meta information is typically compiled away.
This is somewhat similar to what happens with templates in a C++-based \VM.
The \VM frameworks themselves behave like a static language on their own.
As a result, the final \VM artifact has no access to the underlying definition anymore.

As an example we might have several \VM components represented as high-level objects at compile or \VM generation time.
These objects have a class and methods attached, information that is reflectively accessible.
However, once the \VM is compiled down to native code, most of this information is lost.
What is left is native code with low-level instructions that allow little or no reasoning about the original high-level structure.
\sm{This is way to general and unspecific, and  I am not sure why that's discussed here}

We have shown in \secref{background-vm-reflection} how a potential evolution of reflection in a high-level language looks.
We concluded that the evolution of language-side reflection implies a similar evolution at \VM-level.
More behavioral reflection at language-side requires more concepts to be reified in the \VM itself.
This requirement is conflicting with the previously described loss of reification at \VM-level.

We are now going to present \VMs that behave significantly different.
Unlike the previous ones, they no longer make a clear distinction between the static \VM and the dynamic language-side.


% ------------------------------------------------------------------------------
\subsubsection*{\DwarfPython}

\DwarfPython \cite{Kell11a} is a \Python implementation that aims at a barrier-free low-level interaction.
It emerged from an earlier \textsc{Parathon} which used \Dwarf debugging information from external libraries to facilitate foreign function interfaces.
\DwarfPython takes this idea further.
Additionally to describe low-level code, \DwarfPython uses the \Dwarf metamodel to describe \Python code and data.
This is depicted in \figref{background-dwarf-python}.
%
\begin{figure}[h]
	\centering
	\includegraphics[scale=\imagescale]{vm-dwarf-python}
	\caption[\DwarfPython Low-level Reification]{
		\DwarfPython reifies the low-level \VM by using the \Dwarf Debugging at runtime.
		The \Dwarf information is generated by the default C compiler (\texttt{CC}) for C debuggers.}
	\figlabel{background-dwarf-python}
\end{figure}
%
This approach has the advantage that the very same debugging mechanism applies for low-level code, for instance written in C, and for high-level \Python code.
Thus \DwarfPython essentially unifies the previously decoupled \VM with the language-side.

% ------------------------------------------------------------------------------
\subsubsection*{\Pinocchio \VM}
\seclabel{background-pinocchio}

\P \cite{Verw11a} is a research \ST environment that directly uses native code instead of bytecodes.
The only execution base is native code which is directly generated by the language-side compiler.

\P is built from a kernel derived originally from a \PH image.
For the bootstrap classes, objects and methods are exported into binary, native images and linked together with a standard C linker to a final executable.
For simplicity we also rely on a very small part of C code to provide essential primitive, for instance used for file handling.
Additionally we specified part of the bootstrap for the \ST object model in plain C code.
However, besides that, all the other code is written and developed directly in \ST.
%
\begin{figure}[h]
	\centering
	\includegraphics[scale=\imagescale]{vm-pinocchio-bootstrap}
	\caption[\P Bootstrap]{
	\P's Bootstrap directly generates binary images (\texttt{.o}) and combines them with a simple kernel compile from C sources using a standard C linker (\texttt{LD}).}
	\figlabel{background-pinocchio-bootstrap}
\end{figure}

An important aspect of \P is that the method lookup is expressed in terms of normal \ST code.
Typically this code statically resides in the \VM, thus at a different meta-level.
Hence this implies for most systems that the lookup can not be modified without altering the \VM itself.
However, expressing the lookup in terms for normal language-side code introduces a recursive dependencies during the bootstrap.
In order to run the lookup code expressed in \ST code, we have to perform message sends.
These, in return, require an already working lookup mechanism.
Hence, without a taking special care, a language-side lookup method will lead to infinite recursion during startup.
We resolved this problem in \P by directly interacting with the low-level execution format which among other things relies on inline caches to improve performance.
The important property of inline caches is that they bypass the slow language-side lookup by directly jumping to the last activated method at a send-site.
This is exactly the behavior we need to prevent recursion during the startup.
Hence, when generating the native code for the bootstrap, we prefill all the inline caches of the methods required to perform a full method lookup.
As a result, when running requiring the first real method lookup, the lookup code itself is running perfectly on the prefilled inline caches.
What we achieve is a flexible connection between the low-level world and the high-level language-side.
During execution the \VM jumps freely between what previously was native \VM-level code and interpretation of language-side code.

From an architectural point of view, \P is performing almost a direct bootstrap.
Besides the small C kernel, the language-side code is directly compiled to native code.
As a result, \P only requires a single compiler for native code, during bootstrap and at runtime.
Hence, a separate \JIT implementation is not required.

The most obvious shortcoming of \P is the lack of its own garbage collector.
Instead of investing time into a separate well-defined \GC \P relies on the conservative \urlfootnote{Boehm \GC}{http://www.hpl.hp.com/personal/Hans_Boehm/gc/} built for C programs.
The Boehm \GC is sufficiently fast to run \P as a prototype.
\P lacks the necessary reification at level of the object layout to properly implement a \GC.
All the notion about the object layout in memory are hard-coded in the compiler in several places.
Work was undertaken to put first-class object layouts in place and delegate memory allocation and field access to these meta objects.
Yet, at the current state \P has not incorporated this in the compiler core.

\P is self-aware in the sense that it controls  native code generation and lookup at a single abstraction level.
There is no distinction between \VM-level code and language-side code.


% ------------------------------------------------------------------------------
\subsubsection*{\MIST a C-less \ST Implementation}
\urlfootnote{\MIST}{http://mist-project.org/} is another prototype \ST \VM that follows similar goals as the \P \VM.
As well, it no longer uses a bytecode interpreter but only relies on native code.
However, it goes one step further than \P by not relying on any C-based infrastructure.
\MIST implements its own linker to build the final executable.
Hence unlike \P it does not require kernel primitives written in C.
\MIST brings its own implementation to directly perform system calls from within the language.
\todo{along the track: less dependency to existing wrapped/hidden infrastrcture => more self-aware}
\sm{Why is it relevant? It is barely a paragraph, feels like it doesn't fit in or is underrepresented here. Also, instead of being indirect about the goals, it would help to restate them}

% ------------------------------------------------------------------------------
\subsubsection*{\Klein \VM}
\seclabel{background-klein}

\urlfootnote{\Klein}{http://kleinvm.sourceforge.net/} is a metacircular \VM for the \Self programming language that has no separation into \VM and language \cite{Unga05a}.
\Klein performs a direct bootstrap (see \figref{background-metacircular-bootstrap}) much like the aforementioned \P or \MIST \VM.
Hence \Klein does not use an intermediate low-level language to bootstrap the system.
\sm{I still don't get the point about the low-level IR/byte code being one of the hints you pick on.

It is an exchange format, a serialization, what ever, I think you want to make the point that it is not being executed? Well, I am not to sure about that for Klein...}

It is important to point out that the reification of the \VM-level survives the code generation or compilation time.
Instead the \VM structures are represented as real \Self objects.
Hence the \Klein \VM supports true \VM-level reflection since there is only a single code base.

Additionally to the advances in reflection and metacircularity, \Klein focuses on fast compilation turnarounds to allow for a responsive development process.
Which is unlike for instance the \Squeak \VM where a full \VM bootstrap takes an order of minutes on modern hardware.
\Klein also supports advanced mirror-based debugging tools to inspect and modify a remote \VM.

Development on the \Klein \VM stopped in 2009 and left the \Klein \VM in fairly usable state.
Like \P it currently lacks a dedicated \GC.
Yet, it proved that it is possible and build a language-runtime without the classical separation of the language-side and the \VM.
From the literature presented about the \Klein project we see a strong focus on the improvements of the development tools.
The fact that the language-runtime allows \VM-level reflection to change the \VM dynamically is not directly mentioned in the literature.
While we see the practical limitations of changing the \VM at runtime we would like to open the doors to this new form of reflection.

\section{Virtualization}

\subsection{Application Co-existence Techniques}

\subsubsection*{\textsc{Class Loaders}}
In Java, classes are loaded dynamically through a class loader~\cite{Lian98a}. A class loader is a first-class entity responsible of loading classes: create their runtime representation, loading their methods and linking their class references. A class loader remembers all classes it loaded, and it is responsible for loading all classes related to  them. Class loaders define namespaces: different class loaders can load different classes with the same name. These classes will be isolated in the sense that they will not be visible to the others.

Class loaders can be specialized and extended to provide custom behavior. For example, Fong et. al.~\cite{Fong10a} use the class loading mechanism to enforce scoping rules and determine the visibility of names in various region of the program. They allow the user to control untrusted
namespaces and classes, they have defined a language to define security policy. Jensen et. al.~\cite{Jens98a} provide a formalization of the class loader with the means to enforce security. They also use a bytecode verifier on class loading to check if a class' bytecode doesn't try to perform overflow or underflow operations.

The isolating mechanism of class loaders is useful in the scoping context. New versions of code can be loaded in a different class loader and coexist at runtime. However, its basic mechanism does not support a way to manage changes, or update an application.

\subsubsection*{\textsc{Changeboxes}}

Changeboxes~\cite{Denk07c} is a change model designed to encapsulate and scope changes. Its main purpose is to allow several versions of a system to coexist at runtime \ie the existence in the same environment of different versions of the same classes and methods. In changeboxes, a \emph{changebox} is a first-class entity that encapsulates changes made on elements~(classes and methods) and an executable version of the system with its changes applied. The system can contain many changeboxes at the same time, and applications can be scoped to run within different changeboxes. This notion of dynamically scoping an application to a changebox allows one to have co-existing environments~(\eg testing, development, production), increasing the developer's efficiency. Furthermore, it eases application update and migration to new versions, and reduces its update down-time as the application does not have to be stopped to be updated.

A Changeboxes prototype was developed in Smalltalk and its scoping mechanisms were implemented as follows:

\begin{description}
\item[Message send interception.] Message sends can activate different methods, within different changeboxes. A MethodWrapper~\cite{Bran98a} is placed instead of the method that has multiple versions, and it delegates the execution to the method that corresponds to the currently valid changebox.

\item[Class access interception.] Smalltalk resolves class names at compile time, inserting a reference to the given class inside the method's literal array. However, accessing a class should yield different class objects if the changebox contains a different version of it. To resolve this, class accesses affected by a changebox are postponed until runtime, and the code is recompiled in such a way: instead of putting the class inside the literal array, the class is dynamically looked-up from the class table when it is accessed.
\end{description}

Changeboxes model proves sound to update and migrate application and framework classes. However, it has the main drawback of not affecting critical classes in the system. Changeboxes prototype does not work on classes such as \ct{Array} or \ct{CompiledMethod} as the underlying infrastructure~(the VM) restricts the system to the existence of only one of them at the same time. The changes model does not provide neither a solution for this problem, as it focuses on application code update, leaving this as an open problem.

\subsubsection*{\textsc{Caja\textbackslash Cajita}}

\subsubsection*{\textsc{Worlds}}
Worlds~\cite{Wart08a} provide a way to control and scope side-effects in Javascript. Side-effects are limited to a first-class environment.

\subsubsection*{\textsc{Gemstone}}
Gemstone \cite{Otis91a} provides the concept of class versions. Classes are
automatically versioned, but existing instances keep the class (shape and
behavior) of the original definition. Instances can be migrated at any time.
Gemstone provides (database) transaction semantics, thus state can be rolled
back should the migration fail.
Gemstone's class versions extend the usual Smalltalk class evolution mechanism for robustness, 
large datasets, and domain-specific migration policies. In contrast, ObjectSpaces target general 
reflective access and bootstrap-like evolutions of code that is critical to the environment.

\subsubsection*{KaffeOS}
KaffeOS \cite{Back00a} makes an explicit domain separation in memory by using different memory heaps in the virtual machine. They enforce domain separation by using memory write barriers. Cross-domain references become cross-heap references, and thus, they need special virtual machine support.
KaffeOS presents a model where resource accounting is handled at the level of the virtual machine.

\subsubsection*{J-Kernel}
J-Kernel \cite{Hawb98a}

\subsubsection*{Luna}
 and Luna \cite{Hawb02a} present a solution similar to ours regarding the memory usage. They are Java solution for isolating object graphs with security purposes. In them, each object graph is called a \emph{protection domain}. All protection domains loaded in a system, and their objects, share the same memory space. 

The J-Kernel enforces the separation between domains by using the Java type system, the inability of the Java language to forge object references, and by providing capability objects\cite{Levy84a,Mill03a,Spoo00a} enabling remote messaging and controlling the communication. This same separation in Luna \cite{Hawb02a} is achieved by the modification of the type system and the addition in the virtual machine of the \emph{remote reference} concept.

\subsection{Application Control}

\subsubsection*{JVMTI}
Java also provides JPDA, a remote debugging architecture that specifies a native interface on the debuggee VM, and a matching API for the debugger front-end, running in a separate VM. However, JDPA only supports introspection features like  inspection and monitoring, and very limited intercession \cite{jdpa}.

\subsection*{Xen and Operating System Virtualization techniques}
Amongst the most important works on virtualization,we can found approaches like Xen~\cite{Chis07xen}. Xen is a Virtual Machine Monitor~(VMM) that allows to control and manage Virtual Machines in a high performance and resource-managed way. This approach targets the virtualization of full and unmodified operating systems, to facilitate their adoption in industrial/productive environments. Virtualization, in these terms, enables the following applications:

\begin{description}

\item[Server Consolidation.]
\item[Co-located hosting facilities.]
\item[Distributed web services.]
\item[Secure computing platforms.]
\item[Application Mobility.]

\end{description}

This virtualization approach has another main characteristic: it considers its guest systems as a black box, and it does not allow one to observe or modify its internals.

\subsubsection*{MVM: a Multi User Virtual Machine}
The Multi-user Virtual Machine~\cite{Czaj03a,Czaj01a} is a general purpose virtual machine for the Java language that allows the co-existence of different applications, potentially from different users. Each application running on top of the MVM is an \emph{isolate} based on the Java Application Isolation API specification~\cite{JSR121}.

Many isolates co-exist not inteferring each other, as they believe they own their private JVM: the runtime is modified, so state is not shared between them by default. MVM allows several communication mechanisms to securely communicate isolates: from standard mechanisms such as sockets, up to \emph{links}, a low-level isolate-to-isolate mechanism introduced by the Isolate API.

MVM can run any normal Java application. Additionally, MVM-aware applications can use the API it provides to control the life-cycle~(\eg creation, suspension, resuming and and termination) and the available resources of other isolates.

%\subsection*{Java Isolates}
%Java Isolates \cite{JSR121} allow multiple applications to run inside the same Java virtual machine.
%Nothing is shared between the different applications. Resources like CPU time, memory are controlled 
%and restricted. Isolates can communicate through channel, since nothing is shared the data are copied. 
%Java Isolates are defined in the Java Specification Request 121, but no commercials Java virtual 
%machine implement the specification.

% =============================================================================
\input{chapter-footer.tex}