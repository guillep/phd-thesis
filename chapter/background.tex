\input{chapter-header.tex}
% ===========================================================================
\chapter{State of the Art}
\chaplabel{background}
\minitoc
% ===========================================================================
\introduction
% ===========================================================================
	
	
% ===========================================================================
\newpage
% ===========================================================================
\section{Scoping Changes}

\subsection*{Changeboxes} \cite{Denk07c} encapsulate and scope changes, allowing several versions of a system to coexist in a single runtime environment, effectively adapting version control from static source code to running systems. Changeboxes scope code changes, while ObjectSpaces scope generic object references; also, Changeboxes do not directly address the problem of applying changes to code that is critical to the runtime system itself.

Scoping side-effects has been the focus of two recents works. Worlds~\cite{Wart08a} provide a way to control and scope side-effects in Javascript. Similar to
ObjectSpaces, side-effects are limited to a first-class environment.  Tanter proposed a more flexible scheme: contextual values~\cite{Tant08b} are scoped 
by a very general context function.

Gemstone \cite{Otis91a} provides the concept of class versions. Classes are
automatically versioned, but existing instances keep the class (shape and
behavior) of the original definition. Instances can be migrated at any time.
Gemstone provides (database) transaction semantics, thus state can be rolled
back should the migration fail.
Gemstone's class versions extend the usual Smalltalk class evolution mechanism for robustness, 
large datasets, and domain-specific migration policies. In contrast, ObjectSpaces target general 
reflective access and bootstrap-like evolutions of code that is critical to the environment.

In Java, new class definitions can be loaded using a class loader
\cite{Lian98a}. Class loaders define namespaces, a class type is defined by the
name of the class and its class loader. Thus the type system will prohibit
references between namespaces defined by two different loaders. Class loaders
can be used to load new versions of code and allow for these versions to coexist
at runtime, but they do not provide a first-class model of change.
Java also provides JPDA, a remote debugging architecture that specifies a native interface on the debuggee VM, and a matching API for the debugger front-end, running in a separate VM. However, JDPA only supports introspection features like  inspection and monitoring, and very limited intercession \cite{jdpa}.

\subsection*{Reflectivity}One problem meta-circular architectures is that meta-objects rely on the same code they reflect upon; therefore there is a risk of infinite meta-recursion when the meta-level instruments code that it relies upon.
In \cite{Denk08b}, Denker et al solve this problem by tracking the degree of metaness of the execution context. Meta-objects can only reflect on objects of a lower metaness, thus simulating the semantics of an infinite tower of distinct meta-interpreters. The existing work on Meta-context is only concerned with scoping behavioral changes. More work is needed to extend this work to structure. We plan to explore how ObjectSpaces can be used to provide a way to control structural reflective change.

\section{Virtualization Techniques}

The most related family of work is virtualization approaches like Xen \cite{Chis07xen}. Virtualization makes it possible to run several operating systems at once on a single physical machine. As these approaches target full operating systems, they rely on support from the hardware platform, and in some cases from the guest OS; they also concentrate on performance and production features, and consider the guest system mostly as a black box. In contrast, ObjectSpaces provide full control and reflective access to their contents.

\section{Metacircular VMs}

% ===========================================================================
\section{Summary and Outlook}
% ===========================================================================


% =============================================================================
\input{chapter-footer.tex}