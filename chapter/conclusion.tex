\input{chapter-header.tex}
% =============================================================================
\chapter{Conclusion}
\chaplabel{conclusion}
\minitoc
% =============================================================================
\introduction

This thesis focuses on the ease of manipulation of application runtimes for Object-Oriented languages, particularly reflective ones. Having access and the power to change such application runtimes is indeed an issue for both language and application developers.

Application runtimes are at the center of the activity of language developers. Implementing a language requires designing its execution model and its runtime representation besides its syntax and exposed concepts. Moreover, extending an existing language to add new features requires its runtime to be easily extensible. Also, reflective languages add the challenge of circularities. To perform these tasks, language developers need tools that allow them to modify a language runtime, extend it and address its circularities.

Application developers are also exposed to the modification and specialization of application runtimes. Particularly, the spreading of new constrained devices such as embedded systems or sensor networks, require application developers to reduce the memory occupied by their applications. To address these concerns, application developers should have access and control on the application runtimes they develop. They should be able to shrink them.

To address these issues we introduced \Vtt: an infrastructure for application runtime virtualization. We show that runtime virtualization is a general purpose tool that can be used for different purposes. In particular we use it to explore the two challenges presented above. Our first scenario is language bootstrapping. The second is the tailoring of application runtimes.

% =============================================================================
\section{Contributions}
% =============================================================================

\subsection{\Vtt}
The main contribution of this thesis is \Vtt, a \emph{language virtualization infrastructure}. In \Vtt, a first-class application runtime, namely an object space, allows the manipulation, control and monitoring of a virtualized runtime through a clear API. A first-class language hypervisor implements such runtime manipulations with the expression power and abstractions of the high-level language we are manipulating.

\Vtt allows application runtime manipulation and control through several techniques. Mirrors allow the direct manipulation of objects inside an object space while enforcing the invariants of the \VM execution model. The hypervisor can execute an object space in cycles and perform custom operations between each of those cycles. This allows the hypervisor to control the execution inside an object space. Finally, to execute arbitrary code inside an object space \Vtt provides with process injection and virtual execution. The first one injects code inside an object space so it can run normally. The latter is based on code interpretation and while slower provides finer control on what code is executed. 

\subsection{Bootstrapping}
Bootstrapping is commonly known by its usage on compiler building, where a compiler can compile itself.
It can be generalized to the introduction of any software system to its own building process.
A bootstrap process has the property of describing the system under construction in terms of the system itself. This allows us to easily change and extend this system, taking advantage of its abstractions and tools.

We applied the idea of a bootstrap in object-oriented languages by providing a circular language definition \ie a definition of the language runtime defined in itself.
Our virtualization infrastructure eases the execution of such language definition. First, an object space provides a clear VM-language interface to helps with the manipulation of the language runtime under creation. Second, a bootstrapping virtual interpreter allows the execution of the language definition when the language cannot still be executed.

We used this infrastructure to bootstrap three different object-oriented languages with different programming models: 
\begin{description}
\item[Candle.] A minimal Smalltalk with implicit metaclasses.
\item[Pharo.] The core of the Pharo language which is defined by traits, first-class layouts and first-class variables.
\item[MetaTalk.] A Smalltalk based language that decomposes reflection into mirrors. The language meta information is hosted inside the meta level of the language, which can be dynamically removed.
\end{description}

\subsection{RFG Tailoring} 

Application tailoring is a technique reduces the memory footprint of an application by removing code bloat. Code bloat is an issue in constrained scenarios, when the size of an application limits its deployment. For example, devices with limited memory or web-applications in slow networks. Application tailoring reduces unused code units~(\eg classes, methods) to produce a specialized version of an application for its deployment in such scenarios.

Using \Vtt, we developed run-fail-grow~(RFG), an approach for dynamic application tailoring. RFG tailors an application by starting it inside an initially empty object space. We can additionally ensure a set of code units inside our application by introducing them inside the seed. Then, a set of application entry points describing where the application starts are installed inside the object space and executed. As the application executes, the application will \emph{fail} due to missing code units. RFG reacts to missing code failures by installing the required code units. Then, code units are only installed on demand inside the virtualized runtime. Using this technique, we ensure that only the needed code units are introduced.

By performing during execution, RFG tailors programs written in dynamically typed languages and using features such as reflection and polymorphism. It works transparently, being able to tailor legacy and third-party code without modifying it. Tornado, our RFG implementation, succeeds to produce applications with minimal footprint for deployment. Our results show that we can aggressively tailor challenging cases. In our experiments we observe memory reductions from 95.02\% to 99.99\% when comparing with the production ready Pharo distribution.

% =============================================================================
%\section{Published Papers}
%% =============================================================================
%
%\subsection{Journals}
%
%Guillermo Polito, Stéphane Ducasse, Luc Fabresse, Noury Bouraqadi, and Benjamin Ryseghem. Bootstrapping Reflective Systems: The Case of Pharo. In Science of Computer Programming, 2013. Impact Factor: 0,548\newline
%
%
%
%tornado (under submission spe)
%
%\subsection{Workshops}
%
%Clara Allende, Guillermo Polito. Virtual Smalltalk Images: Model and Applications. In WISIT - Workshop de Ingeniería en Sistemas y Tecnologías de la Información, 2014.
%
%Guillermo Polito, Stéphane Ducasse, Luc Fabresse, and Noury Bouraqadi. Understanding Pharo’s global state to move programs through time and space. In IWST - International Workshop on Smalltalk Technology, Co-located within the 22th International Smalltalk Conference - 2014, 2014.
%
%Guillermo Polito, Stéphane Ducasse, Luc Fabresse, and Noury Bouraqadi. Virtual Smalltalk Images: Model and Applications. In IWST - International Workshop on Smalltalk Technology, Co-located within the 21th International Smalltalk Conference - 2013, 2013.

\section{Future Work}

\Vtt presents a language runtime virtualization model that opens several directions for future work that we consider for exploration.

\begin{description}

\item[Security.] Virtualization opens the door to easily forbid or constraint operations to the virtualized runtime. Sandboxing can then be transparent to the virtualized application, which may believe that it \emph{owns} the entire machine for itself.

\item[Resource Control.] An application could be virtualized to restrict its consumption of critical resources such as CPU or memory. Doing so in a flexible way with our first class runtimes would simplify \eg the creation of simulators for specific platforms such as constrained devices.

\item[Application distribution and migration.] The virtual language runtime API exposed by an object space encapsulates the internals of the implementation. This same API may provide transparent access to remote application runtimes residing in different processes/machines. This could open the possibility of exploring application distribution and migration in the \emph{cloud} at the language level.

\item[Dynamic Adaptation.] Language virtualization can be also useful in the context of dynamic adaptation of applications with almost zero downtime. An object space API is general enough to allow updates to occur at runtime. Additionally, the execution cycles provides with atomicity for doing such changes.

\item[VM-Language Co-Evolution.] Our approach doesn't address the co-evolution of language and VM. However, object spaces make explicit the border-line between the VM and the language to change the latter. A co-evolution implies changing not only language and VM but also their interface.

\end{description}

% =============================================================================
%\section{Software Artifacts}
%
%The research that appears in this thesis was supported and validated by several prototypes implemented in the course of the P.h.d. The produced software artifacts are the following:
%
%\begin{description}
%
%\item[\Vtt.] \Vtt's prototype is the main software artifact that results from this research. This prototype included changes in the Pharo \VM to allow the described co-existence, manipulation and control of a language runtime. \Vtt also includes several language libraries that expose the \VM behavior: object spaces and its mirrros encapsulate a language runtime; a \Vtt-based AST interpreter was created from an existent AST interpreter in Pharo; a heap exporter and importer allowed us to work on existing smalltalk images.
%
%\item[Hazelnut/Seed.] Hazelnut/Seed is our bootstrapping solution for the Pharo language. Hazelnut describes a language runtime and allow its creation through a bootstrapping interpreter based on \Vtt's one.
%
%\item[Tornado.] Tornado is our tailoring implementation on top of \Vtt. Tornado uses the Ghost proxies model to implement execution traps and the \Vtt execution cycle to install code when missing code is detected.
%
%\end{description}

% =============================================================================
\input{chapter-footer.tex}