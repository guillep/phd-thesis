\input{chapter-header.tex}
% =============================================================================
\chapter{Introduction}
\chaplabel{introduction}
\minitoc
% =============================================================================

lala

% =============================================================================
\section{Problem Statement}
% =============================================================================
Following the problem description listed in the above introduction we identified the following abstract concerns with existing reflective languages and their \VMs.
%
\begin{itemize}
	\item Reflection and in special behavioral reflection comes at a significant cost due to reification overhead.
		
	\item Intercession is limited to language-side.
	\VMs are not accessible from language-side and they are usually have no reflective properties at runtime. 
	
	\item Existing approaches to a unified model between the \VM and the language-side are holistic, there is no intermediate solution available.
\end{itemize}

\noindent Out of these general problems concerning reflection in high-level languages we see that they have a low-level root.
To address the unification of the language-side and \VM-side we have to grant more access to the language-side.
This includes interacting directly with low-level native instructions.
A similar problem has been solved by applying high-level low-level programming in a more static environment \cite{Fram09a,Graal}.
The approach outlined by Frampton et al. uses a high-level framework to generate native code at compile-time.
We see that their approach has not yet been applied in a more dynamic environment where native code has to be generated at runtime.
Hence we focus on the following concrete problems we wish to solve in this thesis.

\begin{description}
	\item[Problem 1:] High-level low-level programming is not available at runtime and from language-side.
	
	\item[Problem 2:] Intercession is limited to language-side.
	\VMs are not accessible from language-side and they are usually have no reflective properties at runtime.
	
	\item[Problem 3:] High-level low-level programming has not yet been provided as an incremental extension to an existing language runtime.
\end{description}

% =============================================================================
\section{Contributions}
% =============================================================================

To support high-level low-level programming in a dynamic context we identify that native code generation at language-side is essential.
Hence we validate this concept by implementing a language-side framework for native code generation and execution with minimal \VM changes.
Using this framework we identify the limitation of such an approach by evaluating several typical \VM-level applications.


% =============================================================================
\section{Artifacts}
% =============================================================================

We present now our contributions of this dissertation addressing the previously identified problems concerning high-level low-level programming in a dynamic language:

\begin{description}
	\item[\B] is a high-level low-level programming framework written in \urlfootnote{\PH}{http://pharo.org/}.
	The core functionality of \B is to dynamically execute native-code generated at language-side.
	Our framework requires minimal changes to an existing \VM and three custom primitives to support dynamic code activation, the majority of \B is implemented as accessible language-side code.
	\B allows us to directly communicate with the low-level world and thus hoist typical \VM-level applications to the language-side.
		
	\item[\NB] is a \B-based foreign function interface (\FFI).
	\NB generates customized native code at language-side, both being flexible and efficient at the same time.
	\NB outperforms other existing \FFI solutions on the \PH platform, making it an ideal evaluation for the \B framework.
	
	\item[\NBJ] is a prototype \JIT compiler based on \B.
	\NBJ generates the same native code as the \VM-level \JIT by compiling the high-level bytecode intermediate format at language-side.
	Our \B-based \JIT prototype reuses existing \VM-level infrastructure and focuses only on the dynamic code generation.
	However, since there is no well-defined interface with the \VM \NBJ requires an extended \VM with an improved \JIT interface to dynamically install native code.
	
	\item[\AsmJIT] is a assembler framework written in \PH.
	\AsmJIT is the low-level backend for the previously mentioned \B framework.
	We extended the existing assembler framework to support the full 64-bit x86 instruction set.

%	\item[\Eye] is a high-level inspector framework that has now been adopted in \PH.
%	Inspectors are crucial when interacting with the low-level world which typically lacks the structural abstractions present at language-side.
%	\Eye allows us to rapidly define customized views for \PH objects.
%	Our inspector framework seamlessly integrates into the existing \PH debugger.
%	\sm{Eye is very different, the introspection aspect is very important for low level programming, I think you'll need to adapt your story line to include that aspect explicitly, if you want to include eye/emphasise it}
	
\end{description}


% =============================================================================
\section{Outline}
% =============================================================================
%\sm{This dissertation structure is different to what I am used to. At least the way you announce the purpose of the chapters is not what I would expect.
%In my diss, everything revolves around one thesis, here, it is a number of things listed one after another, don't see the central motive I would expect}

\begin{description}
\item[\chapref{background}] sheds light on the context of this work.
	We present a quick overview of language-side reflection followed by a development of \VM-level reflection.
	We find that mostly metacircular \VMs provide limited \VM-level reflection and thus we present several high-level language \VMs falling into this category.
	We conclude that there is only two research \VM that has a uniform model for \VM and language-side.
	Among them is \P a research \ST \VM we contributed to previous to working on this dissertation.

%\item[\chapref{reification}] focuses on language-side applications that simplify the interaction with the underlying \VM.
%	We present a custom inspector framework that is now used by default in \PH.
%	As a second part we explain how we introduced first-class layouts and slots to \PH to reify the low-level structural layout of objects.
%	Both projects are crucial for metacircular \VM development and are direct results from the research conducted on the \P \VM.

\item[\chapref{benzo}] describes a high-level low-level programming framework named \B.
	The core functionality of \B is to dynamically execute native-code generated at language-side.
	\B allows us to hoist typical \VM plugins to the language-side.
	Furthermore we show how code caching makes \B efficient and users essentially only pay a one-time overhead for generating the native code.
	
\item[\chapref{ffi}] presents \NB, a stable foreign function interface (\FFI) implementation that is entirely written at language-side using \B.
	\NB is a real-world validation of \B as it combines both language-side flexibility with \VM-level performance.
	We show in detail how \NB outperforms other existing \FFI solutions on \PH.

\item[\chapref{validation}] focuses on two further \B applications.
	In the first part we present \WF a framework for dynamically generating primitives at runtime.
	\WF extends the concept of metacircularity to the running language by reusing the same sources for dynamic primitives that were previously used to generate the static \VM artifact.
	In a first validation we show how \WF outperforms other reflective language-side solutions to instrument primitives.
	
	In a second part of \chapref{validation} we present \NBJ a prototype \JIT compiler that is based on \B.
	\NBJ shows the limitations of the \B approach as it required a customized \VM to communicate with the existing \JIT interface for native code.
	Our prototype implementation generates the same native code as the existing \VM-level \JIT, however, it is currently limited to simple expressions.
	\NBJ shows that for certain applications a well-define interface with the low-level components of the \VM is required.

\item[\chapref{future}] summarizes the limitations of \B and its application.
Furthermore we list undergoing efforts on the \B infrastructure and future work.

\item[\chapref{conclusion}] concludes the dissertation.

\end{description}

% =============================================================================
\input{chapter-footer.tex}
% =============================================================================